\chapter{Entwicklung einer klaren Erwartungshaltung}
\label{chap:erwartungshaltung}

\section{Beschreibung der Interviews mit Führungskräften zur Identifizierung von Stakeholder-Erwartungen}
\label{sec:beschreibung-der-interviews}

Zur Beantwortung der Forschungsfragen 2 und 3 dieser Arbeit werden
Experteninterviews mit E-Learning Experten durchgeführt. Nachdem für die
Beantwortung der Forschungsfrage eins bereits auf die Theorie eingegangen wurde, soll
nun diese durch praktische Erfahrungen ergänzt werden. In Forschungsfrage 2 werden
die Anforderungen der Praktiker an einen kultursensitiven Leitfaden herausgearbeitet.
Um diese Anforderungen angemessen herauszuarbeiten ist eine ausführliche
Vorbereitung notwendig. Diese Vorbereitung wird in diesem Kapitel erläutert. Zu
Beginn dieses Kapitels wird allgemein auf die qualitative Forschung eingegangen und
im Anschluss daran auf die Vorbereitung der Interviews sowie die Vorgehensweise bei
der Durchführung der Interviews. Das Kapitel schließt mit einer kurzen Einordnung der
Interviewpartner, bezüglich Haupttätigkeitsfeld im E-Learning und der
Mitarbeiteranzahl des Unternehmens, ab. 


Methodik erklären, siehe Wirtschaftsinformatik Bachelorarbeit (S.34)
genau die Schritte der Fragen erklären. Warum diese Reihenfolge

nochmal eine bedarfsmeldung genau erklären und die schritte wie bedarfsmeldung erhalten, gepflegt etc wird erklären

\begin{enumerate}
	\item Welche Art von Projekten sind typischerweise in Ihrem Unternehmen an der Tagesordnung? Können Sie uns Beispiele für verschiedene Arten von Projekten geben, die \emph{adesso} durchführt?
	\item Wie werden Projektbedarfe und -anforderungen innerhalb von \emph{adesso} typischerweise kommuniziert und dokumentiert?
	\item Welche Informationen halten Sie in einer Bedarfsmeldung für besonders wichtig oder unverzichtbar?
	\item Wie detailliert sollten Projektbeschreibungen Ihrer Meinung nach sein? Sind bestimmte Schlüsselaspekte oder -informationen in jeder Bedarfsmeldung enthalten?
	\item Welche Herausforderungen oder Schwierigkeiten sind bei unklaren oder unvollständigen Bedarfsmeldungen aufgetreten?
	\item Wer sind die typischen Stakeholder bei der Erstellung von Bedarfsmeldungen und welche Rolle spielen sie?
	\item Wie wird die Qualität von Bedarfsmeldungen bei \emph{adesso} bewertet? Gibt es bestimmte Kriterien oder Standards, anhand derer Bedarfsmeldungen beurteilt werden?
	\item Wie können Sie die Qualität und Klarheit von Bedarfsmeldungen verbessern?
	\item Welche Auswirkungen haben unklare oder fehlende Informationen in Bedarfsmeldungen auf die Effizienz und den Erfolg von Projekten?
	\item Wie können Sie sicherstellen, dass die Bedürfnisse und Anforderungen aller relevanten Stakeholder in einer Bedarfsmeldung angemessen berücksichtigt werden?
\end{enumerate}
\newpage
g
\newpage
g
\newpage
g
\newpage
g
\newpage

\section{Analyse der Ergebnisse und Entwicklung einer klaren Erwartungshaltung für die Bedarfsmeldungen}

1. Transkription der Interviews:\\
Falls du die Interviews aufgezeichnet hast, transkribiere sie vollständig und genau. Dadurch hast du eine schriftliche Version der Aussagen der Experten, die du leichter analysieren kannst.\\

2. Codierung der Daten:\\
Gehe durch die transkribierten Interviews und markiere oder kodiere relevante Themen, Aussagen oder Muster. Verwende dabei Codes oder Kategorien, die sich auf deine Forschungsfragen beziehen.\\

3. Thematische Analyse:\\
Führe eine thematische Analyse durch, indem du die kodierten Daten systematisch durchgehst und nach wiederkehrenden Themen oder Mustern suchst. Identifiziere Gemeinsamkeiten, Unterschiede oder interessante Einsichten, die sich aus den Aussagen der Experten ergeben.\\

4. Triangulation:\\
Vergleiche die Ergebnisse der Experteninterviews mit anderen Quellen, wie beispielsweise der Literatur, Fallstudien oder empirischen Daten. Durch die Triangulation kannst du die Glaubwürdigkeit und Validität deiner Ergebnisse erhöhen.\\

5. Interpretation der Ergebnisse:\\
Interpretiere die identifizierten Themen oder Muster im Kontext deiner Forschungsfragen und -ziele. Versuche zu verstehen, welche Bedeutung oder Implikationen die Aussagen der Experten für deine Forschung haben könnten.\\

6. Reflexion und Kritik:\\
Reflektiere kritisch über die Aussagen der Experten und die gewonnenen Erkenntnisse. Berücksichtige mögliche Einschränkungen oder Bias in den Interviews und betrachte die Ergebnisse aus verschiedenen Perspektiven.\\

7. Integration in die Gesamtanalyse:\\
Integriere die Ergebnisse der Experteninterviews in deine Gesamtanalyse deiner Bachelorarbeit. Verknüpfe sie mit anderen Forschungsergebnissen, theoretischen Konzepten oder empirischen Daten, um ein umfassendes Verständnis deines Forschungsthemas zu entwickeln.\\

8. Darstellung der Ergebnisse:\\
Präsentiere die wichtigsten Ergebnisse und Erkenntnisse aus den Experteninterviews in deiner Bachelorarbeit. Verwende geeignete Zitate oder Beispiele, um die Aussagen der Experten zu veranschaulichen und deine Argumentation zu unterstützen.
\cite{maguire2002user}

im anhang sind die transskripte
wenn man nicht ne größere anzahl an infos hat gucken ob man das halb automatisch evaluieren. Vielleicht kategorisieren. Infos die wichtig sind gucken ob die dann auch nach dem preprocessing drin sind. Regressive tests schreiben.

transformation von bedarfsmeldung zu guter bedarfsmeldung, was ist der fokus von der bedarfsmeldung, wie gut machen die ansätze das, und muss man das dann noch weiter verarbeiten, haben wir alles was wir brauchen mit nur einem algorithmus, inferenz falls parameter fehlt, gibt es einen der alles löst

Fragen in das proposal aufnehmen, führungskraft vorher fragen ob die fragen nice sind.
\newpage
g
\begin{enumerate}
	\item Wer sind die typischen Stakeholder bei der Erstellung von Bedarfsmeldungen und welche
	Rolle spielen sie?
	\item Welche Art von Projekten sind typischerweise in Ihrem Unternehmen an der Tagesordnung?
	Können Sie uns Beispiele für verschiedene Arten von Projekten geben, die adesso
	durchführt?
	\item Wie werden Projektbedarfe und -anforderungen innerhalb von adesso typischerweise
	kommuniziert und dokumentiert?
	\item Welche Informationen halten Sie in einer Bedarfsmeldung für besonders wichtig oder
	unverzichtbar?
	\item Wie detailliert sollten Projektbeschreibungen Ihrer Meinung nach sein? Sind bestimmte
	Schlüsselaspekte oder -informationen in jeder Bedarfsmeldung enthalten?
	\item Wie wird die Qualität von Bedarfsmeldungen bei adesso bewertet? Gibt es bestimmte
	Kriterien oder Standards, anhand derer Bedarfsmeldungen beurteilt werden?
	\item Wie können Sie die Qualität und Klarheit von Bedarfsmeldungen verbessern?
	\item Welche Herausforderungen oder Schwierigkeiten sind bei unklaren oder unvollständigen
	Bedarfsmeldungen aufgetreten?
	\item Welche Auswirkungen haben unklare oder fehlende Informationen in Bedarfsmeldungen
	auf die Effizienz und den Erfolg von Projekten?
	\item Wie können Sie sicherstellen, dass die Bedürfnisse und Anforderungen aller relevanten
	Stakeholder in einer Bedarfsmeldung angemessen berücksichtigt werden?
\end{enumerate}
\newpage
g
\newpage