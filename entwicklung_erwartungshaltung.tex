\chapter{Entwicklung einer klaren Erwartungshaltung}
\label{chap:erwartungshaltung}
Dieses Kapitel befasst sich mit der Methodologie und Durchführung von Experteninterviews mit dem Ziel ...

\section{Beschreibung der Interviews mit Führungskräften zur Identifizierung von Stakeholder-Erwartungen}
\label{sec:beschreibung-der-interviews}

Im Rahmen der vorliegenden Ausarbeitung werden halbstrukturierte Interviews mit Experten aus dem Bereich <> durchgeführt. Sinn und Zweck von Experteninterviews ist die Rekonstruktion spezifischer Wissensbestände oder besonders exklusiver, detaillierter oder umfassender Kenntnisse über bestimmte Wissensbestände und Praktiken.\\

Der Begriff \emph{Experte} bezeichnet eine Person, die über einen privilegierten Zugang zu Informationen verfügt \cite{pfadenhauer2009eye}. Die Expertise eines Experten ist jedoch nicht allein durch die Informationen definiert, über die er exklusiv verfügt \cite{pfadenhauer2009eye}. Auch die Verantwortung für Problemlösungsentscheidungen ist ein entscheidender Faktor \cite{pfadenhauer2009eye}. Diesbezüglich ist Kompetenz erforderlich, die mit Verantwortung und mit Fähigkeiten sowie mit der Bereitschaft, Verantwortung zu übernehmen, verbunden ist \cite{pfadenhauer2009eye}. Dabei ist zu beachten, dass Verantwortung, Fähigkeiten und Bereitschaft in der Regel zusammenfallen \cite{pfadenhauer2009eye}.\\ 

Die Experteninterviews in dieser Ausarbeitung zielen darauf ab, qualitative Daten zu erheben. Die Interviews werden als Einzelinterviews durchgeführt, wodurch der Fokus auf das spezifische Wissen jedes Befragten gerichtet werden kann. Jeder Interviewpartner reagiert individuell aufgrund seines eigenen Vorwissens auf die Interviewfragen und beeinflusst daher nicht die Aussagen anderer Interviewteilnehmer. Die Ergebnisse der Interviews bilden die Grundlage für die Formulierung der Anforderungen einer optimalen \emph{Bedarfsmeldung}, welche als Basis für das zu entwickelnde System eingesetzt wird. Die Interviews wurden in Teams abgehalten und jedes Interview wird zu Dokumentationszwecken aufgezeichnet. Um die Interviews strukturiert für die qualitative Inhaltsanalyse vorzubereiten, ist es erforderlich, sie vorher in schriftliche Transkripte umzuwandeln. Zur ersten Umwandlung in Text wurde das Transkriptionstool von Teams verwendet. Ungenaue Umwandlungen wurden mit der Videoaufnahme nachgebessert. Grundsätzlich wird die einfache Transkribierung nach Dresing und Pehl angewandt \cite{dresing2015praxisbuch}. Im Rahmen dieses Schritts wird der Text vom Umgangssprachlichem in einen gut Lesbaren Text ohne Lücken übersetzt. Der Interviewer wurde mit einem „I“ und die jeweils befragte Person mit „B“ gekennzeichnet. Im Rahmen des Transkriptionsprozesses werden die Aussagen der Interviews anonymisiert. Dies bedeutet, dass sämtliche personenbezogenen Informationen, wie Vor- und Nachname, durch neutrale Bezeichnungen ersetzt werden.\\

Im Vorfeld der Durchführung der Interviews wurde eine Überprüfung der inhaltlichen Verständlichkeit der Fragen sowie ihrer Beantwortbarkeit vorgenommen. Zudem wurde Feedback zur Reihenfolge der Fragen eingeholt. Zu diesem Zweck wurden die Fragen vorab an eine Führungsperson geschickt und schriftlich beantwortet. Die Fragen wurden den Experten vorab inklusive Kontext des Interviews geschickt, damit diese sich bei Bedarf Gedanken machen können. Zur zeitlichen Begrenzung wird das Interview auf zehn Fragen reduziert. Trotz der vorgegebenen Strukturierung des Interviews wird Raum für spontane Fragen gelassen, um eine natürliche Gesprächsführung zu ermöglichen. Die Fragen dienen als Orientierungshilfe und Leitfaden durch das Interview. Der Leitfaden ist dabei lediglich als inhaltliche Richtlinie zu verstehen, von der situativ abgewichen werden kann.
\section{Ablauf der Interviews}
Im Rahmen der Interviews erfolgt zunächst eine Einführung in die Problemstellung sowie das Ziel der Ausarbeitung. Auf diese Weise wird sichergestellt, dass die Interviewpartner den Sinn und Zweck des Interviews nachvollziehen können. Des Weiteren erfolgt eine definitorische Erläuterung des Begriffs \emph{Bedarfsmeldung}. Auch wenn dies die fachliche Bezeichnung darstellt, ist sie nicht jedem einzelnen Experten geläufig. Im Rahmen der Einführung erfolgt zunächst eine Erörterung der Rolle des Experten bei \emph{adesso}. Im Rahmen dessen erfolgt eine Klärung der genauen Aufgaben des Experten bei \emph{adesso}. Im Anschluss erfolgt eine Erörterung der Rolle des Experten im Kontext des Staffing-Prozesses sowie der Bearbeitung der \emph{Bedarfsmeldung}. Die Interviewfragen sind so konzipiert, dass sie zunächst allgemein gehalten sind und im Verlauf des Interviews zunehmend präziser werden. Im Anschluss an die Erörterung der Frage, welche Art von Projekten über \emph{Bedarfsmeldungen} erfasst wird, wird die Frage aufgeworfen, auf welche Weise diese kommuniziert und dokumentiert werden. Im Anschluss erfolgt eine Klärung der für die Erfassung von \emph{Bedarfsmeldungen} relevanten Informationen. Dabei wird erörtert, welche Informationen von besonderer Bedeutung sind und folglich nicht fehlen dürfen. Die Herausforderungen und Bewertungskriterien geben Aufschluss über die bereits genutzten Ansätze zur Standardisierung der \emph{Bedarfsmeldungen}. Des Weiteren können bereits unternommene Maßnahmen zur Qualitätssicherung als hilfreicher Ansatz zur Identifizierung von Anforderungen der \emph{Bedarfsmeldungen} herangezogen werden. Die Beantwortung der Fragen erfordert insbesondere die Entwicklung eigener Ideen und Konzepte, die im weiteren Verlauf des Prozesses zur Erfassung von \emph{Bedarfsmeldung} gegebenenfalls noch nicht zum Einsatz gekommen sind.
\section{Übersicht der Experten}

\begin{tabularx}{1\textwidth} { 
		| >{\raggedright\arraybackslash}X 
		| >{\raggedright\arraybackslash}X
		| >{\raggedright\arraybackslash}X
		| >{\raggedright\arraybackslash}X | }
	\hline
	Nr. und Datum
	& Profil der Befragten & Durchführungsart & Dauer\\
	\hline
	1. 28.03.2024 & CC-Leiter und Softwarearchitekt & schriftlicher Vorabtest & -\\
	\hline
	2. 29.04.2024 & CC-Leiter und Projektleiter & Video-Interview & 20min\\
	\hline
	3. 29.04.2024 & CC-Leiter und Delivery Manager & Video-Interview & 45min\\
	\hline
	4. 30.04.2024 & CC-Leiter und Softwarearchitekt & Video-Interview & 25min\\
	\hline
	5. 6.05.2024 & CC-Leiter und Bereichsleiter & Video-Interview & 25min\\
	\hline
\end{tabularx}\\

Die erste Spalte der Tabelle zeigt die vergebene Nummerierung sowie das Datum der Durchführung. Dies dient der vereinfachten Referenzierung innerhalb der nachfolgenden Analyse. Die zweite Spalte der Tabelle enthält die Rolle bzw. Tätigkeit der Befragten bei \emph{adesso}. Alle Befragten haben, oder hatten eine Leitende Rolle mit Erfahrungen in der Personaleinsatzplanung. Dementsprechend hat jeder Befragte in irgend einer Form Berührungspunkte mit Bedarfsmeldungen gehabt. Interview 2 und 3 haben zusätzlich noch die Perspektive zur Erstellung und Verwaltung von Bedarfsmeldungen, da ihre Hauptaufgaben genau in diesem Bereich Fallen. Die 3. Spalte zeigt die Art der Befragung. Abgesehen von dem Vorabtest wurden alle Interviews in einem Video-Call abgehalten.\\

Die erste Befragte Person aus dem schriftlichen Vorabtest ist studierter Informatiker mit Diplomabschluss. Bei \emph{adesso} nimmt er die Rolle des CC-Leiters ein. Dies ist die Bezeichnung für Führungspersonen mit Zuständigkeiten für Mitarbeiter bei \emph{adesso}. CC-Leiter haben mit Bedarfsmeldungen zu tun, da sie ihre eigenen Mitarbeiter auf Projekte einstellen müssen. Neben der Tätigkeit als Führungsperson ist er Softwarearchitekt und leitet teilweise Projekte.\\

Die zweite befragte Person hat eine Ausbildung zum Fachinformatiker für Anwendungsentwicklung in einem Softwarehaus abgeschlossen. 7 Jahren Erfahrungen als Consultant wurde er innerhalb von 12 Jahren vom Softwareentwickler zum IT-Leiter einer mittelständigen Bank. Nun ist er als CC-Leiter und Projektleiter bei \emph{adesso} tätig.\\

Die dritte befragte Person war während seiner Laufbahn bei \emph{adesso} als CC-Leiter tätig. Aktuell ist er in einer Mischform von Projektleitung und Produktunterschiede um interne Projekte. Zudem übernimmt er auch noch die Rolle des Delivery Managers, bei dem er eine Schnittstellenposition zwischen \emph{adesso} und den Kunden einnimmt. Dabei geht es unter anderem auch um die Verwaltung der \emph{Bedarfsmeldungen}.\\

Die vierte Befragte Person ist Studierter Kerninformatiker mit einem Diplomabschluss. 2003 startete die Karriere als Sotfware Entwickler und seitdem ist er Software Architekt und CC-Leiter bei \emph{adesso}.\\

Die fünfte Person ist studierter Informatiker mit einem Diplomabschluss. Mit Umwegen ist er im Bereich der Dienstleistung gewechselt. Bei \emph{adesso} ist er CC-Leiter und Bereichsleiter. Er übernimmt das Kunden Management und ist für die Gemeinsame Gestaltung der Zusammenarbeit mit Kunden, aber nicht so sehr für einzelne Bedarfsmeldungen zuständig.\\




%Zur Beantwortung der Forschungsfragen 2 und 3 dieser Arbeit werden
%Experteninterviews mit E-Learning Experten durchgeführt. Nachdem für die
%Beantwortung der Forschungsfrage eins bereits auf die Theorie eingegangen wurde, soll
%nun diese durch praktische Erfahrungen ergänzt werden. In Forschungsfrage 2 werden
%die Anforderungen der Praktiker an einen kultursensitiven Leitfaden herausgearbeitet.
%Um diese Anforderungen angemessen herauszuarbeiten ist eine ausführliche
%Vorbereitung notwendig. Diese Vorbereitung wird in diesem Kapitel erläutert. Zu
%Beginn dieses Kapitels wird allgemein auf die qualitative Forschung eingegangen und
%im Anschluss daran auf die Vorbereitung der Interviews sowie die Vorgehensweise bei
%der Durchführung der Interviews. Das Kapitel schließt mit einer kurzen Einordnung der
%Interviewpartner, bezüglich Haupttätigkeitsfeld im E-Learning und der
%Mitarbeiteranzahl des Unternehmens, ab. 


%Methodik erklären, siehe Wirtschaftsinformatik Bachelorarbeit (S.34)
%genau die Schritte der Fragen erklären. Warum diese Reihenfolge

%nochmal eine bedarfsmeldung genau erklären und die schritte wie bedarfsmeldung erhalten, gepflegt etc wird erklären

%\begin{enumerate}
%	\item Welche Art von Projekten sind typischerweise in Ihrem Unternehmen an der Tagesordnung? Können Sie uns Beispiele für verschiedene Arten von Projekten geben, die \emph{adesso} durchführt?
%	\item Wie werden Projektbedarfe und -anforderungen innerhalb von \emph{adesso} typischerweise kommuniziert und dokumentiert?
%	\item Welche Informationen halten Sie in einer Bedarfsmeldung für besonders wichtig oder unverzichtbar?
%	\item Wie detailliert sollten Projektbeschreibungen Ihrer Meinung nach sein? Sind bestimmte Schlüsselaspekte oder -informationen in jeder Bedarfsmeldung enthalten?
%	\item Welche Herausforderungen oder Schwierigkeiten sind bei unklaren oder unvollständigen Bedarfsmeldungen aufgetreten?
%	\item Wer sind die typischen Stakeholder bei der Erstellung von Bedarfsmeldungen und welche Rolle spielen sie?
%	\item Wie wird die Qualität von Bedarfsmeldungen bei \emph{adesso} bewertet? Gibt es bestimmte Kriterien oder Standards, anhand derer Bedarfsmeldungen beurteilt werden?
%	\item Wie können Sie die Qualität und Klarheit von Bedarfsmeldungen verbessern?
%	\item Welche Auswirkungen haben unklare oder fehlende Informationen in Bedarfsmeldungen auf die Effizienz und den Erfolg von Projekten?
%	\item Wie können Sie sicherstellen, dass die Bedürfnisse und Anforderungen aller relevanten Stakeholder in einer Bedarfsmeldung angemessen berücksichtigt werden?
%\end{enumerate}

\section{Analyse der Ergebnisse und Entwicklung einer klaren Erwartungshaltung für die Bedarfsmeldungen}

1. Transkription der Interviews:\\
Falls du die Interviews aufgezeichnet hast, transkribiere sie vollständig und genau. Dadurch hast du eine schriftliche Version der Aussagen der Experten, die du leichter analysieren kannst.\\

2. Codierung der Daten:\\
Gehe durch die transkribierten Interviews und markiere oder kodiere relevante Themen, Aussagen oder Muster. Verwende dabei Codes oder Kategorien, die sich auf deine Forschungsfragen beziehen.\\

3. Thematische Analyse:\\
Führe eine thematische Analyse durch, indem du die kodierten Daten systematisch durchgehst und nach wiederkehrenden Themen oder Mustern suchst. Identifiziere Gemeinsamkeiten, Unterschiede oder interessante Einsichten, die sich aus den Aussagen der Experten ergeben.\\

4. Triangulation:\\
Vergleiche die Ergebnisse der Experteninterviews mit anderen Quellen, wie beispielsweise der Literatur, Fallstudien oder empirischen Daten. Durch die Triangulation kannst du die Glaubwürdigkeit und Validität deiner Ergebnisse erhöhen.\\

5. Interpretation der Ergebnisse:\\
Interpretiere die identifizierten Themen oder Muster im Kontext deiner Forschungsfragen und -ziele. Versuche zu verstehen, welche Bedeutung oder Implikationen die Aussagen der Experten für deine Forschung haben könnten.\\

6. Reflexion und Kritik:\\
Reflektiere kritisch über die Aussagen der Experten und die gewonnenen Erkenntnisse. Berücksichtige mögliche Einschränkungen oder Bias in den Interviews und betrachte die Ergebnisse aus verschiedenen Perspektiven.\\

7. Integration in die Gesamtanalyse:\\
Integriere die Ergebnisse der Experteninterviews in deine Gesamtanalyse deiner Bachelorarbeit. Verknüpfe sie mit anderen Forschungsergebnissen, theoretischen Konzepten oder empirischen Daten, um ein umfassendes Verständnis deines Forschungsthemas zu entwickeln.\\

8. Darstellung der Ergebnisse:\\
Präsentiere die wichtigsten Ergebnisse und Erkenntnisse aus den Experteninterviews in deiner Bachelorarbeit. Verwende geeignete Zitate oder Beispiele, um die Aussagen der Experten zu veranschaulichen und deine Argumentation zu unterstützen.
\cite{maguire2002user}

im anhang sind die transskripte
wenn man nicht ne größere anzahl an infos hat gucken ob man das halb automatisch evaluieren. Vielleicht kategorisieren. Infos die wichtig sind gucken ob die dann auch nach dem preprocessing drin sind. Regressive tests schreiben.

transformation von bedarfsmeldung zu guter bedarfsmeldung, was ist der fokus von der bedarfsmeldung, wie gut machen die ansätze das, und muss man das dann noch weiter verarbeiten, haben wir alles was wir brauchen mit nur einem algorithmus, inferenz falls parameter fehlt, gibt es einen der alles löst

Fragen in das proposal aufnehmen, führungskraft vorher fragen ob die fragen nice sind.
\newpage
g
\begin{enumerate}
	\item Wer sind die typischen Stakeholder bei der Erstellung von Bedarfsmeldungen und welche
	Rolle spielen sie?
	\item Welche Art von Projekten sind typischerweise in Ihrem Unternehmen an der Tagesordnung?
	Können Sie uns Beispiele für verschiedene Arten von Projekten geben, die adesso
	durchführt?
	\item Wie werden Projektbedarfe und -anforderungen innerhalb von adesso typischerweise
	kommuniziert und dokumentiert?
	\item Welche Informationen halten Sie in einer Bedarfsmeldung für besonders wichtig oder
	unverzichtbar?
	\item Wie detailliert sollten Projektbeschreibungen Ihrer Meinung nach sein? Sind bestimmte
	Schlüsselaspekte oder -informationen in jeder Bedarfsmeldung enthalten?
	\item Wie wird die Qualität von Bedarfsmeldungen bei adesso bewertet? Gibt es bestimmte
	Kriterien oder Standards, anhand derer Bedarfsmeldungen beurteilt werden?
	\item Wie können Sie die Qualität und Klarheit von Bedarfsmeldungen verbessern?
	\item Welche Herausforderungen oder Schwierigkeiten sind bei unklaren oder unvollständigen
	Bedarfsmeldungen aufgetreten?
	\item Welche Auswirkungen haben unklare oder fehlende Informationen in Bedarfsmeldungen
	auf die Effizienz und den Erfolg von Projekten?
	\item Wie können Sie sicherstellen, dass die Bedürfnisse und Anforderungen aller relevanten
	Stakeholder in einer Bedarfsmeldung angemessen berücksichtigt werden?
\end{enumerate}
\newpage
g
\newpage