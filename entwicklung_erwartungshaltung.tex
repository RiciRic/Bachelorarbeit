\chapter{Strukturierte Bedarfsmeldung}
\label{chap:erwartungshaltung}
In diesem Kapitel werden Methodiken und die Durchführung von Experteninterviews erörtert, mit dem Ziel, wichtige Aspekte einer \emph{Bedarfsmeldung} durch eine qualitative Inhaltsanalyse zu identifizieren. Auf Basis dieser Erkenntnisse wird eine strukturierte Vorlage von \emph{Bedarfsmeldungen} entwickelt, die die Anforderungen an das zu entwickelnde System zur Extraktion relevanter Informationen aus semistrukturierten \emph{Bedarfsmeldungen} definiert.
\section{Bedarfsmeldungen}\mbox{} \\
Eine \emph{Bedarfsmeldung} bezeichnet eine Projektbeschreibung, die Anforderungen an ein zu entwickelndes System enthält. Sie umfasst mehrere wesentliche Aspekte, die dafür sorgen klar und effektiv zu sein. Dazu gehört eine Überschrift, die die Projektart und den Bedarf kurz zusammenfasst. Die Beschreibung spiegelt den Bedarf wider und durch den Einsatzkontext können wichtige Hintergrundinformationen zum Umfeld des Projekts erklärt werden. Das Datum gibt Aufschluss über den Zeitplan, in welchem das Projekt durchgeführt werden soll. Durch das Volumen können benötigte Ressourcen und den Umfang des Bedarfs dokumentiert werden. Eine \emph{Bedarfsmeldung} sollte den Tagessatz sowie den Einsatzbereich klar definieren. Zudem ist es wichtig, die Dauer des Einsatzes und den Technologie-Stack anzugeben, der zum Einsatz kommt. Darüber hinaus sollten die Muss- und Kann-Kriterien für die gesuchten Fachkräfte klar formuliert werden. Einarbeitungszeiträume und Lieferverpflichtungen sind ebenfalls entscheidende Bestandteile, die in einer \emph{Bedarfsmeldung} nicht fehlen dürfen.\\
\section{Experteninterviews}
\label{sec:beschreibung-der-interviews}
Im Rahmen der vorliegenden Ausarbeitung werden halbstrukturierte Interviews mit Experten aus dem Bereich des \emph{Staffings} durchgeführt. Sinn und Zweck von Experteninterviews ist die Rekonstruktion spezifischer Wissensbestände oder besonders exklusiver, detaillierter oder umfassender Kenntnisse über bestimmte Wissensbestände und Praktiken. Der Begriff \emph{Experte} bezeichnet hier eine Person, die über einen privilegierten Zugang zu Informationen verfügt \cite{pfadenhauer2009eye}.\\ 

Die Experteninterviews in dieser Ausarbeitung zielen darauf ab, qualitative Daten zu erheben. Die Interviews werden als Einzelinterviews durchgeführt, wodurch der Fokus auf das spezifische Wissen jedes Befragten gerichtet werden kann. Die Ergebnisse der Interviews bilden die Grundlage für die Formulierung der Anforderungen einer optimalen \emph{Bedarfsmeldung}, die als Basis für das zu entwickelnde System eingesetzt wird. Alle Interviews wurden in Microsoft Teams abgehalten. Die Experten sind einerseits vielfach unterwegs und nicht immer in der Geschäftsstelle anzufinden. Andererseits verfolgen sie einen strikten Zeitplan wodurch Meetings auch spontan verschoben werden. Zur Zeiteinsparanis und Spontanität wurden diese dadurch Online abgehalten. Jedes Interview wird zu Dokumentationszwecken aufgezeichnet. Um die Interviews strukturiert für die qualitative Inhaltsanalyse vorzubereiten, ist es erforderlich sie in schriftliche Transkripte umzuwandeln. Zur ersten Textumwandlung wurde das Transkriptionstool von Teams verwendet. Ungenaue Umwandlungen wurden mit der Videoaufnahme nachgebessert. Grundsätzlich wird die einfache Transkribierung nach Dresing und Pehl angewandt, da nur wert auf den Inhalt der Interviews gelegt wird \cite{dresing2015praxisbuch}. Im Rahmen dieses Schritts wird der Text vom Umgangssprachlichem in einen gut Lesbaren Text ohne Lücken übersetzt \cite{dresing2015praxisbuch}. Der Interviewer wurde mit einem „I“ und die jeweils befragte Person mit „B“ gekennzeichnet. Im Rahmen des Transkriptionsprozesses werden die Aussagen der Interviews anonymisiert. Personenbezogenen Informationen werden somit durch neutrale Bezeichnungen ersetzt.\\

Im Vorfeld der Durchführung der Interviews wird eine Überprüfung der inhaltlichen Verständlichkeit der Fragen sowie ihrer Beantwortbarkeit vorgenommen. Zudem wird Feedback zur Reihenfolge der Fragen eingeholt. Zu diesem Zweck sind die Fragen vorab an eine Führungsperson geschickt und schriftlich beantwortet worden. Die Fragen werden den Experten vorab inklusive Kontext des Interviews geschickt, damit diese sich bei Bedarf Gedanken machen können. Zur zeitlichen Begrenzung wird das Interview auf zehn Fragen reduziert. Trotz der vorgegebenen Strukturierung des Interviews wird Raum für spontane Fragen gelassen, um eine natürliche Gesprächsführung zu ermöglichen. Die Fragen dienen als Orientierungshilfe und Leitfaden durch das Interview. Der Leitfaden ist dabei lediglich als inhaltliche Richtlinie zu verstehen, von der gegebenenfalls abgewichen werden kann.
\section{Ablauf der Interviews}
\label{sec:ablaufexperteninterviews}
Im Rahmen der Interviews erfolgt zunächst eine Einführung in die Problemstellung sowie das Ziel der Ausarbeitung. Auf diese Weise wird sichergestellt, dass die Interviewpartner den Sinn und Zweck des Interviews nachvollziehen können. Des Weiteren erfolgt eine definitorische Erläuterung des Begriffs \emph{Bedarfsmeldung}. Auch wenn dies die fachliche Bezeichnung darstellt, ist sie nicht jedem einzelnen Experten geläufig. Im Rahmen der Einführung erfolgt zunächst eine Erörterung der Rolle des Experten bei \emph{adesso}. Im Rahmen dessen erfolgt eine Klärung der genauen Aufgaben des Experten bei \emph{adesso}. Im Anschluss erfolgt eine Erörterung der Rolle des Experten im Kontext des \emph{Staffing}-Prozesses sowie der Bearbeitung der \emph{Bedarfsmeldung}. Die Interviewfragen sind so konzipiert, dass sie zunächst allgemein gehalten sind und im Verlauf des Interviews zunehmend präziser werden. Im Anschluss an die Erörterung der Frage, welche Art von Projekten über \emph{Bedarfsmeldungen} erfasst wird, wird die Frage aufgeworfen, auf welche Weise diese kommuniziert und dokumentiert werden. Im Anschluss erfolgt eine Klärung der für die Erfassung von \emph{Bedarfsmeldungen} relevanten Informationen. Dabei wird erörtert, welche Informationen von besonderer Bedeutung sind und folglich nicht fehlen dürfen. Die Herausforderungen und Bewertungskriterien geben Aufschluss über die bereits genutzten Ansätze zur Standardisierung der \emph{Bedarfsmeldungen}. Des Weiteren können bereits unternommene Maßnahmen zur Qualitätssicherung als hilfreicher Ansatz zur Identifizierung von Anforderungen der \emph{Bedarfsmeldungen} herangezogen werden. Die Beantwortung der Fragen erfordert insbesondere die Entwicklung eigener Ideen und Konzepte, die im weiteren Verlauf des Prozesses zur Erfassung von \emph{Bedarfsmeldung} gegebenenfalls noch nicht zum Einsatz gekommen sind.
\section{Übersicht der Experten}
\label{sec:experten}
Die Übersicht der interviewten Experten ist in der Tabelle \ref{tab:experten} aufgeführt.
\begin{center}
	\begin{tabularx}{1\textwidth} { 
			| >{\raggedright\arraybackslash}X 
			| >{\raggedright\arraybackslash}X
			| >{\raggedright\arraybackslash}X
			| >{\raggedright\arraybackslash}X
			| >{\raggedright\arraybackslash}X | }
		\hline
		Nr. und Datum
		& Profil der Befragten & Durchführungs-art & Dauer & Transkripte (Anhang)\\
		\hline
		\hline
		1. 28.03.2024 & CC-Leiter und Softwarearchitekt & schriftlicher Vorabtest & - & \ref{interview1}\\
		\hline
		2. 29.04.2024 & CC-Leiter und Projektleiter & Video-Interview & 15min & \ref{interview2}\\
		\hline
		3. 29.04.2024 & CC-Leiter und Delivery Manager & Video-Interview & 45min & \ref{interview3}\\
		\hline
		4. 30.04.2024 & CC-Leiter und Softwarearchitekt & Video-Interview & 15min & \ref{interview4}\\
		\hline
		5. 6.05.2024 & Bereichsleiter & Video-Interview & 25min & \ref{interview5}\\
		\hline
	\end{tabularx}\\
	\captionof{table}{Übersicht der Experten}
	\label{tab:experten}
\end{center}
Die erste Spalte der Tabelle \ref{tab:experten} zeigt die vergebene Nummerierung sowie das Datum der Durchführung. Dies dient der vereinfachten Referenzierung innerhalb der Ausarbeitung. Die zweite Spalte der Tabelle enthält die Rolle bzw. Tätigkeit der Befragten bei \emph{adesso}. Alle Befragten haben, oder hatten eine leitende Rolle mit Erfahrungen in der Personaleinsatzplanung. Dementsprechend hat jeder Befragte in irgend einer Form Berührungspunkte mit Bedarfsmeldungen gehabt. Interview 2 und 3 haben zusätzlich noch die Perspektive zur Erstellung und Verwaltung von Bedarfsmeldungen, da ihre Hauptaufgaben genau in diesem Bereich Fallen. Die 3. Spalte zeigt die Art der Befragung. Abgesehen von dem Vorabtest wurden alle Interviews in einem Video-Call abgehalten. Die vierte Spalte zeigt die Dauer der Interviews. Die letzte Spalte ist eine Referenzierung zu den Transkripten der Interviews im Anhang. Die befragten Personen wurden so ausgewählt, dass diese durch unterschiedliche Aufgabenbereiche differenzierte Bezüge zu \emph{Bedarfsmeldungen} pflegen. Alle befragten Personen hatten zu einem Zeitpunkt in ihrer Laufbahn die Rolle einer Führungsperson übernommen und somit Berührungspunkte mit \emph{Bedarfsmeldungen} zur Projektfindung für die eigenen Mitarbeiter hatten. Darüber hinaus hat jeder Experte durch die weiteren Aufgaben innerhalb von \emph{adesso} eine andere Sicht auf die \emph{Bedarfsmeldungen}. Softwarearchitekten legen potenziell mehr Wert auf technischere Aspekte wie benötigte Expertise in Technologien. Ein Delivery Manager, der die Schnittstelle zwischen Unternehmen und Kunde darstellt versucht wiederum wichtige Aspekte aus beiden Sichten zusammenzuführen. Ein Projektleiter hat potenziell wiederum ein anderen Anspruch an eine \emph{Bedarfsmeldung}, da dieser seine eigenen Vorstellungen und Ansprüche an das Projekt durchbringen möchte. Wie sich das genau verhält, wird erst durch die Interviews deutlicher.\\

Die erste Befragte Person aus dem schriftlichen Vorabtest ist studierter Informatiker mit Diplomabschluss. Bei \emph{adesso} nimmt er die Rolle des CC-Leiters ein. Dies ist die Bezeichnung für Führungspersonen mit Zuständigkeiten für Mitarbeiter bei \emph{adesso}. CC-Leiter haben mit Bedarfsmeldungen zu tun, da sie ihre eigenen Mitarbeiter auf Projekte einstellen müssen. Neben der Tätigkeit als Führungsperson ist er Softwarearchitekt und leitet teilweise Projekte.\\

Die zweite befragte Person hat eine Ausbildung zum Fachinformatiker für Anwendungsentwicklung in einem Softwarehaus abgeschlossen. Nach 7 Jahren Erfahrungen als Consultant wurde er innerhalb von 12 Jahren vom Softwareentwickler zum IT-Leiter einer mittelständigen Bank. Nun ist er als CC-Leiter und Projektleiter bei \emph{adesso} tätig.\\

Die dritte befragte Person war während seiner Laufbahn bei \emph{adesso} als CC-Leiter tätig. Aktuell ist er in einer Mischform von Projektleitung und Produktunterschiede um interne Projekte. Zudem übernimmt er auch noch die Rolle des Delivery Managers, bei dem er eine Schnittstellenposition zwischen \emph{adesso} und den Kunden einnimmt. Dabei geht es unter anderem auch um die Verwaltung der \emph{Bedarfsmeldungen}.\\

Die vierte Befragte Person ist Studierter Kerninformatiker mit einem Diplomabschluss. 2003 startete die Karriere als Sotfware Entwickler und seitdem ist er Software Architekt und CC-Leiter bei \emph{adesso}.\\

Die fünfte Person ist promovierter Informatiker mit einem Diplomabschluss. Mit Umwegen ist er im Bereich der Dienstleistung gewechselt. Bei \emph{adesso} ist er CC-Leiter und Bereichsleiter. Er übernimmt das Kunden Management und ist für die Gemeinsame Gestaltung der Zusammenarbeit mit Kunden, aber nicht so sehr für einzelne Bedarfsmeldungen zuständig.\\




%Zur Beantwortung der Forschungsfragen 2 und 3 dieser Arbeit werden
%Experteninterviews mit E-Learning Experten durchgeführt. Nachdem für die
%Beantwortung der Forschungsfrage eins bereits auf die Theorie eingegangen wurde, soll
%nun diese durch praktische Erfahrungen ergänzt werden. In Forschungsfrage 2 werden
%die Anforderungen der Praktiker an einen kultursensitiven Leitfaden herausgearbeitet.
%Um diese Anforderungen angemessen herauszuarbeiten ist eine ausführliche
%Vorbereitung notwendig. Diese Vorbereitung wird in diesem Kapitel erläutert. Zu
%Beginn dieses Kapitels wird allgemein auf die qualitative Forschung eingegangen und
%im Anschluss daran auf die Vorbereitung der Interviews sowie die Vorgehensweise bei
%der Durchführung der Interviews. Das Kapitel schließt mit einer kurzen Einordnung der
%Interviewpartner, bezüglich Haupttätigkeitsfeld im E-Learning und der
%Mitarbeiteranzahl des Unternehmens, ab. 


%Methodik erklären, siehe Wirtschaftsinformatik Bachelorarbeit (S.34)
%genau die Schritte der Fragen erklären. Warum diese Reihenfolge

%nochmal eine bedarfsmeldung genau erklären und die schritte wie bedarfsmeldung erhalten, gepflegt etc wird erklären

%\begin{enumerate}
%	\item Welche Art von Projekten sind typischerweise in Ihrem Unternehmen an der Tagesordnung? Können Sie uns Beispiele für verschiedene Arten von Projekten geben, die \emph{adesso} durchführt?
%	\item Wie werden Projektbedarfe und -anforderungen innerhalb von \emph{adesso} typischerweise kommuniziert und dokumentiert?
%	\item Welche Informationen halten Sie in einer Bedarfsmeldung für besonders wichtig oder unverzichtbar?
%	\item Wie detailliert sollten Projektbeschreibungen Ihrer Meinung nach sein? Sind bestimmte Schlüsselaspekte oder -informationen in jeder Bedarfsmeldung enthalten?
%	\item Welche Herausforderungen oder Schwierigkeiten sind bei unklaren oder unvollständigen Bedarfsmeldungen aufgetreten?
%	\item Wer sind die typischen Stakeholder bei der Erstellung von Bedarfsmeldungen und welche Rolle spielen sie?
%	\item Wie wird die Qualität von Bedarfsmeldungen bei \emph{adesso} bewertet? Gibt es bestimmte Kriterien oder Standards, anhand derer Bedarfsmeldungen beurteilt werden?
%	\item Wie können Sie die Qualität und Klarheit von Bedarfsmeldungen verbessern?
%	\item Welche Auswirkungen haben unklare oder fehlende Informationen in Bedarfsmeldungen auf die Effizienz und den Erfolg von Projekten?
%	\item Wie können Sie sicherstellen, dass die Bedürfnisse und Anforderungen aller relevanten Stakeholder in einer Bedarfsmeldung angemessen berücksichtigt werden?
%\end{enumerate}

\section{Auswertungsmethode}
Zur systematischen Analyse der Experteninterviews wurde die qualitative Inhaltsanalyse nach Philipp Mayring angewendet. Das Ziel der Inhaltsanalyse in dieser Ausarbeitung ist die Zusammentragung der Informationen aus den Interviews. Grundsätzlich unterscheidet sich die Inhaltsanalyse in drei Grundtechniken: (i)\emph{Zusammenfassung} (Materialreduktion, um die wesentlichen Inhalte zu erhalten), (ii)\emph{Explikation} (Verständniserweiterung durch Herantragung von Zusatzmaterial) und (iii)\emph{Strukturierung} (Herausfilterung von Aspekten und Einschätzung von Kriterien) \cite{mayring1994qualitative}.\\

Im Folgenden wird sich auf die Technik der Zusammenfassung konzentriert, da die Anforderungen an \emph{Bedarfsmeldungen} nur durch eine Übersicht der Interviewergebnisse erstellt werden kann.\\

Zur strukturierten Analyse werden die Arbeitsschritte mit Hilfe des Ablaufmodells von Mayring aufgeteilt \cite{mayring2019qualitative}.
%\begin{figure}[H]%htb
%	\centering  
%	\includegraphics[scale=1]{Abbildungen/Ablaufmodell.png}
%	\caption{Allgemeines Ablaufmodell qualitativer Inhaltsanalyse (Mayring, 1988) \cite{mayring2019qualitative}}
%	\label{fig:ablaufmodell}
%\end{figure}\mbox{} \\
\begin{figure}[H]
	\begin{minipage}[b]{.4\linewidth} % [b] => Ausrichtung an \caption
		\includegraphics[scale=1.2]{Abbildungen/Ablaufmodell.png}
		\caption{UML-Aktivitätsdiagramm zum Ablaufmodell der qualitativen Inhaltsanalyse nach Mayring \cite{mayring2019qualitative}}
		\label{fig:ablaufmodell}
	\end{minipage}
	\hspace{.1\linewidth}% Abstand zwischen Bilder
	\begin{minipage}[b]{.4\linewidth} % [b] => Ausrichtung an \caption
		\includegraphics[scale=1.2]{Abbildungen/zusammenfassendeInhaltsanalyse.png}
		\caption{UML-Aktivitätsdiagramm zum Ablaufmodell der zusammenfassenden Inhaltsanalyse nach Mayring \cite{mayring2019qualitative}}
		\label{fig:zusammenfassendeinhaltsanalyse}
	\end{minipage}
\end{figure}
Mayring beschreibt ein allgemeines Ablaufmodell mit Schritten für die qualitative Inhaltsanalyse \cite{mayring2019qualitative}. Das UML-Aktivitätsdiagramm aus der Abbildung \ref{fig:ablaufmodell} beschreibt diesen Ablauf bis zum Schritt \emph{Analyseschritte mittels Kategoriesystem}. In diesem Schritt wird sich auf eine konkrete Analyseform festgelegt. Diese Analyse durchläuft die zusammenfassende Inhaltsanalyse nach Mayring \cite{mayring2019qualitative}. Dadurch wird der Ablauf spezifischer und kann somit nicht mehr mit dem allgemeinen Ablaufmodell dargestellt werden. Die Schritte der zusammenfassenden Inhaltsanalyse nach Mayring werden im UML-Aktivitätsdiagramm in Abbildung \ref{fig:zusammenfassendeinhaltsanalyse} dargestellt.
\section{Durchführung der qualitativen Inhaltsanalyse}
Die Struktur aus den Abbildungen \ref{fig:ablaufmodell} und \ref{fig:zusammenfassendeinhaltsanalyse} wird nachfolgend Schrittweise auf die Interviews angewendet.
\paragraph{1. Festlegung des Materials}\mbox{} \\
Die Materialien sind aus 4 Experteninterviews und einem schriftlichen Vorabtest entnommen worden. Dazu wurden 10 Interviewfragen als Leitfaden angewendet.
\paragraph{2. Analyse der Entstehungssituation}\mbox{} \\
Die Interviewteilnehmer sind Führungskräfte, die bereits mit \emph{Bedarfsmeldungen} Berührungspunkte hatten. Vereinzelt haben diese auch näheren Kontakt mit \emph{Bedarfsmeldungen}. Dies ist in Kapitel \ref{sec:experten} näher erläutert.
\paragraph{3. Formale Charakteristika des Materials}\mbox{} \\
Die Interviews wurden über Microsoft Teams mit der Aufnahmefunktion aufgezeichnet und mit der Teams Transkriptionsfunktion transkribiert. Undeutliche Transkriptionsbereiche wurden mit der Aufnahme nachgebessert. Die Interviews liegen in Textform im Anhang zur Verfügung.
\paragraph{4. Richtung der Analyse}\mbox{} \\
Das Ziel der Analyse ist die Informationsgewinnung aus den Interviews, um die wichtigsten Aspekte von \emph{Bedarfsmeldungen} zu identifizieren. Dabei soll der Fokus auf den Inhalt gelegt werden, wodurch emotionale und sprachliche Faktoren nicht einbezogen werden.
\paragraph{5. Theoretische Differenzierung der Fragestellung}\mbox{} \\
Die Interviews sind auf die Aussagen der Experten aus Teilbereichen der Fachkräfteorganisation und der Akquirierung und Bearbeitung von Projekten und Projektbedarfen zugeschnitten. Es geht um den Informationsgehalt von \emph{Bedarfsmeldungen}. Dafür wurden Themenfelder festgelegt, die Aussagen über die Relevanz und Konsistenz der \emph{Bedarfsmeldungen} behandeln. Der Ablauf und die dazugehörigen Themenfelder wurden bereits in Kapitel \ref{sec:ablaufexperteninterviews} erläutert und spiegeln sich in den Fragen wider.
\paragraph{6. Bestimmung der Analysetechnik und Festlegung des konkreten Ablaufmodells}\mbox{} \\
Zur Extraktion der relevanten Informationen wurde die zusammenfassende Inhaltsanalyse nach Mayring verwendet. Innerhalb der Analyse wird der Text auf wesentliche Inhalte reduziert und in Kategorien unterteilt. Die einzelnen Schritte des Ablaufmodells für eine zusammenfassende Inhaltsanalyse ist in Abbildung \ref{fig:zusammenfassendeinhaltsanalyse} abgebildet.
%\begin{figure}[H]%htb
%	\centering  
%	\includegraphics[scale=0.5]{Abbildungen/zusammenfassendeInhaltsanalyse.png}
%	\caption{Ablaufmodell zusammenfassender Inhaltsanalyse (Mayring, 1988) \cite{mayring2019qualitative}}
%	\label{fig:zusammenfassendeinhaltsanalyse}
%\end{figure}\mbox{} \\
\paragraph{7.Bestimmung der Analyseeinheiten}\mbox{} \\
Mayring beschreibt drei Analyseeinheiten. Die (i)\emph{Auswertungseinheit} definiert, welche Textteile jeweils nacheinander kodiert werden \cite{mayring1994qualitative}. Die (ii)\emph{Kodiereinheit} definiert, welche minimale Materialmenge ausgewertet werden darf und welcher Kategorie sie zugeordnet werden kann \cite{mayring1994qualitative}. Die (iii)\emph{Kontexteinheit} definiert die maximale Textmenge, die unter eine Kategorie fallen kann \cite{mayring1994qualitative}.\\

Die Auswertungseinheit umfasst alle Interviewtranskripte. Die kleinste Textmenge, die kodiert wird betragen einzelne Wörter aus einzelnen Absätzen. Die Kontexteinheit bezieht sich auf die jeweiligen Absätze zu den einzelnen Fragen aus den Interviews.
%\paragraph{Interview Zusammenfassung}\mbox{} \\
%\todo{ich glaube ich brauche diesen Abschnitt nicht...}\\
%In diesem Abschnitt werden die Schritte 2 bis 5 aus dem Ablaufmodell der Abbildung \ref{fig:zusammenfassendeinhaltsanalyse} durchgeführt. Dabei werden alle Interviews heruntergebrochen und paraphrasiert. Die Schritte werden aufgrund der Menge an Materialien in einem Schritt dargestellt.
%\begin{enumerate}
%	\item Wer sind die typischen Stakeholder bei der Erstellung von Bedarfsmeldungen und welche
%	Rolle spielen sie?\\
%	
%	-1. Experte Sales/PL: Anforderer mit den technischen Informationen, Maitre: kümmert sich um das Staffing bzw. die eigentliche Besetzung
%	-Stakeholder sind klassisch die Fachverantwortlichen beim Kunden, aber auch die Entscheider. Sprich also deren Vorgesetzte die quasi fachlich vielleicht das ganze nicht so bewerten können, aber das Budget dafür hergeben müssen und natürlich dann im Zweifelsfall auch CO, CEO oder sogar Geschäftsführer
%	-Kunde, Delivery Manager, als Projektleiter, als Programm Manager, als Account Manager oder Vertriebler mit Kunden sprechen, in Projekte reinschauen, Projektorganisationen definieren und selbst in diesen Rollen eine Bedarfsmeldung erfassen
%	-für die Erstellung der Bedarfsmeldungen der Projektleiter und Maitre eigentlich relevant
%	-Sales oder der interne Maitre dafür zuständig ist, stellt quasi eine Bedarfsmeldung im Normalfall auf Basis von Anforderungen direkt vom Kunden, Oder es kommt konkret im Projekt aus mündlich genannten Bedarfsmeldungen. Das kann auch durch ein Scrum Master oder Product Owner entsprechend entstehen. Da ist dann derjenige, der die Bedarfsmeldung erstellt auch zuständig
%	\item Welche Art von Projekten sind typischerweise in Ihrem Unternehmen an der Tagesordnung?
%	Können Sie uns Beispiele für verschiedene Arten von Projekten geben, die adesso
%	durchführt?\\
%	
%	-Software-Entwicklungsprojekte, angefangen von Projekten in dem ein adessi in einem Kundenprojekt arbeitet über gemischte Teams aus adessi und Kunde bis hin zur kompletten Lieferung von %Projekleitern, Testern, Requirements Engineer und Entwicklern
%	-2 Arten von Projekten teilen. Aus einer sind Time Material Projekte, wo quasi der Kunde mit einer Idee kommt, wo wir gut unterstützen können, Festpreis Projekten haben wir eher die Staffing Hoheit. Das heißt, wir können entscheiden wen wir in das Projekt einsetzen
%	-gibt Projekte, die ein Kunde eine Kundenorganisation aufsetzt und durchführt, an denen wir uns dann beteiligen, indem wir beispielsweise in bestimmte Rollen an bestimmte Stellen dort Menschen reinbringen, Das andere ist, wenn wir im Kundenauftrag Projekte aufsetzen --> Projekte unter unserer Kontrolle im Kundenauftrag und dann gibt es natürlich auch interne Projekte, Consulting
%	-über das Jira erfasst und entsprechend auch mit allen beteiligten Parteien geteilt
%	-demjenigen der es einstellt, werden die Kriterien definiert und in einem JIRA-Ticket überführt --> Felder, die strukturiert sind. Z.B zu welchem Tagessatz das ganze angeboten wird, wann das Ganze startet, wie hoch das Volumen also an Tagen ist. Es gibt einige Freitextfelder, bei dem drinsteht, was die Aufgaben usw. sind. Zum Beispiel bei einer Rahmenvereinbarung, die wir gerade machen gibt es dann ein Excel, was ausgefüllt werden muss mit seiner Selbstbeurteilung, nutzen wir doch JIRA, weil es im Normalfall nicht genau den einen Case gibt, wie wir Bedarfsmeldung reinkriegen
%	
%	TEXT
%	\item Wie werden Projektbedarfe und -anforderungen innerhalb von adesso typischerweise
%	kommuniziert und dokumentiert?\\
%	
%	-Initial über den Maitre, der das Staffing übernimmt bzw. auch Vorschläge von Projektleitenden zum Staffing annimt, teilweise auch über das eigene Netzwerk zwischen Führungskräften, im CC, Bereich oder der LoB. Am Ende über das Staffing Jira
%	-
%	
%	TEXT
%	\item Welche Informationen halten Sie in einer Bedarfsmeldung für besonders wichtig oder
%	unverzichtbar?\\
%
%	TEXT
%	\item Wie detailliert sollten Projektbeschreibungen Ihrer Meinung nach sein? Sind bestimmte
%	Schlüsselaspekte oder -informationen in jeder Bedarfsmeldung enthalten?\\
%	
%	TEXT
%	\item Wie wird die Qualität von Bedarfsmeldungen bei adesso bewertet? Gibt es bestimmte
%	Kriterien oder Standards, anhand derer Bedarfsmeldungen beurteilt werden?\\
%	
%	TEXT
%	\item Wie können Sie die Qualität und Klarheit von Bedarfsmeldungen verbessern?\\
%	
%	TEXT
%	\item Welche Herausforderungen oder Schwierigkeiten sind bei unklaren oder unvollständigen
%	Bedarfsmeldungen aufgetreten?\\
%	
%	TEXT
%	\item Welche Auswirkungen haben unklare oder fehlende Informationen in Bedarfsmeldungen
%	auf die Effizienz und den Erfolg von Projekten?\\
%	
%	TEXT
%	\item Wie können Sie sicherstellen, dass die Bedürfnisse und Anforderungen aller relevanten
%	Stakeholder in einer Bedarfsmeldung angemessen berücksichtigt werden?\\
%	
%	TEXT
%\end{enumerate}
\subsection*{Zusammenfassende Inhaltsanalyse}\mbox{} \\
Die Interviews werden auf die wichtigsten Informationen heruntergebrochen und mit Berücksichtigung der Verfahrensregeln nach Mayring paraphrasiert, generalisiert, reduziert und als Kategoriesystem zusammengetragen \cite{mayring1994qualitative}. Die Verfahrensregeln beschreiben Arbeitsschritte bei der Durchführung der zusammenfassenden Inhaltsanalyse \cite{mayring1994qualitative}. Die Ergebnisse der Schritte aus dem Ablaufmodell der Abbildung \ref{fig:zusammenfassendeinhaltsanalyse} sind im Anhang \ref{sec:reduktion} in der Tabelle \ref{tab:kategorien} dargestellt. Diese Tabelle beinhaltet in der ersten Spalte den \emph{Fall}, der das jeweilige Interview widerspiegelt. Fall 1 entspricht Interview 1, usw. Die zweite Spalte ist die \emph{Nummer} der jeweiligen Zeile der Tabelle, bei der paraphrasiert, generalisiert, reduziert wurde. Diese ist dazu da um inhaltlich zusammengehörende Abschnitte referenzieren zu können. Die dritte Spalte enthält die Paraphrasierung. Die vierte Spalte umfasst die Generalisierung und die fünfte Spalte enthält die Kategorien mit den reduzierten Informationen. Die Paraphrasierung, Generalisierung, sowie erste und zweite Reduktion wurden in einem Schritt durchgeführt. Dies kann bei größeren Datenmengen getan werden \cite{mayring2019qualitative}.
\paragraph{1. Paraphrasierung der inhaltstragenden Textstellen}\mbox{} \\
Zu Beginn wurden die Interviews paraphrasiert. Dabei wurden die Verfahrensregeln für die Paraphrasierung nach Mayring beachtet \cite{mayring2019qualitative}. Dabei wurden alle nicht oder wenig inhaltstrangede Textbestandteile wie ausschmückende, wiederholende oder verdeutlichende Wendungen aus den Interviews gestrichen \cite{mayring2019qualitative}. Außerdem wurden die inhaltstragenden Textstellen auf eine einheitliche Sprachebene gebracht und auf eine grammatikalische Kurzform transformiert \cite{mayring2019qualitative}. Das Abstraktionsniveau ist dabei hoch. Die Aussagen wurden sinngemäß aus den Interviews extrahiert.\\
\paragraph{2. Generalisierung der Paraphrasen}\mbox{} \\
Bei der Generalisierung wurden die Verfahrensregeln für die Generalisierung auf das Abstraktionsniveau nach Mayring angewendet \cite{mayring2019qualitative}. Dabei wurden die Gegenstände der Paraphrasen auf die definierte Abstraktionsebene generalisiert \cite{mayring2019qualitative}. Alle Satzaussagen wurden auf die gleiche Weise generalisiert \cite{mayring2019qualitative}. Paraphrasen, die über dem angestrebten Abstraktionsniveau liegen wurden belassen \cite{mayring2019qualitative}. Bei Zweifelsfällen wurden theoretische Vorannahmen zu Hilfe genommen \cite{mayring2019qualitative}.
\paragraph{3. Reduktion}\mbox{} \\
Die Verfahrensregeln für die erste und zweite Reduktion wurden bei der Durchführung beachtet \cite{mayring2019qualitative}. Bei der ersten Reduktion wurden bedeutungsgleiche Paraphrasen innerhalb der Auswertungseinheiten gestrichen \cite{mayring2019qualitative}. Zudem wurden Paraphrasen gestrichen, die nicht wesentlich inhaltstragend sind \cite{mayring2019qualitative}. Paraphrasen die weiterhin inhaltstragend sind wurden übernommen \cite{mayring2019qualitative}. Die zweite Reduktion umfasst die Zusammenfassung von Paraphrasen mit ähnlichem Gegenstand oder mehreren Aussagen \cite{mayring2019qualitative}. Zudem wurden Paraphrasen mit gleichem Gegenstand und verschiedener Aussagen zusammengefasst \cite{mayring2019qualitative}.
\paragraph{4. Zusammenstellung der Aussagen als Kategoriesystem}\label{sec:kategorien}\mbox{} \\
Die Kategorien sind induktiv aus dem Material erstellt worden \cite{mayring2012qualitative}. Dabei wurden alle Interviews heruntergebrochen, paraphrasiert, generalisiert, reduziert und in Kategorien (K1-K7) überführt. Bei der Durchführung der Schritte 1-3 der zusammenfassenden Inhaltsanalyse haben sich Themenschwerpunkte durch die Interviewfragen entwickelt. Aussagen konnten zusammengefasst werden, wodurch sich daraus die Kategorien gebildet haben. In jeder Kategorie sind Stichpunktartig die Informationen aus den Paraphrasen enthalten. \\
\begin{longtable}{| p{0.5cm} | p{4cm} | p{8.5cm} |}
	\hline
	Nr. & Überschrift & Stichpunkte\\
	\hline
	\hline
	K1 & Arten von Projekten & \begin{itemize}
		\itemsep-0.5em
		\item[-] Kundenprojekte
		\item[-] Softwareentwicklungsprojekte
		\item[-] Time Material
		\item[-] Festpreis
	\end{itemize}\\
	\hline
	K2 & Stakeholder & \begin{itemize}
		\itemsep-0.5em
		\item[-] Maitre
		\item[-] Führungskräftenetzwerk
		\item[-] Sales
		\item[-] Projektleiter
		\item[-] Fachverantwortliche
		\item[-] Entscheider
		\item[-] Geschäftsführer
		\item[-] Delivery Manager
		\item[-] Account Manager
	\end{itemize}\\
	\hline
	K3 & Wichtige Informationen & \begin{itemize}
		\itemsep-0.5em
		\item[-] Tagessatz
		\item[-] Einsatz
		\item[-] Dauer
		\item[-] Tech Stack
		\item[-] Muss/Kann Kriterien
		\item[-] Einarbeitungszeiträume
		\item[-] Lieferverpflichtung
	\end{itemize}\\
	\hline
	K4 & Bedarfsmeldung & \begin{itemize}
		\itemsep-0.5em
		\item[-] Überschrift
		\item[-] Beschreibung
		\item[-] Einsatzkontext
		\item[-] Datum
		\item[-] Volumen
		\item[-] in Jira gespeichert
		\item[-] gewichtete Fähigkeiten
		\item[-] keine feste Struktur
	\end{itemize}\\
	\hline
	K5 & Qualitätsbewertung & \begin{itemize}
		\itemsep-0.5em
		\item[-] keine vorhanden
		\item[-] Erfahrung
		\item[-] über Projektleitung
		\item[-] intensives Lesen
		\item[-] Rückfragen stellen
		\item[-] grober Rahmen durch Jira
		\item[-] regelmäßige Meetings
	\end{itemize}\\
	\hline
	K6 & Qualitätsverbesserung & \begin{itemize}
		\itemsep-0.5em
		\item[-] klare Vorgaben
		\item[-] weniger Freitext
		\item[-] Reviewprozess
		\item[-] Verständnisübereinstimmung
		\item[-] Beseitigung Missverständnisse
		\item[-] strukturierte Datenerfassung
		\item[-] KI-gestützte Prüfungen
	\end{itemize}\\
	\hline
	K7 & Auswirkungen & \begin{itemize}
		\itemsep-0.5em
		\item[-] Überblick verlieren
		\item[-] längerer Staffing-Prozess
		\item[-] Umbesetzung
		\item[-] erhöhter Aufwand
		\item[-] Mehrfachbeantwortung
		\item[-] Missverständnis
		\item[-] unpassendes Personal
	\end{itemize}\\
	\hline
	\caption{Ergebnisse der qualitativen Inhaltsanalyse.}
	\label{tab:kategorientabelle}
\end{longtable}
In Tabelle \ref{tab:kategorientabelle} sind die Ergebnisse der qualitativen Inhaltsanalyse als Kategorien abgebildet. Die Spalte eins beinhaltet die Kategorienummer (K1-K7) der jeweiligen Kategorie. Die zweite Spalte enthält die Überschrift der Kategorien. Die dritte Spalte umfasst die stichpunktartigen Inhalte und Aussagen aus der zusammenfassenden Inhaltsanalyse.
%\begin{itemize}
%	\item[K1] Arten von Projekten
%	\begin{itemize}
%		\item[-] Kundenprojekte
%		\item[-] Softwareentwicklungsprojekte
%		\item[-] Time Material
%		\item[-] Festpreis
%	\end{itemize}
%	\item[K2] Stakeholder
%	\begin{itemize}
%		\item[-] Maitre
%		\item[-] Führungskräftenetzwerk
%		\item[-] Sales
%		\item[-] Projektleiter
%		\item[-] Fachverantwortliche
%		\item[-] Entscheider
%		\item[-] Geschäftsführer
%		\item[-] Delivery Manager
%		\item[-] Account Manager
%	\end{itemize}
%	\item[K3] Wichtige Informationen
%	\begin{itemize}
%		\item[-] Tagessatz
%		\item[-] Einsatz
%		\item[-] Dauer
%		\item[-] Tech Stack
%		\item[-] Muss/Kann Kriterien
%		\item[-] Einarbeitungszeiträume
%		\item[-] Lieferverpflichtung
%	\end{itemize}
%	\item[K4] Bedarfsmeldung
%	\begin{itemize}
%		\item[-] Überschrift
%		\item[-] Beschreibung
%		\item[-] Einsatzkontext
%		\item[-] Datum
%		\item[-] Volumen
%		\item[-] in Jira gespeichert
%		\item[-] gewichtete Fähigkeiten
%		\item[-] keine feste Struktur
%	\end{itemize}
%	\item[K5] Qualitätsbewertung
%	\begin{itemize}
%		\item[-] keine vorhanden
%		\item[-] Erfahrung
%		\item[-] über Projektleitung
%		\item[-] intensives Lesen
%		\item[-] Rückfragen stellen
%		\item[-] grober Rahmen durch Jira
%		\item[-] regelmäßige Meetings
%	\end{itemize}
%	\item[K6] Qualitätsverbesserung
%	\begin{itemize}
%		\item[-] klare Vorgaben
%		\item[-] weniger Freitext
%		\item[-] Reviewprozess
%		\item[-] Verständnisübereinstimmung
%		\item[-] Beseitigung Missverständnisse
%		\item[-] strukturierte Datenerfassung
%		\item[-] KI-gestützte Prüfungen
%	\end{itemize}
%	\item[K7] Auswirkungen
%	\begin{itemize}
%		\item[-] Überblick verlieren
%		\item[-] längerer Staffing-Prozess
%		\item[-] Umbesetzung
%		\item[-] erhöhter Aufwand
%		\item[-] Mehrfachbeantwortung
%		\item[-] Missverständnis
%		\item[-] unpassendes Personal
%	\end{itemize}
%\end{itemize}
\paragraph{5. Rücküberprüfung des Kategoriesystems am Ausgangsmaterial}\mbox{} \\
In diesem Schritt wurde das Ausgangsmaterial ein weiteres Mal durchgesehen. Dadurch wurde sichergestellt, dass alle relevanten Inhalte durch das Kategoriesystem abgedeckt werden. Zudem wurden Unklarheiten und Lücken identifiziert und ausgebessert.
\section{Interpretation der Ergebnisse}
Die Kategorien zeigen, dass \emph{Bedarfsmeldungen} Bestandteil eines komplexen und vielschichtigen Prozesses sind, der verschiedene Arten von Projekten, eine Vielzahl von Stakeholdern und detaillierte Informationen umfasst. 
\begin{figure}[H]%htb
	\centering  
	\includegraphics[scale=0.8]{Abbildungen/bedarfsmeldungsprozess.png}
	\caption{Vereinfachte Darstellung zur Erfassung und Durchführung eines Projektes bei \emph{adesso}.}
	\label{fig:bedarfsmeldungsprozess}
\end{figure}\mbox{} \\
Die Abbildung \ref{fig:bedarfsmeldungsprozess} zeigt eine vereinfachte Darstellung des Prozesses zur Erfassung und Durchführung von Projekten bei \emph{adesso}. Dieser Prozess ist in der Hinsicht vereinfacht, dass dieser im Detail von Projekt zu Projekt unterschiedlich sein kann. Grundsätzlich wird nach dem Erhalt eines Auftrages zwischen einem Kunden- oder internen Projekt unterschieden. Dabei wird differenziert, ob die Anforderungen vom Kunden oder intern definiert werden. Aus den Anforderungen wird die \emph{Bedarfsmeldung} erstellt. Anschließend wird auf Basis dieser \emph{Bedarfsmeldung} passende Mitarbeiter manuell identifiziert. Das Vergütungsmodell unterscheidet sich zwischen einem Festpreis- und Time and Material Projekt. Das Projekt wird mit den ermittelten Mitarbeitern durchgeführt und überwacht. Entsprechend des Vergütungsmodells und ob es ein Gewerk ist, wird entweder eine fertige Software oder eine Dienstleistung erbracht. Dieser Prozess ist im Detail vielschichtig und komplex. Viele schritte können und dürfen nicht von Software übernommen werden. Dennoch kann der Prozess zur Besetzung der Projekte durch Systeme unterstützt werden.\\

Die Qualität der \emph{Bedarfsmeldungen} ist entscheidend für den Erfolg der Projekte, und es existieren Möglichkeiten zur Verbesserung durch Strukturierung, Automatisierung und klarer Kommunikationswege. Die Auswirkungen von unzureichender Pflege und fehlendem Informationsgehalt der \emph{Bedarfsmeldungen} sind weitreichend und können zu erheblicher Ineffizienz und Problemen innerhalb von \emph{adesso} führen.
\paragraph{Aktuelle Struktur einer Bedarfsmeldung}\mbox{} \\
Die \emph{Bedarfsmeldungen} werden in Jira gespeichert und erhalten dadurch einen groben Rahmen, wie diese gepflegt werden.
\begin{figure}[H]%htb
	\centering  
	\includegraphics[scale=1]{Abbildungen/jiraBefore.png}
	\caption{Mockup des Aufbaus einer \emph{Bedarfsmeldung} in Jira.}
	\label{fig:jirabefore}
\end{figure}\mbox{} \\
Diese Struktur ist in Abbildung \ref{fig:jirabefore} abgebildet. Enthalten sind vordefinierte Felder zur Beschreibung von \emph{Bedarfsmeldungen}. Der \emph{Status} beschreibt in welchem Bearbeitungslage das Projekt zur \emph{Bedarfsmeldung} ist. Dabei wird zwischen \emph{Offen}, \emph{Eskaliert} und \emph{Geschlossen} unterschieden. Dadurch wird signalisiert in welcher Form die \emph{Bedarfsmeldung} Aufmerksamkeit benötigt. Die \emph{Labels} helfen bei der Suche nach den \emph{Bedarfsmeldungen} und geben eine abstrakte Übersicht des Themengebiets. Die \emph{Rolle} stellt die Anforderungen bezüglich der Verantwortung an die gesuchten Mitarbeiter. Die Felder \emph{Aufgaben} und \emph{Skills} beinhalten konkrete Anforderungen an die Mitarbeiter in Form von Kriterien und Fähigkeiten. Diese bestehen aus unstrukturiertem Informationsgehalt. Die Informationen sind insofern unstrukturiert, als dass sie ohne vordefinierte Struktur in Form eines Volltexts vorliegen. Infolgedessen kann es zu Abweichungen hinsichtlich der Ausgestaltung von \emph{Bedarfsmeldungen} kommen. Das \emph{Skill-Level} gibt Auskunft über die benötigte Kompetenzstufe in der gesuchten Rolle. Das Feld \emph{Kunde} enthält Informationen zum Auftraggeber der \emph{Bedarfsmeldungen}. Die Felder \emph{Einsatzort}, \emph{Beginn Einsatz} und \emph{Ende Einsatz} beinhalten Zeitliche und ortsspezifische Informationen, in welchem die \emph{Bedarfsmeldung} durchgeführt wird. Der \emph{Tagessatz} gibt Auskunft über die Vergütung des Mitarbeiters. Des weiteren existieren spezifischere Informationen wie \emph{Option auf Verlängerung}, \emph{Reisekosten vergütet}, \emph{ANÜ}, \emph{Freelancer}, \emph{Smartshore-fähig}, \emph{EU-Shoring}, \emph{Deutschsprachig} und \emph{Bemerkung/Sonstiges} die individuell spezifisch zu bestimmten \emph{Bedarfsmeldungen} ausgefüllt werden. Diese Felder werden dennoch nicht regelmäßig ausgefüllt, wodurch diese oftmals leer bleiben. Zudem gelten diese für den Staffing nicht als relevante Felder.\\
\section{Strukturierung von Bedarfsmeldungen}
\label{sec:strukturierungbedarfsmeldung}
Die Problemstellung umfasst eine Reihe von Punkten, die im Rahmen der Ausarbeitung zu behandeln sind. Aufgrund der unstrukturierten und mit fehlenden Informationen versehenen \emph{Bedarfsmeldungen} ist eine Strukturierung von besonderer Relevanz. Dies würde die Extraktion relevanter Informationen erleichtern und somit die Effizienz des Systems verbessern. Auf Basis der Informationen aus den Interviews und der Kategorien im Kapitel \ref{sec:kategorien} wurde die Semi-Strukturierung aus der Abbildung \ref{fig:jirabefore} angepasst, reduziert und konkretisiert.
\begin{figure}[H]%htb
	\centering  
	\includegraphics[scale=1]{Abbildungen/jiraAfter.png}
	\caption{Mockup einer strukturierten \emph{Bedarfsmeldung}.}
	\label{fig:jiraafter}
\end{figure}\mbox{} \\
In Abbildung \ref{fig:jiraafter} ist die Struktur dargestellt, in die die Bedarfsmeldungen durch das zu entwickelnde System überführt werden sollen. Der Einsatzbeginn und das Einsatzende werden in einem Feld zusammengefasst. Des Weiteren werden die Aufgaben und Skills nicht als Freitext, sondern als stichpunkthaltige Zusammenfassungen reduziert. Fachexperten benötigen laut den Kategorien aus Kapitel \ref{sec:kategorien} prägnante Listen mit gekürzten Informationen wie benötigte Expertise in Technologien und Muss- und Kann-Kriterien. Mitarbeiterprofile sind abstrakt gesehen auch Listen mit Skills. Dadurch soll die \emph{Bedarfsmeldung} in eine vergleichbare Struktur überführt werden. Im Rahmen der Auswertung der Interviews wurden diejenigen Felder ermittelt, die als besonders Relevant deklariert sind. Neben diesen Feldern wurden alle weiteren unrelevanten und potenziell leeren Felder entfernt.\\

Im Rahmen der Entwicklung einer Software zur automatisierten Strukturierung einer \emph{Bedarfsmeldung} ist die Extraktion der erforderlichen Informationen aus den Volltexten erforderlich. In diesem Kontext existiert bereits eine Reihe an Methoden und Ansätzen, die sich in der Forschung bewährt haben. Die unstrukturierten Volltexte müssen annäherungsweise in eine strukturierte inhaltliche Aufteilung in einzelne Sequenzen und Stichpunkte überführt werden.\\

%-mock in erwartungshaltung --> was soll mein system leisten können (standardisierung? vielleicht in eine bestimmte form bringen?)

%mock zeigen wie das in jira systemisch strukturiert ist (felder erklären)

%--> am ende wäre es geiler wenn es stukukturiert, es ist ein gamble ob sachen drin sind, wireframe

%--> am ende des tages ist das ein post wo was sinnvolles drin stehen kann, (gloryfied postet)

%erklären was eigentlich strukturiert ist und was freitext ist
%-wenn wir ergebnisse von interviews nehmen, müsste bedarfsmeldung so aussehen (neuer mock, paar parameter mehr als fixe jira felder)

%-->techstack soll systemisch eigene kategorien sein, muss kann kriterien (macht es da sinn wenn die richtig als feld vorgegeben ist, wir geben die form vor)

%template was felder vorgibt und freitext strukturiert, besser definierte semantik --> macht für alle dinge leichter --> ist dann zusammenfassendes kapitel von kapitel 3


%\cite{maguire2002user}

%im anhang sind die transskripte
%wenn man nicht ne größere anzahl an infos hat gucken ob man das halb automatisch evaluieren. Vielleicht kategorisieren. Infos die wichtig sind gucken ob die %dann auch nach dem preprocessing drin sind. Regressive tests schreiben.

%transformation von bedarfsmeldung zu guter bedarfsmeldung, was ist der fokus von der bedarfsmeldung, wie gut machen die ansätze das, und muss man das dann noch weiter verarbeiten, haben wir alles was wir brauchen mit nur einem algorithmus, inferenz falls parameter fehlt, gibt es einen der alles löst

%Fragen in das proposal aufnehmen, führungskraft vorher fragen ob die fragen nice sind.
