\chapter{Anhang}
\label{chap:ergebnisse}
\section{Interviewtranskripte}
\subsection{Vorabtest Fragen und Antworten}
1. Welche Art von Projekten sind typischerweise in Ihrem Unternehmen an der Tagesordnung? Können Sie uns Beispiele für verschiedene Arten von Projekten geben, die adesso
durchführt?\\
Software-Entwicklungsprojekte, angefangen von Projekten in dem ein adessi in einem Kundenprojekt arbeitet über gemischte Teams aus adessi und Kunde bis hin zur kompletten Lieferung von Projekleitern, Testern, Requirements Engineer  und Entwicklern\\

2. Wie werden Projektbedarfe und -anforderungen innerhalb von adesso typischerweise kommuniziert und dokumentiert?\\
Initial über den Maitre, der das Staffing übernimmt bzw. auch Vorschläge von Projektleitenden zum Staffing annimt, teilweise auch über das eigene Netzwerk zwischen Führungskräften, im CC, Bereich oder der LoB. Am Ende über das Staffing Jira\\

3. Welche Informationen halten Sie in einer Bedarfsmeldung für besonders wichtig oder unverzichtbar?\\
Senioritätslevel, Tagessatz, Remote/on Site Einsatz, Dauer, Technischer Stack (Muss- und Kann Kriterien), Einarbeitungszeiträume (ist es verrechenbar oder nicht?), Lieferverpflichtung\\


4. Wie detailliert sollten Bedarfsmeldungen Ihrer Meinung nach sein? Sind bestimmte Schlüsselaspekte oder -informationen in jeder Bedarfsmeldung enthalten?\\
Es sollte aussagefähig sein zumindest welche technischen Kompetenzen wichtig sind und welche Tagessätze, ob Remote möglich ist und die Dauer mindestens in der Bedarfsmeldung vorhanden sein.\\ 


5. Welche Herausforderungen oder Schwierigkeiten sind bei unklaren oder unvollständigen Bedarfsmeldungen aufgetreten?\\
Der Anforderer muss ggf. Fragen mehrfach beantworten, der Kanal über den kommuniziert wird (Teams Chat, Anruf, im Ticket, ...). Dadurch verliert man ggf. den Überblick\\

6. Wer sind die typischen Stakeholder bei der Erstellung von Bedarfsmeldungen und welche Rolle spielen sie?\\
Sales/PL: Anforderer mit den technischen Informationen, Maitre: kümmert sich um das Staffing bzw. die eigentliche Besetzung\\

7. Wie wird die Qualität von Bedarsmeldungen bei \emph{adesso} bewertet? Gibt es bestimmte Kriterien oder Standards, anhand derer Bedarfsmeldungen beurteilt werden?\\
%7. Wie wird die Qualität von Bedarfsmeldungen bei \emph{adesso} bewertet? Gibt es bestimmte Kriterien oder Standards, anhand derer Bedarfsmeldungen beurteilt werden? \\
Gar nicht meines Wissens nach.\\

8. Wie können Sie die Qualität und Klarheit von Bedarfsmeldungen verbessern?\\
Zukünftig: Durch klarere Vorgaben und weniger Freitext, aktuell: durch Nachfragen und Bitten um nachträgliche Pflege, ggf. durch Reviewprozesse bei eigenen Bedarfsmeldungen \\

9. Welche Auswirkungen haben unklare oder fehlende Informationen in Projektbeschreibungen auf die Effizienz und den Erfolg von Projekten?\\
Das Staffing dauert länger und ggf. werden die Stellen durch andere Diensteleister besetzt\\

10. Wie können Sie sicherstellen, dass die Bedürfnisse und Anforderungen aller relevanten Stakeholder in einer Bedarfsmeldung angemessen berücksichtigt werden?\\
Gute Abstimmungen bevor die Bedarfsmeldung erstellt wird, ggf. durch ein Quality-Gate (Review).\\
\subsection{Interview 2}
0:1:13.840 --> 0:1:27.60
Valente de Matos, Ricardo
Du bist zum einen CC Leiter. Das heißt du hast auch Menschen unter deiner Leitung, die du auf Projekte zuweist.
0:1:33.180 --> 0:1:33.820
Bürger, Marco
Genau ja.
0:1:29.780 --> 0:1:36.620
Valente de Matos, Ricardo
Du weiß also zu Projekten zu, übernimmst auch die Projektleitung.
0:1:37.40 --> 0:1:37.630
Bürger, Marco
Ja, genau.
0:1:38.800 --> 0:1:49.260
Valente de Matos, Ricardo
Genau da würde ich dann erst mal gerne wissen: Was sind so die typischen Stakeholder bei der Erstellung von Bedarfsmeldungen und welche Rolle hast du dabei grundsätzlich?
0:1:50.110 --> 0:1:59.390
Bürger, Marco
In den Situationen nehme ich immer die Rolle des Beraters erstmal an. Die Stakeholder sind klassisch die Fachverantwortlichen beim Kunden, aber auch die Entscheider. Sprich also deren Vorgesetzte die quasi fachlich vielleicht das ganze nicht so bewerten können, aber das Budget dafür hergeben müssen und natürlich dann im Zweifelsfall auch CO, CEO oder sogar Geschäftsführer.
0:2:41.310 --> 0:2:48.230
Valente de Matos, Ricardo
Kannst du ein paar Beispiele nennen, welche Arten von Projekten adesso so erhält und durchführt.
0:2:48.870 --> 0:3:21.890
Bürger, Marco
Ja, also im Prinzip kannst du das in 2 Arten von Projekten teilen. Aus einer sind Teil Material Projekte, wo quasi der Kunde mit einer Idee kommt, wo wir gut unterstützen können. Beispielsweise bei Bestandsprojekten. Oder vielleicht weil ihnen selbst die Ressourcen dafür fehlen. Zum anderen hast du halt Festpreis Projekte, wo wir bestimmtes Gewerk für den Kunden abschätzen und das Ganze auch dann gänzlich liefern. In Festpreis Projekten haben wir eher die Staffing Hoheit. Das heißt, wir können entscheiden wen wir in das Projekt einsetzen? Wobei bei einigen Projekten durchaus auch der Kunde sich die Profile mit anschaut und dann auch entscheidet. Anhand von Interviews was macht Sinn, was passt bei mir ins Team vielleicht und denke ich, dass das am besten für mich wäre?
0:3:50.340 --> 0:3:58.320
Valente de Matos, Ricardo
Wie werden diese Bedarfsmeldungen und Anforderungen denn so typischer Weise kommuniziert und dokumentiert? Gibt es da eine Art Ablauf, oder wie wird das gemacht?
0:4:2.620 --> 0:4:8.130
Bürger, Marco
Hängt auch immer ehrlicherweise vom jeweiligen Kunden ab. Bei manchen reicht ein Interview, welches du führst und dann schreibst du es in ein Word Dokument als Anforderungsbeschreibung nieder. Dann lässt man das gegen Zeichnen und dann ist gut. Manchmal muss man aber auch ein paar mehr Integration fahren und dann nochmal genau abzustecken, was denn Bestandteil der Beauftragung ist und was nicht. Also was die Bedarfsmeldung. Da muss man auch mal durchaus eins tiefer bohren, weil da teilweise der Gedanke, was der Kunde möchte nicht mit dem übereinstimmt, was eigentlich gebraucht wird. Das ist so, weil du klassisch irgendein Anforderungsdokument dafür fertig machst, wenn das immer ein bisschen höher geht Richtung Management. Beim Management mit dem Kunden kann es auch durchaus mal eine Präsentation sein, wo das Ganze nochmal ein bisschen aufbereitet ist. Ein bisschen klassisch, ein bisschen bunter und mit weniger Infos und mit weniger Tiefe präsentiert.
0:5:16.340 --> 0:5:23.480
Valente de Matos, Ricardo
Welche Informationen sind denn für dich in Bedarfsmeldungen besonders wichtig oder auch unverzichtbar? Also zum Beispiel gibt es ja auch das Senioritätslevel.
0:5:38.820 --> 0:5:45.70
Bürger, Marco
Ja genau, also Senioritätslevel ist ehrlicherweise erstmal das zweitrangige. Das müssen wir im Nachgang einmal prüfen. Wichtig ist halt was der Text Deck und was vom Kunden kommt. Und quasi anhand der Bedarfsmeldung an sich. Wie hoch ist der Aufwand, der dahinter steckt? Sowohl durch den Kunden als auch das, was wir teilweise schätzen. Dann kann durchaus sein, dass der Kunde sagt, was gebraucht wird, er aber eigentlich keine Ahnung hat. Es kann vorkommen, dass der Kunde gerne in 2 Monaten durch ist, aber es keinen Sinn macht und wir mindestens ein halbes Jahr benötigt. Dann trifft man sich irgendwo in der Mitte und muss aber noch abgrenzen was Sinn macht und was nicht. Wenn das alles klar ist, dann kannst du überlegen was du eher brauchst. Machst du z.B. nur was mit Senioren? Oder reicht ein Junior. Das hängt dann immer eher von der Gesamtsituation ab.
0:6:50.110 --> 0:6:52.600
Valente de Matos, Ricardo
Wie detailliert sollte dann eine Bedarfsmeldung sein? Gibt es Aspekte oder Informationen, die eigentlich in jeder Bedarfsmeldung drin sein sollten und müssen?
0:7:5.130 --> 0:7:6.880
Bürger, Marco
Ich glaube der Umfang ist immer ganz wichtig. Die Erwartungshaltung sollte immer detailliert sein. Und die Technologien ebenfalls.
0:7:24.970 --> 0:7:33.660
Valente de Matos, Ricardo
Welche Herausforderungen oder oder Schwierigkeiten sind dann bei zum Beispiel unklaren oder unvollständigen Bedarfsmeldungen aufgetreten? Hattest du so etwas schon mal?
0:7:35.450 --> 0:7:42.600
Bürger, Marco
Ja durchaus, das sind immer Lernprozesse sowohl beim Kunden als auch bei dem, der die Anforderungen aufnimmt. Hängt immer vom Reifegrad des jeweiligen Konterparts ab. Das heißt auch, dass umso mehr du dich in der Situation meldest und Sachen aufgenommen hast, umso genauer kannst du mal nachfragen und hörst auch die Unsicherheit auf der einen oder anderen Seite heraus.
0:8:8.110 --> 0:8:17.50
Valente de Matos, Ricardo
Gibt es denn irgendwie ein Mechanismus, wie die Qualität von Bedarfs Meldungen bewertet wird, oder gibt es irgendwelche Kriterien oder Standards womit dann beurteilt wird, ob eine Bedarfsmeldung gut oder eher schlecht ist?
0:8:25.100 --> 0:8:32.870
Bürger, Marco
Ich bin ein Fan davon auch Sachen mal querlesen zu lassen und dann vielleicht mal die ein oder andere Meinung einzuholen und auch noch mit dem Kunden, der die Bedarfsmeldungen stellt ganz nah dran zu bleiben und dann zu gucken, dass das immer funktioniert. Damit auch zum Schluss das rauskommt, was am Anfang vielleicht schon ne Idee gewesen ist.
0:8:58.820 --> 0:9:7.190
Valente de Matos, Ricardo
Wie würdest du denn die Qualität von Bedarfsmeldungen verbessern?
0:9:11.70 --> 0:9:34.240
Bürger, Marco
Auch wirklich mit dem Kunden das Ganze einmal durchexerzieren und festzustellen, ob das Verständnis auf allen Seiten das gleiche ist. Weil nur dann kann es auch gut und produktiv werden. Wenn direkt am Anfang schon irgendwie Unklarheit da ist, dann kannst du auch davon ausgehen, dass da Diskussionsbedarf entsteht.
0:9:41.370 --> 0:9:49.980
Valente de Matos, Ricardo
Welche Auswirkungen haben denn unklare und fehlende Informationen in Bedarfsmeldungen in Bezug auf die Effizienz und den Erfolg des Projektes? Hast du da irgendwie Erfahrungen machen können?
0:9:52.830 --> 0:10:1.870
Bürger, Marco
Ja, wenn wir uns unklar sind, wird der Aufwand immer um ein Vielfaches erhöhen, weil das irgendwie im Nachgang immer noch mal gerade gezogen werden muss. Wenn du Pech hast, kannst du alles, was du bisher gemacht hast wegschmeißen und nochmal neu anfangen. Bedeutet natürlich auch immer, dass ein großes Diskussionspotenzial zwischen sowohl den Projektbeteiligten als auch Kunde und Projekt existiert.
0:10:31.900 --> 0:10:44.900
Valente de Matos, Ricardo
Wie könnte man theoretisch sicherstellen, dass die Bedürfnisse und Anforderungen aller Leute, die involviert sind, irgendwie angemessen berücksichtigt werden?
0:10:51.310 --> 0:10:56.730
Bürger, Marco
Ich glaube wenn du da mit viel Erfahrung reingehst und dann auch vielleicht genau weißt, wo du drauf zu achten hast. Mir fehlt ein bisschen die Fantasie, aber das ist ja jetzt dann deine Aufgabe da so ein Automatismus zu erkennen. Das wird auf jeden Fall schwierig weil ich auch an der Stelle glaube, dass wenn du so ein System schaffst, die quasi den Match drauf machen können, es trotzdem viel lernen muss. Von daher glaube ich, Erfahrung ist das es ausmacht, um am Ende zu sagen was Sinn macht oder nicht.
\subsection{Interview 3}
0:0:14.550 --> 0:0:22.730
Valente de Matos, Ricardo
Ich habe mir ein paar Infos geholt. Du warst zumindest mal richtig CC-Leiter oder?
0:0:21.980 --> 0:0:42.770
Kirchner, Lars
Ja ich, ich war mal richtig CC Leiter genau. Ich kann zumindest sagen, dass ich sogar mal 2 Kompetenz Center, 1 in München und 1 in Dortmund geleitet habe und für 48 Menschen zuständig war.
0:0:45.580 --> 0:0:51.830
Valente de Matos, Ricardo
Ok, das bedeutet aber du machst teilweise noch CC Leitung. Oder gar nicht mehr?
0:0:49.600 --> 0:1:21.730
Kirchner, Lars
Nein, nein aus in der Tat persönlichen Gründen habe ich vor 2 Jahren die Entscheidung getroffen, dass ich diese Rolle verlassen muss.
0:1:39.350 --> 0:1:44.510
Valente de Matos, Ricardo
Okay, das heißt aber auch du bist aktuell Projektleiter.
0:1:43.640 --> 0:1:47.210
Kirchner, Lars
Ich bin aktuell von der Laufbahnstufe höher. Programm Manager das ist formal noch eine Führungslaufbahn oder Führungsrolle. Das ist etwas anderes zumindest in der adesso Welt als Projektleitung. Ich kümmere mich in der Tat dann in einer Mischform von Projektleitung und Produktunterschiede um interne Projekte. Ich arbeite mit Studierenden zusammen, ich übernehme Angebotsmanagement für große Angebote. Ich vertrete einzelne Themen, bei dem man jemanden braucht, der entsprechend erfahren ist und dann auf der anderen Seite gewisse intellektuelle Fähigkeiten mit sich bringt. Das sind dann aber eher Sonderthemen. Also ein relativ buntes Sammelsurium. Was ich nicht mehr habe ist Personal Verantwortung. Nicht weil ich nicht mit Menschen umgehen kann, sondern sind beispielsweise die sehr verwalterischen Aspekte nicht so angenehm, die nun mal in dieser Rolle mit drinstecken. Ich übernehme auch noch Delivery Management. Das heißt für große Kunden, gibt es eine spezielle Schnittstellen Rolle, die im Prinzip die adesso Organisation vor dem Kunden Kapsel, weil der Kunde gar nicht im Detail wissen soll, wie wir aufgebaut sind. Der Kunde hat genau solche Bedarfsanfragen und sagt, ich brauche einen Senior Java Developer mit diesen Fähigkeiten und ich bin dann als Delivery Manager dafür zuständig.
0:6:1.670 --> 0:6:7.780
Valente de Matos, Ricardo
Dann wäre jetzt meine Frage was die typischen Stakeholder bei einer Erstellung von einer Bedarfsmeldung sind und welche Rolle hast du dabei?
0:6:12.710 --> 0:6:31.180
Kirchner, Lars
Also Bedarfsmeldungen sind ja ein sehr wichtiges Element in einem der adesso Kerngeschäftsprozesse. Wir sind ein IT-Dienstleister. Das bedeutet, wir entwickeln nicht selber etwas im Sinne von Produkten. Das ist etwas anderes und nennt sich Produkt Geschäft. Das heißt also, wenn wir etwas tun, Entwicklungstätigkeiten aufnehmen, beraten, brauchen wir und benötigen wir immer einen Auftrag. Also einen Kunden. Wenn wir Kundenaufträge haben, weil wir ja ein IT Dienstleister sind, heißt das unsere Kerngeschäftsobjekte sind die Mitarbeitenden, die mit unterschiedlichen Qualifizierungen daneben in genau diesen Kundenaufträgen tätig sind. Das können vom Kunden durchgeführte Projekte sein. Ein Kundenauftrag kann aber auch sein, dass adesso ein Projekt für den Kunden durchführt. Wir müssen wie gesagt diese Projektstellen gezielt besetzen. Das heißt, wir haben immer eine Projektorganisation, mal ist es auch eine Programmorganisation, wo diese Stellen mit entsprechenden rollen oder Stellenanforderungen beschrieben werden. Entweder kommen die vom Kunden direkt, oder wir formulieren die selbst. Diese Beschreibung der Anforderungen, also welche Fähigkeiten und welche Erfahrung und in welchem Umfang Menschen diese mitbringen müssen, damit sie genau diese Projektstelle besetzen können und damit eine ganz bestimmte Rolle in einem Projekt Kontext einnehmen oder in einem Programm Kontext… Diese Beschreibung ist es, was wir als Bedarfsmeldung bezeichnen. Die Quellen dafür sind entweder Kundenorganisationen, dass die dann durch große Accounts wie e-on, die Deutsche Bahn, die ihre ganz eigenen Systeme haben und dann in einer, nicht normierten, aber in einer semi strukturierten Form diese Anforderungen dokumentiert werden. Die gehen dann mehr oder weniger 1 zu 1 an uns über und wir müssen damit weiterarbeiten. Bis dahin, dass wir als Delivery Manager, als Projektleiter, als Programm Manager, als Account Manager oder Vertriebler mit Kunden sprechen, in Projekte reinschauen, Projektorganisationen definieren und selbst in diesen Rollen eine Bedarfsmeldung erfassen. Es geht allerdings jedes Mal darum, innerhalb eines Projektes oder Programms eine bestimmte Stelle zu besetzen.
0:10:12.650 --> 0:10:20.560
Valente de Matos, Ricardo
Welche Art von Projekten hat adesso typischerweise? Hast du da vielleicht ein paar Beispiele?
0:10:26.940 --> 0:10:35.600
Kirchner, Lars
Die erstmal abstrakteste und wichtige oder grundsätzliche Unterscheidung ist: Es gibt Projekte, die ein Kunde eine Kundenorganisation aufsetzt und durchführt, an denen wir uns dann beteiligen, indem wir beispielsweise in bestimmte Rollen an bestimmte Stellen dort Menschen reinbringen. Das heißt, wir arbeiten dann aber in einem extern definierten und normalerweise auch gesteuerten und kontrollierten Projektkontext. Das andere ist, wenn wir im Kundenauftrag Projekte aufsetzen. Da könnte man dann auch nochmal differenzieren. Einmal Projekte unter unserer Kontrolle im Kundenauftrag und dann gibt es natürlich auch interne Projekte. Da kann man aber relativ schnell sagen, dann ist es halt eine interne Stakeholder Position. Zum Beispiel kann die HR-Abteilung auch sowas beauftragen. Das unterscheidet sich dann nicht wesentlich. Was allerdings da dann besser in unserer Kontrolle liegt, ist in der Regel, wenn wir selbst die Projekte in der Organisation in der Durchführung verantworten. Dann haben wir auch die Ausprägung der Projekt Organisation, das Vorgehensmodell, usw. in der Hand. Da haben wir sehr viel mehr Flexibilität, was eben die Definition von Rollen, Anforderungen usw. angeht. Das ist so der wesentliche Unterschied in Bezug auf externe Projekte und von uns durchgeführte Projekte. Es gibt dann aber nochmal einen wesentlichen Unterschied in Bezug auf die Vergütung. Wir unterscheiden da grundsätzlich zwischen Time und Material Aufträgen oder Beauftragung und Festpreis Beauftragung. Festpreis bedeutet, dass es von unserer Seite eine bestimmte, klar bemessene und abgegrenzte Leistung angeboten wird und die Auftraggeber Seite verhandelt mit uns dafür einen festgesetzten Preis. Damit muss man bei der Durchführung beachten, dass man nicht so lange vor sich hinarbeiten kann, bis dann zum Beispiel die Auftraggeberseite sagt OK, wir sind jetzt zufrieden. Sondern man hat halt ein beschränktes Budget und in der Regel auch beschränkte Zeit und trägt damit auch ein größeres Risiko. Dann gibt es da entsprechend die timen Material oder t und m Kontexte. Da trägt die Auftraggeberseite in der Regel das größere Risiko, weil wir abstrakt bezeichnet, erstmal nur dazu verpflichtet sind, nach durchschnittlicher Qualität und Güte solche einzelnen Leistungen über zum Beispiel Mitarbeitende beizusteuern, zu erbringen. Trotzdem hat auch da die Erfahrung gezeigt: Wir sind immer an unserer Kunden Zufriedenheit oder langfristigen Kundenbindung interessiert, dass es uns auch gar nicht hilft, wenn wir in einem t und m Kontext nur durchschnittlich oder vielleicht auch mal schlechte Arbeit leisten und dann hinterher sagen, das war jetzt aber gar nicht unsere Verantwortung. Das ist euer Risiko gewesen. Damit kommen wir auch nicht durch. Und es ist in den seltensten Fällen auch so, dass ein Kunde unerschöpfliche Geldmittel hat. Selbst wenn dieser Kunde diese unerschöpflichen Geldmittel hätte, sagen sie aus rein wirtschaftlichen Aspekten natürlich auch zu irgendeinem Zeitpunkt das reicht jetzt, wir möchten nicht noch mehr Geld ausgeben. Das ist doch mal eine grundsätzliche Unterscheidung, was das Bezahlen angeht. Mit hinein spielt auch noch eine Unterscheidung in die Art der Projekte. Das ist nämlich einmal ein Gewerk, wo wir Gewährleistung übernehmen und das auch entsprechend kalkulieren müssen. Ganz häufig werden Gewerke in Kombination mit Festpreisen angeboten und durchgeführt. Interessanterweise müssen sie es aber gar nicht zwingend. Im Umkehrschluss meistens, wenn ich nach t und m arbeite, handelt es sich auch dann von unserer Leistung um Dienstleistung. Das sind aber teilweise, wenn man da wirklich genau draufschaut oder wenn man da juristisch drauf schauen würde feine Unterschiede. Wenn man beispielsweise bei der Beauftragung oder in der Kommunikation ganz bestimmte Begriffe verwendet und es würde irgendwann vor Gericht landen, egal ob man einen Dienstleistungsvertrag abgeschlossen hat und die ganze Zeit meinte auch als Dienstleistungen zu arbeiten, könnte zum Beispiel ein Gericht aufgrund von Formulierungen usw. hinterher feststellen, dass es sich doch um ein Gewerk gehandelt hat. Das sind dann teilweise eher juristische Unterschiede. Am Ende des Tages bedeutet Gewerk natürlich, dass wir an einem Stück Software gearbeitet haben und eben für die konsistente Fehler freie gesamte Funktionalität dann entsprechende Gewährleistung anbieten und übernehmen. Also damit der Kunde für das Geld, dass die Organisation gezahlt hat, eine einsetzbare Lösung bekommt. Wohingegen bei Dienstleistung eine gar nicht funktionale Lösung bei rauskommen muss. Beispiel typischer und klassischer Bereich für Dienstleistungsgeschäft ist ja Consulting. Consulting bedeutet qualifizierte, erfahrene Menschen zu bestimmten Themen kommunizieren, was ausarbeiten, etwas aufschreiben, können etwas spezifizieren können. Aber letztendlich ist da die menschliche Arbeit beziehungsweise der Erkenntnisgewinn im Zentrum und es wird keine Lösung in dem Sinne geschaffen und bereitgestellt. Dass sind die für mich zumindest relevanten Unterscheidung. 
0:18:3.520 --> 0:18:16.250
Valente de Matos, Ricardo
Wie werden dann Bedarfsmeldungen und die Anforderungen typischerweise bei Adesso kommuniziert und auch dokumentiert. Gibt es einen Ablauf, wie das genau gehandhabt wird?
0:18:23.200 --> 0:18:36.560
Kirchner, Lars
Für den Staffing-Prozess an sich gibt es einen definierten Ablauf, der aber in Großteilen aus manueller Arbeit und manuellem Arbeitseinsatz besteht. Es gibt keine normierte Form einer Bedarfsmeldung. Es hat sich allerdings zumindest eine grobe thematische Struktur etabliert. Das bedeutet in der Regel hat man so etwas wie eine Überschrift und Bezeichnung. Dann hat man in der Regel einen Bereich, der den Einsatz Kontext ein wenig allgemeiner beschreibt. Dann hat man einen Bereich, der auf die individuell geforderten Fähigkeiten und Erfahrungen eingeht. Man hat normalerweise eine Gewichtung dieser Skills. Das bedeutet in Bezug auf Expertise, die da erwartet wird oder mal solche Dinge wie Primary in Secondary usw. Unterteilungen. Wir haben in diesen Bedarfsmeldungen dann aber auch wirtschaftlich relevante Informationen damit verknüpft. Das ist typisch für ein Dienstleistungsunternehmen. Das bedeutet uns interessiert dann die vertraglichen Konditionen im Sinne von Tagessatz, was für eine Organisation für adesso, im Sinne von Dienstleistung auch absolut relevant ist. Das sind dann immer diese Parameter. Ab wann ist der Einsatz gewünscht? Und für wie lange? Weil wir immer prüfen müssen, selbst wenn wir beispielsweise sehr gut fachlich passende Mitarbeitende finden, sind sie aber eventuell schon in anderen Projekten eingesetzt und können dementsprechend gar nicht dort leisten. Es mag dann je nach Kundenkontext noch weitere Informationsblöcke geben. Wir arbeiten mittlerweile auch mit einem Leveling-System, weil man sich als Organisation oder eben als Mensch innerhalb einer Organisation ein bisschen besser in Abfragen organisieren kann, wenn an solchen detaillierteren Bedarfsmeldungen gewisse Labels dran stehen wie beispielsweise Azure als Technologie oder oder Java. Dann kann ich eine potenziell größere Menge von Bedarfsmeldungen, die ich manuell durchsuche besser filtern und einschränken. Was in Bezug auf unserem Staffing-Prozess sehr relevant ist, ist beispielsweise dann der Status einer solchen Bedarfsmeldung, weil die Organisationen nicht alle Bedarfsmeldungen interessiert. Manche Bedarfsmeldungen sind schon verarbeitet und erfüllt. Manche sind nur dokumentiert und sollen aber gar nicht weiter beachtet werden und erst wenn sie beispielsweise bei uns eskaliert werden, sollte eigentlich die relevante Organisation darauf schauen. Wir haben eine vertikale Einteilung des Unternehmens in Branchen, bzw. technologisch getriebenen Organisationseinheiten. Das sind unsere sogenannten Line of Business. Mittlerweile sind das sogar Business Areas und es ist an der Stelle durchaus relevant, welche adesso Organisationseinheit für so eine Business Area oder Line of Business zuständig ist. Wenn wir an der Stelle mit einem gewissen Vorkaufsrecht feststellen, dass wir es nicht bedienen können, werden natürlich alle gefragt und es dürfen beispielsweise aus der Line of Business Motiv selbstverständlich auch mitarbeiten in Projekten der Line Cross Industries und umgekehrt getätigt werden. Aber umso mehr ist es wichtig, dass dann die jeweils geforderten Skills und Erfahrungen und wirtschaftlichen Konditionen möglichst präzise beschrieben werden, damit diese Fragen die aufkommen nicht immer wieder gestellt werden und immer wieder beantwortet werden müssen. Das ist aber, wie gesagt schon Kern Staffing-Prozess.
0:24:4.210 --> 0:24:10.830
Valente de Matos, Ricardo
Du hast auch schon einige Punkte in Richtung Verfügbarkeit genannt, dass das sehr wichtige Aspekte sind. Gibt es besonders wichtiger oder auch unverzichtbare Punkte in einer Bedarfsmeldung, die in jeder drin sein sollte?
0:24:22.530 --> 0:24:23.920
Kirchner, Lars
Das kommt auf die Perspektive an. Wenn ich jetzt eine rein fachliche Perspektive einnehme, ist es da unverzichtbar, dass aufgelistet oder benannt wird, welche Fähigkeiten oder Skills konkret mitgebracht werden müssen und idealerweise auch in welcher Erfahrung und Güte das ist. Aus der Perspektive zum Beispiel des Dienstleistungsunternehmen adesso ist relevant, dass ein Beginn des Einsatzes, ein voraussichtlicher Einsatzzeitraum, ein Tagessatz dran steht. Dass da dran steht, ob Freelancer, ob Smartphone, smartshore Fähigkeit gegeben ist oder near Shore, oder wie die sprachliche Ausrichtung ist. Also ob das zum Beispiel deutschsprachig ist, ob Englisch. Entsprechend zulässige Kommunikationsmittel sind wichtige Zusatzinformationen, die genau darauf abzielen, dass alle Menschen, die potenzielle Kandidaten/Kandidatinnen auf so eine Bedarfsmeldung gefiltert werden. Wenn es nämlich auf der Ebene, dass die Verfügbarkeit aber auch darüberhinausgehend nicht passt oder wenn es zum Beispiel von einem Tagessatz, aus welchen Gründen auch immer sehr unattraktiv ist, es dazu führen kann, dass dann die Organisation bestimmte Personen nicht anbietet. Das ist aber wie gesagt eine Perspektive, die durch die Natur von adesso als Dienstleister, und wir sind ein auslastungsgetriebenes unternehmen, was halt auch versucht wirtschaftlich zu arbeiten, darüber reinkommen. Idealerweise möchte zum gewünschten Einsatzbeginn auch dann wirklich eine Person nicht nur theoretisch benannt, sondern idealerweise interviewt, geprüft, eingeführt und wie auch immer wurde und dann wirklich los laufen kann.
0:28:46.70 --> 0:28:51.420
Valente de Matos, Ricardo
Gibt es Herausforderungen und Schwierigkeiten bei unklaren oder unvollständigen Bedarfsmeldungen? Hast du da Erfahrungen machen können?
0:28:51.490 --> 0:29:23.70
Kirchner, Lars
Ja. Bei einem Auftraggeber handelt es sich in der Regel um Branchen, die eben nicht als Kerngeschäft IT Software Entwicklung betreiben. Dementsprechend fällt es ihnen teilweise schwer, präzise Bedarfsmeldungen zu formulieren, die inhaltlich genau das transportieren oder beinhalten oder umfassen, was sie eigentlich an Fähigkeiten und Erfahrung benötigen. Das liegt einfach daran, dass sie in Teilen nicht für jeden Kunden gilt, aber in Teilen den Prinzip Menschen etwas beschreiben lassen, die davon nicht wirklich Ahnung haben. Das wiederum führt dazu oder kann dazu führen, dass der Kunde etwas anfordert, was der Kundenorganisation, dem Projekt oder wie auch immer im schlimmsten Falle sogar gar nicht hilft. Oder aber das ist eine andere Ausprägung, dass dann Kombinationen von Erfahrung und Fähigkeiten gesucht werden, die es in der realen Welt einfach so nicht gibt. Das ist die berühmte eierlegende Wollmilchsau, wo man drauf schaut und sagt, diese Menschen hätten wir auch gerne als Mitarbeitende. Vielleicht gibt es auf diesem Planeten auch eine Handvoll davon, aber das ist unrealistisch. Und das kann wie gesagt darin begründet sein, dass auf der auftraggebenden Seite jetzt Menschen damit beauftragt werden, solche Bedarfsmeldungen zu formulieren, die das nicht wirklich können. Eigentlich haben wir an der Stelle schon immer einen idealerweise von uns moderierten Prozess. Deswegen macht zum Beispiel auch die Rolle Delivery Management Sinn. Das hängt wie gesagt auch sehr mit der insgesamten Qualifikation oder Qualität von der Auftraggeberseite in diesem Kontext zusammen. Bei manchen Kundenorganisationen ist das trotzdem sehr gut eingespielt und etabliert und funktioniert auch so. Bei manchen muss man da früh ansetzen und sagen wir sprechen miteinander, und ich arbeite dann beispielsweise in diesem Gespräch heraus, was der Kunde wirklich benötigt, was abgrenzbar eine sinnvolle Bedarfsmeldung ist, was es für eine Stelle beinhaltet und womit wir dann weiterarbeiten können. Das ist oder kann ein Problem sein, wenn wir nicht entsprechend damit umgehen. Nicht nur die fachliche Qualität oder auch die die Passgenauigkeit, die wir dort anbieten können ist da wichtig, sondern auch vor allem die Schnelligkeit vom Staffing-Prozess. Bedeutet, wenn ein Unternehmen sich so aufgestellt hat, dass eine Bedarfsmeldung, die reinkommt innerhalb von sehr kurzer Zeit bearbeitet wird, sagen wir mal 2 Stunden, dann habe ich einen klaren Vorteil gegenüber zum Beispiel einem Anbieter, der dafür 2 Tage braucht oder eine Woche oder 2 Wochen. Denn für die Auftraggebende Seite ist klar, wenn ich dann Rückmeldungen bekomme und ich schaue da rein und diese Rückmeldungen sind plausibel… Was hält mich davon ab, dann zu sagen ich beauftrage ihn jetzt. Es muss nicht perfekt sein, aber wenn es plausibel ist und die Konditionen sind gut, dann ist der Auftrag ausgesprochen und die anderen gehen logischerweise leer aus. Wenn wir eben entsprechend in unserem Matching entweder nicht gut arbeiten oder überfordert sind und beispielsweise einen Software Architekt für Java mit bestimmten weiteren Anforderungen gefordert ist und wir bieten da ein Profil drauf an, wo man darauf schaut und feststellt, dass es ein .Net Software Developer ist, ist es bei diesem extremen Beispiel so, dass es glücklicherweise auch sofort auffällt. Aber das würde natürlich dann zu Irritationen auf Kundenseite führen. Bedeutet: Wir sollten nicht nur dafür sorgen, dass die Erwartungen oder Anforderungen möglichst gut an der Realität sind, sondern wenn wir dann wiederum auch da etwas matchen und einreichen, dass das dann auch diesen Anforderungen nach Möglichkeit entspricht und, dass innerhalb von möglichst kurzer Zeit so geschieht. Und die unglücklichste Variante ist, wenn man ein Prozess durchläuft und man bietet da jemanden an und der wird sogar genommen und wird eingearbeitet und dann stellt man irgendwie fest der macht irgendwie Unsinn. Man hat nie eine Garantie. Letztendlich sind es Menschen die dort arbeiten.
0:39:18.140 --> 0:39:22.580
Valente de Matos, Ricardo
Welche Auswirkungen haben unklare oder auch fehlende Informationen in Bedarfsmeldungen jetzt aber konkret in Bezug auf die Effizienz und den Erfolg von Projekten.
0:39:27.60 --> 0:39:36.170
Kirchner, Lars
Im Idealfall wird, möglichst früh erkannt, dass eine Bedarfsmeldung lückenhaft, unpräzise wie auch immer formuliert ist. Dann muss nachgefragt werden. Ich schaue mir etwas an, versuche zu verstehen, was die andere Seite sucht und wenn das für mich dann nicht konsistent auf die zumindest für mich bekannten Rollenlösung ist, dann muss ich nachfragen. Alles andere ist eine Interpretation. Dann läuft man mit sehr großer Wahrscheinlichkeit in die von mir gerade beschriebenen Probleme rein. Die schlechteste Variante ist, dass ich sage ich nehme die Informationen, die ich jetzt da vorgelegt bekommen habe, interpretieren sie nach besten Wissen und Gewissen und dann ist es aber mehr oder weniger ein Glücksspiel. Das heißt, wenn ich dann Profile finde, könnten sie immer noch zufällig das sein, was der Kunde eigentlich wollte und gesucht hat.
0:40:49.150 --> 0:41:1.310
Valente de Matos, Ricardo
Die letzte Frage hast du im Grunde auch schon mit beantwortet. Wie könnte man sicherstellen, dass Bedürfnisse und Anforderungen aller relevanten Stakeholder in einer Bedarfsmeldung Berücksichtigt werden?
0:41:4.190 --> 0:41:20.600
Kirchner, Lars
Zwei Möglichkeiten. Man könnte, da glaube ich aber nicht dran, natürlich den Prozess standardisieren und stark formalisieren. Also das im Prinzip von einer öffentlichen Stelle aus gesagt wird: Alle Dienstleister und Auftraggeber dieser Welt wenn ihr in dieser Art Geschäft betreiben wollt, müsst ihr so ein Format einreichen. Also Bürokratie pur. Das würde uns nichts verbessern, aber das ist eine Möglichkeit. Die andere Möglichkeit ist meines Erachtens, dass dann für genau solche Prozesse entsprechend versierte Menschen diesen Prozess, das heißt die Anforderungserhebung, die Dokumentation, das erfüllen diese Anforderungen komplett begleiten und moderieren. Das ist Delivery Management. Das ist Business Development. Manchmal auch Account Management.
0:42:26.240 --> 0:42:30.90
Valente de Matos, Ricardo
Dann sind wir eigentlich schon durch mit den Fragen. Hast du noch zu irgendeinem Punkt irgendwelche Fragen oder irgendwas, was vielleicht noch für mich in dem Themenbereich interessant sein könnte?
0:43:1.90 --> 0:43:6.710
Kirchner, Lars
Die größte Schwierigkeit liegt darin, dass wir Informationen über eine natürliche Sprache transportieren. Das ist zwar flexibel, weil die Sprache an der Stelle ja eben nicht formalisiert ist. Was aber immer das Risiko mit sich bringt, dass etwas nicht präzise beschrieben, abgegrenzt oder interpretierbar wird und genau bei solchen Prozessen, wo wir eigentlich präzise arbeiten wollen, haben wir genau diese große Herausforderung, dass die bisher benutzte Art, um diese Informationen zu ermitteln und die gerade wieder zu lesen zu interpretieren eben ein Stück weit ungenügende Mittel, nämlich den natürlichen sprachigen Raum verwendet. Da ist zwar mit gesundem Menschenverstand gearbeitet worden. Das heißt, es haben sich Semistrukturen gebildet. Aber es ist wirklich sehr individuell unterschiedlich in was für einer Qualität oder was für einer Realitätsnähe solche Bedarfsmeldungen formuliert werden. Wenn wir auf unserer Seite jemanden sitzen haben, der oder die eben auch in diesem Umfeld relativ wenig Ahnung und Erfahrung hat, dann haben wir auch nochmal ein Risiko, dass selbst wenn die Bedarfsmeldungen präzise und realitätsnah formuliert ist, bei uns in der Interpretation etwas schiefgeht. Das bedeutet, dass halt die Menschen, die auf unserer Seite Bedarfsmeldungen lesen und versuchen zu bedienen, leider auch hinreichend viel Erfahrung in der Projekt IT haben müssen.

\newpage
