\chapter{Anhang}
\label{chap:ergebnisse}
\section{Interviewtranskripte}
\subsection{Vorabtest Fragen und Antworten}
\label{interview1}
1. Welche Art von Projekten sind typischerweise in Ihrem Unternehmen an der Tagesordnung? Können Sie uns Beispiele für verschiedene Arten von Projekten geben, die adesso durchführt?\\

Software-Entwicklungsprojekte, angefangen von Projekten in dem ein adessi in einem Kundenprojekt arbeitet über gemischte Teams aus adessi und Kunde bis hin zur kompletten Lieferung von Projekleitern, Testern, Requirements Engineer und Entwicklern\\

2. Wie werden Projektbedarfe und -anforderungen innerhalb von adesso typischerweise kommuniziert und dokumentiert?\\

Initial über den Maitre, der das Staffing übernimmt bzw. auch Vorschläge von Projektleitenden zum Staffing annimt, teilweise auch über das eigene Netzwerk zwischen Führungskräften, im CC, Bereich oder der LoB. Am Ende über das Staffing Jira\\

3. Welche Informationen halten Sie in einer Bedarfsmeldung für besonders wichtig oder unverzichtbar?\\

Senioritätslevel, Tagessatz, Remote/on Site Einsatz, Dauer, Technischer Stack (Muss- und Kann Kriterien), Einarbeitungszeiträume (ist es verrechenbar oder nicht?), Lieferverpflichtung\\

4. Wie detailliert sollten Bedarfsmeldungen Ihrer Meinung nach sein? Sind bestimmte Schlüsselaspekte oder -informationen in jeder Bedarfsmeldung enthalten?\\

Es sollte aussagefähig sein zumindest welche technischen Kompetenzen wichtig sind und welche Tagessätze, ob Remote möglich ist und die Dauer mindestens in der Bedarfsmeldung vorhanden sein.\\ 

5. Welche Herausforderungen oder Schwierigkeiten sind bei unklaren oder unvollständigen Bedarfsmeldungen aufgetreten?\\

Der Anforderer muss ggf. Fragen mehrfach beantworten, der Kanal über den kommuniziert wird (Teams Chat, Anruf, im Ticket, ...). Dadurch verliert man ggf. den Überblick\\

6. Wer sind die typischen Stakeholder bei der Erstellung von Bedarfsmeldungen und welche Rolle spielen sie?\\

Sales/PL: Anforderer mit den technischen Informationen, Maitre: kümmert sich um das Staffing bzw. die eigentliche Besetzung\\

7. Wie wird die Qualität von Bedarsmeldungen bei \emph{adesso} bewertet? Gibt es bestimmte Kriterien oder Standards, anhand derer Bedarfsmeldungen beurteilt werden?\\

Gar nicht meines Wissens nach.\\

8. Wie können Sie die Qualität und Klarheit von Bedarfsmeldungen verbessern?\\

Zukünftig: Durch klarere Vorgaben und weniger Freitext, aktuell: durch Nachfragen und Bitten um nachträgliche Pflege, ggf. durch Reviewprozesse bei eigenen Bedarfsmeldungen \\

9. Welche Auswirkungen haben unklare oder fehlende Informationen in Projektbeschreibungen auf die Effizienz und den Erfolg von Projekten?\\

Das Staffing dauert länger und ggf. werden die Stellen durch andere Diensteleister besetzt\\

10. Wie können Sie sicherstellen, dass die Bedürfnisse und Anforderungen aller relevanten Stakeholder in einer Bedarfsmeldung angemessen berücksichtigt werden?\\

Gute Abstimmungen bevor die Bedarfsmeldung erstellt wird, ggf. durch ein Quality-Gate (Review).\\
\subsection{Interview 2}
\label{interview2}
%Bürger, Marco\\
0:1:13 --> 0:1:33\\
I:\\
Du bist zum einen CC Leiter. Das heißt du hast auch Menschen unter deiner Leitung, die du auf Projekte zuweist.\\

0:1:33 --> 0:1:29\\
B:\\
Genau ja.\\

0:1:29 --> 0:1:37\\
I:\\
Du weiß also zu Projekten zu, übernimmst auch die Projektleitung.\\

0:1:37 --> 0:1:38\\
B:\\
Ja, genau.\\

0:1:38 --> 0:1:50\\
I:\\
Genau da würde ich dann erst mal gerne wissen: Was sind so die typischen Stakeholder bei der Erstellung von Bedarfsmeldungen und welche Rolle hast du dabei grundsätzlich?\\

0:1:50 --> 0:2:41\\
B:\\
In den Situationen nehme ich immer die Rolle des Beraters erstmal an. Die Stakeholder sind klassisch die Fachverantwortlichen beim Kunden, aber auch die Entscheider. Sprich also deren Vorgesetzte die quasi fachlich vielleicht das ganze nicht so bewerten können, aber das Budget dafür hergeben müssen und natürlich dann im Zweifelsfall auch CO, CEO oder sogar Geschäftsführer.\\

0:2:41 --> 0:2:48\\
I:\\
Kannst du ein paar Beispiele nennen, welche Arten von Projekten adesso so erhält und durchführt.\\

0:2:48 --> 0:3:50\\
B:\\
Ja, also im Prinzip kannst du das in 2 Arten von Projekten teilen. Aus einer sind Time Material Projekte, wo quasi der Kunde mit einer Idee kommt, wo wir gut unterstützen können. Beispielsweise bei Bestandsprojekten. Oder vielleicht weil ihnen selbst die Ressourcen dafür fehlen. Zum anderen hast du halt Festpreis Projekte, wo wir bestimmtes Gewerk für den Kunden abschätzen und das Ganze auch dann gänzlich liefern. In Festpreis Projekten haben wir eher die Staffing Hoheit. Das heißt, wir können entscheiden wen wir in das Projekt einsetzen. Wobei bei einigen Projekten durchaus auch der Kunde sich die Profile mit anschaut und dann auch entscheidet. Anhand von Interviews was macht Sinn, was passt bei mir ins Team vielleicht und denke ich, dass das am besten für mich wäre?\\

0:3:50 --> 0:4:2\\
I:\\
Wie werden diese Bedarfsmeldungen und Anforderungen denn so typischer Weise kommuniziert und dokumentiert? Gibt es da eine Art Ablauf, oder wie wird das gemacht?\\

0:4:2 --> 0:5:16\\
B:\\
Hängt auch immer ehrlicherweise vom jeweiligen Kunden ab. Bei manchen reicht ein Interview, welches du führst und dann schreibst du es in ein Word Dokument als Anforderungsbeschreibung nieder. Dann lässt man das gegen Zeichnen und dann ist gut. Manchmal muss man aber auch ein paar mehr Integration fahren und dann nochmal genau abzustecken, was denn Bestandteil der Beauftragung ist und was nicht. Also was die Bedarfsmeldung. Da muss man auch mal durchaus eins tiefer bohren, weil da teilweise der Gedanke, was der Kunde möchte nicht mit dem übereinstimmt, was eigentlich gebraucht wird. Das ist so, weil du klassisch irgendein Anforderungsdokument dafür fertig machst, wenn das immer ein bisschen höher geht Richtung Management. Beim Management mit dem Kunden kann es auch durchaus mal eine Präsentation sein, wo das Ganze nochmal ein bisschen aufbereitet ist. Ein bisschen klassisch, ein bisschen bunter und mit weniger Infos und mit weniger Tiefe präsentiert.\\

0:5:16 --> 0:5:38\\
I:\\
Welche Informationen sind denn für dich in Bedarfsmeldungen besonders wichtig oder auch unverzichtbar? Also zum Beispiel gibt es ja auch das Senioritätslevel.\\

0:5:38 --> 0:6:50\\
B:\\
Ja genau, also Senioritätslevel ist ehrlicherweise erstmal das zweitrangige. Das müssen wir im Nachgang einmal prüfen. Wichtig ist der Tech Stack und was vom Kunden kommt. Und quasi anhand der Bedarfsmeldung an sich. Wie hoch ist der Aufwand, der dahinter steckt? Sowohl durch den Kunden als auch das, was wir teilweise schätzen. Dann kann durchaus sein, dass der Kunde sagt, was gebraucht wird, er aber eigentlich keine Ahnung hat. Es kann vorkommen, dass der Kunde gerne in 2 Monaten durch ist, aber es keinen Sinn macht und wir mindestens ein halbes Jahr benötigt. Dann trifft man sich irgendwo in der Mitte und muss aber noch abgrenzen was Sinn macht und was nicht. Wenn das alles klar ist, dann kannst du überlegen was du eher brauchst. Machst du z.B. nur was mit Senioren? Oder reicht ein Junior. Das hängt dann immer eher von der Gesamtsituation ab.\\

0:6:50 --> 0:7:5\\
I:\\
Wie detailliert sollte dann eine Bedarfsmeldung sein? Gibt es Aspekte oder Informationen, die eigentlich in jeder Bedarfsmeldung drin sein sollten und müssen?\\

0:7:5 --> 0:7:24\\
B:\\
Ich glaube der Umfang ist immer ganz wichtig. Die Erwartungshaltung sollte immer detailliert sein. Und die Technologien ebenfalls.\\

0:7:24 --> 0:7:33\\
I:\\
Welche Herausforderungen oder oder Schwierigkeiten sind dann bei zum Beispiel unklaren oder unvollständigen Bedarfsmeldungen aufgetreten? Hattest du so etwas schon mal?\\

0:7:33 --> 0:8:8\\
B:\\
Ja durchaus, das sind immer Lernprozesse sowohl beim Kunden als auch bei dem, der die Anforderungen aufnimmt. Hängt immer vom Reifegrad des jeweiligen Konterparts ab. Das heißt auch, dass umso mehr du dich in der Situation meldest und Sachen aufgenommen hast, umso genauer kannst du mal nachfragen und hörst auch die Unsicherheit auf der einen oder anderen Seite heraus.\\

0:8:8 --> 0:8:17\\
I:\\
Gibt es denn irgendwie ein Mechanismus, wie die Qualität von Bedarfs Meldungen bewertet wird, oder gibt es irgendwelche Kriterien oder Standards womit dann beurteilt wird, ob eine Bedarfsmeldung gut oder eher schlecht ist?\\

0:8:17 --> 0:8:58\\
B:\\
Ich bin ein Fan davon auch Sachen mal querlesen zu lassen und dann vielleicht mal die ein oder andere Meinung einzuholen und auch noch mit dem Kunden, der die Bedarfsmeldungen stellt ganz nah dran zu bleiben und dann zu gucken, dass das immer funktioniert. Damit auch zum Schluss das rauskommt, was am Anfang vielleicht schon ne Idee gewesen ist.\\

0:8:58 --> 0:9:7\\
I:\\
Wie würdest du denn die Qualität von Bedarfsmeldungen verbessern?\\

0:9:7 --> 0:9:41\\
B:\\
Auch wirklich mit dem Kunden das Ganze einmal durchexerzieren und festzustellen, ob das Verständnis auf allen Seiten das gleiche ist. Weil nur dann kann es auch gut und produktiv werden. Wenn direkt am Anfang schon irgendwie Unklarheit da ist, dann kannst du auch davon ausgehen, dass da Diskussionsbedarf entsteht.\\

0:9:41 --> 0:9:49\\
I:\\
Welche Auswirkungen haben denn unklare und fehlende Informationen in Bedarfsmeldungen in Bezug auf die Effizienz und den Erfolg des Projektes? Hast du da irgendwie Erfahrungen machen können?\\

0:9:49 --> 0:10:31\\
B:\\
Ja, wenn wir uns unklar sind, wird der Aufwand immer um ein Vielfaches erhöhen, weil das irgendwie im Nachgang immer noch mal gerade gezogen werden muss. Wenn du Pech hast, kannst du alles, was du bisher gemacht hast wegschmeißen und nochmal neu anfangen. Bedeutet natürlich auch immer, dass ein großes Diskussionspotenzial zwischen sowohl den Projektbeteiligten als auch Kunde und Projekt existiert.\\

0:10:31 --> 0:10:44\\
I:\\
Wie könnte man theoretisch sicherstellen, dass die Bedürfnisse und Anforderungen aller Leute, die involviert sind, irgendwie angemessen berücksichtigt werden?\\

0:10:44 --> 0:14:56\\
B:\\
Ich glaube wenn du da mit viel Erfahrung reingehst und dann auch vielleicht genau weißt, wo du drauf zu achten hast. Mir fehlt ein bisschen die Fantasie, aber das ist ja jetzt dann deine Aufgabe da so ein Automatismus zu erkennen. Das wird auf jeden Fall schwierig weil ich auch an der Stelle glaube, dass wenn du so ein System schaffst, die quasi den Match drauf machen können, es trotzdem viel lernen muss. Von daher glaube ich, Erfahrung ist das es ausmacht, um am Ende zu sagen was Sinn macht oder nicht.\\
\subsection{Interview 3}
\label{interview3}
%Kirchner, Lars\\
0:0:14 --> 0:0:22\\
I:\\
Ich habe mir ein paar Infos geholt. Du warst zumindest mal richtig CC-Leiter oder?\\

0:0:22 --> 0:0:45\\
B:\\
Ja ich war mal richtig CC Leiter genau. Ich kann zumindest sagen, dass ich sogar mal 2 Kompetenz Center, 1 in München und 1 in Dortmund geleitet habe und für 48 Menschen zuständig war.\\

0:0:45 --> 0:0:51\\
I:\\
Ok, das bedeutet aber du machst teilweise noch CC Leitung. Oder gar nicht mehr?\\

0:0:51 --> 0:1:21\\
B:\\
Nein, aus in der Tat persönlichen Gründen habe ich vor 2 Jahren die Entscheidung getroffen, dass ich diese Rolle verlassen muss.\\

0:1:39 --> 0:1:44\\
I:\\
Okay, das heißt aber auch du bist aktuell Projektleiter.\\

0:1:44 --> 0:6:1\\
B:\\
Ich bin aktuell von der Laufbahnstufe höher. Programm Manager das ist formal noch eine Führungslaufbahn oder Führungsrolle. Das ist etwas anderes zumindest in der adesso Welt als Projektleitung. Ich kümmere mich in der Tat dann in einer Mischform von Projektleitung und Produktunterschiede um interne Projekte. Ich arbeite mit Studierenden zusammen, ich übernehme Angebotsmanagement für große Angebote. Ich vertrete einzelne Themen, bei dem man jemanden braucht, der entsprechend erfahren ist und dann auf der anderen Seite gewisse intellektuelle Fähigkeiten mit sich bringt. Das sind dann aber eher Sonderthemen. Also ein relativ buntes Sammelsurium. Was ich nicht mehr habe ist Personal Verantwortung. Nicht weil ich nicht mit Menschen umgehen kann, sondern sind beispielsweise die sehr verwalterischen Aspekte nicht so angenehm, die nun mal in dieser Rolle mit drinstecken. Ich übernehme auch noch Delivery Management. Das heißt für große Kunden, gibt es eine spezielle Schnittstellen Rolle, die im Prinzip die adesso Organisation vor dem Kunden Kapsel, weil der Kunde gar nicht im Detail wissen soll, wie wir aufgebaut sind. Der Kunde hat genau solche Bedarfsanfragen und sagt, ich brauche einen Senior Java Developer mit diesen Fähigkeiten und ich bin dann als Delivery Manager dafür zuständig.\\

0:6:1 --> 0:6:12\\
I:\\
Dann wäre jetzt meine Frage was die typischen Stakeholder bei einer Erstellung von einer Bedarfsmeldung sind und welche Rolle hast du dabei?\\

0:6:12 --> 0:10:12\\
B:\\
Also Bedarfsmeldungen sind ja ein sehr wichtiges Element in einem der adesso Kerngeschäftsprozesse. Wir sind ein IT-Dienstleister. Das bedeutet, wir entwickeln nicht selber etwas im Sinne von Produkten. Das ist etwas anderes und nennt sich Produkt Geschäft. Das heißt also, wenn wir etwas tun, Entwicklungstätigkeiten aufnehmen, beraten, brauchen wir und benötigen wir immer einen Auftrag. Also einen Kunden. Wenn wir Kundenaufträge haben, weil wir ja ein IT Dienstleister sind, heißt das unsere Kerngeschäftsobjekte sind die Mitarbeitenden, die mit unterschiedlichen Qualifizierungen daneben in genau diesen Kundenaufträgen tätig sind. Das können vom Kunden durchgeführte Projekte sein. Ein Kundenauftrag kann aber auch sein, dass adesso ein Projekt für den Kunden durchführt. Wir müssen wie gesagt diese Projektstellen gezielt besetzen. Das heißt, wir haben immer eine Projektorganisation, mal ist es auch eine Programmorganisation, wo diese Stellen mit entsprechenden rollen oder Stellenanforderungen beschrieben werden. Entweder kommen die vom Kunden direkt, oder wir formulieren die selbst. Diese Beschreibung der Anforderungen, also welche Fähigkeiten und welche Erfahrung und in welchem Umfang Menschen diese mitbringen müssen, damit sie genau diese Projektstelle besetzen können und damit eine ganz bestimmte Rolle in einem Projekt Kontext einnehmen oder in einem Programm Kontext… Diese Beschreibung ist es, was wir als Bedarfsmeldung bezeichnen. Die Quellen dafür sind entweder Kundenorganisationen, dass die dann durch große Accounts wie e-on, die Deutsche Bahn, die ihre ganz eigenen Systeme haben und dann in einer, nicht normierten, aber in einer semi strukturierten Form diese Anforderungen dokumentiert werden. Die gehen dann mehr oder weniger 1 zu 1 an uns über und wir müssen damit weiterarbeiten. Bis dahin, dass wir als Delivery Manager, als Projektleiter, als Programm Manager, als Account Manager oder Vertriebler mit Kunden sprechen, in Projekte reinschauen, Projektorganisationen definieren und selbst in diesen Rollen eine Bedarfsmeldung erfassen. Es geht allerdings jedes Mal darum, innerhalb eines Projektes oder Programms eine bestimmte Stelle zu besetzen.\\

0:10:12 --> 0:10:20\\
I:\\
Welche Art von Projekten hat adesso typischerweise? Hast du da vielleicht ein paar Beispiele?\\

0:10:26 --> 0:18:3\\
B:\\
Die erstmal abstrakteste und wichtige oder grundsätzliche Unterscheidung ist: Es gibt Projekte, die ein Kunde eine Kundenorganisation aufsetzt und durchführt, an denen wir uns dann beteiligen, indem wir beispielsweise in bestimmte Rollen an bestimmte Stellen dort Menschen reinbringen. Das heißt, wir arbeiten dann aber in einem extern definierten und normalerweise auch gesteuerten und kontrollierten Projektkontext. Das andere ist, wenn wir im Kundenauftrag Projekte aufsetzen. Da könnte man dann auch nochmal differenzieren. Einmal Projekte unter unserer Kontrolle im Kundenauftrag und dann gibt es natürlich auch interne Projekte. Da kann man aber relativ schnell sagen, dann ist es halt eine interne Stakeholder Position. Zum Beispiel kann die HR-Abteilung auch sowas beauftragen. Das unterscheidet sich dann nicht wesentlich. Was allerdings da dann besser in unserer Kontrolle liegt, ist in der Regel, wenn wir selbst die Projekte in der Organisation in der Durchführung verantworten. Dann haben wir auch die Ausprägung der Projekt Organisation, das Vorgehensmodell, usw. in der Hand. Da haben wir sehr viel mehr Flexibilität, was eben die Definition von Rollen, Anforderungen usw. angeht. Das ist so der wesentliche Unterschied in Bezug auf externe Projekte und von uns durchgeführte Projekte. Es gibt dann aber nochmal einen wesentlichen Unterschied in Bezug auf die Vergütung. Wir unterscheiden da grundsätzlich zwischen Time und Material Aufträgen oder Beauftragung und Festpreis Beauftragung. Festpreis bedeutet, dass es von unserer Seite eine bestimmte, klar bemessene und abgegrenzte Leistung angeboten wird und die Auftraggeber Seite verhandelt mit uns dafür einen festgesetzten Preis. Damit muss man bei der Durchführung beachten, dass man nicht so lange vor sich hinarbeiten kann, bis dann zum Beispiel die Auftraggeberseite sagt OK, wir sind jetzt zufrieden. Sondern man hat halt ein beschränktes Budget und in der Regel auch beschränkte Zeit und trägt damit auch ein größeres Risiko. Dann gibt es da entsprechend die timen Material oder t und m Kontexte. Da trägt die Auftraggeberseite in der Regel das größere Risiko, weil wir abstrakt bezeichnet, erstmal nur dazu verpflichtet sind, nach durchschnittlicher Qualität und Güte solche einzelnen Leistungen über zum Beispiel Mitarbeitende beizusteuern, zu erbringen. Trotzdem hat auch da die Erfahrung gezeigt: Wir sind immer an unserer Kunden Zufriedenheit oder langfristigen Kundenbindung interessiert, dass es uns auch gar nicht hilft, wenn wir in einem t und m Kontext nur durchschnittlich oder vielleicht auch mal schlechte Arbeit leisten und dann hinterher sagen, das war jetzt aber gar nicht unsere Verantwortung. Das ist euer Risiko gewesen. Damit kommen wir auch nicht durch. Und es ist in den seltensten Fällen auch so, dass ein Kunde unerschöpfliche Geldmittel hat. Selbst wenn dieser Kunde diese unerschöpflichen Geldmittel hätte, sagen sie aus rein wirtschaftlichen Aspekten natürlich auch zu irgendeinem Zeitpunkt das reicht jetzt, wir möchten nicht noch mehr Geld ausgeben. Das ist doch mal eine grundsätzliche Unterscheidung, was das Bezahlen angeht. Mit hinein spielt auch noch eine Unterscheidung in die Art der Projekte. Das ist nämlich einmal ein Gewerk, wo wir Gewährleistung übernehmen und das auch entsprechend kalkulieren müssen. Ganz häufig werden Gewerke in Kombination mit Festpreisen angeboten und durchgeführt. Interessanterweise müssen sie es aber gar nicht zwingend. Im Umkehrschluss meistens, wenn ich nach t und m arbeite, handelt es sich auch dann von unserer Leistung um Dienstleistung. Das sind aber teilweise, wenn man da wirklich genau draufschaut oder wenn man da juristisch drauf schauen würde feine Unterschiede. Wenn man beispielsweise bei der Beauftragung oder in der Kommunikation ganz bestimmte Begriffe verwendet und es würde irgendwann vor Gericht landen, egal ob man einen Dienstleistungsvertrag abgeschlossen hat und die ganze Zeit meinte auch als Dienstleistungen zu arbeiten, könnte zum Beispiel ein Gericht aufgrund von Formulierungen usw. hinterher feststellen, dass es sich doch um ein Gewerk gehandelt hat. Das sind dann teilweise eher juristische Unterschiede. Am Ende des Tages bedeutet Gewerk natürlich, dass wir an einem Stück Software gearbeitet haben und eben für die konsistente Fehler freie gesamte Funktionalität dann entsprechende Gewährleistung anbieten und übernehmen. Also damit der Kunde für das Geld, dass die Organisation gezahlt hat, eine einsetzbare Lösung bekommt. Wohingegen bei Dienstleistung eine gar nicht funktionale Lösung bei rauskommen muss. Beispiel typischer und klassischer Bereich für Dienstleistungsgeschäft ist ja Consulting. Consulting bedeutet qualifizierte, erfahrene Menschen zu bestimmten Themen kommunizieren, was ausarbeiten, etwas aufschreiben, können etwas spezifizieren können. Aber letztendlich ist da die menschliche Arbeit beziehungsweise der Erkenntnisgewinn im Zentrum und es wird keine Lösung in dem Sinne geschaffen und bereitgestellt. Dass sind die für mich zumindest relevanten Unterscheidung. \\

0:18:3 --> 0:18:16\\
I:\\
Wie werden dann Bedarfsmeldungen und die Anforderungen typischerweise bei Adesso kommuniziert und auch dokumentiert. Gibt es einen Ablauf, wie das genau gehandhabt wird?\\

0:18:16 --> 0:24:4\\
B:\\
Für den Staffing-Prozess an sich gibt es einen definierten Ablauf, der aber in Großteilen aus manueller Arbeit und manuellem Arbeitseinsatz besteht. Es gibt keine normierte Form einer Bedarfsmeldung. Es hat sich allerdings zumindest eine grobe thematische Struktur etabliert. Das bedeutet in der Regel hat man so etwas wie eine Überschrift und Bezeichnung. Dann hat man in der Regel einen Bereich, der den Einsatz Kontext ein wenig allgemeiner beschreibt. Dann hat man einen Bereich, der auf die individuell geforderten Fähigkeiten und Erfahrungen eingeht. Man hat normalerweise eine Gewichtung dieser Skills. Das bedeutet in Bezug auf Expertise, die da erwartet wird oder mal solche Dinge wie Primary in Secondary usw. Unterteilungen. Wir haben in diesen Bedarfsmeldungen dann aber auch wirtschaftlich relevante Informationen damit verknüpft. Das ist typisch für ein Dienstleistungsunternehmen. Das bedeutet uns interessiert dann die vertraglichen Konditionen im Sinne von Tagessatz, was für eine Organisation für adesso, im Sinne von Dienstleistung auch absolut relevant ist. Das sind dann immer diese Parameter. Ab wann ist der Einsatz gewünscht? Und für wie lange? Weil wir immer prüfen müssen, selbst wenn wir beispielsweise sehr gut fachlich passende Mitarbeitende finden, sind sie aber eventuell schon in anderen Projekten eingesetzt und können dementsprechend gar nicht dort leisten. Es mag dann je nach Kundenkontext noch weitere Informationsblöcke geben. Wir arbeiten mittlerweile auch mit einem Leveling-System, weil man sich als Organisation oder eben als Mensch innerhalb einer Organisation ein bisschen besser in Abfragen organisieren kann, wenn an solchen detaillierteren Bedarfsmeldungen gewisse Labels dran stehen wie beispielsweise Azure als Technologie oder oder Java. Dann kann ich eine potenziell größere Menge von Bedarfsmeldungen, die ich manuell durchsuche besser filtern und einschränken. Was in Bezug auf unserem Staffing-Prozess sehr relevant ist, ist beispielsweise dann der Status einer solchen Bedarfsmeldung, weil die Organisationen nicht alle Bedarfsmeldungen interessiert. Manche Bedarfsmeldungen sind schon verarbeitet und erfüllt. Manche sind nur dokumentiert und sollen aber gar nicht weiter beachtet werden und erst wenn sie beispielsweise bei uns eskaliert werden, sollte eigentlich die relevante Organisation darauf schauen. Wir haben eine vertikale Einteilung des Unternehmens in Branchen, bzw. technologisch getriebenen Organisationseinheiten. Das sind unsere sogenannten Line of Business. Mittlerweile sind das sogar Business Areas und es ist an der Stelle durchaus relevant, welche adesso Organisationseinheit für so eine Business Area oder Line of Business zuständig ist. Wenn wir an der Stelle mit einem gewissen Vorkaufsrecht feststellen, dass wir es nicht bedienen können, werden natürlich alle gefragt und es dürfen beispielsweise aus der Line of Business Motiv selbstverständlich auch mitarbeiten in Projekten der Line Cross Industries und umgekehrt getätigt werden. Aber umso mehr ist es wichtig, dass dann die jeweils geforderten Skills und Erfahrungen und wirtschaftlichen Konditionen möglichst präzise beschrieben werden, damit diese Fragen die aufkommen nicht immer wieder gestellt werden und immer wieder beantwortet werden müssen. Das ist aber, wie gesagt schon Kern Staffing-Prozess.\\

0:24:4 --> 0:24:22\\
I:\\
Du hast auch schon einige Punkte in Richtung Verfügbarkeit genannt, dass das sehr wichtige Aspekte sind. Gibt es besonders wichtiger oder auch unverzichtbare Punkte in einer Bedarfsmeldung, die in jeder drin sein sollte?\\

0:24:22 --> 0:28:46\\
B:\\
Das kommt auf die Perspektive an. Wenn ich jetzt eine rein fachliche Perspektive einnehme, ist es da unverzichtbar, dass aufgelistet oder benannt wird, welche Fähigkeiten oder Skills konkret mitgebracht werden müssen und idealerweise auch in welcher Erfahrung und Güte das ist. Aus der Perspektive zum Beispiel des Dienstleistungsunternehmen adesso ist relevant, dass ein Beginn des Einsatzes, ein voraussichtlicher Einsatzzeitraum, ein Tagessatz dran steht. Dass da dran steht, ob Freelancer, ob Smartphone, smartshore Fähigkeit gegeben ist oder near Shore, oder wie die sprachliche Ausrichtung ist. Also ob das zum Beispiel deutschsprachig ist, ob Englisch. Entsprechend zulässige Kommunikationsmittel sind wichtige Zusatzinformationen, die genau darauf abzielen, dass alle Menschen, die potenzielle Kandidaten/Kandidatinnen auf so eine Bedarfsmeldung gefiltert werden. Wenn es nämlich auf der Ebene, dass die Verfügbarkeit aber auch darüberhinausgehend nicht passt oder wenn es zum Beispiel von einem Tagessatz, aus welchen Gründen auch immer sehr unattraktiv ist, es dazu führen kann, dass dann die Organisation bestimmte Personen nicht anbietet. Das ist aber wie gesagt eine Perspektive, die durch die Natur von adesso als Dienstleister, und wir sind ein auslastungsgetriebenes unternehmen, was halt auch versucht wirtschaftlich zu arbeiten, darüber reinkommen. Idealerweise möchte zum gewünschten Einsatzbeginn auch dann wirklich eine Person nicht nur theoretisch benannt, sondern idealerweise interviewt, geprüft, eingeführt und wie auch immer wurde und dann wirklich los laufen kann.\\

0:28:46 --> 0:28:51\\
I:\\
Gibt es Herausforderungen und Schwierigkeiten bei unklaren oder unvollständigen Bedarfsmeldungen? Hast du da Erfahrungen machen können?\\

0:28:51 --> 0:39:18\\
B:\\
Ja. Bei einem Auftraggeber handelt es sich in der Regel um Branchen, die eben nicht als Kerngeschäft IT Software Entwicklung betreiben. Dementsprechend fällt es ihnen teilweise schwer, präzise Bedarfsmeldungen zu formulieren, die inhaltlich genau das transportieren oder beinhalten oder umfassen, was sie eigentlich an Fähigkeiten und Erfahrung benötigen. Das liegt einfach daran, dass sie in Teilen nicht für jeden Kunden gilt, aber in Teilen den Prinzip Menschen etwas beschreiben lassen, die davon nicht wirklich Ahnung haben. Das wiederum führt dazu oder kann dazu führen, dass der Kunde etwas anfordert, was der Kundenorganisation, dem Projekt oder wie auch immer im schlimmsten Falle sogar gar nicht hilft. Oder aber das ist eine andere Ausprägung, dass dann Kombinationen von Erfahrung und Fähigkeiten gesucht werden, die es in der realen Welt einfach so nicht gibt. Das ist die berühmte eierlegende Wollmilchsau, wo man drauf schaut und sagt, diese Menschen hätten wir auch gerne als Mitarbeitende. Vielleicht gibt es auf diesem Planeten auch eine Handvoll davon, aber das ist unrealistisch. Und das kann wie gesagt darin begründet sein, dass auf der auftraggebenden Seite jetzt Menschen damit beauftragt werden, solche Bedarfsmeldungen zu formulieren, die das nicht wirklich können. Eigentlich haben wir an der Stelle schon immer einen idealerweise von uns moderierten Prozess. Deswegen macht zum Beispiel auch die Rolle Delivery Management Sinn. Das hängt wie gesagt auch sehr mit der insgesamten Qualifikation oder Qualität von der Auftraggeberseite in diesem Kontext zusammen. Bei manchen Kundenorganisationen ist das trotzdem sehr gut eingespielt und etabliert und funktioniert auch so. Bei manchen muss man da früh ansetzen und sagen wir sprechen miteinander, und ich arbeite dann beispielsweise in diesem Gespräch heraus, was der Kunde wirklich benötigt, was abgrenzbar eine sinnvolle Bedarfsmeldung ist, was es für eine Stelle beinhaltet und womit wir dann weiterarbeiten können. Das ist oder kann ein Problem sein, wenn wir nicht entsprechend damit umgehen. Nicht nur die fachliche Qualität oder auch die die Passgenauigkeit, die wir dort anbieten können ist da wichtig, sondern auch vor allem die Schnelligkeit vom Staffing-Prozess. Bedeutet, wenn ein Unternehmen sich so aufgestellt hat, dass eine Bedarfsmeldung, die reinkommt innerhalb von sehr kurzer Zeit bearbeitet wird, sagen wir mal 2 Stunden, dann habe ich einen klaren Vorteil gegenüber zum Beispiel einem Anbieter, der dafür 2 Tage braucht oder eine Woche oder 2 Wochen. Denn für die Auftraggebende Seite ist klar, wenn ich dann Rückmeldungen bekomme und ich schaue da rein und diese Rückmeldungen sind plausibel… Was hält mich davon ab, dann zu sagen ich beauftrage ihn jetzt. Es muss nicht perfekt sein, aber wenn es plausibel ist und die Konditionen sind gut, dann ist der Auftrag ausgesprochen und die anderen gehen logischerweise leer aus. Wenn wir eben entsprechend in unserem Matching entweder nicht gut arbeiten oder überfordert sind und beispielsweise einen Software Architekt für Java mit bestimmten weiteren Anforderungen gefordert ist und wir bieten da ein Profil drauf an, wo man darauf schaut und feststellt, dass es ein .Net Software Developer ist, ist es bei diesem extremen Beispiel so, dass es glücklicherweise auch sofort auffällt. Aber das würde natürlich dann zu Irritationen auf Kundenseite führen. Bedeutet: Wir sollten nicht nur dafür sorgen, dass die Erwartungen oder Anforderungen möglichst gut an der Realität sind, sondern wenn wir dann wiederum auch da etwas matchen und einreichen, dass das dann auch diesen Anforderungen nach Möglichkeit entspricht und, dass innerhalb von möglichst kurzer Zeit so geschieht. Und die unglücklichste Variante ist, wenn man ein Prozess durchläuft und man bietet da jemanden an und der wird sogar genommen und wird eingearbeitet und dann stellt man irgendwie fest der macht irgendwie Unsinn. Man hat nie eine Garantie. Letztendlich sind es Menschen die dort arbeiten.\\

0:39:18 --> 0:39:22\\
I:\\
Welche Auswirkungen haben unklare oder auch fehlende Informationen in Bedarfsmeldungen jetzt aber konkret in Bezug auf die Effizienz und den Erfolg von Projekten.\\

0:39:27 --> 0:40:49\\
B:\\
Im Idealfall wird, möglichst früh erkannt, dass eine Bedarfsmeldung lückenhaft, unpräzise wie auch immer formuliert ist. Dann muss nachgefragt werden. Ich schaue mir etwas an, versuche zu verstehen, was die andere Seite sucht und wenn das für mich dann nicht konsistent auf die zumindest für mich bekannten Rollenlösung ist, dann muss ich nachfragen. Alles andere ist eine Interpretation. Dann läuft man mit sehr großer Wahrscheinlichkeit in die von mir gerade beschriebenen Probleme rein. Die schlechteste Variante ist, dass ich sage ich nehme die Informationen, die ich jetzt da vorgelegt bekommen habe, interpretieren sie nach besten Wissen und Gewissen und dann ist es aber mehr oder weniger ein Glücksspiel. Das heißt, wenn ich dann Profile finde, könnten sie immer noch zufällig das sein, was der Kunde eigentlich wollte und gesucht hat.\\

0:40:49 --> 0:41:1\\
I:\\
Die letzte Frage hast du im Grunde auch schon mit beantwortet. Wie könnte man sicherstellen, dass Bedürfnisse und Anforderungen aller relevanten Stakeholder in einer Bedarfsmeldung Berücksichtigt werden?\\

0:41:4 --> 0:42:26\\
B:\\
Zwei Möglichkeiten. Man könnte, da glaube ich aber nicht dran, natürlich den Prozess standardisieren und stark formalisieren. Also das im Prinzip von einer öffentlichen Stelle aus gesagt wird: Alle Dienstleister und Auftraggeber dieser Welt wenn ihr in dieser Art Geschäft betreiben wollt, müsst ihr so ein Format einreichen. Also Bürokratie pur. Das würde uns nichts verbessern, aber das ist eine Möglichkeit. Die andere Möglichkeit ist meines Erachtens, dass dann für genau solche Prozesse entsprechend versierte Menschen diesen Prozess, das heißt die Anforderungserhebung, die Dokumentation, das erfüllen diese Anforderungen komplett begleiten und moderieren. Das ist Delivery Management. Das ist Business Development. Manchmal auch Account Management.\\

0:42:26 --> 0:42:30\\
I:\\
Dann sind wir eigentlich schon durch mit den Fragen. Hast du noch zu irgendeinem Punkt irgendwelche Fragen oder irgendwas, was vielleicht noch für mich in dem Themenbereich interessant sein könnte?\\

0:42:30 --> 0:45:6\\
B:\\
Die größte Schwierigkeit liegt darin, dass wir Informationen über eine natürliche Sprache transportieren. Das ist zwar flexibel, weil die Sprache an der Stelle ja eben nicht formalisiert ist. Was aber immer das Risiko mit sich bringt, dass etwas nicht präzise beschrieben, abgegrenzt oder interpretierbar wird und genau bei solchen Prozessen, wo wir eigentlich präzise arbeiten wollen, haben wir genau diese große Herausforderung, dass die bisher benutzte Art, um diese Informationen zu ermitteln und die gerade wieder zu lesen zu interpretieren eben ein Stück weit ungenügende Mittel, nämlich den natürlichen sprachigen Raum verwendet. Da ist zwar mit gesundem Menschenverstand gearbeitet worden. Das heißt, es haben sich Semistrukturen gebildet. Aber es ist wirklich sehr individuell unterschiedlich in was für einer Qualität oder was für einer Realitätsnähe solche Bedarfsmeldungen formuliert werden. Wenn wir auf unserer Seite jemanden sitzen haben, der oder die eben auch in diesem Umfeld relativ wenig Ahnung und Erfahrung hat, dann haben wir auch nochmal ein Risiko, dass selbst wenn die Bedarfsmeldungen präzise und realitätsnah formuliert ist, bei uns in der Interpretation etwas schiefgeht. Das bedeutet, dass halt die Menschen, die auf unserer Seite Bedarfsmeldungen lesen und versuchen zu bedienen, leider auch hinreichend viel Erfahrung in der Projekt IT haben müssen.\\
\subsection{Interview 4}
\label{interview4}
%Herbermann, Sebastian\\
0:0:49 --> 0:1:5\\
I:\\
Erstmal vorweg zu deiner Person. Ich weiß, dass du CC Leiter bist und auch Softwarearchitekt bist. Kannst du vielleicht kurz erzählen, wie deinen Werdegang. Was du studiert hast und wie du dann bei adesso gelandet bist?\\

0:1:5 --> 0:1:38\\
B:\\
Studiert habe ich Kerninformatik an der TU Dortmund mit einem Diplomabschluss. Danach bin ich 2003 im Bereich der Software Entwicklung gestartet. Und habe dann seitdem Software Architekten und Führungskraft gemacht.\\

0:1:38 --> 0:1:39\\
I:\\
Was sind denn dann so die typischen Stakeholder bei der Erstellung von Bedarfsmeldungen und was für eine Rolle hast du dabei?\\

0:1:57 --> 0:2:0\\
B:\\
Stakeholder bei der Erstellung von Bedarfsmeldungen?\\

0:2:0 --> 0:2:4\\
I:\\
Wer sind so Person, die da dran grundsätzlich beteiligt sind.\\

0:2:11 --> 0:3:4\\
B:\\
Wenn wir Bedarfsmeldungen erstellen, sind wir über den Akquise Prozess schon hinaus. Das heißt, der Vertrieb ist raus. Damit ist für die Erstellung der Bedarfsmeldungen der Projektleiter und Maitre eigentlich relevant. Also hängt immer davon ab. Der Projektleiter hat ein Bedarf und stimmt den mit dem internen Maitre ab, wer dann für das Staffing verantwortlich ist. Das heißt, sie sind auch für die Bedarfserstellung verantwortlich. Daneben hast du dann noch Leute, die Input geben für die Bedarfsmeldung, also Input, der jetzt nicht zum Beispiel durch den Vertrag geregelt ist, wäre so etwas wie das Geld Profil, was dann vom Architekten zum Beispiel zugeliefert wird.\\

0:3:4 --> 0:3:14\\
I:\\
Was sind denn so typische Projekte, die bei Adesso angenommen und bearbeitet werden? Hast du ein paar Beispiele für Projekte, die so durchgeführt werden?\\

0:3:20 --> 0:3:27\\
B:\\
Typische Projekte alles rund um die Software Entwicklung, also sowohl in den Development als auch Consulting rollen.\\

0:3:34 --> 0:3:47\\
I:\\
Und wie werden Projektbedarfe und Anforderungen innerhalb von adesso kommuniziert und dokumentiert? Also gibt es da irgendwie einen Ablauf?\\

0:3:47 --> 0:3:55\\
B:\\
Ja, und die Bedarfsmeldungen werden über das Jira erfasst und entsprechend auch mit allen beteiligten Parteien geteilt.\\

0:3:55 --> 0:4:19\\
I:\\
Was sind Aspekte, die in einer Bedarfsmeldung besonders wichtig sind? Gibt es unverzichtbare Punkte, die immer in einer Bedarfsmeldung drin sein sollten.\\

0:4:19 --> 0:5:47\\
B:\\
Rahmenbedingungen wie Kunde, Einsatzbeginn, Laufzeit oder Ort und Tagessatz. Und die notwendigen Skills sowie eine Aufgabenbeschreibung damit klar ist, was tatsächlich inhaltlich zu tun ist und was vom Kunden oder in dem Fall für ein Profil gefordert wird. Indem Moment von die Aufgabenbeschreibung oder die geforderten Skills fehlen verstehe ich gar nicht, was gesucht wird und kann dementsprechend auch keine passenden Profile anbieten. Die anderen Rahmenbedingungen brauche ich natürlich, um auch zu prüfen, ob das Personal entsprechend überhaupt zum Beispiel verfügbar ist.\\

0:5:47 --> 0:5:53\\
I:\\
Welche Herausforderungen hat man denn bei unklaren oder unvollständigen Bedarfsmeldungen?\\

0:5:53 --> 0:6:2\\
B:\\
Eine Herausforderungen ist, dass das angebotene Personal gegebenenfalls nicht zu dem Einsatz passt oder bei dem Einsatz nicht verfügbar ist. Das heißt am Ende, dass die Rückmeldungen auf die Bedarfsmeldungen gar nicht zielführend ist.\\

0:6:17 --> 0:6:26\\
I:\\
Gibt es Kriterien oder Standards, um die Qualität der Bedarfsmeldungen sicher zu stellen?\\

0:6:26 --> 0:7:4\\
B:\\
Es gibt einerseits die technische Validierung über die Daten im Jira, die man eingibt. Weitergehende Validierung erfolgt dann, nicht immer aber meistens nochmal im 4 Augen Prinzip. Die Projektleitung meldet sich da zur Abstimmung nochmal.\\

0:7:4 --> 0:7:36\\
I:\\
Wie würdest du denn die Qualität und Klarheit von Bedarfsmeldungen verbessern? Was für Mechanismen würdest du denn anwenden, die eventuell auch noch gar nicht von jedem benutzt werden?\\

0:7:36 --> 0:8:41\\
B:\\
Ich glaube, man könnte einige Daten strukturierter erfassen. Bei der inhaltlichen Beschreibung ist es schwierig, das tatsächlich noch automatisiert weiter zu verbessern. Die Herausforderung ist jetzt nicht 15 Java Entwickler Profil zu finden, sondern die besonderen Profile, die wir selten haben, die dann aus der Struktur raus sein, wo du halt zum Beispiel nicht sagen kannst, du Hinterlegst die Skills in der Liste und sagt, ich brauche irgendwie Java. Die Herausforderungen ist natürlich alles, was aus dem Raster herausfällt.\\

0:8:41 --> 0:8:48\\
I:\\
Im Bezug auf den Projekterfolg hast du da schon Erfahrungen gemacht, welche Auswirkungen unklare und fehlende Informationen in Bedarfsmeldungen dann wirklich am Ende im Projekt haben?\\

0:8:48 --> 0:9:32\\
B:\\
Die Herausforderung ist, dass die Projektmitarbeitenden dann den Aufgaben im Zweifelsfall nicht gewachsen sind, weil notwendiges Know How in den Technologien fehlt. Weil andere Skills auch unternommen und auch Softskills eventuell fehlen. Und somit das Projekt gar nicht sicher, oder der Projekterfolg gar nicht sichergestellt werden kann.\\

0:9:32 --> 0:9:45\\
I:\\
Wie kannst du sicherstellen, dass die Bedürfnisse und Anforderungen aller involvierten Personen angemessen berücksichtigt werden?\\

0:9:56 --> 0:10:34\\
B:\\
Je nachdem in welche Rolle und in welchen Schritten im ganzen Prozess durch intensives Lesen und drauf schauen. Oder Rückfragen stellen. Rückfrage heißt immer das eine Informationen gefehlt hat.\\

0:10:34 -->0:11:26\\
I:\\
Ich hätte noch eine spezielle Frage an dich, weil du auch Software Architekt bist und mit vielen Technologien zu tun hast. Hättest du eine Idee, wie du grundsätzlich das Problem der Informationsgewinnung von unstrukturierten Bedarfsmeldungen angehen würdest. jetzt auch vielleicht in Bezug auf Technologien oder Ansätzen. Hast du da irgendwelche Erfahrungen irgendwann mal machen können?\\

0:11:26 --> 0:14:27\\
B:\\
Man könnte natürlich gerade in der schwierigen Erfassung von den Freitextinformationen, die nicht stark strukturiert sind im Zweifel weitere Validierungen bauen, die auch dann vielleicht über das hinausgehen, was uns Jira an Validierung anbietet. In den Freitextinformationen könnte ich mir vorstellen, dass man da KI gestützt Prüfungen macht, um eine Vollständigkeit sicherzustellen.\\
\subsection{Interview 5}
\label{interview5}
%Neubauer, Johannes\\
0:4:52 --> 0:4:58\\
I:\\
Wer sind denn die typischen Stakeholder bei der Erstellung von Bedarfsmeldungen und was für eine Rolle hast du dabei?\\

0:5:14 --> 0:5:19\\
B:\\
Bei adesso ist es im Normalfall so, dass Sales oder der interne Maitre dafür zuständig ist. Er stellt quasi eine Bedarfsmeldung im Normalfall auf Basis von Anforderungen direkt vom Kunden. Also entweder die kommen aus etwas, was wir angeboten haben, oder bei einer Rahmenvereinbarung kommen sie vielleicht aus einem Wettbewerb. Oder es kommt konkret im Projekt aus mündlich genannten Bedarfsmeldungen. Das kann auch durch ein Scrum Master oder Product Owner entsprechend entstehen. Da ist dann derjenige, der die Bedarfsmeldung erstellt auch zuständig, die Kriterien festzulegen. Was für ein Mitarbeiter oder Mitarbeiterin, mit welchen Fähigkeiten wir brauchen und stellt die dann ein. Ich bin Bereichsleitungsebene. Das heißt Ich bin einerseits in den Call, wo sich alle Bereichsleiter des gesamten Unternehmens treffen, um einmal zu schauen, welche Bedarfe eigentlich gerade eskaliert sind. Das heißt, welche sind gerade, oder müssen in naher Zukunft aufgelöst werden? Vieles wird einfach im direkten Kontakt behandelt. Jemand sieht das Ticket, bietet drauf und alles ist gut. Es gibt manche Bedarfsmeldungen, die schwerer zu bedienen sind. In solchen Fällen bin ich involviert. Und ansonsten bin ich eher in einer Support-Rolle in dem Prozess, weil ich entweder jemanden die Vertretung übernehme, weil ich im Normalfall nicht interner, sondern externer Maitre bin. Das heißt, ich bin eher für das Kunden Management und für die Gemeinsame Gestaltung, wie man zukünftig die Zusammenarbeit gestaltet zuständig, aber nicht so sehr jetzt für einzelne Bedarfsmeldungen. Ansonsten direkte Berührungspunkte mit konkreten Bedarfsmeldungen hab ich nicht.\\

0:10:59 --> 0:11:8\\
I:\\
Wie werden Bedarfsmeldungen und Anforderungen grundsätzlich kommuniziert und dokumentiert?\\

0:11:8 --> 0:14:11\\
B:\\
Von demjenigen der es einstellt, werden die Kriterien definiert und in einem JIRA-Ticket überführt. Das kann sehr unterschiedliche Formen annehmen.Es gibt ein paar Felder, die strukturiert sind. Z.B zu welchem Tagessatz das ganze angeboten wird, wann das Ganze startet, wie hoch das Volumen also an Tagen ist. Es gibt einige Freitextfelder, bei dem drinsteht, was die Aufgaben usw. sind. Zum Beispiel bei einer Rahmenvereinbarung, die wir gerade machen gibt es dann ein Excel, was ausgefüllt werden muss mit seiner Selbstbeurteilung. Wie gut der Kandidat, der jetzt gefunden wird oder gefunden werden soll, darauf passt. So ein Bewertungsschema wird dann mit beigefügt, das die Führungskraft ausfüllen kann. In einem anderen Fall kann das aber ganz anders aussehen, da wird dann einfach der Export aus unserem Profiler, wo also die Profilbeschreibung drinsteht, einfach als Kommentar in das JIRA hinzugefügt und gesagt: Den könnte man jetzt anbieten und das ist eher ein manueller Prozess. Das ist sehr verschieden. Deswegen nutzen wir doch JIRA, weil es im Normalfall nicht genau den einen Case gibt, wie wir Bedarfsmeldung reinkriegen und wie wir diese auch beantworten müssen. Im Normalfall ist es eben so, dass die Bedarfsanfragen in die Welt rausgeschickt werden. Also erstmal in die eigene Organisationseinheit und wenn man dort niemanden hat, der sehr gut passt, dann eben in die gesamte adesso und die Führungskräfte gucken von der anderen Seite da drauf und schauen, ob der eigene Mitarbeiter darauf passt. Im Normalfall ist es nicht so, dass diese Bedarfsmeldungen eine konkrete Suchanfrage sind, sondern die sind eher eine Bedarfsmeldung, wo drauf sich dann jemand melden kann.\\

0:14:11 --> 0:14:21\\
I:\\
Was sind denn besonders wichtige oder auch unverzichtbare Informationen innerhalb einer Bedarfsmeldung?\\

0:14:21 --> 0:16:50\\
B:\\
Wichtig ist für die Führungskraft, die einen Mitarbeiter anbietet, ab wann die dann frei sind. Bis wann, wie lange wieviel Prozent dieser Mitarbeiter oder diese Mitarbeiterin eingesetzt wird. Für denjenigen, der den Bedarf meldet. Ist es am Interessantes oder am wichtigsten was die muss und soll Kriterien sind? Damit der Mitarbeiter möglichst gut Match. Man kann sagen, dass es für beide Seiten wichtig ist. Für die Führungskraft ist es erstmal interessant ob und für wie lange er Mitarbeiter in einem Projekt kriegt. Bekomme ich die passende Ressource auf meine Anfrage? Nachrangiger ist der Tagessatz. Das ist gesamtwirtschaftlich fürs Unternehmen sehr wichtig. Aber dadurch, dass wir eine sehr offene und transparente bereichsübergreifende Zusammenarbeit pflegen, ist so, dass wir, wenn einmal einen Preis, wie z.B. den Tagessatz, von einer Einheit verhandelt wurde, dass das dann von einer anderen Einheit nicht in Frage gestellt wird oder gesagt wird jetzt können wir hier den Mitarbeiter nicht anbieten, weil der Tagessatz zu niedrig ist. Das ist dann so verhandelt und wir sind dann ein Commitment als Firma eingegangen. Wir haben uns selber als Ziel gesetzt, eben hohe Qualität abzuliefern und Kunden zu Partnern zu machen und deswegen bedienen wir dann diese Anfragen.\\

0:16:50 --> 0:17:2\\
I:\\
Welche Herausforderungen oder Schwierigkeiten sind bei unklaren oder auch unvollständigen Bedarfsmeldungen aufgetreten? Hast du da irgendwie Erfahrungen machen können?\\

0:17:2 --> 0:19:17\\
B:\\
Wenn die Bedarfsmeldungen zu wenig Informationen darüber enthalten, was genau die Mitarbeiter, die da drauf angeboten werden sollen, können sollen, wo die vielleicht auch sitzen sollen, idealerweise, ob das ein Angebot einer einzelnen Arbeitskraft ist oder ob das im Rahmen eines Teams ist. Zu schauen ob der Mitarbeiter oder die Mitarbeiterin auf diese Anfrage passt. Wenn ich jetzt einen Junior Mitarbeiter habe, der ganz frisch bei Adesso angefangen hat, dann sollte ich ihn vielleicht nicht auf ein Ticket anbieten, wo diese Person dann ganz alleine zum Kunden geschickt wird. Das wäre unfair der Person gegenüber und wahrscheinlich auch nicht von Erfolg gekrönt. Wenn auf irgendeiner Ebene was fehlt, führt es dazu, dass entweder falsche Zuordnung gemacht werden, oder dass es einen Kommunikationsoverhead gibt. Oder, dass Versprechungen gemacht werden, die dann hinterher nochmal korrigiert werden müssen, weil es dann doch nicht gut passt. Das heißt es ist schon essentiell, dass die Bedarfsmeldungen gut sind, weil nur dann auch die erste Wahl auch eine gute Wahl ist. Wir haben und deswegen in beide Richtungen Schwierigkeiten. wir haben aber teilweise, dass Mitarbeiter angeboten werden, die nicht so gut passen, wo dann die Informationslage auf der anderen Seite schwierig ist. Das ist dann unter Umständen ein großer Overhead und weil man dann auf der Seite sehr viel Aufwand und unter Umständen durch Interviews oder ähnliches betreiben muss. Was dann auch sehr zeitaufwendig sein kann.\\

0:19:17 --> 0:19:21\\
I:\\
Wie wird denn die Qualität von Bedarfsmeldungen bewertet? Gibt es irgendwelche Kriterien oder Standards die ihr einhaltet?\\

0:19:21 --> 0:24:2\\
B:\\
Es gibt, wie gesagt ein paar feste Felder, die auch in dem JIRA eben festgelegt sind, auch ein paar feste Zustände und Übergänge, die einen groben Rahmen geben. Dann gibt es bei größeren Rahmenvereinbarungen feste Rahmenbedingungen, in denen gesagt wird, so übersetzen wir die Anforderungen des Kunden in eine Bedarfsmeldung. Aber es ist nicht so, dass es eine übergreifende Qualitätssicherung über die Tickets gibt, sondern das ist eher ein Lernprozess. Leute die Tickets erzeugen machen, das entweder schon eine Weile oder es wird dort irgendjemanden zur Seite gestellt, der das schon länger macht. Die muss Kriterien was sind die? Soll Kriterien, dass man dabei schreibt. Nicht nur, dass jemand ein Skill hat, sondern auch wie viel Erfahrung er oder sie in dem Skill. Und noch ein paar weitere Punkte, die die dann ja dazu führen, dass die Ticket Qualität hoffentlich gut ist. Aber sie ist auch nicht durchweg gut, das muss man auch dazu sagen. Es gibt sehr gute Tickets und es gibt welche, die nicht ganz so gut sind. Wobei man bei manchen auch fairerweise sagen muss, dass die unter Umständen dann auch aus Rahmenvereinbarungen kommen, wo wir eigentlich nur Anfragen durch Reichen und auch nur teilweise ein zwei Tage Zeit ist, um überhaupt eine Anfrage zu machen und uns nicht sehr viel mehr übrigbleibt als nahezu alles was wir von außen reinbekommen dann 1 zu 1 zu übernehmen. Weil dort nicht genug Raum ist, um einen richtigen Angebotsprozess zu machen. Bei großen Rahmenvereinbarungen gibt es dann auch nochmal auf JIRA Tools, die dabei unterstützen diese Bedarfsmeldungen zu erzeugen, zu tracken bis hin zu einem Angebot zu bringen.\\

0:24:2 --> 0:24:19\\
I:\\
Wie kannst du denn sicherstellen, dass die Bedürfnisse und Anforderungen aller involvierten Leute in Bedarfs Bildung berücksichtigt werden?\\

0:24:19 --> 0:25:43\\
B:\\
Ich glaub, es gibt nicht die eine Person, die das sicherstellen kann. Das ist eine gemeinsame Verantwortung. Damit dann am Ende sichergestellt ist, dass das gut funktioniert, gibt es sicherlich ein paar Punkte, wo man positiv darauf einwirken kann. Wir machen regelmäßig Bereichsmeetings, wo dann auch ein Agenda Punkt im Normalfall ist wie das Staffing läuft. Wir haben auch wöchentlich Jour Fixe, wo wir dann über das Tagesgeschäft reden und was die aktuell laufende Staffing-Prozesse sind? Wie sieht die Auslastung aus usw.? Dort kann man schauen wo denn Probleme sind und dann entsprechend kommunizieren. Viele von den Bedarfen werden gar nicht im JIRA eingetragen, sondern es werden direkt entsprechende Mitarbeiter angesprochen.\\
\newpage
\section{Paraphrasierung und Reduktion}
\label{sec:reduktion}
%\begin{center}
\begin{longtable}{| p{0.5cm} | p{0.5cm} | p{4cm} | p{4cm} | p{4cm} |}
	%{
	%	| >{\hsize=.2\hsize}X 
	%	| >{\hsize=.2\hsize}X
	%	| >{\raggedright\arraybackslash}X
	%	| >{\raggedright\arraybackslash}X
	%	| >{\raggedright\arraybackslash}X | }
	\hline
	Fall & Nr. & Paraphrase & Generalisierung & Reduktion \\
	\hline
	\hline
	\endhead
	%\sout{test generalisierung}
	1 & 1 & Bedarfsmeldungen beschreiben Software-Entwicklungsprojekte, angefangen von Projekten in dem ein adessi in einem Kundenprojekt arbeitet über gemischte Teams bis hin zur kompletten Lieferung. & Bedarfsmeldungen Beschreiben Software-Entwicklungsprojekte wie Kundenprojekte, bei dem in gemischten Teams bis zur Auslieferung gearbeitet wird & \multirow{3}{4cm}{K1 Arten von Projekten \\ -tttttt \\ -ttttttt \\ -t \\ \
		
		K2 Kommunikationswege \\ -tttttt \\ -ttttttt \\ \
		
		K3 Wichtige Informationen \\ -tttttt \\ -ttttttt \\ \
		
		K4 Detaillierungsgrad \\ -tttttt \\ -ttttttt \\ \
		
		K5 Herausforderungen \\ -tttttt \\ -ttttttt \\ \
		
		K6 Qualitätsbewertung \\ -tttttt \\ -ttttttt \\ \
		
		K7 Qualitätsverbesserung \\ -tttttt \\ -ttttttt \\ \
		
		K8 Auswirkungen \\ -tttttt \\ -ttttttt \\} \\
	\cline{1-4}
	1 & 2 & Initial über den Maitre, der das Staffing übernimmt, teilweise auch über das eigene Netzwerk zwischen Führungskräften. Am Ende über das Staffing Jira. & Maitre, Führungskräftenetzwerk, am ende Staffing Jira & \\
	\cline{1-4}
	1 & 3 & Senioritätslevel, Tagessatz, Remote/on Site Einsatz, Dauer, Technischer Stack (Muss- und Kann Kriterien), Einarbeitungszeiträume, Lieferverpflichtung. & Senioritätslevel, Tagessatz, Remote/on Site Einsatz, Dauer, Technischer Stack (Muss- und Kann Kriterien), Einarbeitungszeiträume, Lieferverpflichtung & \\
	\cline{1-4}
	1 & 4 & Bedarfsmeldungen sollten aussagefähig sein, welche technischen Kompetenzen wichtig sind, welche Tagessätze, ob Remote möglich ist und die Dauer. & Aussagefähigkeit von Bedarfsmeldungen durch technischen Kompetenzen, Tagessätze, Remote Möglichkeit und Dauer & \\ 
	\cline{1-4}
	1 & 5 & Der Anforderer muss ggf. Fragen mehrfach beantworten. Dadurch verliert man ggf. den Überblick. & Überblick verlieren durch Mehrfachbeantwortung von Fragen & \\ 
	\cline{1-4}
	1 & 6 & Typische Stakeholder sind Sales/PL: Anforderer mit den technischen Informationen, Maitre: kümmert sich um das Staffing. & Stakeholder sind Sales/PL: Anforderer mit technischen Informationen, Maitre & \\ 
	\cline{1-4}
	1 & 7 & Bewertung der Qualität erfolgt gar nicht. & keine Qualitätsbewertung & \\ 
	\cline{1-4}
	1 & 8 & Qualitätsverbesserung durch klarere Vorgaben und weniger Freitext, aktuell: durch Nachfragen und Bitten um nachträgliche Pflege, ggf. durch Reviewprozesse. & Qualitätsverbesserung durch klare Vorgaben, weniger Freitext, Nachfragen und nachträgliche Pflege (Reviewprozess) & \\ 
	\cline{1-4}
	1 & 9 & Auswirkungen fehlender Informationen sind, dass das Staffing länger dauert und ggf. die Stellen durch andere Dienstleister besetzt werden. & Fehlinformationsauswirkungen sind längerer Staffingprozess und Umbesetzung durch anderen Dienstleister & \\ 
	\cline{1-4}
	1 & 10 & Gute Abstimmungen bevor die Bedarfsmeldung erstellt wird, ggf. durch ein Quality-Gate (Review). & Abstimmung vor Bedarfsmeldungserstellung durch Quality-Gate (Review) & \\ 
	\cline{1-4}
	2 & 11 & In Bedarfsmeldungen sind die typischen Stakeholder Fachverantwortliche, Entscheider und höhere Management-Ebenen wie CEO oder Geschäftsführer. & Stakeholder sind Fachverantwortliche, Entscheider, CEO oder Geschäftsführer & \\ 
	\cline{1-4}
	2 & 12 & Projekte lassen sich in Time Material Projekte und Festpreis Projekte unterteilen, wobei die Art der Projekte die Ressourcenplanung beeinflusst. & Projektunterscheidung ind Time Material und Festpreis Projekte & \\
	\cline{1-4} 
	2 & 13 & Die Dokumentation der Bedarfsmeldungen variiert je nach Kundenanforderungen, von einfachen Word-Dokumenten bis zu detaillierten Präsentationen. & Dokumentation von Bedarfsmeldung variieren (von Word-Dokument bis Präsentation) & \\ 
	\cline{1-4}
	2 & 14 & Wichtig in Bedarfsmeldungen sind das Tech Stack, der Aufwand und die zeitlichen Erwartungen. & Wichtig sind Tech Stack, Aufwand und zeitliche Erwartungen & \\ 
	\cline{1-4}
	2 & 15 & Senioritätslevel ist zweitrangig und abhängig von der Projektsituation. & Senioritätslevel zweitrangig & \\ 
	\cline{1-4}
	2 & 16 & Unklare oder unvollständige Bedarfsmeldungen führen zu erhöhtem Aufwand und möglichen Verzögerungen, da Missverständnisse geklärt werden müssen. & Erhöhter Aufwand und Verzögerungen durch unklare/unvollständige Bedarfsmeldungen & \\ 
	\cline{1-4}
	2 & 17 & Die Qualität von Bedarfsmeldungen kann verbessert werden, indem man sicherstellt, dass das Verständnis auf allen Seiten übereinstimmt und Missverständnisse vermieden werden. & Qualitätssteigerung von Bedarfsmeldungen durch Verständnisübereinstimmung und Beseitigung von Missverständnisse & \\ 
	\cline{1-4}
	2 & 18 & Unklare Informationen in Bedarfsmeldungen erhöhen den Aufwand und führen zu Nacharbeit und potenziellen Konflikten zwischen den Beteiligten. & Unklare Informationen erhöhen Aufwand und resultieren in Nacharbeiten und Konflikten & \\ 
	\cline{1-4}
	2 & 19 & Erfahrung ist entscheidend, um sicherzustellen, dass die Bedürfnisse und Anforderungen aller Beteiligten angemessen berücksichtigt werden. & Erfahrung ist zur Sicherstellung von Bedürfnissen entscheidend & \\ 
	\cline{1-4}
	3 & 20 & Bedarfsmeldungen können entweder vom Kunden direkt kommen oder von uns selbst formuliert werden. & Bedarfsmeldungen können selbst formuliert oder vom Kunden stammen  & \\ 
	\cline{1-4}
	3 & 21 & Bedarfsmeldungen enthalten wirtschaftlich relevante Informationen wie vertragliche Konditionen, Tagessatz, Einsatzbeginn und voraussichtlicher Einsatzzeitraum. & Bedarfsmeldungsinfomationen sind vertragliche Konditionen, Tagessatz, Einsatzbeginn und Einsatzzeitraum & \\ 
	\cline{1-4}
	3 & 22 & Die typische Struktur einer Bedarfsmeldung beinhaltet eine Überschrift, eine allgemeine Beschreibung des Einsatzkontexts, eine Liste der geforderten Fähigkeiten und Erfahrungen sowie deren Gewichtung. & Bedarfsmeldungen enthalten eine Überschrift, eine Beschreibung des Einsatzkontexts, eine Liste der Fähigkeiten und Erfahrungen und deren Gewichtung & \\ 
	\cline{1-4}
	3 & 23 & Unklare oder unvollständige Bedarfsmeldungen können zu Missverständnissen führen und den Staffing-Prozess verlangsamen. & Unvollständige Bedarfsmeldungen führen zu Missverständnissen und langsamen Staffing-Prozess & \\ 
	\cline{1-4}
	3 & 24 & Projekte können entweder vom Kunden gesteuert werden oder von uns im Kundenauftrag durchgeführt werden. & Bedarfe sind entweder Kundengesteuert oder Kundenaufträge & \\ 
	\cline{1-4}
	3 & 25 & Die Art der Projektvergabe (Time and Material oder Festpreis) beeinflusst die Anforderungen und den Ablauf. & Beeinflussung der Anforderungen und Ablauf durch Art der Projektvergabe & \\ 
	\cline{1-4}
	3 & 26 & Der Staffing-Prozess bei adesso ist manuell und nicht normiert, aber es gibt eine grobe thematische Struktur, die sich etabliert hat. & Keine feste, aber grobe Struktur zur Erfassung des Staffing-Prozesses & \\ 
	\cline{1-4}
	3 & 27 & Die Schnelligkeit im Staffing-Prozess ist ein Wettbewerbsvorteil. Eine schnelle und präzise Bearbeitung von Bedarfsmeldungen ist entscheidend für den Erfolg. & schnelle und präzise Bearbeitung von Bedarfsmeldungen ist Erfolgsentscheidend & \\ 
	\cline{1-4}
	3 & 28 & Versierte Personen wie Delivery Manager oder Account Manager sollten den Prozess der Anforderungserhebung und Dokumentation begleiten und moderieren, um sicherzustellen, dass alle relevanten Informationen erfasst werden. & Begleitung des Bedarfsmeldungsdokumentierung durch Delivery Manager oder Account Manager & \\ 
	\cline{1-4}
	4 & 29 & Die Erstellung von Bedarfsmeldungen erfolgt durch Projektleiter und interne Maitre. Diese stimmen den Bedarf ab und sind für das Staffing verantwortlich. & Staffingverantwortliche sind Projektleiter und Maitre & \\ 
	\cline{1-4}
	4 & 30 & Rahmenbedingungen wie Kunde, Einsatzbeginn, Laufzeit, Ort und Tagessatz sowie notwendige Skills und Aufgabenbeschreibungen sind unverzichtbar in Bedarfsmeldungen. & Unverzichtbare Bedarfsmeldungsinformationen sind Rahmenbedingungen, Einsatzbeginn, Laufzeit, Ort und Tagessatz, Skills und Aufgabenbeschreibungen & \\ 
	\cline{1-4}
	4 & 31 & Bedarfsmeldungen werden über Jira erfasst und mit allen beteiligten Parteien geteilt. & Bedarfsmeldungserfassung in Jira & \\ 
	\cline{1-4}
	4 & 32 & Unklare oder unvollständige Bedarfsmeldungen führen dazu, dass das angebotene Personal möglicherweise nicht passt oder verfügbar ist. & Unvollständige/Unklare Bedarfsmeldungen führen zu unpassenden Personal & \\ 
	\cline{1-4}
	4 & 33 & Technische Validierung der Daten erfolgt über Jira, weitergehende Validierung im Vier-Augen-Prinzip durch die Projektleitung. & Qualitätssicherung erfolgt über Jira und der Projektleitung & \\ 
	\cline{1-4}
	4 & 34 & Verbesserungsvorschlag: Strukturiertere Erfassung der Daten und KI-gestützte Prüfungen für Freitextinformationen. & Qualitätssicherungsvorschlag durch strukturierte Datenerfassung und KI-gestützte Prüfungen & \\ 
	\cline{1-4}
	4 & 35 & Fehlende Informationen in Bedarfsmeldungen können dazu führen, dass Mitarbeitende den Aufgaben nicht gewachsen sind, was den Projekterfolg gefährdet. & fehlende Informationen führen zu fehlplatzierten Mitarbeitern und gefährdeten Projekterfolg & \\ 
	\cline{1-4}
	4 & 36 & Durch intensives Lesen, Rückfragen und die Berücksichtigung aller involvierten Personen werden Bedürfnisse und Anforderungen sichergestellt. & Sicherstellung der Anforderungen und Bedürfnisse durch intensives Lesen, Rückfragen und Berücksichtigung involvierter Personen & \\ 
	\cline{1-4}
	5 & 37 & Bedarfsmeldungen werden normalerweise von Sales oder dem internen Maitre erstellt, basierend auf Kundenanforderungen. & Bedarfsmeldungserstellung durch Maitre oder Sales basierend auf Kundenanforderungen. & \\ 
	\cline{1-4}
	5 & 38 & Der Ersteller der Bedarfsmeldung legt auch die Kriterien für den benötigten Mitarbeiter fest. & Festlegung der Kiterien durch den Ersteller der Bedarfsmeldung & \\ 
	\cline{1-4}
	5 & 39 & Die Bedarfsmeldungen werden in JIRA-Tickets überführt, die verschiedene Felder enthalten, wie Tagessatz, Startdatum, Volumen und Aufgaben. Es gibt auch Freitextfelder und Excel-Dokumente zur Selbstbeurteilung des Kandidaten. & Überführung der Bedarfsmeldung in Jira in die Felder Tagessatz, Startdatum, Volumen, Aufgaben, Freitext und Selbstbeurteilung & \\ 
	\cline{1-4}
	5 & 40 & Wichtige Informationen in einer Bedarfsmeldung sind das Startdatum, die Dauer und der Prozentsatz der Einsatzzeit des Mitarbeiters sowie die Muss- und Soll-Kriterien, um die Passgenauigkeit des Mitarbeiters sicherzustellen. & Wichtige Bedarfsmeldungsinformationen sind Startdatum, Prozentsatz, Einsatzzeit und Muss-/Soll- Kriterien & \\ 
	\cline{1-4}
	5 & 41 & Unklare oder unvollständige Bedarfsmeldungen führen zu falschen Zuordnungen, Kommunikationsaufwand und notwendigen Korrekturen. Dies kann auch zu einem erheblichen Overhead und Zeitaufwand durch Interviews führen. & Unklare/Unvollständige Bedarfsmeldungen führen zu falscher Zuordnung, Kommunikations-/Zeitaufwand und Korrekturen & \\ 
	\cline{1-4}
	5 & 42 & Es gibt feste Felder und Zustände in Jira, die einen groben Rahmen bieten, sowie feste Rahmenbedingungen bei größeren Vereinbarungen. Es gibt jedoch keine übergreifende Qualitätssicherung, sondern es handelt sich um einen Lernprozess. & Trotz fehlender Qualitätssicherung groben Rahmen durch feste Felder in Jira und Erfahrung & \\ 
	\cline{1-4}
	5 & 43 & Die Bedürfnisse und Anforderungen aller Beteiligten werden durch regelmäßige Meetings, wöchentliche Jour Fixe und direkte Kommunikation berücksichtigt. Viele Bedarfe werden direkt angesprochen und nicht im JIRA eingetragen. & Bedarfe durch regelmäßigen Meetings, Jour Fixe und Kommunikation besprochen und berücksichtigt & \\ 
	\hline
	\caption{Erstellung der Kategorien aus den Experteninterviews.\label{tab:kategorien}}\\
\end{longtable}
%\captionof{table}{Kategorien}
%\label{tab:Erstellung der Kategorien aus den Experteninterviews}
%\end{center}
