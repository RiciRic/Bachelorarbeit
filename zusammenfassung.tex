\chapter{Zusammenfassung und Ausblick}
\label{chap:ergebnisseausblick}

ergebnis der arbeit: diese modelle in der reihenfolge kommen am nähesten an die bedarfsmeldung

Fragestellung beantworten

- ist deterministisch

schluss --> was habe ich gebaut und wie performt das, system funktioniert hinreichend gut,

schluss--> es macht sinn oder nicht das system weiter zu erforschen

prototyp proof of concept --> gucken ob es vielversprechend



\subsection*{Ausblick}
Im Rahmen künftiger Überlegungen könnte die Ermittlung der Schlüsselwörter alternativ mittels eines durch Experten angelegten Katalogs mit aus der Branche enthaltenen Schlüsselwörtern erfolgen. Eine Steigerung der Ergebnisqualität ist potenziell zu erwarten, wenn präzisere Schlüsselwörter verwendet werden. Ein weiterer Evaluationsschritt in Bezug auf die NLP-Modelle könnte in einem Austausch der Modellgrößen bestehen. Eine Steigerung der Ergebnisqualität könnte durch den Einsatz von größeren und komplexeren Modelle erzielt werden. Des Weiteren besteht die Möglichkeit, den Übersetzungsschritt durch andere Anbieter durchführen zu lassen, wodurch fachlich korrektere Übersetzungen gewährleistet werden können. In Bezug auf das Recommender-System zur Mitarbeiterempfehlung kann eine Evaluierung erfolgen, ob eine Anwendung dieses Systems auch für die Mitarbeiterprofile denkbar wäre.
\newpage
