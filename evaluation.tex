\chapter{Evaluierung des entwickelten Systems}
\label{chap:evaluation}

vielleicht erklären warum precision, recall, f1 score nicht gehen -\\

nicht überlegen wie evaluieren sonder was will ich evaluieren,\\
was sind die fragen die ich beantworten möchte, was sind die aussagen die ich machen will. hypothesen belegen, wiederlegen\\
was möchte ich zeigen, (den expertenprozess abbilden, expertenprozess ist ideal, mein prozess hat diese abweichung)\\

z.b. erwartungshaltung formulieren und mit cosine similarity gucken was näher dran ist,

wie machen das andere ansätze,

\section{Evaluationsmetrik}
-cosine similarity
-performance, zeit

\section{Beschreibung des verwendeten Datensatzes}
\todo{darauf eingehen welche Felder in der JSON von JIRA sind und worauf sich genau konzentriert wird.}\\
\newpage

\section{Beschreibung des verwendeten Datensatzes}

überlegung ob tfidf unterschied macht alle bedarfsmeldungen mit einer zu vergleichen und daraus wichtige wörter identifizieren oder eine für sich alleine reicht.

gucken was tokenisierung wirklich macht
\section{Präsentation und Diskussion der Ergebnisse}
\newpage
g
\newpage
g
\newpage
g
\newpage
Zeit und Leistung Übersicht
\newpage
g
\newpage

\section{Vergleich des Systems mit einem Large Language Model-Ansatz}
\newpage
g
\newpage
g
\newpage

g
\newpage

\section{Analyse von Abweichungen, Ähnlichkeiten und Verbesserungspotenzialen des Systems}
\newpage
g
\newpage