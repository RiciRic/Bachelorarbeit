\chapter{Grundlagen}
\label{chap:literaturüberblick}
In diesem Kapitel werden die für das Thema notwendigen Grundlagen und bereits erforschten Themengebiete im Kontext von Recommender Systemen und Informationsverarbeitung behandelt, die für das weitere Verständnis der Arbeit notwendig sind. Es wird ein Einblick in die Art und Weise gegeben, wie andere Autoren Information Retrieval und Filtering einsetzen und kombinieren.
\section{Kontext}
Als IT-Dienstleister wird \emph{adesso} von Kunden unter anderem mit der Entwicklung individueller Softwarelösungen beauftragt. Derzeit verbringen Führungskräfte jedoch viel Zeit damit, interne Mitarbeiterinnen und Mitarbeiter manuell für Kundenprojekte zu suchen und diese dann aufgrund ihrer Erfahrungen und Fähigkeiten auszuwählen und entsprechend einzusetzen. Dieser Prozess soll durch eine KI-Lösung unterstützt werden. Da es sich bei der Personalsuche um einen geschäftskritischen Prozess handelt, ist der Spielraum für Fehler gering. Im internen Projekt \emph{adesso Staffing Advisor} wird eine durch Large Language Model-gestützte Anwendung entwickelt, die Führungskräfte bei der Suche nach geeignetem Personal für ausgewählte Projekte unterstützt. Der Ansatz des Large Language Modeling ist jedoch nicht deterministisch. Es besteht die Gefahr, dass bei gleichem Input unterschiedliche Ergebnisse erzielt werden. Daher versucht \emph{adesso} durch den Einsatz von Methoden und Technologien neben dem Large Language Model-Ansatz deterministische Ergebnisse zu erzielen, die auf einem ähnlichen Niveau liegen.\\

Die vorliegende Ausarbeitung befasst sich mit der Informationsgewinnung in \emph{Bedarfsmeldungen}.
\todo{gucken wie ich das hier mache}
\section{Recommender Systems Historie und aktueller Stand der Forschung}
Auch wenn die Erstellung eines Recommender Systems nicht Gegenstand der vorliegenden Ausarbeitung ist, stellt die Nutzung von Information Retrieval und Filtering ein entscheidener Schritt in Richtung eines funktionierenden Recommender Systems dar. Das Verständnis der Funktionsweise eines Recommender Systems sowie dessen Entwicklung in den vergangenen Jahren ist daher für das Verständnis des Teilbereichs dieser Thematik von Nutzen.\\

Recommender Systems existieren bereits seit vielen Jahren. Im Jahr 1992 führten Belkin und Croft eine Analyse und einen Vergleich des Information Retrievals und Filtering durch \cite{dong2022brief}. Das Information Retrieval behandelt die grundlegende Technologie der Suchmaschine \cite{dong2022brief}. Das Recommender System basiert hauptsächlich auf der Technologie des Information Filtering. Im selben Jahr präsentierte Goldberg das Tapestry-System, welches das erste System zur Informationsfilterung darstellt, das auf kollaboratives Filtern durch menschliche Bewertung basiert. Die Mehrheit der frühen Empfehlungsmodelle basiert auf kollaborativer Empfehlungen, wobei K-Nearest-Neighbor (KNN)-Modelle eine besondere Rolle einnehmen. Diese Modelle prognostizieren die Nachbarn eines Zielnutzers, indem sie eine Ähnlichkeit zwischen den vorherigen Präferenzen und den Präferenzen der anderen Nutzer berechnen \cite{dong2022brief}. Die Studie von Goldberg inspirierte einige Forscher des Massachusetts Institute of Technology (MIT) und der University of Minnesota (UMN) dazu, einen Nachrichtenempfehlungsdienst mit dem Namen \emph{GroupLens} zu entwickeln. Die Hauptkomponente dieses Dienstes ist ein Modell zur kollaborativen Filterung zwischen Nutzern \cite{dong2022brief}. Das gleichnamige Forschungslabor kann somit als Pionier auf dem Gebiet der Recommender Systems bezeichnet werden. Die dort durchgeführten Forschungen bilden die Grundlage für nachfolgende Musik- und Video-Ähnlichkeitsempfehlungen \cite{dong2022brief}. \\

Recommender Systeme haben in den letzten Jahren verschiedene Definitionen erhalten. Eine dieser Definitionen wird in dem Artikel von Resnick und Varian (1997) sinngemäß so beschrieben, dass ein typisches Recommender System Empfehlungen durch Personen als Eingabe erhält, die das System dann zusammenschließt und an geeignete Empfänger weiterleitet \cite{burke2011recommender}. In einigen Fällen besteht die primäre Transformation in der Zusammenführung, in anderen Fällen liegt die Fähigkeit des Systems darin, gute Übereinstimmungen zwischen Empfehlungsgebern und Empfehlungsempfängern herzustellen \cite{burke2011recommender}. Empfehlungssysteme stellen ein Instrument zur Interaktion mit umfangreichen und vielschichtigen Informationen dar. Sie ermöglichen eine personalisierte Sicht auf diese Informationen, indem sie die für den Nutzer wahrscheinlich relevanten Inhalte aufbereiten \cite{burke2011recommender}. Besonders im Handelsverkehr im Internet sind Recommender Systeme ein häufiger Einsatzgebiet. Dabei werden Recommender Systeme als Werkzeuge zum Suchen und Filtern von Informationen verwendet, die dem Benutzer Vorschläge unterbreiten, die für ihn nützlich sein könnten. Sie sind in einer Vielzahl von Internetanwendungen weit verbreitet und helfen den Nutzern, bessere Entscheidungen bei der Suche nach Nachrichten, Musik, Urlaubsangeboten oder Geldanlagen zu treffen \cite{ricci2014recommender}. Eine spezifisches Recommender System konzentriert sich normalerweise auf eine Art von Themengebiet wie z. B. Filme oder Nachrichten \cite{ricci2014recommender}. Darüber hinaus sind sie zu einem entscheidenden Faktor in der Entscheidungsfindung von Organisationen geworden \cite{chartron2014general}. Unternehmen wie \emph{adesso} bauen immer weiter auf Recommender System unterstützte System auf, um Prozesse zu beschleunigen oder zu vereinfachen. Grundsätzlich können die Methoden in die Typen (i)\emph{collaborative Filtering-based} (kollaborative Empfehlungssysteme), (ii)\emph{content-based} (inhaltsbasierte Empfehlungssysteme), (iii)\emph{knowledge-based} (wissensbasiert Empfehlungssysteme) und (iv)\emph{hybrid} (hybride Empfehlungssysteme) unterteilt werden.\\

Jede Empfehlungsmethode hat ihre Vorteile und Grenzen \cite{lu2020recommender}. Insbesondere das inhaltsbasierte Empfehlungssystem bring eine hohe Relevanz für das Mitarbeiterempfehlungssystem. Die Grundprinzipien inhaltsbasierter Empfehlungssysteme sind zum einen die Analyse der Beschreibung der von einem bestimmten Benutzer bevorzugten \emph{Items}, um die gemeinsamen Hauptattribute (Präferenzen) zu identifizieren, die diese \emph{Items} unterscheiden. Diese Präferenzen werden in einem \emph{Benutzerprofil} gespeichert \cite{lu2020recommender}. Zusätzlich werden die Eigenschaften jedes \emph{Items} mit dem \emph{Benutzerprofil} verglichen, so dass nur \emph{Items} empfohlen werden, die eine hohe Ähnlichkeit mit dem \emph{Benutzerprofil} aufweisen \cite{lu2020recommender}. Bei der Idee der Mitarbeiterempfehlung kann also die \emph{Bedarfsmeldung} mit den benötigten Projektskills und Anforderung als \emph{Benutzerprofil} angesehen werden. Die Mitarbeiterprofile sind dabei die \emph{Items}. Die Attribute werden verglichen (Skills der Mitarbeiter mit den Skills und Anforderungen der \emph{Bedarfsmeldung}) und ähnliche \emph{Items} werden vorgeschlagen. Mit Hilfe traditioneller Methoden des Information Retrievals, wie z.B. dem Kosinus-Ähnlichkeitsmaß, werden dann Empfehlungen generiert \cite{lu2020recommender}. Darüber hinaus generieren sie Empfehlungen mit Hilfe von statistischen und maschinelle Lernverfahren, die in der Lage sind, Nutzerinteressen aus historischen Nutzerdaten zu lernen \cite{lu2020recommender}.
\section{Information Retrieval und Information Filtering}
%Diese Arbeit beschreibt den Unterschied zwischen Information Filtering und Information Retrieval\cite{belkin1992information}
Im Allgemeinen wird einem Informationssystem die Funktion zugeschrieben, den Benutzer zu den Dokumenten zu führen, die seinen Informationsbedarf am besten decken \cite{belkin1992information}. Allgemeiner ausgedrückt ist das Ziel eines Informationssystems, dem Benutzer Informationen aus der Wissensressource zur Verfügung zu stellen, die ihm helfen, ein Problem zu lösen \cite{belkin1992information}. Auf der anderen Seite ist unter Filtern das Entfernen von Daten aus einem eingehenden Datenstrom zu verstehen und nicht das Auffinden von Daten in diesem Datenstrom \cite{belkin1992information}. Filtersysteme verarbeiten große Datenmengen \cite{belkin1992information}. Typische Anwendungen betreffen Gigabytes von Text oder weitaus größere Mengen anderer Medien \cite{belkin1992information}. Während es bei dem Information Retrieval typischerweise um die einmalige Nutzung des Systems durch eine Person mit einem einmaligen Ziel und einer einmaligen Anfrage geht, befasst sich die Informationsfilterung mit der wiederholten Nutzung des Systems durch eine oder mehrere Personen mit langfristigen Zielen oder Interessen \cite{belkin1992information}.
\section{Data-Mining}
Data Mining ist ein interdisziplinäres Teilgebiet der Informatik, das sich mit der rechnergestützten Entdeckung von Mustern in großen Datenbeständen befasst \cite{jain2013data}. Ziel dieses fortgeschrittenen Analyseverfahrens ist es, Informationen aus einem Datensatz zu extrahieren und in eine für die weitere Verwendung verständliche Struktur umzuwandeln \cite{jain2013data}. Die verwendeten Methoden liegen an der Schnittstelle zwischen künstlicher Intelligenz, maschinellem Lernen, Statistik, Datenbanksystemen und Business Intelligence \cite{jain2013data}. Beim Data Mining geht es um die Lösung von Problemen durch die Analyse von Daten, die bereits in Datenbanken vorhanden sind \cite{jain2013data}.