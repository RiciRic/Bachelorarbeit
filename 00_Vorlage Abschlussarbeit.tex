%Dokumentklasse
\documentclass[a4paper,12pt]{scrreprt}
\usepackage[left= 3.5cm,right = 2cm, bottom = 2 cm]{geometry}
\addtolength{\footskip}{-0.5cm}
\usepackage[onehalfspacing]{setspace}
% ============= Packages =============

% Dokumentinformationen
\usepackage[hyphens]{url}
\usepackage[
pdfsubject={},
pdfauthor={Ricardo Valente de Matos},
pdfkeywords={},	
%Links nicht einrahmen
hidelinks,
breaklinks=true
]{hyperref}
% Standard Packages
\usepackage[utf8]{inputenc}
\usepackage[ngerman]{babel}
\usepackage[T1]{fontenc}
\usepackage{graphicx, subfig}
\graphicspath{{img/}}
\usepackage{fancyhdr}
\usepackage{lmodern}
\usepackage{color}

\usepackage{dirtree}

\usepackage[style=ieee, sorting=nty, urldate =comp, backend=bibtex]{biblatex}
\addbibresource{Literatur.bib}

%\usepackage[numbers]{natbib}
%\bibpunct{(}{)}{;}{a}{,}{,}

\usepackage{listings}

% zusätzliche Schriftzeichen der American Mathematical Society
\usepackage{amsfonts}
\usepackage{amsmath}
\usepackage{float}

\usepackage{tabularx}
\usepackage{multirow}
\usepackage{enumitem}

%nicht einrücken nach Absatz
\setlength{\parindent}{0pt}
\RedeclareSectionCommand[beforeskip=0pt]{chapter}
\usepackage{listings}
\usepackage{color}
\definecolor{lightgray}{rgb}{.9,.9,.9}
\definecolor{darkgray}{rgb}{.4,.4,.4}
\definecolor{purple}{rgb}{0.65, 0.12, 0.82}

\lstdefinelanguage{JavaScript}{
	keywords={typeof, new, true, false, catch, function, return, null, catch, switch, var, if, in, while, do, else, case, break},
	keywordstyle=\color{blue}\bfseries,
	ndkeywords={class, export, boolean, throw, implements, import, this},
	ndkeywordstyle=\color{darkgray}\bfseries,
	identifierstyle=\color{black},
	sensitive=false,
	comment=[l]{//},
	morecomment=[s]{/*}{*/},
	commentstyle=\color{purple}\ttfamily,
	stringstyle=\color{red}\ttfamily,
	morestring=[b]',
	morestring=[b]"
}

\lstset{
	language=JavaScript,
	backgroundcolor=\color{lightgray},
	extendedchars=true,
	basicstyle=\footnotesize\ttfamily,
	showstringspaces=false,
	showspaces=false,
	numbers=left,
	numberstyle=\footnotesize,
	numbersep=9pt,
	tabsize=2,
	breaklines=true,
	showtabs=false,
	captionpos=b
}


% ============= Kopf- und Fußzeile =============
\pagestyle{fancy}
%
\lhead{}
\chead{}
\rhead{\slshape \leftmark}
%%
\lfoot{}
\cfoot{\thepage}
\rfoot{}
%%
\renewcommand{\headrulewidth}{0.4pt}
\renewcommand{\footrulewidth}{0pt}

% ============= Package Einstellungen & Sonstiges ============= 
%Besondere Trennungen
\hyphenation{De-zi-mal-tren-nung}

\newcommand{\hiddenchapter}[1]{
	\chapter*{{#1}}
}

%-------------

\newcommand{\todo}[1]{\textcolor{red}{ToDo:} #1\marginpar{<--hier}}

% ============= Dokumentbeginn =============

\begin{document}
%Seiten ohne Kopf- und Fußzeile sowie Seitenzahl
\pagestyle{empty}

\begin{center}
	\begin{tabular}{p{\textwidth}}
		
		\begin{center}
			\textbf{\Large{Bachelorarbeit}}
		\end{center} \\ \\
		
		\begin{center}
			\LARGE{\textsc{
					%\textit{\emph{adesso Staffing Advisor Lab}}\\
					%Konzeption und prototypische Entwicklung der Struktur und Architektur einer Softwareplattform für Transparenz in KI-Anwendungen
					...
			}}
		\end{center}
		
		\\
		
		
		
		\begin{center}
			von
		\end{center}
		
		\begin{center}
			\large{\textbf{Ricardo Valente de Matos}}
		\end{center}
	
	\begin{center}
		\large{Matrikelnummer: 7203677} \\
		\large{im Studiengang Wirtschaftsinformatik \\
			der Fachhochschule Dortmund \\}
	\end{center}
		
		
		\\
		
		\\
		
		\begin{center}
			\begin{tabular}{lll}
				\textbf{Erstprüfer:} & & Prof. Dr.-Ing. Guy Vollmer\\
				\textbf{Zweitprüfer:} & & Stephan Schmeißer, M. Sc., Adessoplatz 1, 44269 Dortmund\\
			\end{tabular}
		\end{center}
	
	\\ \\
	
	\begin{center}
		\large{Dortmund, den \today}
	\end{center}
		
	\end{tabular}
\end{center}

%\setcounter{page}{1}
%\pagestyle{plain}

%letztes kapitel zusammenfassung und ausblick
\pagestyle{fancy}
\pagenumbering{Roman}
\tableofcontents
%\listoffigures
%\lstlistoflistings
\newpage
%-NOCH ZU ERLEDIGEN-\\

%\chapter*{Lesehinweis}
%Aus Gründen der besseren Lesbarkeit werden Wörter und Wortgruppen, die hervorgehoben werden oder mehrfach auftauchen, durch \emph{kursiven} Text kenntlich gemacht. Zudem wird in dieser Projektarbeit die Sprachform des generischen Maskulinums angewandt. Sämtliche Ausführungen sind jedoch geschlechtsunabhängig und beziehen sich damit auf alle Geschlechter.
\newpage

\setcounter{page}{1}
\pagestyle{fancy}
\pagenumbering{arabic}
\setcounter{chapter}{0}
\newpage

%ai act verabschiedet, gabs diskussion --> vielleicht gibt es da vorgaben
%gemini --> google hat was gefaked, advanced model, video war aber gefaked

%ich gehe in themensondierung
%vielleicht finde ich dabei was
%primärliteratur finden
%related work kapitel

%prblemstellung
%grundlagen um problem
%ausprägung des problems
%um problem zu lösen brauche ich diese anforderungen(gibt es andere die ähnliches problem haben und wie machen die das, und bewerte die herangehensweisen über meine anforderungen)

%empfehlungssysteme auswärten
%ki zweig, recommender systems

%gpt fragen ob es related work

-quantifizierung
was macht die metriken, wofür sind die da,
genauer checken,


related work genauer, wie hast du das verwendet

was will ich erreichen. semistrukturiert, unstrukturierte daten für matching


----------------

\cite{said2013evaluating}

\cite{lerche2017implizites}

\cite{silveira2019good}

----------------


\cite{andersson2022comparison}
distance and similarity measures: euclidean distance, cosine similarity, ...,
um dinge richtig zu vergleichen heuristic success metric, statistic analysis (t-test) performance(time, memory usage), ethical( energieverbrauch, )

vergleich zwischen traditionellen metriken wie cosine, euclidean, etc mit bert, performance tracken

\cite{navrozidis2020using}

python flask anwendung --> evaluation mit 17 trainees, implementationsdetails (wie die modelle implementiert sind, rource of models ), preprocessing teil, evaluation method ist precision, recall,

er hat cosine similarity genutzt um dokument nach relevant und unrelevant zu sortieren --> die dann als similar oder not similar eingestuft,



\cite{lavi2021consultantbert}
roc-auc, precision, recall, f1

\cite{jamshidian2023evaluation}
precision, recall, f1

optimierung preprocessing,

\cite{henderi2021text}
vergleich cosine, jaccard und dice's coefficient
preprocessing techniques

correlation and mae comparison(pearson correlation, mean absolute error )

\cite{gross2016funktionieren}
jaccard, pearson, precision recall

\cite{ahrendt2016genauigkeit}
genauigkeitsmetriken: MAE und RMSE, klassifikationsmetriken: precision, recall, f1, durchführung mit eigenen daten, daten vorbereiten

\cite{karypis2001evaluation}
top n empfehlungen

\cite{fard2013recommender}
precision, recall, f1

\cite{sondur2016similarity}
vergleich jaccard, pearson, cosine similarity, etc


-----------------

ich will die metriken in das frontend einbauen und anwenden, 

mehr ansätze hinzu nehmen? Pearson, jaccard sind andere ansätze für similarity --> evaluieren(precision, recall f1 )


drei schritte:

runtergebrochen vergleich von cosine similarity mit sbert und sklearn. Augenmerk auf die architektur und die fähigkeit mitarbeiter mit einer bedarfsmeldung zu vergleichen, vergleich der ansätze mit zeit und speichernutzung

-ansätze und metriken erforschen (genauigkeitsmetriken: MAE und RMSE, klassifikationsmetriken: precision, recall, f1, daten vorbereiten)
-implementationsdetails(visuell darstellen im lab, was steht bereits)
-evaluieren (metriken anwenden --> vergleich der drei ansätze bzw vielleicht nehme ich ansätze hinzu, performance (zeit und speichernutzung) ) --> daraus schlüsse ziehen welcher ansatz am geeignetsten für Mitarbeiterempfehlung ist


------------

welcher ansatz repräsentiert die bereits verteilten vorschlägen,

ergebnisgüte bewerten
gucken ob ansätze ergebnis

Anforderungskatalog erarbeiten --> entscheidung macht sinn oder nicht sinn
--> möglichkeiten quantifiziert --> man fragt leute die das tagtäglich machen, statistische verfahren, man braucht kontrolle, reproduzieren können ob ergebnis liefert was man haben will, bedarfsmeldung und person ob stelle bekommen hat, crossvalidation, ethische verwendung von ki in bewerbung


-->  oder cc leiter fragen ob die vielleicht datenlage haben, da mal fragen, projektleiter, aquise, personal in der hinterhand, lars ist in aquisen, systematisch aufarbeiten, mechanismus zum durchjagen der daten, deterministische ergebnis wenn sich profilker nicht ändert, ergebnisanalyse, mechanismus zum wiederholen um aussage zu treffen
ziel: bei neuen ansätzen neue erkenntnisse ziehen. Quasi einen riesen Test/benchmark erstellen für die dinger, wiederholbar, Benchmarking system (kann man viel drüber philosophieren, t-test ist sehr basic, interessante frage was eignet sich da eigentlich um sowas zu bewerten), was für ne beurteilungsmetrik kann man da verwenden. Diskurs mit welchen mitteln geht man da ran ob ergebnis brauchbar oder nicht brauchbar. Welche art der probleme haben wir --> welche lösungen können wir benutzen (einmal durcharbeiten)
rapidminer, panda python, welche existierenden verfahren haben wir, vor und nachteile --> vielleicht reicht ein verfahren nicht, wir müssen diese drei auswertungen machen, anderer diskurs wäre kann man das für reinforced learning nutzen, bevor nutzer ergebnis sieht um bewertung vorher zu machen.

projekt eingefrieren, repo darf sich nicht mehr ändern durch andere commits. ich selbnst arbeite da dann nur dran

ai ergebnisse interpretieren, amazon ai bewerbung hireing interessanter ansatz (einfach mal reingucken, wie stellen die fest dass system so kacke ist, rassistisch, auch weil ähnlicher anwendungsfall, qualität von ai ergebnissen sicherstellen(statistik etc)),



-------------------------------------------------------
fachsprache berücksichtigung, text bestimmte form,

ki modell neu entwickeln

macht das ein unterschied ob der inhalt fachlich ist.



-------------------------------------------------------

zu jedem zeitpunkt das system verwenden können,

----------------------

information retrieval

\cite{kobayashi2000information}

\cite{singhal2001modern}

\cite{croft2000combining}

\cite{horesh2016information}

\cite{belkin1992information}

information filtering
\cite{lanquillon2001enhancing}

preprocessing
\cite{alasadi2017review}

-------
spam-filter
\cite{shafi2017review}
\cite{khorsi2007overview}
\cite{tretyakov2004machine}
----



\chapter{Einleitung}
\label{chap:einleitung}

\newpage
\section{Problemstellung}
\label{sec:problemstellung}

\newpage
\section{Ziele und Ergebnisse der Arbeit}
\label{sec:zieleundergebnis}

\section{Aufbau der Arbeit}

\newpage

\chapter{Grundlagen}
\label{chap:grundlagen}

\section{Künstliche Intelligenz}
\label{sec:kunstlicheintelligenz}

1. Starke KI beinhaltet Problemlösungen genereller Art. Das, was am Ehesten an sowas heran kommt ist ChatGPT. Dennoch ist das Konzept einer starken KI ein Produkt aus Science-Fiction. Die Idee ist, dass die Maschine eine Art Bewusstsein hat und ein selbstständiges Verständnis unterschiedlicher Wissensbereiche entwickelt. 
2. Schwache KI beinhaltet meist die Problemlösung einer konkreten Art. KI ist ein Konstrukt aus komplexen Algorithmen. Wenn von KI gesprochen wird, ist immer eine schwache KI gemeint.

\section{Recommender Systems}
\label{sec:recommendersystems}

\section{Warum Testen und Überwachen der KI}
\label{sec:testen}

1. KI-Systeme übernehmen bereits kritische Aufgaben. Identifizierung von Unfällen, Feuer oder Naturkatastrophen sind Aufgaben, die von einer KI schneller, besser und effizienter erledigt werden kann. Bei kritischen Prozessen ist es wichtig, dass die KI die vorgesehenden Leistungen erbringt.
\newpage







\chapter{Verwandte Arbeiten}
\label{chap:verwandtearbeiten}

\newpage

\chapter{Adesso Staffing Advisor}
\label{chap:staffingadvisor}

\section{Aufbau des Projekts}
\label{sec:ausgangssituation}

\section{Preprocessing}
\label{sec:preprocessing}

\subsection{Keyword-Extraction}

\subsection{Normalizing}

\subsection{Large Language Models}

\section{KI-Modelle}
\label{sec:similaritycalculation}

\subsection{spacy}

\subsection{sbert}

\section{Nutzung von Daten}
\label{chap:ui}

-Welche Informationen der Ergebnisse des KI-Ansatzes sind vorhanden und werden gebraucht. (preprocessing, Ergebnis, similarity-Werte)

\subsection{Welche Daten werden in die KI gegeben}

\subsection{Welche Daten werden von den KI-Ansätzen erstellt}

\subsection{Welche Daten werden von der KI zurückgegeben}

\newpage

\chapter{Ähnlichkeitsmetriken}
\label{chap:lösungsansatz}

Ähnlichkeitsmetriken:

Überprüfe die Genauigkeit der Ähnlichkeitsmetriken, die im Recommender-System verwendet werden. Dazu gehören beispielsweise Kosinus-Ähnlichkeit, Pearson-Korrelation, Jaccard-Ähnlichkeit oder andere, je nach Kontext.

Top-N-Empfehlungen:
Evaluieren Sie, wie gut das Recommender-System in der Lage ist, relevante Elemente unter den Top-N-Empfehlungen zu platzieren. Dies ist eine gängige Metrik, um die praktische Anwendbarkeit des Systems zu bewerten.

--------------------------
Repräsentation der Merkmale:
Untersuche, wie gut die Merkmale (Features) der Elemente im System repräsentiert sind. Eine gute Ähnlichkeitsberechnung hängt oft von der Qualität und Relevanz der Merkmale ab.


Diversität der Empfehlungen:

Prüfe, ob die Ähnlichkeitsbasierten Empfehlungen zu vielfältig sind. Eine zu starke Konzentration auf ähnliche Elemente könnte zu eintönigen Empfehlungen führen.
Benutzerbewertungen und Feedback:

Integriere Benutzerbewertungen und -feedback in die Evaluierung, um sicherzustellen, dass die Ähnlichkeitsberechnungen den tatsächlichen Vorlieben der Benutzer entsprechen.
Cold Start-Szenarien:

Teste das System unter Bedingungen des "Cold Start", um sicherzustellen, dass es auch effektive Empfehlungen machen kann, wenn es nur begrenzte Daten gibt.
Auswirkungen von Merkmalen:

Analysiere, wie sich das Hinzufügen oder Entfernen von Merkmalen auf die Empfehlungen auswirkt. Dies kann helfen, die Sensitivität des Systems gegenüber verschiedenen Merkmalen zu verstehen.
Nutzerinteraktion:

%Untersuche, wie gut das Recommender-System auf Veränderungen in der Benutzerinteraktion reagiert. Dies könnte Änderungen in den Präferenzen der Benutzer oder neue Interaktionen mit dem System einschließen.
Es ist wichtig, die spezifischen Anforderungen deines Recommender-Systems zu berücksichtigen und die Evaluierungsmethoden entsprechend anzupassen. Kombiniere mehrere Metriken, um ein umfassenderes Bild der Leistung des Systems zu erhalten.

%-Welche Art und Weisen des Testens und der Überwachung existieren (preprocessing/similarity visualisieren, Ergebnis und Input gegenüberstellen, etc.)

%Keywords visualisieren: word cloud, Circle packing, The horn of plenty, Treemap, donut chart, Grid of bar charts

%Daten gegenüberstellen: Alle Eingabedaten und alle Mitarbeiterinformationen nebeneinander
%Vielleicht mit einer „similarity“-Matrix.
%Vielleicht Gegenüberstellung von Ergebnissen eines supervised und unsupervised Ansatzes. (Matrix)
%Visualisierung von Daten:
%1.	Kreisdiagramm
%2.	Balkendiagramm
%3.	Säulendiagramm
%4.	Kurvendiagramm
%5.	Punktediagramm
%Vergleichstypen:
%1.	Strukturvergleich(Welcher Anteil verschiedene Komponenten macht an einer Gesamtheit (z. B. in Prozent) aus)
%2.	Rangfolgevergleich(Die verschiedenen Objekte können im Vergleich zueinander z. B. kleiner, größer, besser, schlechter oder gleich sein)
%3.	Häufigkeitsvergleich(Größenklassen bilden)
%4.	Korrelationsvergleich(Vergleich, ob zwischen zwei Variablen ein quantitativer Zusammenhang bzw. 


\section{Genauigkeit der Ähnlichkeit}

\section{Qualität und Relevanz der Merkmale}

\section{Eintönige Empfehlungen}

\section{Benutzerbewertungen und -feedback}

\section{Cold Start}

\section{Sensitivität des Systems}
\newpage

%\chapter{Umsetzung der Visualisierungsmethoden}
\label{chap:implementation}
-Wie können diese Art und Weisen mit dem „adesso Staffing Advisor“ Lab implementiert werden

\section{World Cloud}

\newpage

\chapter{Evaluierung des entwickelten Systems}
\label{chap:evaluation}

vielleicht erklären warum precision, recall, f1 score nicht gehen -\\

nicht überlegen wie evaluieren sonder was will ich evaluieren,\\
was sind die fragen die ich beantworten möchte, was sind die aussagen die ich machen will. hypothesen belegen, wiederlegen\\
was möchte ich zeigen, (den expertenprozess abbilden, expertenprozess ist ideal, mein prozess hat diese abweichung)\\

z.b. erwartungshaltung formulieren und mit cosine similarity gucken was näher dran ist,

wie machen das andere ansätze,

\section{Evaluationsmetrik}
-cosine similarity
-performance, zeit

\section{Beschreibung des verwendeten Datensatzes}
\todo{darauf eingehen welche Felder in der JSON von JIRA sind und worauf sich genau konzentriert wird.}\\
\newpage

\section{Beschreibung des verwendeten Datensatzes}

überlegung ob tfidf unterschied macht alle bedarfsmeldungen mit einer zu vergleichen und daraus wichtige wörter identifizieren oder eine für sich alleine reicht.

gucken was tokenisierung wirklich macht
\section{Präsentation und Diskussion der Ergebnisse}
\newpage
g
\newpage
g
\newpage
g
\newpage
Zeit und Leistung Übersicht
\newpage
g
\newpage

\section{Vergleich des Systems mit einem Large Language Model-Ansatz}
\newpage
g
\newpage
g
\newpage

g
\newpage

\section{Analyse von Abweichungen, Ähnlichkeiten und Verbesserungspotenzialen des Systems}
\newpage
g
\newpage

\chapter{Zusammenfassung und Ausblick}
\label{chap:ergebnisseausblick}

ergebnis der arbeit: diese modelle in der reihenfolge kommen am nähesten an die bedarfsmeldung

\subsection*{Ausblick}
-keywords präzisieren und keywordskatalog anlegen
-genauer untersuchen wie einzelne ansätze mit anderen nlp vortrainierten datensätzen abschneidet
-in bezug zu recommender systems beschreiben wie damit weiter gemacht werden könnte, die keyword extraction auch für die profile nutzen 
\newpage


%%\addcontentsline{toc}{section}{Eidesstattliche Erklärung}
\hiddenchapter{Eidesstattliche Erklärung}
Hiermit erkläre ich, dass ich die vorliegende Arbeit selbstständig
und ohne Benutzung anderer als der angegebenen Hilfsmittel angefertigt
sowie die aus fremden Quellen direkt oder indirekt übernommenen
Gedanken als solche kenntlich gemacht habe. \\ \\
Die Arbeit wurde bisher in gleicher oder ähnlicher Form keiner
anderen Prüfungsbehörde vorgelegt und auch noch nicht veröffentlicht. \\ \\
Dortmund, den \today \\ \\ \\
\begin{tabular}{@{}l@{}}\hline
	Ricardo Valente de Matos
\end{tabular}
\setcounter{page}{6}

%\bibliographystyle{IEEEtranS}
%\bibliography{Literatur}
\raggedright
\printbibliography

\end{document}
