%Dokumentklasse
\documentclass[a4paper,12pt]{scrreprt}
%\usepackage[left= 3.5cm,right = 2cm, bottom = 2 cm]{geometry}
\usepackage[left= 4.5cm,right = 1.5cm, bottom = 2.5 cm]{geometry}
\addtolength{\footskip}{-0.5cm}
\usepackage[onehalfspacing]{setspace}
% ============= Packages =============

% Dokumentinformationen
\usepackage[hyphens]{url}
\usepackage[
pdfsubject={},
pdfauthor={Ricardo Valente de Matos},
pdfkeywords={},	
%Links nicht einrahmen
hidelinks,
breaklinks=true
]{hyperref}
% Standard Packages
\usepackage[utf8]{inputenc}
\usepackage[ngerman]{babel}
\usepackage[T1]{fontenc}
\usepackage{graphicx, subfig}
\graphicspath{{img/}}
\usepackage{fancyhdr}
\usepackage{lmodern}
\usepackage{color}

\usepackage{dirtree}

\usepackage[style=ieee, sorting=nty, urldate =comp, backend=bibtex]{biblatex}
\addbibresource{Literatur.bib}

%\usepackage[numbers]{natbib}
%\bibpunct{(}{)}{;}{a}{,}{,}

%----------Checkbox
\usepackage{enumitem,amssymb}
\newlist{todolist}{itemize}{2}
\setlist[todolist]{label=$\square$}
\usepackage{pifont}
\newcommand{\cmark}{\ding{51}}%
\newcommand{\xmark}{\ding{55}}%
\newcommand{\done}{\rlap{$\square$}{\raisebox{2pt}{\large\hspace{1pt}\cmark}}%
	\hspace{-2.5pt}}
\newcommand{\wontfix}{\rlap{$\square$}{\large\hspace{1pt}\xmark}}
%--------------

\usepackage{listings}

% zusätzliche Schriftzeichen der American Mathematical Society
\usepackage{amsfonts}
\usepackage{amsmath}
\usepackage{float}

\usepackage{tabularx}
\usepackage{longtable}
\usepackage{multirow}
\usepackage{enumitem}

%\usepackage{ulem}

%nicht einrücken nach Absatz
\setlength{\parindent}{0pt}
\RedeclareSectionCommand[beforeskip=0pt]{chapter}
\usepackage{listings}
\usepackage{color}
\definecolor{lightgray}{rgb}{.9,.9,.9}
\definecolor{darkgray}{rgb}{.4,.4,.4}
\definecolor{purple}{rgb}{0.65, 0.12, 0.82}

\lstdefinelanguage{JavaScript}{
	keywords={typeof, new, true, false, catch, function, return, null, catch, switch, var, if, in, while, do, else, case, break},
	keywordstyle=\color{blue}\bfseries,
	ndkeywords={class, export, boolean, throw, implements, import, this},
	ndkeywordstyle=\color{darkgray}\bfseries,
	identifierstyle=\color{black},
	sensitive=false,
	comment=[l]{//},
	morecomment=[s]{/*}{*/},
	commentstyle=\color{purple}\ttfamily,
	stringstyle=\color{red}\ttfamily,
	morestring=[b]',
	morestring=[b]"
}

\lstset{
	language=JavaScript,
	backgroundcolor=\color{lightgray},
	extendedchars=true,
	basicstyle=\footnotesize\ttfamily,
	showstringspaces=false,
	showspaces=false,
	numbers=left,
	numberstyle=\footnotesize,
	numbersep=9pt,
	tabsize=2,
	breaklines=true,
	showtabs=false,
	captionpos=b
}


% ============= Kopf- und Fußzeile =============
\pagestyle{fancy}
%
\lhead{}
\chead{}
\rhead{\slshape \leftmark}
%%
\lfoot{}
\cfoot{\thepage}
\rfoot{}
%%
\renewcommand{\headrulewidth}{0.4pt}
\renewcommand{\footrulewidth}{0pt}

% ============= Package Einstellungen & Sonstiges ============= 
%Besondere Trennungen
\hyphenation{De-zi-mal-tren-nung}

\newcommand{\hiddenchapter}[1]{
	\chapter*{{#1}}
}

%-------------

\newcommand{\todo}[1]{\textcolor{red}{ToDo:} #1\marginpar{<--hier}}

% ============= Dokumentbeginn =============

\begin{document}
%Seiten ohne Kopf- und Fußzeile sowie Seitenzahl
\pagestyle{empty}

\begin{center}
	\begin{tabular}{p{\textwidth}}
		
		\begin{center}
			\textbf{\Large{Bachelorarbeit}}
		\end{center} \\ \\
		
		\begin{center}
			\LARGE{\textsc{
					%\textit{\emph{adesso Staffing Advisor Lab}}\\
					%Konzeption und prototypische Entwicklung der Struktur und Architektur einer Softwareplattform für Transparenz in KI-Anwendungen
					...
			}}
		\end{center}
		
		\\
		
		
		
		\begin{center}
			von
		\end{center}
		
		\begin{center}
			\large{\textbf{Ricardo Valente de Matos}}
		\end{center}
	
	\begin{center}
		\large{Matrikelnummer: 7203677} \\
		\large{im Studiengang Wirtschaftsinformatik \\
			der Fachhochschule Dortmund \\}
	\end{center}
		
		
		\\
		
		\\
		
		\begin{center}
			\begin{tabular}{lll}
				\textbf{Erstprüfer:} & & Prof. Dr.-Ing. Guy Vollmer\\
				\textbf{Zweitprüfer:} & & Stephan Schmeißer, M. Sc., Adessoplatz 1, 44269 Dortmund\\
			\end{tabular}
		\end{center}
	
	\\ \\
	
	\begin{center}
		\large{Dortmund, den \today}
	\end{center}
		
	\end{tabular}
\end{center}

%\setcounter{page}{1}
%\pagestyle{plain}

%letztes kapitel zusammenfassung und ausblick
\pagestyle{fancy}
\pagenumbering{Roman}
\chapter*{Abstract}
\todo{Abstract erstellen}
\newpage

\chapter*{Lesehinweis}
Aus Gründen der besseren Lesbarkeit werden Wörter und Wortgruppen, die hervorgehoben werden oder mehrfach auftauchen, durch \emph{kursiven} Text kenntlich gemacht. Zudem wird in dieser Ausarbeitung die Sprachform des generischen Maskulinums angewandt. Sämtliche Ausführungen sind jedoch geschlechtsunabhängig und beziehen sich damit auf alle Geschlechter.
\newpage
\tableofcontents
\listoffigures
\listoftables
\lstlistoflistings
\newpage

\setcounter{page}{1}
\pagestyle{fancy}
\pagenumbering{arabic}
\setcounter{chapter}{0}
\newpage

\chapter{Einleitung}
\label{chap:einleitung}

\newpage
\section{Problemstellung}
\label{sec:problemstellung}

\newpage
\section{Ziele und Ergebnisse der Arbeit}
\label{sec:zieleundergebnis}

\section{Aufbau der Arbeit}

\newpage

%\section{Verwandte Arbeiten}
\chapter{Literaturüberblick}
\label{sec:literaturueberblick}
Es gibt eine Reihe an verwandten Arbeiten, die sich mit unterschiedlichen Aspekten des \emph{Staffing}-Prozesses und der Nutzung von Information Retrieval und Filtering zur Informationsgewinnung beschäftigen. Dennoch beschäftigt sich keine Arbeit mit dem spezifischen Problem der Informationsgewinnung aus \emph{Bedarfsmeldungen}, in der Art, wie sie \emph{adesso} verwendet. Es wird ein Einblick in die Art und Weise gegeben wie andere Autoren Information Retrieval einsetzen und kombinieren.\\
\section{Techniken}
Nachfolgend werden Details zu verschiedenen Arbeiten absatzweise dargestellt. Die Quellen stehen jeweils am Anfang jedes Absatzes und beziehen sich aufgetroffene Aussagen innerhalb der Absätze. Diese befassen sich mit unterschiedlichen Techniken, die Lösungen für die Implementierungsebene des zu entwickelnden Systems bieten.
\subsection*{Automatisiertes Staffing}
In der Arbeit \citetitle{horesh2016information}\cite{horesh2016information} beschreiben \citeauthor{horesh2016information} einen Ansatz zur Ableitung von Unternehmensdaten und digitalen Fußabdrücken von Mitarbeitern. Mit Hilfe eines Big-Data-Workflows, der die Komponenten Information Retrieval und Suche, Datenfusion, Matrixvervollständigung und ordinale Regression nutzt, können Informationen zur Expertise automatisch zusammengeführt und für die Nutzung durch Experten aufbereitet werden. Das System soll Fähigkeiten, Talente und Fachwissen der Mitarbeiter in einem breiten Bereich wie cloud computing oder cybersecurity einschätzen. Beim Ansatz des Information Retrieval und -fusion wird eine Liste von Suchbegriffen erstellt, die sich auf das breite Fachgebiet der Mitarbeiter beziehen. Die Suche wird nach jedem dieser Abfragebegriffe durchgeführt, um Zusammenhänge zwischen Mitarbeiter und Datenquellen zu finden. Die verschiedenen Zusammenhänge werden zusammengefügt, gewichtet und nach der Abfrage sortiert. Die Mitarbeiter werden nach Daten gewichtet und bewertet, um einen einzigen Wert (sehr niedrig, niedrig, moderat, etwas, begrenzt) für ihr Fachwissen in diesem breiten Bereich zu erhalten.\\
\subsection*{Vorverarbeitung}
Die Autoren \citeauthor{jain2013data} der Arbeit \citetitle{jain2013data}\cite{jain2013data} beschreiben Data-Mining als ein interdisziplinäres Teilgebiet der Informatik, das sich mit der rechnergestützten Entdeckung von Mustern in großen Datenbeständen befasst \cite{jain2013data}. Ziel dieses fortgeschrittenen Analyseverfahrens ist es, Informationen aus einem Datensatz zu extrahieren und in eine für die weitere Verwendung verständliche Struktur umzuwandeln \cite{jain2013data}. Die verwendeten Methoden liegen an der Schnittstelle zwischen künstlicher Intelligenz, maschinellem Lernen, Statistik, Datenbanksystemen und Business Intelligence \cite{jain2013data}. Beim Data Mining geht es um die Lösung von Problemen durch die Analyse von Daten, die bereits in Datenbanken vorhanden sind \cite{jain2013data}.\\

In der Arbeit \citetitle{alasadi2017review}\cite{alasadi2017review} zeigen die Autoren \citeauthor{alasadi2017review} Wege und Schritte zur Aufbereitung von Datensätzen auf. Die Arbeit umfasst Data-Mining Vorverarbeitungsmethoden, um die Qualität der Daten zu verbessern. Diese weisen wichtige Schritte auf, um die Effizienz in der Datensammlung zu verbessern. Die Datenvorverarbeitung (preprocessing) stellt eine der essenziellen Data-Mining-Aufgaben dar, die die Vorbereitung und Umwandlung von Daten in eine geeignete Form umfasst. Die Datenvorverarbeitung zielt unter anderem darauf ab, die Datenmenge zu reduzieren und Daten zu standardisieren. Die Datenvorverarbeitung umfasst eine Reihe von Techniken, darunter Datenbereinigung, -integration, -transformation und -reduktion. \\

In der Arbeit mit dem Titel \citetitle{kroha2000preprocessing}\cite{kroha2000preprocessing} von dem Autor \citeauthor{kroha2000preprocessing} wird der Teil des Anforderungsspezifikationsprozesses diskutiert, der zwischen der textuellen Anforderungsdefinition und den dazugehörigen Diagrammen der Anforderungsspezifikation liegt. Es wird die These aufgestellt, dass die Erstellung einer textuellen Anforderungsbeschreibung, die das Verständnis des Analysten für das Problem darstellt, die Effizienz der Anforderungsvalidierung durch den Benutzer verbessert. Die vorliegende Idee ist laut dem Autor aus dem Problem entstanden, dass Software-Entwickler nicht immer über die erforderlichen Kenntnisse in den fachlichen Abläufen der Themengebiete verfügen, die für die Erstellung der Software relevant sind. Im Rahmen der Anforderungsdefinition erfolgt eine textuelle Verfeinerung, die als Anforderungsbeschreibung bezeichnet werden kann. Bei der Arbeit mit dem unterstützten Werkzeug \emph{Tessi} ist der Analytiker durch die genannten Vorgaben gezwungen, Anforderungen zu vervollständigen und zu erklären sowie die Rollen der Wörter im Text im Sinne der objektorientierten Analyse zu spezifizieren. Im Rahmen der Vorverarbeitung erfolgt eine Transformation der Requirements durch Templates.
\subsection*{Schlüsselwörter identifizieren}
In der Arbeit \citetitle{ramos2003using}\cite{ramos2003using} haben die Autoren \citeauthor{ramos2003using} die \emph{TF-IDF}-Methode (\emph{Term Frequency-Inverse Document Frequency}) zur Ermittlung der Häufigkeit von Wörtern in einem bestimmten Dokument im Vergleich zum Anteil dieses Wortes im gesamten Dokumenten ermittelt. Die Berechnung erlaubt eine Einschätzung der Relevanz eines bestimmten Wortes in einem bestimmten Dokument. Die Grundidee des Ansatzes besteht darin, dass Wörter, die in einem einzigen Dokument oder in einer kleinen Gruppe von Dokumenten häufig vorkommen, tendenziell höhere \emph{TF-IDF}-Werte aufweisen als häufig vorkommende Wörter wie Artikel und Präpositionen. \emph{TF-IDF} stellt laut den Autoren ein effizientes Verfahren zum Abgleich von Wörtern in einer Anfrage mit Dokumenten dar. Bei Eingabe einer Abfrage zu einem bestimmten Thema durch einen Benutzer kann \emph{TF-IDF} relevante Informationen zu dieser Abfrage in Dokumenten finden. Trotz der Stärken von \emph{TF-IDF}, sind auch seine Grenzen zu berücksichtigen. In Bezug auf Synonyme ist zu beachten, dass \emph{TF-IDF} nicht auf die Beziehung zwischen den Wörtern eingeht. Des Weiteren werden laut den Autoren unterschiedliche Schreibweisen von Wörtern nicht berücksichtigt, was dazu führen kann, dass Wörter fälschlicherweise als nicht so häufig auftauchend deklariert werden, obwohl sie mit leicht abgewandelter Schreibweise häufiger vorkommen.\\
Die Autoren haben die \emph{TF-IDF}-Methode anhand von 1400 Dokumenten getestet. Dazu wurden die \emph{TF-IDF}-Werte berechnet und die ersten 100 Dokumente zurückgegeben. Die zurückgegebenen Dokumente werden in absteigender Reihenfolge zurückgegeben, wobei die Dokumente mit höheren Gewichtssummen zuerst erscheinen. Um die Ergebnisse zu vergleichen, wurde zu mehreren Dokumenten die Anzahl einer bestimmten Anfrage ermittelt.\\

Die Autoren \citeauthor{bafna2016document} der Arbeit \citetitle{bafna2016document}\cite{bafna2016document} haben im Rahmen der Textanalyse die \emph{TF-IDF}-Methode angewandt, um häufige Begriffe zu eliminieren und lediglich die relevantesten Begriffe aus einem Textkorpus zu extrahieren. In der Untersuchung wird der \emph{TF-IDF}-Algorithmus zusammen mit dem \emph{Fuzzy K-means} und dem \emph{hierarchischen} Algorithmus verwendet. In einem ersten Schritt werden Experimente mit einem kleinen Datensatz durchgeführt und eine Clusteranalyse vorgenommen. Im Anschluss erfolgt die Anwendung des vorher besten ermittelten Algorithmus auf den erweiterten Datensatz. In Kombination mit den verschiedenen Clustern der verwandten Dokumente werden der resultierende \emph{Silhouettenkoeffizient}, die \emph{Entropie} sowie der \emph{F1-Score} dargestellt, um das Verhalten des Algorithmus für den Datensatz zu veranschaulichen.\\
\subsection*{Wortketten bilden}
Die Autoren \citeauthor{majumder2002n} beschreiben in der Arbeit \citetitle{majumder2002n}\cite{majumder2002n} \emph{N-Gramme} als Folgen von Zeichen oder Wörtern, die aus einem Text extrahiert werden. Diese lassen sich in zwei Kategorien unterteilen: i) zeichenbasiert und ii) wortbasiert. Ein Zeichen-\emph{N-Gramm} bezeichnet eine Folge von n aufeinanderfolgenden Zeichen, die aus einem Wort extrahiert werden. Die Hauptmotivation hinter diesem Ansatz besteht darin, dass ähnliche Wörter einen hohen Anteil an \emph{N-Grammen} gemeinsam haben werden. In der Regel umfasst ein \emph{N-Gramm} lediglich die am häufigsten auftretenden Wortpaare und verwendet einen Backoff-Mechanismus, um die Wahrscheinlichkeit zu berechnen, die bei der Suche nach dem gewünschten Wortpaar nicht erfolgreich war. Die Analyse von \emph{N-Grammen} erlaubt die Identifikation häufig vorkommender Phrasen oder Begriffe, die als potenzielle Schlüsselwörter bezeichnet werden können. In der Durchführung des Experimentes werden von den Autoren \emph{N-Gramme} verwendet, um die indische Sprache aus mehrsprachigen Dokumentensammlungen zu identifizieren. Dazu werden zunächst \emph{N-Gramm}-Profile für 10 indische Sprachen erstellt. Ein \emph{N-Gramm}-Häufigkeitsprofil wird durch Zählen aller \emph{N-Gramme} in einer Reihe von Dokumenten in einer bestimmten Sprache und deren Sortierung in absteigender Reihenfolge erstellt. Im Falle der Identifizierung einer neuen Dokumentsprache wird ein \emph{N-Gramm}-Profil des Dokuments erstellt und anschließend der Abstand zwischen dem neuen Dokumentprofil und den Sprachprofilen berechnet. Der Abstand wird mit dem \emph{out-of-place measure} zwischen den beiden Profilen berechnet. Der kürzeste Abstand wird ausgewählt und es wird vorhergesagt, ob das bestimmte Dokument zu dieser Sprache gehört. Zur Vermeidung einer Fehlklassifizierung wurde ein Schwellenwert eingeführt, bei dessen Überschreitung das System die Aussage trifft, dass die Sprache des Dokuments nicht bestimmt werden kann.
\subsection*{Wortgruppen identifizieren}
Die Autoren \citeauthor{kumawat2015pos} der Arbeit \citetitle{kumawat2015pos}\cite{kumawat2015pos} beschreiben unterschiedliche Ansätze von \emph{POS-Tagging} (\emph{Part-of-Speech-Tagging}), um einen für indische Sprache zu erstellen. Sie beschreiben, dass die Katalogisierung von Wortarten (\emph{POS}) einen Prozess bezeichnet, bei dem jedem einzelnen Wort eines Satzes ein Wortart-Tag oder ein anderes philologisches Klassenzeichen zugeordnet wird. Die Vorverarbeitungsaufgabe des \emph{Taggings} von Sprachbestandteilen stellt laut den Autoren einen essenziellen Schritt in \emph{NLP} dar.\\

\emph{NLP} (Natural Language Processing) stellt einen zentralen Aspekt im Bereich der künstlichen Intelligenz sowie der Computerwissenschaften dar \cite{kang2020natural}. Studien in diesem Bereich umfassen Theorien und Methoden, die eine Kommunikation zwischen Menschen und Computern in natürlicher Sprache ermöglichen \cite{kang2020natural}. \emph{NLP} vereint die Gebiete Informatik, Linguistik und Mathematik mit dem primären Ziel, menschliche Sprache in Befehle zu übersetzen, die von Computern ausgeführt werden können \cite{kang2020natural}.\\

Die Zuordnung von Wortarten stellt laut den Autoren \citeauthor{kumawat2015pos} der Arbeit \citetitle{kumawat2015pos}\cite{kumawat2015pos} eine grundlegende Aufgabe bei der Verarbeitung natürlicher Sprache dar. Die Erstellung erfolgt unter Zuhilfenahme linguistischer Theorien, zufälliger Muster sowie einer Kombination aus beidem. Ein \emph{POS-Tagger} ist definiert als ein Teil einer Software, der jedem Wort einer Sprache, das er liest, eine Wortart zuordnet. Die Ansätze des \emph{POS-Tagging} lassen sich in drei Kategorien unterteilen: \emph{regelbasiertes Tagging}, \emph{statistisches Tagging} und \emph{hybrides Tagging}. Im Rahmen der Zuweisung von \emph{POS-Tags} zu Wörtern im regelbasierten \emph{POS}-System erfolgt die Verwendung einer Reihe von handgeschriebenen Regeln in Kombination mit Kontextinformationen. Der Nachteil dieses Systems besteht darin, dass es nicht funktioniert, wenn der Text nicht bekannt ist. Das Problem besteht darin, dass das System nicht in der Lage ist, den passenden Text vorherzusagen. Um eine höhere Effizienz und Genauigkeit in diesem System zu erreichen, ist es daher empfehlenswert, einen umfassenden Satz von handkodierten Regeln zu verwenden. Die Häufigkeit und Wahrscheinlichkeit sind in dem statistischen Ansatz einbezogen. Der grundlegende statistische Ansatz basiert auf der am häufigsten verwendeten Markierung für ein bestimmtes Wort in den annotierten Trainingsdaten. Diese Information wird auch zur Markierung dieses Wortes im Text verwendet.\\
Das Experiment aus der Arbeit umfasst einen Datensatz aus 20000 Sätzen mit manuellen Tags in der Sprache Marathi. Damit kann das entwickelte \emph{POS-Tagging}-System getestet werden. 

%\section{Text-Ranking-Algorithmus}
%Text-Ranking-Algorithmen: Text-Ranking-Algorithmen wie TextRank oder YAKE (Yet Another Keyword Extractor) verwenden graphenbasierte Methoden, um Schlüsselwörter in einem Text zu identifizieren. Die Algorithmen bewerten die Wichtigkeit von Wörtern basierend auf ihrer Verbindung zu anderen Wörtern im Text und extrahieren Schlüsselwörter entsprechend ihrer Rangfolge.\\ \cite{mihalcea2004textrank}\cite{zhang2020empirical}\cite{pay2019ensemble}\\

%Die Autoren \citeauthor{mihalcea2004textrank} verwenden in der Arbeit \citetitle{mihalcea2004textrank}\cite{mihalcea2004textrank} die Methode \emph{TextRank}, um Schlüsselwörter in einem natürlichen Text zu ermitteln. Dabei handelt es sich um ein graphenbasierten Text-Ranking-Algorithmus, der eine Möglichkeit darstellt, die Wichtigkeit eines Knotens innerhalb eines Graphen zu bestimmen. Dabei werden globale Informationen berücksichtigt, die rekursiv aus dem gesamten Graphen berechnet werden. Im Gegensatz zu anderen Methoden, die sich lediglich auf lokale, knotenspezifische Informationen stützen, ermöglicht dies eine objektivere Bewertung. Der \emph{TextRank}-Algorithmus zur Schlüsselwortextraktion
%ist ein \emph{unsupervised}-Ansatz. Zunächst wird der Text im Vorverarbeitungsschritt \emph{tokenisiert} und mit \emph{Part-of-Speech-Tags} annotiert, der für die Anwendung syntaktischer Filter notwendig ist. Um ein Wachstum des Graphen durch Hinzufügen aller Kombinationen von Sequenzen, die aus mehr als einem \emph{N-Gramm} bestehen, zu vermeiden, werden nur einzelne Wörter zum Hinzufügung in den Graphen in Betracht gezogen. Anschließend werden alle lexikalischen Einheiten, in den Graphen eingefügt und es wird eine Kante zwischen den lexikalischen Einheiten eingefügt. Nach der Erstellung des Graphen wird jedem Knoten eine Punktzahl mit einen Anfangswert von 1 gegeben. Der Ranking-Algorithmus wird anschließend in mehreren Iterationen auf den Graphen angewendet. Im Anschluss an die Ermittlung der Punktzahlen für jeden Knoten im Graphen erfolgt eine Sortierung der Knoten in umgekehrter Reihenfolge ihrer Punktzahl. Die Knoten mit der höchsten Punktzahl werden für die Nachbearbeitung ausgewählt. Bei Nachbearbeitung werden alle vom \emph{TextRank}-Algorithmus als potenzielle Schlüsselwörter im Text markiert. Anschließend werden Sequenzen benachbarter Schlüsselwörter zusammengefasst.\\

%\cite{zhang2020empirical}\cite{pay2019ensemble}
\subsection*{Datum und Zeitangaben identifizieren}
% \cite{nadeau2007survey}\cite{partalidou2019design}\\
Die Arbeit \citetitle{mansouri2008named}\cite{mansouri2008named} der Autoren \citeauthor{mansouri2008named} vergleichen verschiedene \emph{NER}-Ansätze (Named Entity Recognition) auf die Genauigkeit, um die Stärken und Schwächen einzelner Methoden zu identifizieren. \emph{NER} stellt einen Teilbereich der Informationsextraktion dar und umfasst die Verarbeitung sowohl strukturierter als auch unstrukturierter Dokumente und die Identifizierung von Wörtern, die sich auf Personen, Orte, Organisationen und Unternehmen beziehen. \emph{NER} stellt eine grundlegende Aufgabe eines Systems im Bereich des \emph{NLP} dar. \emph{NER} umfasst laut den Autoren zwei Aufgaben. Die erste Aufgabe besteht in der Identifizierung von Eigennamen im Text. Die zweite Aufgabe ist die Klassifizierung dieser Namen in eine Reihe von vordefinierten Kategorien. Dazu zählen (i)Personennamen, (ii)Organisationen, wie Unternehmen, Regierungsorganisationen, Ausschüsse usw., (iii)Orte, wie Städte, Länder, Flüsse usw., (iv)Datumsangaben und (v)Zeitangaben.\\
%Die Arbeit vergleicht Ansätze mithilfe von den Methoden \emph{Precision}, \emph{Recall} und \emph{F1-Score}
\subsection*{Hybride Ansätze}
In der Untersuchung \citetitle{croft2000combining}\cite{croft2000combining} wird von den Autoren \citeauthor{croft2000combining} die Entwicklung von Kombinationen im Bereich des Information Retrievals analysiert. Dabei werden sowohl experimentelle Ergebnisse als auch die Retrieval-Modelle, die als formale Rahmen für die Kombination vorgeschlagen wurden, berücksichtigt. In der Untersuchung wird aufgezeigt, dass Kombinationsansätze für die Informationssuche als Kombination der Ergebnisse mehrerer Klassifikatoren auf der Grundlage einer oder mehrerer Darstellungen modelliert werden können. Zudem wird dargelegt, dass dieses einfache Modell Erklärungen für viele der experimentellen Ergebnisse liefern kann.\\

Die Arbeit \citetitle{chiny2021lstm}\cite{chiny2021lstm} von den Autoren \citeauthor{chiny2021lstm} kombiniert drei Ansätze des Information Retrievals mit dem Ziel, relevante Informationen aus Produktreviews zu extrahieren. Der Ansatz \emph{TF-IDF} wird mit einem sogenannten \emph{CLASSIFIER} Model kombiniert. Das Klassifikationsmodell verarbeitet drei Eingaben der Modelle \emph{LSTM}, \emph{VADER} und \emph{TF-IDF}. Die Werte dieser Eingaben liegen im Bereich von [0,1]. Die Ausgabe des Klassifikationsmodells ist binär und gibt eine Vorhersage des vollständigen Textes der Modelleingabe aus (positiv oder negativ). Aus einem Datensatz wurden 5000 zufällige Bewertungen ausgewählt, die sich von den für die LSTM- und \emph{TF-IDF}-Modelle verwendeten Trainings- und Testdatensätzen unterscheiden. Die Autoren \citeauthor{chiny2021lstm} haben das System durch die Eingabe des globalen Modells laufen lassen, um die Vorhersagen zu erhalten, die von den \emph{LSTM}-, \emph{VADER}- und \emph{TF-IDF}-Modellen zurückgegeben werden. Anschließend teilten sie diese Ergebnisse in zwei Stapel auf (75 \% für die Trainingsmenge und 25 \% für die Testmenge), um das binäre Klassifizierungsmodell zu trainieren und zu bewerten. Die Evaluation erfolgt durch den Einsatz der Methoden \emph{Precision}, \emph{Recall} und \emph{F1-Score}.\\

Die Arbeit \citetitle{suhasini2021hybrid}\cite{suhasini2021hybrid} von den Autoren \citeauthor{suhasini2021hybrid} befasst sich mit der Filterung von Falschmeldungen. In diesem Beitrag werden hybride Verfahren zur Gewinnung von Merkmalen untersucht, die in dem Gebiet noch nicht gründlich erforscht wurden. Die Anwendung von Hybridsystemen hat sich in einer Vielzahl von Anwendungsbereichen als nützlich erwiesen und zeigen eine Tendenz, die Fehlerquote zu reduzieren, indem sie Techniken wie \emph{TF-IDF} und \emph{N-Grams} verwenden. Es wurden Experimente unter Verwendung von Echtzeit-Twitterdaten durchgeführt. Der Datensatz umfasste ca. 5.800 Tweets, die sich auf Donald-Drummond-Geschichten bezogen. Die Sammlung und Verarbeitung der Tweets erfolgt mit Python. Der Datensatz umfasste Original-Tweets, die als gefälscht und echt gekennzeichnet wurden. Die Genauigkeit der Prognose wurde anhand der verschiedenen Nachrichten evaluiert, die für das Training verwendet wurden und am Ende mit \emph{Precision}, \emph{Recall} und \emph{F1-Score} evaluiert.\\

Im Rahmen der Studie mit dem Titel \citetitle{darmawan2015hybrid}\cite{darmawan2015hybrid} wurde von den Autoren \citeauthor{darmawan2015hybrid} ein hybrider Algorithmus zur Extraktion von Schlüsselwörtern und Kosinusähnlichkeit zur Verbesserung der Satzkohäsion bei der Textzusammenfassung vorgeschlagen. Die vorgeschlagene Methode basiert auf einer Komprimierung von 50 \%, 30 \% und 20 \%, um Kandidaten für die Zusammenfassung zu erstellen. Die Auswertung des Ergebnisses mittels \emph{t-Test} zeigt, dass die vorgeschlagene Methode den Kohäsionsgrad signifikant erhöht.\\
Der Ablauf umfasst die Analyse eines Dokuments mithilfe eines Extraktionsalgorithmus sowie die Berechnung der \emph{TF-IDF}-Werte für jeden Begriff. Anschließend werden alle \emph{TF-IDF}-Werte für jeden Satz summiert. Im nächsten Schritt werden alle Sätze anhand der Summe von \emph{TF-IDF} eingestuft. Das Kompressionsverhältnis bestimmt die Position des Satzrangs. In dieser Studie wird eine Kompression von 50 \% verwendet, was bedeutet, dass die Satzzusammenfassung um 50 \% des Originaltextes gekürzt wird. Nach der Auswahl des Satzes wird dessen Berechnung durchgeführt. Die Ähnlichkeit wird mit der \emph{Cosinus-Ähnlichkeitsmethode} berechnet. Anschließend werden alle Sätze anhand ihrer \emph{Cosinus-Ähnlichkeit} von der höchsten zur niedrigsten sortiert. Der resultierende Text mit neuer Satzanordnung stellt die finale Zusammenfassung dar.\\

\subsection*{Pipeline}
In der Arbeit \citetitle{pirk2019implementierung} \cite{pirk2019implementierung} wird von dem Autor \citeauthor{pirk2019implementierung} eine Pipeline entwickelt, die die \emph{N-Gramm}-Analyse verwendet, um Schlagwörter aus einem Text zu extrahieren und mit verschiedenen Ansätzen von Word-Clouds zu visualisieren. Der Fokus dieser Studie liegt dennoch eher auf der Visualisierung als auf der Informationsgewinnung eines Textes.\\

Die Arbeit \citetitle{lavin2019analyzing}\cite{lavin2019analyzing} von dem Autor \citeauthor{lavin2019analyzing} präsentiert eine Anleitung zur Erstellung einer Pipeline mit Python und \emph{TF-IDF}. Darüber hinaus wird die Relevanz von \emph{TF-IDF} als Vorverarbeitung beim maschinellen Lernen erörtert. Im Vergleich zur rohen Termhäufigkeit weist \emph{TF-IDF} in der Regel einen höheren Vorhersagewert auf. Die Gewichtung von Themenwörtern wird erhöht, um die Bedeutung von Wörtern zu erhöhen, während die Gewichtung von hochfrequenten Funktionswörtern verringert wird. Es werden Verfahren zur Vorverarbeitung von Texten vorgestellt, die eine Umformung in die gewünschte Darstellungsform ermöglichen. Zudem werden Methoden zur Interpretation der Ergebnisse des \emph{TF-IDF}-Verfahrens erörtert. Die vorliegende Arbeit widmet sich zunächst einer detaillierten Betrachtung der zugrundeliegenden Algorithmen und ihrer Funktionsweise. Im Anschluss erfolgt die Implementierung in Python. Die Verwendung der Bibliothek \emph{sklearn} ist dabei von zentraler Bedeutung.\\

In dem Beitrag mit dem Titel \citetitle{partalidou2019design}\cite{partalidou2019design} wird von den Autoren \citeauthor{partalidou2019design} ein maschineller Lernansatz für die Bereiche \emph{POS-Tagging} und \emph{NER} für die griechische Sprache unter Verwendung von \emph{spaCy} erarbeitet und evaluiert. Die Verarbeitung natürlicher Sprache wirft insbesondere bei der Analyse unüblicher Sprachen wie Griechisch Schwierigkeiten auf.\\ Der Datensatz wurde aus Texten einer griechischen Zeitung extrahiert. Die Artikel der Zeitung wurden in verschiedene Kategorien wie beispielsweise Sport, Gesundheit, Wirtschaft und politische Nachrichten eingeteilt. Die Daten bestehen aus einer Reihe von XML-Dateien, die Informationen auf der Ebene von Absätzen, Sätzen und Wörtern enthalten. Im Rahmen der Evaluation wurden verschiedene Parameter getestet, um das optimale Ergebnis zu erzielen. Der Datensatz wurde gemischt und in einen Trainingssatz, einen Testsatz und einen Validierungssatz aufgeteilt. Zur Evaluierung wurden die Methoden des \emph{Precision}, \emph{Recalls} und \emph{F1-Scores} angewendet.

\section{Ergebnis}
In der Literatur finden sich verschiedene Ansätze zur Extraktion relevanter Stichpunkte aus einem Volltext. Eine Methode zur Ermittlung wichtiger Schlüsselwörter in Texten stellt die \emph{TF-IDF}-Methode dar. Innerhalb einer oder mehrerer \emph{Bedarfsmeldungen} lassen sich mit dieser Methode häufig auftauchende Wörter ermitteln. Es besteht somit die Möglichkeit, Wörter aus einer \emph{Bedarfsmeldung} mit anderen \emph{Bedarfsmeldungen} zu vergleichen und die Häufigkeit der Wörter zu berechnen, um somit potenzielle Schlüsselwörter zu ermitteln. %Des Weiteren können graphenbasierte Methoden wie \emph{TextRank} zur Identifizierung von Schlüsselwörtern in Texten herangezogen werden. 
Ein Ansatz zur Bildung von Wortketten ist die Nutzung von \emph{N-Grammen}. Dabei lassen sich Sätze und Absätze zur weiteren Verarbeitung separieren, wodurch Wörter und ihre Zusammenhänge identifizieren lassen. Auch grammatische Kategorien von Wörtern können Rückschlüsse auf potenzielle Schlüsselwörter zulassen. Die \emph{POS-Tagging}-Methode stellt eine Möglichkeit dar, um dieses Ziel zu erreichen. Des Weiteren kann \emph{NER} dazu beitragen Zeitangaben zu identifizieren. Hybride, die sich durch besondere Zusammensetzungen auszeichnen, zeigen Überlegenheiten gegenüber den einzelnen Ansätzen, die für sich genommen jeweils nur eine Teilkompetenz abdecken.

%spam-filter (Empfinde das Thema eventuell als zu unpassend)\\
%-Überblick über verfügbare Methoden, Herausforderungen und zukünftige Forschungsrichtungen im Bereich der Spam-Erkennung, Filterung und Eindämmung von SMS-Spam. Dabei werden auch Methodiken der keyword frequency ratio und Herunterbrechung auf keyword components behandelt \cite{shafi2017review}\\

%In diesem Beitrag werden Studien zu Technologien vorgestellt, die für die Suche und das Abrufen von Informationen im Web nützlich sind. Es wird aufgezeigt, dass Information Retrieval und Ranking im Web-Kontext anders Funktioniert als in einer statischen Datenbank. \cite{kobayashi2000information}\\

%konzeptbasiertes recruiting --> neuer Ansatz

%\section{Relevante Methoden und Techniken im Bereich Information Retrieval und Data-Mining}
%\label{sec:relevante-methoden}
\newpage







\chapter{Entwicklung einer klaren Erwartungshaltung}
\label{chap:erwartungshaltung}
Dieses Kapitel befasst sich mit der Methodologie und Durchführung von Experteninterviews mit dem Ziel 

\section{Beschreibung der Interviews mit Führungskräften zur Identifizierung von Stakeholder-Erwartungen}
\label{sec:beschreibung-der-interviews}

Im Rahmen der vorliegenden Ausarbeitung werden halbstrukturierte Interviews mit Experten aus dem Bereich <> durchgeführt. Sinn und Zweck von Experteninterviews ist die Rekonstruktion spezifischer Wissensbestände oder besonders exklusiver, detaillierter oder umfassender Kenntnisse über bestimmte Wissensbestände und Praktiken.\\

Der Begriff \emph{Experte} bezeichnet eine Person, die über einen privilegierten Zugang zu Informationen verfügt. Die Expertise eines Experten ist jedoch nicht allein durch die Informationen definiert, über die er exklusiv verfügt. Auch die Verantwortung für Problemlösungsentscheidungen ist ein entscheidender Faktor. Diesbezüglich ist Kompetenz erforderlich, die mit Verantwortung und mit Fähigkeiten sowie mit der Bereitschaft, Verantwortung zu übernehmen, verbunden ist. Dabei ist zu beachten, dass Verantwortung, Fähigkeiten und Bereitschaft in der Regel zusammenfallen.\\

Die Experteninterviews zielen darauf ab, qualitative Daten zu erheben, was durch die Transkription der Interviews in Text gelingt. Die Interviews werden als Einzelinterviews durchgeführt, wodurch der Fokus besonders gut auf das spezifische Wissen jedes Befragten gerichtet werden kann. Jeder Interviewpartner reagiert individuell aufgrund seines eigenen Vorwissens auf die Interviewfragen und beeinflusst daher nicht die Aussagen anderer Interviewteilnehmer. Die Ergebnisse der Interviews bilden die Grundlage für die Formulierung der Anforderungen einer optimalen Bedarfsmeldung, welche als Basis für das zu entwickelnde System eingesetzt wird. Jedes Interview wird zu Dokumentationszwecken aufgezeichnet. Um die Interviews strukturiert für die qualitative Inhaltsanalyse vorzubereiten, ist es erforderlich, sie vorher in schriftliche Transkripte umzuwandeln. Im Rahmen dieses Schritts werden der Interviewer mit einem „I“ und die jeweils befragte Person mit „B“ gekennzeichnet. Im Rahmen des Transkriptionsprozesses werden die Aussagen der Interviews anonymisiert. Dies bedeutet, dass sämtliche personenbezogenen Informationen, wie Vor- und Nachname, durch neutrale Bezeichnungen ersetzt werden.\\

Im Vorfeld der Durchführung der Interviews wurde eine Überprüfung der inhaltlichen Verständlichkeit der Fragen sowie ihrer Beantwortbarkeit vorgenommen. Zudem wurde Feedback zur Reihenfolge eingeholt. Zu diesem Zweck wurden die Fragen vorab an eine Führungsperson geschickt und schriftlich beantwortet. Die Fragen wurden den Experten vorab inklusive Kontext des Interviews geschickt, damit diese sich bei Bedarf Gedanken vorab machen können. Zur zeitlichen Begrenzung wird das Interview auf zehn Fragen reduziert. Trotz der vorgegebenen Strukturierung des Interviews wird Raum für spontane Fragen gelassen, um eine natürliche Gesprächsführung zu ermöglichen. Die Fragen dienen als Orientierungshilfe und Leitfaden durch das Interview. Der Leitfaden ist dabei lediglich als inhaltliche Richtlinie zu verstehen, von der situativ abgewichen werden kann.\\

Ablauf der Interviews...

\section{Übersicht der Experten}

\begin{tabularx}{1\textwidth} { 
		| >{\raggedright\arraybackslash}X 
		| >{\raggedright\arraybackslash}X
		| >{\raggedright\arraybackslash}X
		| >{\raggedright\arraybackslash}X | }
	\hline
	Nr. und Datum
	& Profil der Befragten & Durchführungsart & Dauer\\
	\hline
	1. ... & CC-Leiter & schriftlicher Vorabtest & -\\
	\hline
	2. 29.04.2024 & CC-Leiter & Video-Interview & ...min\\
	\hline
	3. 29.04.2024 & CC-Leiter & Video-Interview & ...min\\
	\hline
	4. 30.04.2024 & CC-Leiter & Video-Interview & ...min\\
	\hline
	5. 6.05.2024 & CC-Leiter & Video-Interview & ...min\\
	\hline
\end{tabularx}\\

Die erste Spalte der Tabelle zeigt die vergebene Nummerierung sowie das Datum der Durchführung. Dies dient der vereinfachten Referenzierung innerhalb der nachfolgenden Analyse.
Die zweite Spalte der Tabelle enthält die Rolle bzw. Tätigkeit der Befragten. Aus den Angaben wird ersichtlich, dass zwei von fünf durchgeführten Interviews mit ...




%Zur Beantwortung der Forschungsfragen 2 und 3 dieser Arbeit werden
%Experteninterviews mit E-Learning Experten durchgeführt. Nachdem für die
%Beantwortung der Forschungsfrage eins bereits auf die Theorie eingegangen wurde, soll
%nun diese durch praktische Erfahrungen ergänzt werden. In Forschungsfrage 2 werden
%die Anforderungen der Praktiker an einen kultursensitiven Leitfaden herausgearbeitet.
%Um diese Anforderungen angemessen herauszuarbeiten ist eine ausführliche
%Vorbereitung notwendig. Diese Vorbereitung wird in diesem Kapitel erläutert. Zu
%Beginn dieses Kapitels wird allgemein auf die qualitative Forschung eingegangen und
%im Anschluss daran auf die Vorbereitung der Interviews sowie die Vorgehensweise bei
%der Durchführung der Interviews. Das Kapitel schließt mit einer kurzen Einordnung der
%Interviewpartner, bezüglich Haupttätigkeitsfeld im E-Learning und der
%Mitarbeiteranzahl des Unternehmens, ab. 


%Methodik erklären, siehe Wirtschaftsinformatik Bachelorarbeit (S.34)
%genau die Schritte der Fragen erklären. Warum diese Reihenfolge

%nochmal eine bedarfsmeldung genau erklären und die schritte wie bedarfsmeldung erhalten, gepflegt etc wird erklären

%\begin{enumerate}
%	\item Welche Art von Projekten sind typischerweise in Ihrem Unternehmen an der Tagesordnung? Können Sie uns Beispiele für verschiedene Arten von Projekten geben, die \emph{adesso} durchführt?
%	\item Wie werden Projektbedarfe und -anforderungen innerhalb von \emph{adesso} typischerweise kommuniziert und dokumentiert?
%	\item Welche Informationen halten Sie in einer Bedarfsmeldung für besonders wichtig oder unverzichtbar?
%	\item Wie detailliert sollten Projektbeschreibungen Ihrer Meinung nach sein? Sind bestimmte Schlüsselaspekte oder -informationen in jeder Bedarfsmeldung enthalten?
%	\item Welche Herausforderungen oder Schwierigkeiten sind bei unklaren oder unvollständigen Bedarfsmeldungen aufgetreten?
%	\item Wer sind die typischen Stakeholder bei der Erstellung von Bedarfsmeldungen und welche Rolle spielen sie?
%	\item Wie wird die Qualität von Bedarfsmeldungen bei \emph{adesso} bewertet? Gibt es bestimmte Kriterien oder Standards, anhand derer Bedarfsmeldungen beurteilt werden?
%	\item Wie können Sie die Qualität und Klarheit von Bedarfsmeldungen verbessern?
%	\item Welche Auswirkungen haben unklare oder fehlende Informationen in Bedarfsmeldungen auf die Effizienz und den Erfolg von Projekten?
%	\item Wie können Sie sicherstellen, dass die Bedürfnisse und Anforderungen aller relevanten Stakeholder in einer Bedarfsmeldung angemessen berücksichtigt werden?
%\end{enumerate}
\newpage
g
\newpage
g
\newpage
g
\newpage

\section{Analyse der Ergebnisse und Entwicklung einer klaren Erwartungshaltung für die Bedarfsmeldungen}

1. Transkription der Interviews:\\
Falls du die Interviews aufgezeichnet hast, transkribiere sie vollständig und genau. Dadurch hast du eine schriftliche Version der Aussagen der Experten, die du leichter analysieren kannst.\\

2. Codierung der Daten:\\
Gehe durch die transkribierten Interviews und markiere oder kodiere relevante Themen, Aussagen oder Muster. Verwende dabei Codes oder Kategorien, die sich auf deine Forschungsfragen beziehen.\\

3. Thematische Analyse:\\
Führe eine thematische Analyse durch, indem du die kodierten Daten systematisch durchgehst und nach wiederkehrenden Themen oder Mustern suchst. Identifiziere Gemeinsamkeiten, Unterschiede oder interessante Einsichten, die sich aus den Aussagen der Experten ergeben.\\

4. Triangulation:\\
Vergleiche die Ergebnisse der Experteninterviews mit anderen Quellen, wie beispielsweise der Literatur, Fallstudien oder empirischen Daten. Durch die Triangulation kannst du die Glaubwürdigkeit und Validität deiner Ergebnisse erhöhen.\\

5. Interpretation der Ergebnisse:\\
Interpretiere die identifizierten Themen oder Muster im Kontext deiner Forschungsfragen und -ziele. Versuche zu verstehen, welche Bedeutung oder Implikationen die Aussagen der Experten für deine Forschung haben könnten.\\

6. Reflexion und Kritik:\\
Reflektiere kritisch über die Aussagen der Experten und die gewonnenen Erkenntnisse. Berücksichtige mögliche Einschränkungen oder Bias in den Interviews und betrachte die Ergebnisse aus verschiedenen Perspektiven.\\

7. Integration in die Gesamtanalyse:\\
Integriere die Ergebnisse der Experteninterviews in deine Gesamtanalyse deiner Bachelorarbeit. Verknüpfe sie mit anderen Forschungsergebnissen, theoretischen Konzepten oder empirischen Daten, um ein umfassendes Verständnis deines Forschungsthemas zu entwickeln.\\

8. Darstellung der Ergebnisse:\\
Präsentiere die wichtigsten Ergebnisse und Erkenntnisse aus den Experteninterviews in deiner Bachelorarbeit. Verwende geeignete Zitate oder Beispiele, um die Aussagen der Experten zu veranschaulichen und deine Argumentation zu unterstützen.
\cite{maguire2002user}

im anhang sind die transskripte
wenn man nicht ne größere anzahl an infos hat gucken ob man das halb automatisch evaluieren. Vielleicht kategorisieren. Infos die wichtig sind gucken ob die dann auch nach dem preprocessing drin sind. Regressive tests schreiben.

transformation von bedarfsmeldung zu guter bedarfsmeldung, was ist der fokus von der bedarfsmeldung, wie gut machen die ansätze das, und muss man das dann noch weiter verarbeiten, haben wir alles was wir brauchen mit nur einem algorithmus, inferenz falls parameter fehlt, gibt es einen der alles löst

Fragen in das proposal aufnehmen, führungskraft vorher fragen ob die fragen nice sind.
\newpage
g
\begin{enumerate}
	\item Wer sind die typischen Stakeholder bei der Erstellung von Bedarfsmeldungen und welche
	Rolle spielen sie?
	\item Welche Art von Projekten sind typischerweise in Ihrem Unternehmen an der Tagesordnung?
	Können Sie uns Beispiele für verschiedene Arten von Projekten geben, die adesso
	durchführt?
	\item Wie werden Projektbedarfe und -anforderungen innerhalb von adesso typischerweise
	kommuniziert und dokumentiert?
	\item Welche Informationen halten Sie in einer Bedarfsmeldung für besonders wichtig oder
	unverzichtbar?
	\item Wie detailliert sollten Projektbeschreibungen Ihrer Meinung nach sein? Sind bestimmte
	Schlüsselaspekte oder -informationen in jeder Bedarfsmeldung enthalten?
	\item Wie wird die Qualität von Bedarfsmeldungen bei adesso bewertet? Gibt es bestimmte
	Kriterien oder Standards, anhand derer Bedarfsmeldungen beurteilt werden?
	\item Wie können Sie die Qualität und Klarheit von Bedarfsmeldungen verbessern?
	\item Welche Herausforderungen oder Schwierigkeiten sind bei unklaren oder unvollständigen
	Bedarfsmeldungen aufgetreten?
	\item Welche Auswirkungen haben unklare oder fehlende Informationen in Bedarfsmeldungen
	auf die Effizienz und den Erfolg von Projekten?
	\item Wie können Sie sicherstellen, dass die Bedürfnisse und Anforderungen aller relevanten
	Stakeholder in einer Bedarfsmeldung angemessen berücksichtigt werden?
\end{enumerate}
\newpage
g
\newpage

\chapter{Analyse der Techniken des Information Retrieval und Data-Mining}
\label{chap:staffingadvisor}

\section{Beschreibung der untersuchten Techniken und Ansätze}

\subsection{TF-IDF}
Im Rahmen der Textanalyse wird die TF-IDF-Technik angewendet, welche die häufigsten Begriffe eliminiert und lediglich die relevantesten Begriffe aus einem Textkorpus extrahiert \cite{bafna2016document}. Die TF-IDF-Methode dient der Ermittlung der Häufigkeit von Wörtern in einem bestimmten Dokument im Vergleich zum Anteil dieses Wortes im gesamten Dokumenten \cite{ramos2003using}. Die Berechnung erlaubt eine Einschätzung der Relevanz eines bestimmten Wortes in einem bestimmten Dokument \cite{ramos2003using}. Die Grundidee des Ansatzes besteht darin, dass Wörter, die in einem einzigen Dokument oder in einer kleinen Gruppe von Dokumenten häufig vorkommen, tendenziell höhere TF-IDF-Werte aufweisen als häufig vorkommende Wörter wie Artikel und Präpositionen \cite{ramos2003using}. TF-IDF stellt ein effizientes Verfahren zum Abgleich von Wörtern in einer Anfrage mit Dokumenten dar \cite{ramos2003using}. Bei Eingabe einer Abfrage zu einem bestimmten Thema durch einen Benutzer kann TF-IDF relevante Informationen zu dieser Abfrage in Dokumenten finden \cite{ramos2003using}. In Bezug auf die Bedarfsmeldungen besteht somit die Möglichkeit, Wörter aus einer Bedarfsmeldung mit anderen Bedarfsmeldungen zu vergleichen und die Häufigkeit ihrer Verwendung zu ermitteln, um somit potenzielle Schlüsselwörter zu ermitteln.\\

Trotz der Stärken von TF-IDF, sind auch seine Grenzen zu berücksichtigen. In Bezug auf Synonyme ist zu beachten, dass TF-IDF nicht auf die Beziehung zwischen den Wörtern eingeht. Des Weiteren werden unterschiedliche Schreibweisen von Wörtern nicht berücksichtigt, was dazu führen kann, dass Wörter fälschlicherweise als nicht so häufig auftauchend deklariert werden, obwohl sie mit leicht abgewandelter Schreibweise häufiger vorkommen.
\subsection{Text-Ranking-Algorithmen}
Text-Ranking-Algorithmen: Text-Ranking-Algorithmen wie TextRank oder YAKE (Yet Another Keyword Extractor) verwenden graphenbasierte Methoden, um Schlüsselwörter in einem Text zu identifizieren. Die Algorithmen bewerten die Wichtigkeit von Wörtern basierend auf ihrer Verbindung zu anderen Wörtern im Text und extrahieren Schlüsselwörter entsprechend ihrer Rangfolge.\\ \cite{mihalcea2004textrank}\cite{zhang2020empirical}\cite{pay2019ensemble}\\

Ein graphenbasierter Rangordnungsalgorithmus stellt eine Möglichkeit dar, die Wichtigkeit eines Knotens innerhalb eines Graphen zu bestimmen. Dabei werden globale Informationen berücksichtigt, die rekursiv aus dem gesamten Graphen berechnet werden. Im Gegensatz zu anderen Methoden, die sich lediglich auf lokale, knotenspezifische Informationen stützen, ermöglicht dies eine objektivere Bewertung.\cite{mihalcea2004textrank}

\subsection{N-Gramm}
N-Gramme sind Folgen von Zeichen oder Wörtern, die aus einem Text extrahiert werden. N-Gramme lassen sich in zwei Kategorien unterteilen: i) zeichenbasiert und ii) wortbasiert. Ein Zeichen-N-Gramm bezeichnet eine Folge von n aufeinanderfolgenden Zeichen, die aus einem Wort extrahiert werden. Die Hauptmotivation hinter diesem Ansatz besteht darin, dass ähnliche Wörter einen hohen Anteil an N-Grammen gemeinsam haben werden. In der Regel umfasst ein N-Gramm lediglich die am häufigsten auftretenden Wortpaare und verwendet einen Backoff-Mechanismus, um die Wahrscheinlichkeit zu berechnen, die bei der Suche nach dem gewünschten Wortpaar nicht erfolgreich war. \cite{majumder2002n} Die Analyse von N-Grammen erlaubt die Identifikation häufig vorkommender Phrasen oder Begriffe, die als potenzielle Schlüsselwörter bezeichnet werden können.

\subsection{POS-Tagging}
%Part-of-Speech (POS) Tagging: POS-Tagging wird genutzt, um die grammatischen Kategorien von Wörtern in einem Text zu bestimmen. Durch die Berücksichtigung von Wörtern mit bestimmten POS-Tags wie Substantiven oder Adjektiven können relevante Schlüsselwörter extrahiert werden.\\ \cite{kumawat2015pos}\cite{nakagawa2007hybrid}\\

Die Katalogisierung von Wortarten (POS) bezeichnet einen Prozess, bei dem jedem einzelnen Wort eines Satzes ein Wortart-Tag oder ein anderes philologisches Klassenzeichen zugeordnet wird. Die Vorverarbeitungsaufgabe des Taggings von Sprachbestandteilen stellt einen essenziellen Schritt in der Verarbeitung natürlicher Sprache dar. Die Zuordnung von Wortarten stellt eine grundlegende Aufgabe bei der Verarbeitung natürlicher Sprache dar. Die Erstellung erfolgt unter Zuhilfenahme linguistischer Theorien, zufälliger Muster sowie einer Kombination aus beidem. Ein Part-of-Speech-Tagger (POS-Tagger) ist definiert als ein Teil einer Software, der jedem Wort einer Sprache, das er liest, eine Wortart zuordnet. Die Ansätze des POS-Tagging lassen sich in drei Kategorien unterteilen: regelbasiertes Tagging, statistisches Tagging und hybrides Tagging. Im Rahmen der Zuweisung von POS-Tags zu Wörtern im regelbasierten POS-System erfolgt die Verwendung einer Reihe von handgeschriebenen Regeln in Kombination mit Kontextinformationen. Der Nachteil dieses Systems besteht darin, dass es nicht funktioniert, wenn der Text nicht bekannt ist. Das Problem besteht darin, dass das System nicht in der Lage ist, den passenden Text vorherzusagen. Um eine höhere Effizienz und Genauigkeit in diesem System zu erreichen, ist es daher empfehlenswert, einen umfassenden Satz von handkodierten Regeln zu verwenden. Die Häufigkeit und Wahrscheinlichkeit sind in den statistischen Ansatz einbezogen. Der grundlegende statistische Ansatz basiert auf der am häufigsten verwendeten Markierung für ein bestimmtes Wort in den annotierten Trainingsdaten. Diese Information wird auch zur Markierung dieses Wortes im unannotierten Text verwendet.\cite{kumawat2015pos}

\subsection{Named Entity Recognition}
Named Entity Recognition (NER) \cite{mansouri2008named} \cite{nadeau2007survey}\cite{partalidou2019design}\\

\subsection{Regelbasierte Ansätze}
Regelbasierte Ansätze: Regelbasierte Ansätze verwenden vordefinierte Regeln oder Muster, um Schlüsselwörter zu identifizieren. Dies kann beispielsweise das Extrahieren von Wörtern sein, die häufig im Text vorkommen oder bestimmten Mustern entsprechen.\\

\subsection{Hybride Ansätze}
Hybride Ansätze: Hybride Ansätze kombinieren verschiedene Methoden und Techniken, um eine genauere Extraktion von Schlüsselwörtern zu ermöglichen. (Z.B. Kombination aus TF-IDF-Gewichtung und Text-Ranking-Algorithmen verwendet).\cite{darmawan2015hybrid} \\

\subsection{Data-Mining}
Data-Mining: \cite{jun2001review}\cite{jain2013data}

\subsection{preprocessing}
preprocessing: \cite{garcia2016big}

\subsection{data-fusion}
data-fusion: \cite{famili1997data} \cite{frank2005comparing} \cite{bohne2013data}
\newpage
g
\newpage
g
\newpage
g
\newpage

\section{Bewertung und Auswahl der besten Ansätze für die Extraktion relevanter Inhalte aus Bedarfsmeldungen}

\newpage
g
\newpage

\chapter{Umsetzung}
\label{chap:implementierung}
Auf Basis welcher Methodiken und Ansätze die relevanten Informationen extrahiert werden können, wurde in Kapitel \ref{sec:literaturueberblick} dargestellt. Nun gilt es diese Ansätze in einem System zu implementieren. In diesem Kapitel werden die Funktionalitäten des Systems zusammengetragen. Dabei werden verwendete Technologien und Implementationsaspekte genauer beschrieben.
\section{Beschreibung}
Das System durchläuft verschiedene Schritte, weswegen es als Pipeline implementiert wird. Damit alle Ansätze und Methoden zur Extraktion von Informationen gut funktionieren und vergleichbar bleiben, wird eine Übersetzungsfunktion der \emph{Bedarfsmeldungen} hinzugefügt. Auch wenn die \emph{Bedarfsmeldungen} in den meisten Fällen auf Deutsch sind, hilf es diese zu übersetzen, damit keine Unterschiede in der Ergebnisqualität resultiert, da einige Methoden und Ansätze auf Basis von englischen Trainingssätzen trainiert wurden. Schließlich müssen alle aus Kapitel \ref{sec:literaturueberblick} untersuchten Ansätze implementiert und nutzbar sein. Sie sollen die Möglichkeit haben \emph{Bedarfsmeldungen} als Input zu erhalten und eine strukturierte \emph{Bedarfsmeldung} als Ausgabe zurückzugeben. Zur vereinfachten Entwicklung soll das System modular sein, damit Methoden und Ansätze nach belieben durchgetauscht und verwendet werden können.
\section{Konkreter Ablauf der Pipeline}
Dieses Kapitel beschreibt den Ablauf der Pipeline. Dazu wird das Aktivitätsdiagramm aus der Abbildung \ref{fig:ablaufsystemabstrakt} verfeinert und mit allen analysierten Komponenten aus Kapitel \ref{sec:literaturueberblick} genauer beschrieben. Die Abbildung \ref{fig:flowchart} zeigt das UML-Aktivitätsdiagramm der Python Pipeline zur Strukturierung von \emph{Bedarfsmeldungen}. Im ersten Schritt \emph{unstrukturierte Bedarfsmeldung strukturieren} im Ablauf der Pipeline wird eine \emph{Bedarfsmeldung} ausgewählt und in die festgelegte \emph{Bedarfsmeldungsstruktur} aus Kapitel \ref{sec:strukturierungbedarfsmeldung} umgebaut. Die Felder \emph{Einsatzbeginn} und \emph{Einsatzende} werden zu einem Feld \emph{Einsatz} zusammengefügt. Neben den Feldern \emph{Aufgaben} und \emph{Skills} bleiben alle weiteren Felder bis zum letzten Punkt Ausgabe unverändert. Die Felder \emph{Aufgaben} und \emph{Skills} werden jeweils in die Schleife weitergeleitet und auf Stichpunkte reduziert.
\begin{figure}[H]
	\centering  
	\includegraphics[width=\linewidth]{Abbildungen/flowchart.png}
	\caption{UML-Aktivitätsdiagramm der Pipeline.}
	\label{fig:flowchart}
\end{figure}\mbox{} \\
Der Prozess zur Reduktion durchläuft mehrere Schritte. Zu Beginn werden die Volltexte aus den Feldern im Punkt \emph{ins Englische übersetzen} übersetzt. Anschließend erfolgt im Schritt \emph{Datums-Daten und Zeiten mit NER extrahieren} zum einen die Extrahierung der zeitbezogenen Daten. Darauffolgend wird der Volltext im Punkt \emph{Preprocessing} für die weitere Nutzung vorverarbeitet. Der Grund warum die Extraktion mit \emph{NER} nicht nach der Vorverarbeitung durchgeführt werden kann ist, da im Vorverarbeitungsschritt alle Nummern und somit auch alle zeitbezogenen Daten entfernt werden. Die Resultate des \emph{NER} werden zum Ende hin zurück ins deutsche übersetzt und der fertigen Stichpunktliste beigefügt. Zur Erstellung der Stichpunktliste werden relevante Schlüsselwörter durch die \emph{TF-IDF}-Methode ermittelt. Anschließend wird der vorverarbeitete Text zu \emph{Bi-Grammen} umgeformt. Weshalb für die \emph{n-Gramme} \emph{Bi-Gramme} verwendet werden, wird in Kapitel \ref{sec:ngram} näher erläutert. Beim Punkt \emph{Bi-Gramme auf Schlüsselwörter filtern} werden alle \emph{Bi-Gramme} entfernt, die kein Schlüsselwort aus der \emph{TF-IDF}-Methode enthalten. Somit erhält der Schritt \emph{untypische Wortartenkombinationen mit POS-Tagging entfernen} eine reduzierte Liste mit \emph{Bi-Grammen}, bei dem mindestens eines der beiden \emph{Bi-Gramm}-Wörter ein Schlüsselwort darstellt. Jedes Wort der \emph{Bi-Gramm}-Liste durchläuft eine weitere Filterung. Im Englischen existieren Wortarten, die typischerweise nicht nebeneinander stehen. Durch Entfernung dieser \emph{Bi-Gramme} durch POS-Tagging-Kombinationen erfolgt eine weitere Trennung der Wortketten in der \emph{Bi-Gramm}-Liste. Wörter die nicht zusammengehören, verlieren dadurch die Verbindung zueinander. Im darauf folgenden Schritt \emph{Bi-Gramm zu Stichpunkten überführen} werden die \emph{Bi-Gramme} jeweils zu einem String zusammengefügt, bei dem im darauf Folgendem \emph{Bi-Gramm} ebenfalls ein relevantes Wort enthalten sind. Dadurch formt sich eine Liste mit Stichpunkten bestehend aus eins bis x vielen Wörtern, die Bezug zueinander haben. Im Schritt \emph{ins Deutsche übersetzen} wird die Liste mit Stichpunkten zurück ins Deutsche übersetzt. Dieser Ablauf wird für die beiden Felder \emph{Aufgaben} und \emph{Skills} durchlaufen, da diese Volltextfelder darstellen.
\section{Implementationsdetails}
Dieses Kapitel beschreibt den technischen Entwicklungsprozess zur Umsetzung der Anforderungen des Systems. Die Implementierung fokussiert sich auf die Umsetzungen von Technologien und Funktionsweisen verschiedener Anforderungen. %Zudem wird die Struktur des Projektes aufgezeigt.
\subsection{Pipeline}
Die Pipeline wurde in der Programmiersprache Python umgesetzt. Python hat sich zu einer der populärsten interpretierten Programmiersprachen entwickelt \cite{mckinney2012python}. Die Programmiersprache eignet sich insbesondere für die Erstellung kleiner Programme und Skripte, die zur Automatisierung von Aufgaben eingesetzt werden können \cite{mckinney2012python}. Python hat eine große und aktive Community für wissenschaftliche Berechnungen und Datenanalysen hervorgebracht und hat sich in den letzten Jahren zu einer der wichtigsten Sprachen für Data Science, maschinelles Lernen und allgemeine Softwareentwicklung in Wissenschaft und Industrie entwickelt \cite{mckinney2012python}. Python unterstützt Modularität, wodurch ein Teil der Anforderungen somit abgedeckt werden kann. Die Implementierungen der einzelnen Verfahren aus den Anforderungen werden nicht manuell, sondern auf Basis von bereits existierende Bibliotheken umgesetzt.
\subsection{Projektstruktur}
\label{sec:projektstruktur}
Das Projekt wird in einem git Repository gespeichert und versioniert. %Die Projektstruktur ist ohne zusätzliche Konfigurationsdateien wie folgt aufgebaut:
%\dirtree{%
%	.1 .git/.
%	.2 modules/.
%	.3 ner.py.
%	.3 nGram.py.
%	.3 posTagging.py.
%	.3 preprocessing.py.
%	.3 readRequirements.py.
%	.3 textRankingAlgorithm.py.
%	.3 tfIdf.py.
%	.3 transformRequirements.py.
%	.3 translate.py.
%	.2 requirements/.
%	.3 jiraTickets.json.
%	.3 preprocessedRequirements.json.
%	.3 ....
%	.2 app.py.
%	.2 ....
%}
Die einzelnen Module sind in dem Verzeichnis \url{modules/} in separaten \url{.py} Dateien gelagert. Die \emph{Bedarfsmeldungen} werden im \url{requirements/}-Verzeichnis gespeichert. Alle relevanten \emph{Bedarfsmeldungen} sind in einer \url{jiraTickets.json}-Datei als Liste gespeichert. Die Datei \url{app.py} ist der Kern der Pipeline. Diese importiert alle Module und implementiert die Struktur der Pipeline. Das Projekt kann über den Befehl \lstinline{> py app.py} ausgeführt werden.
\subsection{Modulimplementationen}
Nachfolgend werden Implementationsdetails zu den einzelnen Modulen gegeben. Dabei werden verwendete Bibliotheken und Code-Details näher erläutert.
\paragraph{Strukturierung von Bedarfsmeldungen}\mbox{}\\
Die \emph{Bedarfsmeldungen} wurden über die Jira-Schnittstelle extrahiert und im Verzeichnis \url{requirements/jiraTickets.json} gespeichert. Zur Eingrenzung der Datenmenge wurde ein Filter angewendet, der nur die \emph{offenen} und \emph{eskalierten} \emph{Bedarfsmeldungen} zurückgibt. Diese sind die relevanten und noch aktuellen \emph{Bedarfsmeldungen}. Zudem wurden alle unrelevanten Felder mit einem weiteren Filter herausgenommen. Die Daten aus der Jira-API bestehen namentlich aus \emph{customfields} mit einer angehangenden ID. Der Softwareprototyp lädt in dem Modul \emph{readRequirements.py} die unstrukturierten Fields und formt diese in dem Modul \emph{transformRequirements.py} in \emph{Bedarfsmeldungs}-Objekte um.
\begin{center}
	\begin{tabularx}{1\textwidth} { 
			| >{\raggedright\arraybackslash}X 
			| >{\raggedright\arraybackslash}X
			| >{\raggedright\arraybackslash}X | }
		\hline
		Display-Felder & Jira-API-Felder & Objekt-Felder \\
		\hline
		\hline
		Überschrift & summary & header\\
		\hline
		Rolle & customfield\_15321 & role\\
		\hline
		Aufgaben & customfield\_10288 & tasks\\
		\hline
		Skills & customfield\_10296 & skills\\
		\hline
		Skill-Level & customfield\_15322 & skillLevel\\
		\hline
		Kunde & customfield\_10279 & customer\\
		\hline
		Einsatzort & customfield\_10297 & location\\
		\hline
		Beginn & customfield\_10293 & timePeriod\\
		\hline
		Ende & customfield\_10294 & timePeriod\\
		\hline
		Tagessatz & customfield\_10298 & dailyRate\\
		\hline
	\end{tabularx}\\
	\captionof{table}{Übersicht der Datenfelder}
	\label{tab:jiradaten}
\end{center}
In der Tabelle \ref{tab:jiradaten} ist in der ersten Spalte eine Übersicht der Datenfelder, wie diese in der Abbildung \ref{fig:jiraafter} mit dem Mockup einer standardisierten \emph{Bedarfsmeldung} auftauchen. In der zweiten Spalte sind die dazugehörigen Feldernamen, die in der \url{jiraTickets.json} enthalten sind. Die dritte Spalte spiegelt die jeweiligen Namen innherhalb des  \emph{Bedarfsmeldungs}-Objekts im Prototypen wieder. Das  \emph{Bedarfsmeldungs}-Objekt dient der strukturierten Handhabung der \emph{Bedarfsmeldungs}-Daten innerhalb des Systems.

%Um einen Freitext aus einer einzelnen \emph{Bedarfsmeldung} für die Pipeline zu laden werden Daten mit der Endung \url{.txt} verwendet.
%\begin{lstlisting}[caption={Implementation der Methode read() des Moduls \emph{readRequirements.py}}, label=lst:read]
%	def read(filename):
%		path = os.path.join(PATH, filename)
%		file = open(path, "r", encoding="utf-8")
%		content = file.read()
%		file.close()
%		return content
%\end{lstlisting}
%Im Listing \ref{lst:read} ist die Implementierung der Methode zum laden eines Freitextes dargestellt. Durch ein mitgelieferten Parameter \emph{filename} wird der Name der Datei in Zeile 2 an den Pfad angehängt. Mit der Methode \lstinline{open()}
%aus Zeile 3 kann die Datei geladen werden. Die Methode erhält die Parameter Pfad, \emph{'r'} (read) und das encoding \emph{'utf-8'}. Das encoding ist dabei Entscheidend, damit die unstrukturierten \emph{Bedarfsmeldungen} laden können. Bei der Pflege in Jira wird wenig Wert auf eine einheitliche Struktur. Somit können Zeichen enthalten sein, die beim öffnen nicht erkannt werden und eine Fehlermeldung wird zurückgegeben. Zur Vermeidung dieses Fehlers wird das encoding festgelegt. Nach dem laden durch die Methode \lstinline{read()} wird der Inhalt der \url{.txt} Datei in der Variable \emph{content} gespeichert und zurückgegeben.
\paragraph{Übersetzung}\mbox{}\\
Das Modul \emph{translate.py} ist dazu da, um die \emph{Bedarfsmeldungen} zu übersetzen. Für die Übersetzung wurde die Python Bibliothek \emph{deep-translator} verwendet. Diese bietet Implementationen unterschiedlicher Übersetzungs-APIs von diversen Anbietern. Der Vorteil ist dabei die vereinfachte Möglichkeit Anbieter bei Bedarf zu wechseln.
%\begin{lstlisting}[caption={Implementation des Moduls \emph{translate.py}}, label=lst:translate]
%	from deep_translator import GoogleTranslator
%	
%	def translate(text):
%		translated = GoogleTranslator(source='auto', target='en').translate(text)
%		return translated
%\end{lstlisting}
%Das Listing \ref{lst:translate} zeigt die Implementation des Moduls. 
Es wurde sich für den Google Translator entschieden, da hierfür kein API-Key benötigt wird. Die Methode erhält einen Text als Parameter. Der Google Translator erhält die Parameter \emph{source} und \emph{target}, bei dem \emph{source} angibt in welcher Sprache der Eingabetext ist. Durch Angabe von \emph{'auto'} wird die Sprache ermittelt. Der Grund dafür ist, dass grundsätzlich anderssprachige \emph{Bedarfsmeldung} enthalten sein können. Der Parameter \emph{target} ist die Zielsprache in die der Input übersetzt werden soll. Die Zielsprache ist hier Englisch (\emph{'en'}).
\paragraph{NER}\mbox{}\\
Um die Methode \emph{NER} zu Implementieren wurde die Bibliothek \emph{spaCy} und das Modell \emph{en\_core\_web\_sm} verwendet. 
%\begin{lstlisting}[caption={Implementation des Moduls \emph{ner.py}}, label=lst:ner]
%	import spacy
%	nlp = spacy.load("en_core_web_sm")
%	ner_categories = ["ORG","FAC","GPE","PRODUCT", "EVENT", "LANGUAGE", "DATE", "QUANTITY"]
%	def useNER(requirement):
%		tokenized = nlp(requirement)
%		entities = []
%		for ent in tokenized.ents:
%			if ent.label_ in ner_categories:
%				entities.append((ent.text, ent.label_))
%	return entities
%\end{lstlisting}
Die zu extrahierende Kategorie ist Datum (DATE). Innerhalb der Methode \lstinline{useNER()} wird der Volltext als Parameter übergeben und zu Tokens umgeformt. Anschließend werden alle Tokens durchlaufen und nach ihren Kategorien überprüft. Ist ein Token die definierte Kategorie, wird der Token in eine separate Liste gespeichert und zurückgegeben. Die extrahierten Tokens werden aus dem Volltext entfernt, um am ende keine doppelten Stichpunkte zu erhalten.
\paragraph{Preprocessing}\mbox{}\\
Vor der weiteren Nutzung der Daten innerhalb einer \emph{Bedarfsmeldung}, ist es erforderlich diese von irrelevanten Wörtern, Zeichen und Formatierungen zu befreien. % Zur Entfernung von Satzzeichen wurde die Bibliothek \emph{string} verwendet.
%\begin{lstlisting}[caption={Implementation der Methode removePunctuation() des Moduls \emph{preprocessing.py}}, label=lst:punctuation]
%	import string
%	def removePunctuation(text):
%		content=""
%		for i in text: 
%			if i not in string.punctuation:
%				content+=i    
%		return content
%\end{lstlisting}
%Im Listing \ref{lst:punctuation} ist die Implementierung dargestellt. Die \emph{Bedarfsmeldung} wird in Zeile 2 als Parameter übergeben. In Zeile 4 erfolgt ein Durchlauf jedes Zeichens innerhalb der \emph{Bedarfsmeldung}. Falls innerhalb der Schleife das Aktuelle Zeichen kein Satzzeichen enthält wird dieses in die Variable \emph{content} zwischengespeichert. Nach Abschluss der Schleife wird die Variable zurückgegeben. 
Zur Eliminierung wiederaufgetretener Wörter, die keine Relevanz für den Informationsgehalt aufweisen, wurde die Bibliothek \emph{nltk} verwendet. Diese beinhaltet eine Liste an sogenannten \emph{stopwords}.
%\begin{lstlisting}[caption={Implementation der Methode removeStopwords() des Moduls \emph{preprocessing.py}}, label=lst:stopwords]
%	from nltk.corpus import stopwords
%	def removeStopwords(text):
%		words=[word for word in text.split(" ") if word not in set(stopwords.words('english'))]
%		return " ".join(str(word) for word in words)
%\end{lstlisting}
%Im Listing \ref{lst:stopwords} ist die Implementierung zur Entfernung von \emph{stopwords} dargestellt. In Zeile 3 
Es werden alle mit einem Leerzeichen getrennten Wörter aus der übergebenen \emph{Bedarfsmeldung} in einer Liste aufgeteilt. Dabei wird jeder Listeneintrag mit der \emph{stopword}-Liste von \emph{nltk} verglichen. Stimmt das Wort nicht mit einem Eintrag der \emph{stopwords} überein, wird diese in die Liste \emph{words} hinzugefügt. Zum Schluss werden die Wörter wieder zu einem String zusammengetragen und zurückgegeben. \\

Um weitere Formatierungen und ungewünschte Zeichen zu entfernen wird die Bibliothek \emph{re} verwendet. Diese kann Regular Expression-Patterns anwenden und Bereiche, die zum Pattern passen entfernen.
%\begin{lstlisting}[caption={Implementation der Methode removeTags() und removeSpecialCharactersAndDigits() des Moduls \emph{preprocessing.py}}, label=lst:re]
%	import re
%	def removeTags(text):
%		return re.sub("</?.*?>"," <> ",text)
%	
%	def removeSpecialCharactersAndDigits(text):
%		return re.sub("(\\d|\\W)+"," ",text)
%\end{lstlisting}
%Im Listing \ref{lst:re} sind Implementationsdetails zur Entfernung von Tags und ungewollte Zeichen dargestellt. 
Die Methode \lstinline{removeTags()} erhält die Expression \lstinline{</?.*?>}. Dabei werden \lstinline{<Tags>} ermittelt und mit der Methode \lstinline{sub()} entfernt. Zur Entfernung von Ziffern und Nicht-Alphanummerischen Zeichen wird die Expression \lstinline{(\\d|\\W)+} angewendet. Als Ergebnis des Preprocessing wird ein gesäuberter String ohne Zeichen und Tags zurückgegeben. Zum schluss werden alle Wörter in Kleinbuchstaben umgewandelt. Der Grund dafür ist, dass somit die Vergleichbarkeit der Wörter gefördert wird. Es könnte sonst vorkommen, dass Beispielsweise beim Schlüsselwörterabgleich zwei Wörter nicht als identisch identifiziert werden, da sie einmal mit großem und eimal mit kleinem Anfangsbuchstaben geschrieben wurde.
\paragraph{TF-IDF}\label{tfidf}\mbox{}\\
Vorbereitend für die \emph{TF-IDF}-Methode wurden die \emph{Skills} und \emph{Aufgaben} aus allen \emph{Bedarfsmeldungen} ins Englische übersetzt, Preprocessed und in einem String zusammengefasst. Der Grund dafür ist, dass für die \emph{TF-IDF}-Methode ein Textkorpus benötigt wird, woraus die Schlüsselwörter durch \emph{Term Frequency} ermittelt werden. Damit dieser Prozess nur einmal erfolgen muss, wurden die Ergebnisse in die \url{requirements/preprocessedRequirements.json} gespeichert. Für die Implementierung des \emph{TF-IDF} wurden die Bibliotheken \emph{sklearn} und \emph{numpy} verwendet. Der Textkorpus wird dem \emph{TfidfVectorizer} von \emph{sklearn} beigefügt und die \emph{TF-IDF}-Werte werden berechnet. Anschließend werden alle durchschnittlichen \emph{TF-IDF}-Werte für jedes Wort im gesamten Textkorpus berechnet. Daraus wird eine Liste mit Wörtern und ihren durchschnittlichen \emph{TF-IDF}-Werten erstellt und in absteigender Reihenfolge sortiert. Durch ein vordefinierten Score Threshold können darunterliegende Schlüsselwörter entfernt werden. Dies ist relevant, um Wörter mit niedrigem Scoring und diese Beispielsweise nur einmal im Textkorpus auftauchen nicht als Schlüsselwörter erfasst werden. Als Ergebnis wird eine Liste mit Schlüsselwörtern zurückgegeben.

%\begin{lstlisting}[caption={Implementation des Moduls \emph{tfIdf.py}}, label=lst:postagging]
%import modules.readRequirements as readRequirements
%from sklearn.feature_extraction.text import TfidfVectorizer
%import numpy as np
%def useTfIdf(text):
%	object = readRequirements.loadJson("preprocessedRequirements.json")
%	documents = [object['tasks'], object['skills']]
%	vectorizer = TfidfVectorizer()
%	tfidf_matrix = vectorizer.fit_transform(documents)
%	feature_names = vectorizer.get_feature_names_out()
%	avg_tfidf_scores = np.mean(tfidf_matrix.toarray(), axis=0)
%	tfidf_scores = list(zip(feature_names, avg_tfidf_scores))
%	sorted_tfidf_scores = sorted(tfidf_scores, key=lambda x: x[1], reverse=True)
%	score_threshold = 0.1
%	filtered_keywords = [word for word, score in sorted_tfidf_scores if score > score_threshold]
%	return filtered_keywords
%\end{lstlisting}

%\paragraph{TextRank}\mbox{}\\
%Für die Implementation von \emph{TextRank} wurde die Bibliothek \emph{spaCy} und \emph{pytextrank} verwendet. Dieses bietet die NLP-Pipeline \emph{en\_core\_web\_sm}, das mit Internettext vortrainiert wurde und Vokabeln, Syntax und Entitäten enthält.
%\begin{lstlisting}[caption={Implementation des Moduls \emph{textRankingAlgorithm.py}}, label=lst:textrank]
%	import spacy
%	import pytextrank
%	def useTextRank(requirement):
%		nlp = spacy.load("en_core_web_sm")
%		nlp.add_pipe("textrank")
%		tokenized = nlp(requirement)
%		for phrase in tokenized._.phrases:
%			print(phrase.text)
%			print(phrase.rank, phrase.count)
%			print(phrase.chunks)
%\end{lstlisting}
%Das Listing \ref{textrank} zeigt die Implementierung des \emph{textRankingAlgorithm.py} Moduls. In Zeile 3 wird die \emph{Bedarfsmeldung} als Parameter übergeben. Die Pipeline \emph{en\_core\_web\_sm} wird in Zeile 4 geladen und in Zeile 5 werden die \emph{TextRank} Elemente der Pipeline hinzugefügt. Anschließend wird in Zeile 6 die \emph{Bedarfsmeldung} der Pipeline eingefügt und in Tokens umgewandelt.\\
%\todo{muss noch fertiggestellt werden}\\
\paragraph{N-Gramm}\label{sec:ngram}\mbox{}\\
Zur Bildung von Wortketten wie in Kapitel \ref{sec:anforderungsanalyse} beschrieben, ist es notwendig \emph{bi-Gramme} zu verwenden. Dadurch können immer zwei nebeneinanderstehenden Wörter isoliert betrachtet werden. Bei einem höheren n der \emph{n-Gramme} würde somit das Prinzip einer Wortkette, wie sie im System benötigt wird nicht funktionieren. Die Methode der \emph{n-Gramme} wurde mit der Bibliothek \emph{nltk} implementiert.
%\begin{lstlisting}[caption={Implementation des Moduls \emph{nGram.py}}, label=lst:ngram]
%	from nltk import ngrams
%	n = 2
%	def useNGram(text):
%		nGramList = []
%		nGrams = ngrams(text.split(), n)
%		for grams in nGrams:
%			nGramList.append(grams)
%		return nGramList
%\end{lstlisting}
%Das Listing \ref{lst:ngram} zeigt die Implementation des Moduls \emph{nGram.py}. 
Eine Variable \emph{n} ist definiert, die die Größe eines \emph{n-Gramms} widerspiegelt. Da in der Pipeline \emph{Bi-Gramme} benötigt werden, liegt der Wert von n auf 2. Mit der Methode \lstinline{ngrams()}
können die \emph{n-Gramme} generiert werden. Diese erhalten einen \emph{String} und die Variable \emph{n} als Parameter. Der zu überführende Text wird als Parameter übergeben und durch die Methode \lstinline{split()}
in einer List auf die einzelnen Wörter umgeformt. Die einzelnen Tupel mit den \emph{Bi-Grammen} werden am Ende zurückgegeben.
\paragraph{Entfernung von Bi-Grammen ohne Schlüsselwörter}\mbox{}\\
Zur Entfernung von Bi-Gramm-Tupel, die keine Schlüsselwörter enthalten, wurde die Methode \lstinline{containsKeywords()} erstellt.
\begin{lstlisting}[caption={Implementation der Filterung für Schlüsselwörter in einer Bi-Gramm Liste}, label=lst:bigram]
	def containsKeywords(biGramList, keywordsList):
		filteredBiGram = []
		for tupel in biGramList:
			if any(word in tupel for word in keywordsList):
				filteredBiGram.append(tupel)
		return filteredBiGram
\end{lstlisting}
Das Listing \ref{lst:bigram} implementiert diese Methode. Als Parameter wird die \emph{Bi-Gramm}-Liste und die Schlüsselwörterliste übergeben. In Zeile 3 ist zu sehen, dass jeder Tupel traversiert wird. Dabei wird in Zeile 4 jedes Wort im Tupel mit der Schlüsselwörterliste verglichen. Wenn das aktuelle Wort in der Liste enthalten ist, so wird dieser Tupel in eine neue Liste aufgenommen. Diese wird am Ende zurückgegeben, wodurch eine gefilterte Liste mit ausschließlich enthaltenen Schlüsselwörtern vorhanden ist.
\paragraph{POS-Tagging}\label{postagging} \mbox{}\\
In der Englischen Sprache existieren verschiedene Wortgruppenkombinationen, die zusammen ungewöhnlich klingen und dadurch beim sprechen und schreiben nicht oder nur selten verwendet werden. Wenn diese Kombinationen in den \emph{Bi-Grammen} identifiziert und entfernt werden, können somit Verbindungsketten unterbrochen werden. Dadurch stehen Wörter und spätere Stichpunkte nicht nebeneinander, die sprachlich im betrachteten Kontext wenig Sinn machen. \emph{Bedarfsmeldungen} wurden im professionell seriösen Kontext angefertigt, wodurch einige nebeneinanderstehenden Wörter im Englischen nicht häufig zusammen auftauchen.
\begin{center}
	\begin{tabularx}{1\textwidth} { 
			| >{\raggedright\arraybackslash}X 
			| >{\raggedright\arraybackslash}X
			| >{\raggedright\arraybackslash}X | }
		\hline
		Wortarten & POS-Tagging Abkürzungen & Beispiel \\
		\hline
		\hline
		Nomen + Adjektiv & NN JJ & "time beautiful"\\
		\hline
		Verb + Adjektiv & VB JJ & "think beautiful"\\
		\hline
		Verb + Pronomen & VB PRP & "think him"\\
		\hline
		Verb + Verb & VB VB & "think understand"\\
		\hline
		Adjektiv + Adjektiv & JJ JJ & beautiful strange\\
		\hline
		Adjektiv + Verb & JJ VB & beautiful think\\
		\hline
		Adjektiv + Adverb & JJ RB & "beautiful fast"\\
		\hline
		Adverb + Adjektiv & RB JJ & "fast beautiful"\\
		\hline
	\end{tabularx}\\
	\captionof{table}{Untypische Wortartenkombinationen in der englischen Sprache.}
	\label{tab:wortkombinationen}
\end{center}
In der Tabelle \ref{tab:wortkombinationen} sind einige Beispiele für solche untypischen Zusammensetzungen. In Spalte eins sind die jeweiligen Wortarten beschrieben. Die zweite Spalte zeigt die dazugehörigen Tags, wie diese mit \emph{POS-Tagging} ermittelt werden. Die dritte Spalte zeigt jeweils ein Beispiel dieser Wortkombinationen. Die Identifikation erfolgte durch Betrachtung der Grundregeln der Englischen Grammatik aus der Arbeit \citetitle{ogden1930basic}\cite{ogden1930basic} und den Syntaxregeln aus der Arbeit \citetitle{tayal2014syntax}\cite{tayal2014syntax}. Diese Regeln beinhalten Kombinationen aus Wortarten, die zusammen Erlaubt sind. Dadurch konnten Kombinationen ausgeschlossen werden. Schließlich wurden durch trial and error weitere Kombinationen identifiziert, die schließlich in die Tabelle aufgenommen wurden. Es kann nicht ausgeschlossen werden, dass weitere Kombinationen und Sonderfälle existieren. Dies stellt ein Optimierungspotenzial außerhalb dieser Ausarbeitung dar. Somit werden nur die Kombinationen aus der Tabelle \ref{tab:wortkombinationen} im System beachtet. Dennoch ist dieses so konzipiert, dass weitere Kombinationen in das System hinzugefügt werden können.\\

Zur Implementierung der Methode \emph{POS-Tagging} wurde die Bibliothek \emph{nltk} verwendet. Die \emph{Tags} werden mit dem vortrainierten Modell \emph{averaged\_perceptron\_tagger} ermittelt.
%\begin{lstlisting}[caption={Implementation des Moduls \emph{posTagging.py}}, label=lst:postagging]
%	import nltk
%	nltk.download('averaged_perceptron_tagger')
%	def usePosTagging(text):
%		tokens = nltk.word_tokenize(text)
%		pos_tags = nltk.pos_tag(tokens)
%		return pos_tags
%\end{lstlisting}
%Im Listing \ref{lst:postagging} sind Details zur Implementierung dargestellt. 
Als Parameter wird ein String übergeben, der zwei Wörter aus den \emph{Bi-Grammen} enthält. Dieser String wird in Tokens umgewandelt. Diese werden in die Methode \lstinline{pos_tag()}
übergeben und eine Liste mit Tupeln wird zurückgegeben, bei dem das Wort und der dazugehörige \emph{Tag} enthalten ist.
\paragraph{Entfernung von Bi-Grammen mit untypischen Wortartenkombinationen}\mbox{}\\
Die \emph{Bi-Gramm}-Liste muss im nächsten Schritt weiter reduziert werden. Hierbei werden alle \emph{POS-Tagging}-Kombinationen aus der Tabelle \ref{tab:wortkombinationen} mit den Tupeln aus der \emph{Bi-Gramm}-Liste verglichen.
\begin{lstlisting}[caption={Implementation der Filterung von Wortartenkombinationen}, label=lst:wortarten]
	combinationsToRemove = ["NN JJ", "VB JJ", "VB PRP", "VB VB", "JJ JJ", "JJ VB", "JJ RB", "RB JJ"]
	def removeWordCombination(biGramList):
		filteredList = []
		for left, right in biGramList:
			biGramString = f"{left} {right}"
			biGramStringsWithTags = usePosTagging(biGramString)
			tagString = ' '.join(tag for _, tag in biGramStringsWithTags)
			if tagString not in combinationsToRemove:
				filteredList.append((left, right))
		return filteredList
\end{lstlisting}
Das Listing \ref{lst:wortarten} implementiert diesen Schritt. Dazu werden in Zeile 1 alle Kombinationen als String in eine Liste zwischengespeichert. Als Parameter wird die zu filternde \emph{Bi-Gramm}-Liste übergeben. Diese wird in Zeile 4 traversiert und die Tupel mit den beiden Wörtern aus den \emph{Bi-Grammen} wird in Zeile 5 als String zusammengefügt. Dieser String wird in die \emph{POS-Tagging}-Methode übergeben und als Rückgabewert erhält das System eine Liste mit zwei Tupeln. Diese Tupel beinhalten jeweils eines der Wörter und den dazugehörigen Tag aus dem \emph{POS-Tagging}. Die beiden Tags werden in Zeile 7 zusammengefügt, sodass ein String entsteht, wie es in der \emph{combinationsToRemove}-Liste in Zeile 1 ist. In Zeile 8 wird überprüft ob die Kombination aus Tags eines aus der Liste in Zeile 1 entspricht. Ist dies nicht der Fall, wird der Tupel in eine neue Liste eingefügt, die am Ende als Ausgabe zurückgegeben wird.
\paragraph{Überführung von Bi-Grammen in Stichpunkte}\label{par:stichpunkte}\mbox{}\\
Da der Volltext aus den Bedarfsmeldungen nicht einzelne Wörter, sondern inhaltlich aufeinander bezogene Stichpunkte entstehen lassen soll, werden die um die Schlüsselwörter herumliegenden Wörter aneinander gefügt. Dazu werden diejenigen \emph{Bi-Gramme} zu einem Satz zusammengefügt, die im ersten Tupel auf der rechten Seite und im zweiten Tupel auf der linken Seite das gleiche Wort enthalten. Hinzu kommt eine Überprüfung ob das darauffolgende linke Wort ein Schlüsselwort ist. Somit werden nur die Wörter zu Sätzen verkettet, die aus Schlüsselwörtern bestehen. In dem Fall, bei dem im darauf folgenden Tupel kein Schlüsselwort auf der linken sondern nur auf der rechten besteht, entsteht ein neuer Stichpunkt. Als Beispiel können folgende \emph{Bi-Gramme} betrachtet werden: (\grqq expertise\grqq, \grqq aws\grqq) (\grqq aws\grqq, \grqq technologie\grqq) (\grqq technologie\grqq, \grqq pflicht\grqq) (\grqq pflicht\grqq, \grqq englisch\grqq) (\grqq englisch\grqq, \grqq skills\grqq). Als Schlüsselwörter wurden die Wörter \grqq aws\grqq, \grqq pflicht\grqq, \grqq englisch\grqq ermittelt. Die Idee ist es, die zusammengehörigen \emph{Bi-Gramme} an diesen Schlüsselwörtern zusammenzufügen. Daraus werden die Stichpunkte \grqq expertise aws technologie\grqq und \grqq technologie pflicht englisch skills\grqq.
\begin{lstlisting}[caption={Umformung der Bi-Gramm Liste in Stichpunkte}, label=lst:stichpunkte]
	def combineWords(filteredBiGram, keywords):
		combinedWords = []
		i = 0
		while i < len(filteredBiGram):
			keyPhrase = recursiveCombine(filteredBiGram, i, "", keywords, "")
			combinedWords.append(keyPhrase)
			wordsCount = len(keyPhrase.split(" "))
			wordsCount = max(1, wordsCount - 1)
			i += wordsCount
			if(i > len(filteredBiGram)):
				break
		return combinedWords
	
	def recursiveCombine(filteredBiGram, index, currentString, keywords, lastRightItem):
		left, right = filteredBiGram[index]
		if not currentString:
			currentString = left
		else:
			currentString += " " + left
		if lastRightItem in {left, ""}:
			if right in keywords and index + 1 < len(filteredBiGram):
				currentString = recursivCombine(filteredBiGram, index + 1, currentString, keywords, right)
			else:
				currentString += " " + right
		return currentString
\end{lstlisting}
Die Implementierung zu dieser Idee ist im Listing \ref{lst:stichpunkte} dargestellt. Die Methode \lstinline{combineWords} in Zeile 1 erhält die \emph{Bi-Gramm}-Liste und die Schlüsselwörterliste als Parameter. Die \emph{Bi-Gramm}-Liste wird traversiert und ein string aus zusammengehörigen Wörtern wird rekursiv in Zeile 5 mit der Methode \lstinline{recursiveCombine} erstellt. Dazu wird in jedem rekursiven Schritt überprüft, ob das darauf folgende rechte wort ein Schlüsselwort und das gleiche wie das linke Wort im aktuellen Tupel entspricht. Ist dies der Fall werden die \emph{bi-Gramm} an der Stelle zusammengefügt. Dies wiederholt sich bis dieser Fall nicht mehr eintritt. Dadurch ist die Rekursion vorbei und es wird in Zeile 7 bis 11 geprüft wie viele Wörter in einem String gelandet sind. Das ist wichtig um herauszufinden an welcher Stelle in der \emph{bi-Gramm}-Liste weiter gearbeitet werden muss. Sind alle \emph{bi-Gramme} zusammengesetzt worden, wird eine Liste mit allen zusammengesetzten Stichpunkten zurückgegeben.
\section{Erklärung des Systems anhand eines Beispiels}
Nachfolgend wird zum besseren Verständnis die Schritte zur Verarbeitung eines Volltextes anhand eines konkreten Beispiels dargestellt.\\
\begin{enumerate}
	\itemsep-0.15em
	\item Zu Beginn haben wir folgenden Volltext:\\ \textbf{Für verschiedene IT-Projekte (mit internen und externen Kunden) benötigen wir Unterstützung von einem Projektleiter, mit Erfahrungen in der Planung, Koordination und Durchführung von IT-Migrations- und Implementierungsprojekten im Umfeld von Rechenzentrumsservices. Mindestens 2 Jahre Erfahrung sind erwünscht.}
	\item Dieser Text wird übersetzt:\\ \textbf{For various IT projects (with internal and external customers) we need support from a project manager with experience in planning, coordinating and executing IT migration and implementation projects in the area of data center services. At least 2 years of experience is desired.}
	\item Nun werden aus dem Text Datums- und Zeitangaben extrahiert und aus dem Volltext entnommen: \grqq At least 2 years\grqq \\ Der neue Text beinhaltet: \textbf{For various IT projects (with internal and external customers) we need support from a project manager with experience in planning, coordinating and executing IT migration and implementation projects in the area of data center services. of experience is desired.}
	\item Als nächstes wird der Text Vorverarbeitet:\\ \textbf{for various it projects internal external customers need support project manager experience planning coordinating executing it migration implementation projects area data center services experience desired}
	\item Die Schlüsselwörter bestehen unter anderem aus:\\ \textbf{planning, coordinating, migration, implementation, data, center, services, ...}
	\item Aus dem Volltext werden diese \emph{bi-Gramme} erstellt:\\ \textbf{('for', 'various'), ('various', 'it'), ('it', 'projects'), ('projects', 'internal'), ('internal', 'external'), ('external', 'customers'), ('customers', 'need'), ('need', 'support'), ('support', 'project'), ('project', 'manager'), ('manager', 'experience'), ('experience', 'planning'), ('planning', 'coordinating'),
	('coordinating', 'executing'), ('executing', 'it'), ('it', 'migration'), ('migration', 'implementation'), ('implementation', 'projects'), ('projects', 'area'), ('area', 'data'), ('data', 'center'), ('center', 'services'), ('services', 'experience'), ('experience', 'desired')}
	\item Anschließend werden alle \emph{bi-Gramme} ohne Schlüsselwort entfernt:\\ \textbf{('support', 'project'), ('project', 'manager'), ('manager', 'experience'), ('experience', 'planning'), ('planning', 'coordinating'), ('coordinating', 'executing'), ('it', 'migration'), ('migration', 'implementation'), ('implementation', 'projects'), ('area', 'data'), ('data', 'center'), ('center', 'services'), ('services', 'experience'), ('experience', 'desired')}
	\item Der nächste Schritt beinhaltet die Entfernung von untypischen \emph{POS-Tagging}-Kombinationen. In diesem Beispiel sind alle Kombinationen zulässig, weswegen die \emph{bi-Gramm}-Liste unverändert bleibt.
	\item Nun werden die \emph{bi-Gramme} nach dem Verfahren aus Kapitel \ref{par:stichpunkte} in folgende Stichpunkte zusammengetragen:\\ \textbf{support project manager, manager experience planning coordinating executing, it migration implementation projects, area data Center services experience desired}.\\ \\ Hinzu kommen die Daten aus dem \emph{NER}-Schritt:\\ \textbf{At least 2 years}
	\item Nach der Übersetzung ist das Ergebnis: \begin{itemize}\itemsep-0.3em\item Unterstützung des Projektleiters \item Manager Erfahrung Planung Koordinierung Durchführung \item Projekte zur Implementierung von IT-Migrationen \item Erfahrung im Bereich Rechenzentrumsdienstleistungen erwünscht \item Mindestens 2 Jahre \end{itemize}
\end{enumerate}

%\chapter{Umsetzung der Visualisierungsmethoden}
\label{chap:implementation}
-Wie können diese Art und Weisen mit dem „adesso Staffing Advisor“ Lab implementiert werden

\section{World Cloud}

\newpage

\chapter{Evaluierung des entwickelten Systems}
\label{chap:evaluation}

vielleicht erklären warum precision, recall, f1 score nicht gehen -\\

nicht überlegen wie evaluieren sonder was will ich evaluieren,\\
was sind die fragen die ich beantworten möchte, was sind die aussagen die ich machen will. hypothesen belegen, wiederlegen\\
was möchte ich zeigen, (den expertenprozess abbilden, expertenprozess ist ideal, mein prozess hat diese abweichung)\\

z.b. erwartungshaltung formulieren und mit cosine similarity gucken was näher dran ist,

wie machen das andere ansätze,

\section{Evaluationsmetrik}
-cosine similarity
-performance, zeit

\section{Beschreibung des verwendeten Datensatzes}
\todo{darauf eingehen welche Felder in der JSON von JIRA sind und worauf sich genau konzentriert wird.}\\
\newpage

\section{Beschreibung des verwendeten Datensatzes}

überlegung ob tfidf unterschied macht alle bedarfsmeldungen mit einer zu vergleichen und daraus wichtige wörter identifizieren oder eine für sich alleine reicht.

gucken was tokenisierung wirklich macht
\section{Präsentation und Diskussion der Ergebnisse}
\newpage
g
\newpage
g
\newpage
g
\newpage
Zeit und Leistung Übersicht
\newpage
g
\newpage

\section{Vergleich des Systems mit einem Large Language Model-Ansatz}
\newpage
g
\newpage
g
\newpage

g
\newpage

\section{Analyse von Abweichungen, Ähnlichkeiten und Verbesserungspotenzialen des Systems}
\newpage
g
\newpage

\chapter{Zusammenfassung und Ausblick}
\label{chap:ergebnisseausblick}

ergebnis der arbeit: diese modelle in der reihenfolge kommen am nähesten an die bedarfsmeldung

\subsection*{Ausblick}
-keywords präzisieren und keywordskatalog anlegen
-genauer untersuchen wie einzelne ansätze mit anderen nlp vortrainierten datensätzen abschneidet
-in bezug zu recommender systems beschreiben wie damit weiter gemacht werden könnte, die keyword extraction auch für die profile nutzen 
\newpage



\chapter{Anhang}
\label{chap:ergebnisse}
\section{Interviewtranskripte}
\subsection{Vorabtest Fragen und Antworten}
\label{interview1}
1. Welche Art von Projekten sind typischerweise in Ihrem Unternehmen an der Tagesordnung? Können Sie uns Beispiele für verschiedene Arten von Projekten geben, die adesso durchführt?\\

Software-Entwicklungsprojekte, angefangen von Projekten in dem ein adessi in einem Kundenprojekt arbeitet über gemischte Teams aus adessi und Kunde bis hin zur kompletten Lieferung von Projekleitern, Testern, Requirements Engineer und Entwicklern\\

2. Wie werden Projektbedarfe und -anforderungen innerhalb von adesso typischerweise kommuniziert und dokumentiert?\\

Initial über den Maitre, der das Staffing übernimmt bzw. auch Vorschläge von Projektleitenden zum Staffing annimt, teilweise auch über das eigene Netzwerk zwischen Führungskräften, im CC, Bereich oder der LoB. Am Ende über das Staffing Jira\\

3. Welche Informationen halten Sie in einer Bedarfsmeldung für besonders wichtig oder unverzichtbar?\\

Senioritätslevel, Tagessatz, Remote/on Site Einsatz, Dauer, Technischer Stack (Muss- und Kann Kriterien), Einarbeitungszeiträume (ist es verrechenbar oder nicht?), Lieferverpflichtung\\

4. Wie detailliert sollten Bedarfsmeldungen Ihrer Meinung nach sein? Sind bestimmte Schlüsselaspekte oder -informationen in jeder Bedarfsmeldung enthalten?\\

Es sollte aussagefähig sein zumindest welche technischen Kompetenzen wichtig sind und welche Tagessätze, ob Remote möglich ist und die Dauer mindestens in der Bedarfsmeldung vorhanden sein.\\ 

5. Welche Herausforderungen oder Schwierigkeiten sind bei unklaren oder unvollständigen Bedarfsmeldungen aufgetreten?\\

Der Anforderer muss ggf. Fragen mehrfach beantworten, der Kanal über den kommuniziert wird (Teams Chat, Anruf, im Ticket, ...). Dadurch verliert man ggf. den Überblick\\

6. Wer sind die typischen Stakeholder bei der Erstellung von Bedarfsmeldungen und welche Rolle spielen sie?\\

Sales/PL: Anforderer mit den technischen Informationen, Maitre: kümmert sich um das Staffing bzw. die eigentliche Besetzung\\

7. Wie wird die Qualität von Bedarsmeldungen bei \emph{adesso} bewertet? Gibt es bestimmte Kriterien oder Standards, anhand derer Bedarfsmeldungen beurteilt werden?\\

Gar nicht meines Wissens nach.\\

8. Wie können Sie die Qualität und Klarheit von Bedarfsmeldungen verbessern?\\

Zukünftig: Durch klarere Vorgaben und weniger Freitext, aktuell: durch Nachfragen und Bitten um nachträgliche Pflege, ggf. durch Reviewprozesse bei eigenen Bedarfsmeldungen \\

9. Welche Auswirkungen haben unklare oder fehlende Informationen in Projektbeschreibungen auf die Effizienz und den Erfolg von Projekten?\\

Das Staffing dauert länger und ggf. werden die Stellen durch andere Diensteleister besetzt\\

10. Wie können Sie sicherstellen, dass die Bedürfnisse und Anforderungen aller relevanten Stakeholder in einer Bedarfsmeldung angemessen berücksichtigt werden?\\

Gute Abstimmungen bevor die Bedarfsmeldung erstellt wird, ggf. durch ein Quality-Gate (Review).\\
\subsection{Interview 2}
\label{interview2}
%Bürger, Marco\\
0:1:13 --> 0:1:33\\
I:\\
Du bist zum einen CC Leiter. Das heißt du hast auch Menschen unter deiner Leitung, die du auf Projekte zuweist.\\

0:1:33 --> 0:1:29\\
B:\\
Genau ja.\\

0:1:29 --> 0:1:37\\
I:\\
Du weiß also zu Projekten zu, übernimmst auch die Projektleitung.\\

0:1:37 --> 0:1:38\\
B:\\
Ja, genau.\\

0:1:38 --> 0:1:50\\
I:\\
Genau da würde ich dann erst mal gerne wissen: Was sind so die typischen Stakeholder bei der Erstellung von Bedarfsmeldungen und welche Rolle hast du dabei grundsätzlich?\\

0:1:50 --> 0:2:41\\
B:\\
In den Situationen nehme ich immer die Rolle des Beraters erstmal an. Die Stakeholder sind klassisch die Fachverantwortlichen beim Kunden, aber auch die Entscheider. Sprich also deren Vorgesetzte die quasi fachlich vielleicht das ganze nicht so bewerten können, aber das Budget dafür hergeben müssen und natürlich dann im Zweifelsfall auch CO, CEO oder sogar Geschäftsführer.\\

0:2:41 --> 0:2:48\\
I:\\
Kannst du ein paar Beispiele nennen, welche Arten von Projekten adesso so erhält und durchführt.\\

0:2:48 --> 0:3:50\\
B:\\
Ja, also im Prinzip kannst du das in 2 Arten von Projekten teilen. Aus einer sind Time Material Projekte, wo quasi der Kunde mit einer Idee kommt, wo wir gut unterstützen können. Beispielsweise bei Bestandsprojekten. Oder vielleicht weil ihnen selbst die Ressourcen dafür fehlen. Zum anderen hast du halt Festpreis Projekte, wo wir bestimmtes Gewerk für den Kunden abschätzen und das Ganze auch dann gänzlich liefern. In Festpreis Projekten haben wir eher die Staffing Hoheit. Das heißt, wir können entscheiden wen wir in das Projekt einsetzen. Wobei bei einigen Projekten durchaus auch der Kunde sich die Profile mit anschaut und dann auch entscheidet. Anhand von Interviews was macht Sinn, was passt bei mir ins Team vielleicht und denke ich, dass das am besten für mich wäre?\\

0:3:50 --> 0:4:2\\
I:\\
Wie werden diese Bedarfsmeldungen und Anforderungen denn so typischer Weise kommuniziert und dokumentiert? Gibt es da eine Art Ablauf, oder wie wird das gemacht?\\

0:4:2 --> 0:5:16\\
B:\\
Hängt auch immer ehrlicherweise vom jeweiligen Kunden ab. Bei manchen reicht ein Interview, welches du führst und dann schreibst du es in ein Word Dokument als Anforderungsbeschreibung nieder. Dann lässt man das gegen Zeichnen und dann ist gut. Manchmal muss man aber auch ein paar mehr Integration fahren und dann nochmal genau abzustecken, was denn Bestandteil der Beauftragung ist und was nicht. Also was die Bedarfsmeldung. Da muss man auch mal durchaus eins tiefer bohren, weil da teilweise der Gedanke, was der Kunde möchte nicht mit dem übereinstimmt, was eigentlich gebraucht wird. Das ist so, weil du klassisch irgendein Anforderungsdokument dafür fertig machst, wenn das immer ein bisschen höher geht Richtung Management. Beim Management mit dem Kunden kann es auch durchaus mal eine Präsentation sein, wo das Ganze nochmal ein bisschen aufbereitet ist. Ein bisschen klassisch, ein bisschen bunter und mit weniger Infos und mit weniger Tiefe präsentiert.\\

0:5:16 --> 0:5:38\\
I:\\
Welche Informationen sind denn für dich in Bedarfsmeldungen besonders wichtig oder auch unverzichtbar? Also zum Beispiel gibt es ja auch das Senioritätslevel.\\

0:5:38 --> 0:6:50\\
B:\\
Ja genau, also Senioritätslevel ist ehrlicherweise erstmal das zweitrangige. Das müssen wir im Nachgang einmal prüfen. Wichtig ist der Tech Stack und was vom Kunden kommt. Und quasi anhand der Bedarfsmeldung an sich. Wie hoch ist der Aufwand, der dahinter steckt? Sowohl durch den Kunden als auch das, was wir teilweise schätzen. Dann kann durchaus sein, dass der Kunde sagt, was gebraucht wird, er aber eigentlich keine Ahnung hat. Es kann vorkommen, dass der Kunde gerne in 2 Monaten durch ist, aber es keinen Sinn macht und wir mindestens ein halbes Jahr benötigt. Dann trifft man sich irgendwo in der Mitte und muss aber noch abgrenzen was Sinn macht und was nicht. Wenn das alles klar ist, dann kannst du überlegen was du eher brauchst. Machst du z.B. nur was mit Senioren? Oder reicht ein Junior. Das hängt dann immer eher von der Gesamtsituation ab.\\

0:6:50 --> 0:7:5\\
I:\\
Wie detailliert sollte dann eine Bedarfsmeldung sein? Gibt es Aspekte oder Informationen, die eigentlich in jeder Bedarfsmeldung drin sein sollten und müssen?\\

0:7:5 --> 0:7:24\\
B:\\
Ich glaube der Umfang ist immer ganz wichtig. Die Erwartungshaltung sollte immer detailliert sein. Und die Technologien ebenfalls.\\

0:7:24 --> 0:7:33\\
I:\\
Welche Herausforderungen oder oder Schwierigkeiten sind dann bei zum Beispiel unklaren oder unvollständigen Bedarfsmeldungen aufgetreten? Hattest du so etwas schon mal?\\

0:7:33 --> 0:8:8\\
B:\\
Ja durchaus, das sind immer Lernprozesse sowohl beim Kunden als auch bei dem, der die Anforderungen aufnimmt. Hängt immer vom Reifegrad des jeweiligen Konterparts ab. Das heißt auch, dass umso mehr du dich in der Situation meldest und Sachen aufgenommen hast, umso genauer kannst du mal nachfragen und hörst auch die Unsicherheit auf der einen oder anderen Seite heraus.\\

0:8:8 --> 0:8:17\\
I:\\
Gibt es denn irgendwie ein Mechanismus, wie die Qualität von Bedarfs Meldungen bewertet wird, oder gibt es irgendwelche Kriterien oder Standards womit dann beurteilt wird, ob eine Bedarfsmeldung gut oder eher schlecht ist?\\

0:8:17 --> 0:8:58\\
B:\\
Ich bin ein Fan davon auch Sachen mal querlesen zu lassen und dann vielleicht mal die ein oder andere Meinung einzuholen und auch noch mit dem Kunden, der die Bedarfsmeldungen stellt ganz nah dran zu bleiben und dann zu gucken, dass das immer funktioniert. Damit auch zum Schluss das rauskommt, was am Anfang vielleicht schon ne Idee gewesen ist.\\

0:8:58 --> 0:9:7\\
I:\\
Wie würdest du denn die Qualität von Bedarfsmeldungen verbessern?\\

0:9:7 --> 0:9:41\\
B:\\
Auch wirklich mit dem Kunden das Ganze einmal durchexerzieren und festzustellen, ob das Verständnis auf allen Seiten das gleiche ist. Weil nur dann kann es auch gut und produktiv werden. Wenn direkt am Anfang schon irgendwie Unklarheit da ist, dann kannst du auch davon ausgehen, dass da Diskussionsbedarf entsteht.\\

0:9:41 --> 0:9:49\\
I:\\
Welche Auswirkungen haben denn unklare und fehlende Informationen in Bedarfsmeldungen in Bezug auf die Effizienz und den Erfolg des Projektes? Hast du da irgendwie Erfahrungen machen können?\\

0:9:49 --> 0:10:31\\
B:\\
Ja, wenn wir uns unklar sind, wird der Aufwand immer um ein Vielfaches erhöhen, weil das irgendwie im Nachgang immer noch mal gerade gezogen werden muss. Wenn du Pech hast, kannst du alles, was du bisher gemacht hast wegschmeißen und nochmal neu anfangen. Bedeutet natürlich auch immer, dass ein großes Diskussionspotenzial zwischen sowohl den Projektbeteiligten als auch Kunde und Projekt existiert.\\

0:10:31 --> 0:10:44\\
I:\\
Wie könnte man theoretisch sicherstellen, dass die Bedürfnisse und Anforderungen aller Leute, die involviert sind, irgendwie angemessen berücksichtigt werden?\\

0:10:44 --> 0:14:56\\
B:\\
Ich glaube wenn du da mit viel Erfahrung reingehst und dann auch vielleicht genau weißt, wo du drauf zu achten hast. Mir fehlt ein bisschen die Fantasie, aber das ist ja jetzt dann deine Aufgabe da so ein Automatismus zu erkennen. Das wird auf jeden Fall schwierig weil ich auch an der Stelle glaube, dass wenn du so ein System schaffst, die quasi den Match drauf machen können, es trotzdem viel lernen muss. Von daher glaube ich, Erfahrung ist das es ausmacht, um am Ende zu sagen was Sinn macht oder nicht.\\
\subsection{Interview 3}
\label{interview3}
%Kirchner, Lars\\
0:0:14 --> 0:0:22\\
I:\\
Ich habe mir ein paar Infos geholt. Du warst zumindest mal richtig CC-Leiter oder?\\

0:0:22 --> 0:0:45\\
B:\\
Ja ich war mal richtig CC Leiter genau. Ich kann zumindest sagen, dass ich sogar mal 2 Kompetenz Center, 1 in München und 1 in Dortmund geleitet habe und für 48 Menschen zuständig war.\\

0:0:45 --> 0:0:51\\
I:\\
Ok, das bedeutet aber du machst teilweise noch CC Leitung. Oder gar nicht mehr?\\

0:0:51 --> 0:1:21\\
B:\\
Nein, aus in der Tat persönlichen Gründen habe ich vor 2 Jahren die Entscheidung getroffen, dass ich diese Rolle verlassen muss.\\

0:1:39 --> 0:1:44\\
I:\\
Okay, das heißt aber auch du bist aktuell Projektleiter.\\

0:1:44 --> 0:6:1\\
B:\\
Ich bin aktuell von der Laufbahnstufe höher. Programm Manager das ist formal noch eine Führungslaufbahn oder Führungsrolle. Das ist etwas anderes zumindest in der adesso Welt als Projektleitung. Ich kümmere mich in der Tat dann in einer Mischform von Projektleitung und Produktunterschiede um interne Projekte. Ich arbeite mit Studierenden zusammen, ich übernehme Angebotsmanagement für große Angebote. Ich vertrete einzelne Themen, bei dem man jemanden braucht, der entsprechend erfahren ist und dann auf der anderen Seite gewisse intellektuelle Fähigkeiten mit sich bringt. Das sind dann aber eher Sonderthemen. Also ein relativ buntes Sammelsurium. Was ich nicht mehr habe ist Personal Verantwortung. Nicht weil ich nicht mit Menschen umgehen kann, sondern sind beispielsweise die sehr verwalterischen Aspekte nicht so angenehm, die nun mal in dieser Rolle mit drinstecken. Ich übernehme auch noch Delivery Management. Das heißt für große Kunden, gibt es eine spezielle Schnittstellen Rolle, die im Prinzip die adesso Organisation vor dem Kunden Kapsel, weil der Kunde gar nicht im Detail wissen soll, wie wir aufgebaut sind. Der Kunde hat genau solche Bedarfsanfragen und sagt, ich brauche einen Senior Java Developer mit diesen Fähigkeiten und ich bin dann als Delivery Manager dafür zuständig.\\

0:6:1 --> 0:6:12\\
I:\\
Dann wäre jetzt meine Frage was die typischen Stakeholder bei einer Erstellung von einer Bedarfsmeldung sind und welche Rolle hast du dabei?\\

0:6:12 --> 0:10:12\\
B:\\
Also Bedarfsmeldungen sind ja ein sehr wichtiges Element in einem der adesso Kerngeschäftsprozesse. Wir sind ein IT-Dienstleister. Das bedeutet, wir entwickeln nicht selber etwas im Sinne von Produkten. Das ist etwas anderes und nennt sich Produkt Geschäft. Das heißt also, wenn wir etwas tun, Entwicklungstätigkeiten aufnehmen, beraten, brauchen wir und benötigen wir immer einen Auftrag. Also einen Kunden. Wenn wir Kundenaufträge haben, weil wir ja ein IT Dienstleister sind, heißt das unsere Kerngeschäftsobjekte sind die Mitarbeitenden, die mit unterschiedlichen Qualifizierungen daneben in genau diesen Kundenaufträgen tätig sind. Das können vom Kunden durchgeführte Projekte sein. Ein Kundenauftrag kann aber auch sein, dass adesso ein Projekt für den Kunden durchführt. Wir müssen wie gesagt diese Projektstellen gezielt besetzen. Das heißt, wir haben immer eine Projektorganisation, mal ist es auch eine Programmorganisation, wo diese Stellen mit entsprechenden rollen oder Stellenanforderungen beschrieben werden. Entweder kommen die vom Kunden direkt, oder wir formulieren die selbst. Diese Beschreibung der Anforderungen, also welche Fähigkeiten und welche Erfahrung und in welchem Umfang Menschen diese mitbringen müssen, damit sie genau diese Projektstelle besetzen können und damit eine ganz bestimmte Rolle in einem Projekt Kontext einnehmen oder in einem Programm Kontext… Diese Beschreibung ist es, was wir als Bedarfsmeldung bezeichnen. Die Quellen dafür sind entweder Kundenorganisationen, dass die dann durch große Accounts wie e-on, die Deutsche Bahn, die ihre ganz eigenen Systeme haben und dann in einer, nicht normierten, aber in einer semi strukturierten Form diese Anforderungen dokumentiert werden. Die gehen dann mehr oder weniger 1 zu 1 an uns über und wir müssen damit weiterarbeiten. Bis dahin, dass wir als Delivery Manager, als Projektleiter, als Programm Manager, als Account Manager oder Vertriebler mit Kunden sprechen, in Projekte reinschauen, Projektorganisationen definieren und selbst in diesen Rollen eine Bedarfsmeldung erfassen. Es geht allerdings jedes Mal darum, innerhalb eines Projektes oder Programms eine bestimmte Stelle zu besetzen.\\

0:10:12 --> 0:10:20\\
I:\\
Welche Art von Projekten hat adesso typischerweise? Hast du da vielleicht ein paar Beispiele?\\

0:10:26 --> 0:18:3\\
B:\\
Die erstmal abstrakteste und wichtige oder grundsätzliche Unterscheidung ist: Es gibt Projekte, die ein Kunde eine Kundenorganisation aufsetzt und durchführt, an denen wir uns dann beteiligen, indem wir beispielsweise in bestimmte Rollen an bestimmte Stellen dort Menschen reinbringen. Das heißt, wir arbeiten dann aber in einem extern definierten und normalerweise auch gesteuerten und kontrollierten Projektkontext. Das andere ist, wenn wir im Kundenauftrag Projekte aufsetzen. Da könnte man dann auch nochmal differenzieren. Einmal Projekte unter unserer Kontrolle im Kundenauftrag und dann gibt es natürlich auch interne Projekte. Da kann man aber relativ schnell sagen, dann ist es halt eine interne Stakeholder Position. Zum Beispiel kann die HR-Abteilung auch sowas beauftragen. Das unterscheidet sich dann nicht wesentlich. Was allerdings da dann besser in unserer Kontrolle liegt, ist in der Regel, wenn wir selbst die Projekte in der Organisation in der Durchführung verantworten. Dann haben wir auch die Ausprägung der Projekt Organisation, das Vorgehensmodell, usw. in der Hand. Da haben wir sehr viel mehr Flexibilität, was eben die Definition von Rollen, Anforderungen usw. angeht. Das ist so der wesentliche Unterschied in Bezug auf externe Projekte und von uns durchgeführte Projekte. Es gibt dann aber nochmal einen wesentlichen Unterschied in Bezug auf die Vergütung. Wir unterscheiden da grundsätzlich zwischen Time und Material Aufträgen oder Beauftragung und Festpreis Beauftragung. Festpreis bedeutet, dass es von unserer Seite eine bestimmte, klar bemessene und abgegrenzte Leistung angeboten wird und die Auftraggeber Seite verhandelt mit uns dafür einen festgesetzten Preis. Damit muss man bei der Durchführung beachten, dass man nicht so lange vor sich hinarbeiten kann, bis dann zum Beispiel die Auftraggeberseite sagt OK, wir sind jetzt zufrieden. Sondern man hat halt ein beschränktes Budget und in der Regel auch beschränkte Zeit und trägt damit auch ein größeres Risiko. Dann gibt es da entsprechend die timen Material oder t und m Kontexte. Da trägt die Auftraggeberseite in der Regel das größere Risiko, weil wir abstrakt bezeichnet, erstmal nur dazu verpflichtet sind, nach durchschnittlicher Qualität und Güte solche einzelnen Leistungen über zum Beispiel Mitarbeitende beizusteuern, zu erbringen. Trotzdem hat auch da die Erfahrung gezeigt: Wir sind immer an unserer Kunden Zufriedenheit oder langfristigen Kundenbindung interessiert, dass es uns auch gar nicht hilft, wenn wir in einem t und m Kontext nur durchschnittlich oder vielleicht auch mal schlechte Arbeit leisten und dann hinterher sagen, das war jetzt aber gar nicht unsere Verantwortung. Das ist euer Risiko gewesen. Damit kommen wir auch nicht durch. Und es ist in den seltensten Fällen auch so, dass ein Kunde unerschöpfliche Geldmittel hat. Selbst wenn dieser Kunde diese unerschöpflichen Geldmittel hätte, sagen sie aus rein wirtschaftlichen Aspekten natürlich auch zu irgendeinem Zeitpunkt das reicht jetzt, wir möchten nicht noch mehr Geld ausgeben. Das ist doch mal eine grundsätzliche Unterscheidung, was das Bezahlen angeht. Mit hinein spielt auch noch eine Unterscheidung in die Art der Projekte. Das ist nämlich einmal ein Gewerk, wo wir Gewährleistung übernehmen und das auch entsprechend kalkulieren müssen. Ganz häufig werden Gewerke in Kombination mit Festpreisen angeboten und durchgeführt. Interessanterweise müssen sie es aber gar nicht zwingend. Im Umkehrschluss meistens, wenn ich nach t und m arbeite, handelt es sich auch dann von unserer Leistung um Dienstleistung. Das sind aber teilweise, wenn man da wirklich genau draufschaut oder wenn man da juristisch drauf schauen würde feine Unterschiede. Wenn man beispielsweise bei der Beauftragung oder in der Kommunikation ganz bestimmte Begriffe verwendet und es würde irgendwann vor Gericht landen, egal ob man einen Dienstleistungsvertrag abgeschlossen hat und die ganze Zeit meinte auch als Dienstleistungen zu arbeiten, könnte zum Beispiel ein Gericht aufgrund von Formulierungen usw. hinterher feststellen, dass es sich doch um ein Gewerk gehandelt hat. Das sind dann teilweise eher juristische Unterschiede. Am Ende des Tages bedeutet Gewerk natürlich, dass wir an einem Stück Software gearbeitet haben und eben für die konsistente Fehler freie gesamte Funktionalität dann entsprechende Gewährleistung anbieten und übernehmen. Also damit der Kunde für das Geld, dass die Organisation gezahlt hat, eine einsetzbare Lösung bekommt. Wohingegen bei Dienstleistung eine gar nicht funktionale Lösung bei rauskommen muss. Beispiel typischer und klassischer Bereich für Dienstleistungsgeschäft ist ja Consulting. Consulting bedeutet qualifizierte, erfahrene Menschen zu bestimmten Themen kommunizieren, was ausarbeiten, etwas aufschreiben, können etwas spezifizieren können. Aber letztendlich ist da die menschliche Arbeit beziehungsweise der Erkenntnisgewinn im Zentrum und es wird keine Lösung in dem Sinne geschaffen und bereitgestellt. Dass sind die für mich zumindest relevanten Unterscheidung. \\

0:18:3 --> 0:18:16\\
I:\\
Wie werden dann Bedarfsmeldungen und die Anforderungen typischerweise bei Adesso kommuniziert und auch dokumentiert. Gibt es einen Ablauf, wie das genau gehandhabt wird?\\

0:18:16 --> 0:24:4\\
B:\\
Für den Staffing-Prozess an sich gibt es einen definierten Ablauf, der aber in Großteilen aus manueller Arbeit und manuellem Arbeitseinsatz besteht. Es gibt keine normierte Form einer Bedarfsmeldung. Es hat sich allerdings zumindest eine grobe thematische Struktur etabliert. Das bedeutet in der Regel hat man so etwas wie eine Überschrift und Bezeichnung. Dann hat man in der Regel einen Bereich, der den Einsatz Kontext ein wenig allgemeiner beschreibt. Dann hat man einen Bereich, der auf die individuell geforderten Fähigkeiten und Erfahrungen eingeht. Man hat normalerweise eine Gewichtung dieser Skills. Das bedeutet in Bezug auf Expertise, die da erwartet wird oder mal solche Dinge wie Primary in Secondary usw. Unterteilungen. Wir haben in diesen Bedarfsmeldungen dann aber auch wirtschaftlich relevante Informationen damit verknüpft. Das ist typisch für ein Dienstleistungsunternehmen. Das bedeutet uns interessiert dann die vertraglichen Konditionen im Sinne von Tagessatz, was für eine Organisation für adesso, im Sinne von Dienstleistung auch absolut relevant ist. Das sind dann immer diese Parameter. Ab wann ist der Einsatz gewünscht? Und für wie lange? Weil wir immer prüfen müssen, selbst wenn wir beispielsweise sehr gut fachlich passende Mitarbeitende finden, sind sie aber eventuell schon in anderen Projekten eingesetzt und können dementsprechend gar nicht dort leisten. Es mag dann je nach Kundenkontext noch weitere Informationsblöcke geben. Wir arbeiten mittlerweile auch mit einem Leveling-System, weil man sich als Organisation oder eben als Mensch innerhalb einer Organisation ein bisschen besser in Abfragen organisieren kann, wenn an solchen detaillierteren Bedarfsmeldungen gewisse Labels dran stehen wie beispielsweise Azure als Technologie oder oder Java. Dann kann ich eine potenziell größere Menge von Bedarfsmeldungen, die ich manuell durchsuche besser filtern und einschränken. Was in Bezug auf unserem Staffing-Prozess sehr relevant ist, ist beispielsweise dann der Status einer solchen Bedarfsmeldung, weil die Organisationen nicht alle Bedarfsmeldungen interessiert. Manche Bedarfsmeldungen sind schon verarbeitet und erfüllt. Manche sind nur dokumentiert und sollen aber gar nicht weiter beachtet werden und erst wenn sie beispielsweise bei uns eskaliert werden, sollte eigentlich die relevante Organisation darauf schauen. Wir haben eine vertikale Einteilung des Unternehmens in Branchen, bzw. technologisch getriebenen Organisationseinheiten. Das sind unsere sogenannten Line of Business. Mittlerweile sind das sogar Business Areas und es ist an der Stelle durchaus relevant, welche adesso Organisationseinheit für so eine Business Area oder Line of Business zuständig ist. Wenn wir an der Stelle mit einem gewissen Vorkaufsrecht feststellen, dass wir es nicht bedienen können, werden natürlich alle gefragt und es dürfen beispielsweise aus der Line of Business Motiv selbstverständlich auch mitarbeiten in Projekten der Line Cross Industries und umgekehrt getätigt werden. Aber umso mehr ist es wichtig, dass dann die jeweils geforderten Skills und Erfahrungen und wirtschaftlichen Konditionen möglichst präzise beschrieben werden, damit diese Fragen die aufkommen nicht immer wieder gestellt werden und immer wieder beantwortet werden müssen. Das ist aber, wie gesagt schon Kern Staffing-Prozess.\\

0:24:4 --> 0:24:22\\
I:\\
Du hast auch schon einige Punkte in Richtung Verfügbarkeit genannt, dass das sehr wichtige Aspekte sind. Gibt es besonders wichtiger oder auch unverzichtbare Punkte in einer Bedarfsmeldung, die in jeder drin sein sollte?\\

0:24:22 --> 0:28:46\\
B:\\
Das kommt auf die Perspektive an. Wenn ich jetzt eine rein fachliche Perspektive einnehme, ist es da unverzichtbar, dass aufgelistet oder benannt wird, welche Fähigkeiten oder Skills konkret mitgebracht werden müssen und idealerweise auch in welcher Erfahrung und Güte das ist. Aus der Perspektive zum Beispiel des Dienstleistungsunternehmen adesso ist relevant, dass ein Beginn des Einsatzes, ein voraussichtlicher Einsatzzeitraum, ein Tagessatz dran steht. Dass da dran steht, ob Freelancer, ob Smartphone, smartshore Fähigkeit gegeben ist oder near Shore, oder wie die sprachliche Ausrichtung ist. Also ob das zum Beispiel deutschsprachig ist, ob Englisch. Entsprechend zulässige Kommunikationsmittel sind wichtige Zusatzinformationen, die genau darauf abzielen, dass alle Menschen, die potenzielle Kandidaten/Kandidatinnen auf so eine Bedarfsmeldung gefiltert werden. Wenn es nämlich auf der Ebene, dass die Verfügbarkeit aber auch darüberhinausgehend nicht passt oder wenn es zum Beispiel von einem Tagessatz, aus welchen Gründen auch immer sehr unattraktiv ist, es dazu führen kann, dass dann die Organisation bestimmte Personen nicht anbietet. Das ist aber wie gesagt eine Perspektive, die durch die Natur von adesso als Dienstleister, und wir sind ein auslastungsgetriebenes unternehmen, was halt auch versucht wirtschaftlich zu arbeiten, darüber reinkommen. Idealerweise möchte zum gewünschten Einsatzbeginn auch dann wirklich eine Person nicht nur theoretisch benannt, sondern idealerweise interviewt, geprüft, eingeführt und wie auch immer wurde und dann wirklich los laufen kann.\\

0:28:46 --> 0:28:51\\
I:\\
Gibt es Herausforderungen und Schwierigkeiten bei unklaren oder unvollständigen Bedarfsmeldungen? Hast du da Erfahrungen machen können?\\

0:28:51 --> 0:39:18\\
B:\\
Ja. Bei einem Auftraggeber handelt es sich in der Regel um Branchen, die eben nicht als Kerngeschäft IT Software Entwicklung betreiben. Dementsprechend fällt es ihnen teilweise schwer, präzise Bedarfsmeldungen zu formulieren, die inhaltlich genau das transportieren oder beinhalten oder umfassen, was sie eigentlich an Fähigkeiten und Erfahrung benötigen. Das liegt einfach daran, dass sie in Teilen nicht für jeden Kunden gilt, aber in Teilen den Prinzip Menschen etwas beschreiben lassen, die davon nicht wirklich Ahnung haben. Das wiederum führt dazu oder kann dazu führen, dass der Kunde etwas anfordert, was der Kundenorganisation, dem Projekt oder wie auch immer im schlimmsten Falle sogar gar nicht hilft. Oder aber das ist eine andere Ausprägung, dass dann Kombinationen von Erfahrung und Fähigkeiten gesucht werden, die es in der realen Welt einfach so nicht gibt. Das ist die berühmte eierlegende Wollmilchsau, wo man drauf schaut und sagt, diese Menschen hätten wir auch gerne als Mitarbeitende. Vielleicht gibt es auf diesem Planeten auch eine Handvoll davon, aber das ist unrealistisch. Und das kann wie gesagt darin begründet sein, dass auf der auftraggebenden Seite jetzt Menschen damit beauftragt werden, solche Bedarfsmeldungen zu formulieren, die das nicht wirklich können. Eigentlich haben wir an der Stelle schon immer einen idealerweise von uns moderierten Prozess. Deswegen macht zum Beispiel auch die Rolle Delivery Management Sinn. Das hängt wie gesagt auch sehr mit der insgesamten Qualifikation oder Qualität von der Auftraggeberseite in diesem Kontext zusammen. Bei manchen Kundenorganisationen ist das trotzdem sehr gut eingespielt und etabliert und funktioniert auch so. Bei manchen muss man da früh ansetzen und sagen wir sprechen miteinander, und ich arbeite dann beispielsweise in diesem Gespräch heraus, was der Kunde wirklich benötigt, was abgrenzbar eine sinnvolle Bedarfsmeldung ist, was es für eine Stelle beinhaltet und womit wir dann weiterarbeiten können. Das ist oder kann ein Problem sein, wenn wir nicht entsprechend damit umgehen. Nicht nur die fachliche Qualität oder auch die die Passgenauigkeit, die wir dort anbieten können ist da wichtig, sondern auch vor allem die Schnelligkeit vom Staffing-Prozess. Bedeutet, wenn ein Unternehmen sich so aufgestellt hat, dass eine Bedarfsmeldung, die reinkommt innerhalb von sehr kurzer Zeit bearbeitet wird, sagen wir mal 2 Stunden, dann habe ich einen klaren Vorteil gegenüber zum Beispiel einem Anbieter, der dafür 2 Tage braucht oder eine Woche oder 2 Wochen. Denn für die Auftraggebende Seite ist klar, wenn ich dann Rückmeldungen bekomme und ich schaue da rein und diese Rückmeldungen sind plausibel… Was hält mich davon ab, dann zu sagen ich beauftrage ihn jetzt. Es muss nicht perfekt sein, aber wenn es plausibel ist und die Konditionen sind gut, dann ist der Auftrag ausgesprochen und die anderen gehen logischerweise leer aus. Wenn wir eben entsprechend in unserem Matching entweder nicht gut arbeiten oder überfordert sind und beispielsweise einen Software Architekt für Java mit bestimmten weiteren Anforderungen gefordert ist und wir bieten da ein Profil drauf an, wo man darauf schaut und feststellt, dass es ein .Net Software Developer ist, ist es bei diesem extremen Beispiel so, dass es glücklicherweise auch sofort auffällt. Aber das würde natürlich dann zu Irritationen auf Kundenseite führen. Bedeutet: Wir sollten nicht nur dafür sorgen, dass die Erwartungen oder Anforderungen möglichst gut an der Realität sind, sondern wenn wir dann wiederum auch da etwas matchen und einreichen, dass das dann auch diesen Anforderungen nach Möglichkeit entspricht und, dass innerhalb von möglichst kurzer Zeit so geschieht. Und die unglücklichste Variante ist, wenn man ein Prozess durchläuft und man bietet da jemanden an und der wird sogar genommen und wird eingearbeitet und dann stellt man irgendwie fest der macht irgendwie Unsinn. Man hat nie eine Garantie. Letztendlich sind es Menschen die dort arbeiten.\\

0:39:18 --> 0:39:22\\
I:\\
Welche Auswirkungen haben unklare oder auch fehlende Informationen in Bedarfsmeldungen jetzt aber konkret in Bezug auf die Effizienz und den Erfolg von Projekten.\\

0:39:27 --> 0:40:49\\
B:\\
Im Idealfall wird, möglichst früh erkannt, dass eine Bedarfsmeldung lückenhaft, unpräzise wie auch immer formuliert ist. Dann muss nachgefragt werden. Ich schaue mir etwas an, versuche zu verstehen, was die andere Seite sucht und wenn das für mich dann nicht konsistent auf die zumindest für mich bekannten Rollenlösung ist, dann muss ich nachfragen. Alles andere ist eine Interpretation. Dann läuft man mit sehr großer Wahrscheinlichkeit in die von mir gerade beschriebenen Probleme rein. Die schlechteste Variante ist, dass ich sage ich nehme die Informationen, die ich jetzt da vorgelegt bekommen habe, interpretieren sie nach besten Wissen und Gewissen und dann ist es aber mehr oder weniger ein Glücksspiel. Das heißt, wenn ich dann Profile finde, könnten sie immer noch zufällig das sein, was der Kunde eigentlich wollte und gesucht hat.\\

0:40:49 --> 0:41:1\\
I:\\
Die letzte Frage hast du im Grunde auch schon mit beantwortet. Wie könnte man sicherstellen, dass Bedürfnisse und Anforderungen aller relevanten Stakeholder in einer Bedarfsmeldung Berücksichtigt werden?\\

0:41:4 --> 0:42:26\\
B:\\
Zwei Möglichkeiten. Man könnte, da glaube ich aber nicht dran, natürlich den Prozess standardisieren und stark formalisieren. Also das im Prinzip von einer öffentlichen Stelle aus gesagt wird: Alle Dienstleister und Auftraggeber dieser Welt wenn ihr in dieser Art Geschäft betreiben wollt, müsst ihr so ein Format einreichen. Also Bürokratie pur. Das würde uns nichts verbessern, aber das ist eine Möglichkeit. Die andere Möglichkeit ist meines Erachtens, dass dann für genau solche Prozesse entsprechend versierte Menschen diesen Prozess, das heißt die Anforderungserhebung, die Dokumentation, das erfüllen diese Anforderungen komplett begleiten und moderieren. Das ist Delivery Management. Das ist Business Development. Manchmal auch Account Management.\\

0:42:26 --> 0:42:30\\
I:\\
Dann sind wir eigentlich schon durch mit den Fragen. Hast du noch zu irgendeinem Punkt irgendwelche Fragen oder irgendwas, was vielleicht noch für mich in dem Themenbereich interessant sein könnte?\\

0:42:30 --> 0:45:6\\
B:\\
Die größte Schwierigkeit liegt darin, dass wir Informationen über eine natürliche Sprache transportieren. Das ist zwar flexibel, weil die Sprache an der Stelle ja eben nicht formalisiert ist. Was aber immer das Risiko mit sich bringt, dass etwas nicht präzise beschrieben, abgegrenzt oder interpretierbar wird und genau bei solchen Prozessen, wo wir eigentlich präzise arbeiten wollen, haben wir genau diese große Herausforderung, dass die bisher benutzte Art, um diese Informationen zu ermitteln und die gerade wieder zu lesen zu interpretieren eben ein Stück weit ungenügende Mittel, nämlich den natürlichen sprachigen Raum verwendet. Da ist zwar mit gesundem Menschenverstand gearbeitet worden. Das heißt, es haben sich Semistrukturen gebildet. Aber es ist wirklich sehr individuell unterschiedlich in was für einer Qualität oder was für einer Realitätsnähe solche Bedarfsmeldungen formuliert werden. Wenn wir auf unserer Seite jemanden sitzen haben, der oder die eben auch in diesem Umfeld relativ wenig Ahnung und Erfahrung hat, dann haben wir auch nochmal ein Risiko, dass selbst wenn die Bedarfsmeldungen präzise und realitätsnah formuliert ist, bei uns in der Interpretation etwas schiefgeht. Das bedeutet, dass halt die Menschen, die auf unserer Seite Bedarfsmeldungen lesen und versuchen zu bedienen, leider auch hinreichend viel Erfahrung in der Projekt IT haben müssen.\\
\subsection{Interview 4}
\label{interview4}
%Herbermann, Sebastian\\
0:0:49 --> 0:1:5\\
I:\\
Erstmal vorweg zu deiner Person. Ich weiß, dass du CC Leiter bist und auch Softwarearchitekt bist. Kannst du vielleicht kurz erzählen, wie deinen Werdegang. Was du studiert hast und wie du dann bei adesso gelandet bist?\\

0:1:5 --> 0:1:38\\
B:\\
Studiert habe ich Kerninformatik an der TU Dortmund mit einem Diplomabschluss. Danach bin ich 2003 im Bereich der Software Entwicklung gestartet. Und habe dann seitdem Software Architekten und Führungskraft gemacht.\\

0:1:38 --> 0:1:39\\
I:\\
Was sind denn dann so die typischen Stakeholder bei der Erstellung von Bedarfsmeldungen und was für eine Rolle hast du dabei?\\

0:1:57 --> 0:2:0\\
B:\\
Stakeholder bei der Erstellung von Bedarfsmeldungen?\\

0:2:0 --> 0:2:4\\
I:\\
Wer sind so Person, die da dran grundsätzlich beteiligt sind.\\

0:2:11 --> 0:3:4\\
B:\\
Wenn wir Bedarfsmeldungen erstellen, sind wir über den Akquise Prozess schon hinaus. Das heißt, der Vertrieb ist raus. Damit ist für die Erstellung der Bedarfsmeldungen der Projektleiter und Maitre eigentlich relevant. Also hängt immer davon ab. Der Projektleiter hat ein Bedarf und stimmt den mit dem internen Maitre ab, wer dann für das Staffing verantwortlich ist. Das heißt, sie sind auch für die Bedarfserstellung verantwortlich. Daneben hast du dann noch Leute, die Input geben für die Bedarfsmeldung, also Input, der jetzt nicht zum Beispiel durch den Vertrag geregelt ist, wäre so etwas wie das Geld Profil, was dann vom Architekten zum Beispiel zugeliefert wird.\\

0:3:4 --> 0:3:14\\
I:\\
Was sind denn so typische Projekte, die bei Adesso angenommen und bearbeitet werden? Hast du ein paar Beispiele für Projekte, die so durchgeführt werden?\\

0:3:20 --> 0:3:27\\
B:\\
Typische Projekte alles rund um die Software Entwicklung, also sowohl in den Development als auch Consulting rollen.\\

0:3:34 --> 0:3:47\\
I:\\
Und wie werden Projektbedarfe und Anforderungen innerhalb von adesso kommuniziert und dokumentiert? Also gibt es da irgendwie einen Ablauf?\\

0:3:47 --> 0:3:55\\
B:\\
Ja, und die Bedarfsmeldungen werden über das Jira erfasst und entsprechend auch mit allen beteiligten Parteien geteilt.\\

0:3:55 --> 0:4:19\\
I:\\
Was sind Aspekte, die in einer Bedarfsmeldung besonders wichtig sind? Gibt es unverzichtbare Punkte, die immer in einer Bedarfsmeldung drin sein sollten.\\

0:4:19 --> 0:5:47\\
B:\\
Rahmenbedingungen wie Kunde, Einsatzbeginn, Laufzeit oder Ort und Tagessatz. Und die notwendigen Skills sowie eine Aufgabenbeschreibung damit klar ist, was tatsächlich inhaltlich zu tun ist und was vom Kunden oder in dem Fall für ein Profil gefordert wird. Indem Moment von die Aufgabenbeschreibung oder die geforderten Skills fehlen verstehe ich gar nicht, was gesucht wird und kann dementsprechend auch keine passenden Profile anbieten. Die anderen Rahmenbedingungen brauche ich natürlich, um auch zu prüfen, ob das Personal entsprechend überhaupt zum Beispiel verfügbar ist.\\

0:5:47 --> 0:5:53\\
I:\\
Welche Herausforderungen hat man denn bei unklaren oder unvollständigen Bedarfsmeldungen?\\

0:5:53 --> 0:6:2\\
B:\\
Eine Herausforderungen ist, dass das angebotene Personal gegebenenfalls nicht zu dem Einsatz passt oder bei dem Einsatz nicht verfügbar ist. Das heißt am Ende, dass die Rückmeldungen auf die Bedarfsmeldungen gar nicht zielführend ist.\\

0:6:17 --> 0:6:26\\
I:\\
Gibt es Kriterien oder Standards, um die Qualität der Bedarfsmeldungen sicher zu stellen?\\

0:6:26 --> 0:7:4\\
B:\\
Es gibt einerseits die technische Validierung über die Daten im Jira, die man eingibt. Weitergehende Validierung erfolgt dann, nicht immer aber meistens nochmal im 4 Augen Prinzip. Die Projektleitung meldet sich da zur Abstimmung nochmal.\\

0:7:4 --> 0:7:36\\
I:\\
Wie würdest du denn die Qualität und Klarheit von Bedarfsmeldungen verbessern? Was für Mechanismen würdest du denn anwenden, die eventuell auch noch gar nicht von jedem benutzt werden?\\

0:7:36 --> 0:8:41\\
B:\\
Ich glaube, man könnte einige Daten strukturierter erfassen. Bei der inhaltlichen Beschreibung ist es schwierig, das tatsächlich noch automatisiert weiter zu verbessern. Die Herausforderung ist jetzt nicht 15 Java Entwickler Profil zu finden, sondern die besonderen Profile, die wir selten haben, die dann aus der Struktur raus sein, wo du halt zum Beispiel nicht sagen kannst, du Hinterlegst die Skills in der Liste und sagt, ich brauche irgendwie Java. Die Herausforderungen ist natürlich alles, was aus dem Raster herausfällt.\\

0:8:41 --> 0:8:48\\
I:\\
Im Bezug auf den Projekterfolg hast du da schon Erfahrungen gemacht, welche Auswirkungen unklare und fehlende Informationen in Bedarfsmeldungen dann wirklich am Ende im Projekt haben?\\

0:8:48 --> 0:9:32\\
B:\\
Die Herausforderung ist, dass die Projektmitarbeitenden dann den Aufgaben im Zweifelsfall nicht gewachsen sind, weil notwendiges Know How in den Technologien fehlt. Weil andere Skills auch unternommen und auch Softskills eventuell fehlen. Und somit das Projekt gar nicht sicher, oder der Projekterfolg gar nicht sichergestellt werden kann.\\

0:9:32 --> 0:9:45\\
I:\\
Wie kannst du sicherstellen, dass die Bedürfnisse und Anforderungen aller involvierten Personen angemessen berücksichtigt werden?\\

0:9:56 --> 0:10:34\\
B:\\
Je nachdem in welche Rolle und in welchen Schritten im ganzen Prozess durch intensives Lesen und drauf schauen. Oder Rückfragen stellen. Rückfrage heißt immer das eine Informationen gefehlt hat.\\

0:10:34 -->0:11:26\\
I:\\
Ich hätte noch eine spezielle Frage an dich, weil du auch Software Architekt bist und mit vielen Technologien zu tun hast. Hättest du eine Idee, wie du grundsätzlich das Problem der Informationsgewinnung von unstrukturierten Bedarfsmeldungen angehen würdest. jetzt auch vielleicht in Bezug auf Technologien oder Ansätzen. Hast du da irgendwelche Erfahrungen irgendwann mal machen können?\\

0:11:26 --> 0:14:27\\
B:\\
Man könnte natürlich gerade in der schwierigen Erfassung von den Freitextinformationen, die nicht stark strukturiert sind im Zweifel weitere Validierungen bauen, die auch dann vielleicht über das hinausgehen, was uns Jira an Validierung anbietet. In den Freitextinformationen könnte ich mir vorstellen, dass man da KI gestützt Prüfungen macht, um eine Vollständigkeit sicherzustellen.\\
\subsection{Interview 5}
\label{interview5}
%Neubauer, Johannes\\
0:4:52 --> 0:4:58\\
I:\\
Wer sind denn die typischen Stakeholder bei der Erstellung von Bedarfsmeldungen und was für eine Rolle hast du dabei?\\

0:5:14 --> 0:5:19\\
B:\\
Bei adesso ist es im Normalfall so, dass Sales oder der interne Maitre dafür zuständig ist. Er stellt quasi eine Bedarfsmeldung im Normalfall auf Basis von Anforderungen direkt vom Kunden. Also entweder die kommen aus etwas, was wir angeboten haben, oder bei einer Rahmenvereinbarung kommen sie vielleicht aus einem Wettbewerb. Oder es kommt konkret im Projekt aus mündlich genannten Bedarfsmeldungen. Das kann auch durch ein Scrum Master oder Product Owner entsprechend entstehen. Da ist dann derjenige, der die Bedarfsmeldung erstellt auch zuständig, die Kriterien festzulegen. Was für ein Mitarbeiter oder Mitarbeiterin, mit welchen Fähigkeiten wir brauchen und stellt die dann ein. Ich bin Bereichsleitungsebene. Das heißt Ich bin einerseits in den Call, wo sich alle Bereichsleiter des gesamten Unternehmens treffen, um einmal zu schauen, welche Bedarfe eigentlich gerade eskaliert sind. Das heißt, welche sind gerade, oder müssen in naher Zukunft aufgelöst werden? Vieles wird einfach im direkten Kontakt behandelt. Jemand sieht das Ticket, bietet drauf und alles ist gut. Es gibt manche Bedarfsmeldungen, die schwerer zu bedienen sind. In solchen Fällen bin ich involviert. Und ansonsten bin ich eher in einer Support-Rolle in dem Prozess, weil ich entweder jemanden die Vertretung übernehme, weil ich im Normalfall nicht interner, sondern externer Maitre bin. Das heißt, ich bin eher für das Kunden Management und für die Gemeinsame Gestaltung, wie man zukünftig die Zusammenarbeit gestaltet zuständig, aber nicht so sehr jetzt für einzelne Bedarfsmeldungen. Ansonsten direkte Berührungspunkte mit konkreten Bedarfsmeldungen hab ich nicht.\\

0:10:59 --> 0:11:8\\
I:\\
Wie werden Bedarfsmeldungen und Anforderungen grundsätzlich kommuniziert und dokumentiert?\\

0:11:8 --> 0:14:11\\
B:\\
Von demjenigen der es einstellt, werden die Kriterien definiert und in einem JIRA-Ticket überführt. Das kann sehr unterschiedliche Formen annehmen.Es gibt ein paar Felder, die strukturiert sind. Z.B zu welchem Tagessatz das ganze angeboten wird, wann das Ganze startet, wie hoch das Volumen also an Tagen ist. Es gibt einige Freitextfelder, bei dem drinsteht, was die Aufgaben usw. sind. Zum Beispiel bei einer Rahmenvereinbarung, die wir gerade machen gibt es dann ein Excel, was ausgefüllt werden muss mit seiner Selbstbeurteilung. Wie gut der Kandidat, der jetzt gefunden wird oder gefunden werden soll, darauf passt. So ein Bewertungsschema wird dann mit beigefügt, das die Führungskraft ausfüllen kann. In einem anderen Fall kann das aber ganz anders aussehen, da wird dann einfach der Export aus unserem Profiler, wo also die Profilbeschreibung drinsteht, einfach als Kommentar in das JIRA hinzugefügt und gesagt: Den könnte man jetzt anbieten und das ist eher ein manueller Prozess. Das ist sehr verschieden. Deswegen nutzen wir doch JIRA, weil es im Normalfall nicht genau den einen Case gibt, wie wir Bedarfsmeldung reinkriegen und wie wir diese auch beantworten müssen. Im Normalfall ist es eben so, dass die Bedarfsanfragen in die Welt rausgeschickt werden. Also erstmal in die eigene Organisationseinheit und wenn man dort niemanden hat, der sehr gut passt, dann eben in die gesamte adesso und die Führungskräfte gucken von der anderen Seite da drauf und schauen, ob der eigene Mitarbeiter darauf passt. Im Normalfall ist es nicht so, dass diese Bedarfsmeldungen eine konkrete Suchanfrage sind, sondern die sind eher eine Bedarfsmeldung, wo drauf sich dann jemand melden kann.\\

0:14:11 --> 0:14:21\\
I:\\
Was sind denn besonders wichtige oder auch unverzichtbare Informationen innerhalb einer Bedarfsmeldung?\\

0:14:21 --> 0:16:50\\
B:\\
Wichtig ist für die Führungskraft, die einen Mitarbeiter anbietet, ab wann die dann frei sind. Bis wann, wie lange wieviel Prozent dieser Mitarbeiter oder diese Mitarbeiterin eingesetzt wird. Für denjenigen, der den Bedarf meldet. Ist es am Interessantes oder am wichtigsten was die muss und soll Kriterien sind? Damit der Mitarbeiter möglichst gut Match. Man kann sagen, dass es für beide Seiten wichtig ist. Für die Führungskraft ist es erstmal interessant ob und für wie lange er Mitarbeiter in einem Projekt kriegt. Bekomme ich die passende Ressource auf meine Anfrage? Nachrangiger ist der Tagessatz. Das ist gesamtwirtschaftlich fürs Unternehmen sehr wichtig. Aber dadurch, dass wir eine sehr offene und transparente bereichsübergreifende Zusammenarbeit pflegen, ist so, dass wir, wenn einmal einen Preis, wie z.B. den Tagessatz, von einer Einheit verhandelt wurde, dass das dann von einer anderen Einheit nicht in Frage gestellt wird oder gesagt wird jetzt können wir hier den Mitarbeiter nicht anbieten, weil der Tagessatz zu niedrig ist. Das ist dann so verhandelt und wir sind dann ein Commitment als Firma eingegangen. Wir haben uns selber als Ziel gesetzt, eben hohe Qualität abzuliefern und Kunden zu Partnern zu machen und deswegen bedienen wir dann diese Anfragen.\\

0:16:50 --> 0:17:2\\
I:\\
Welche Herausforderungen oder Schwierigkeiten sind bei unklaren oder auch unvollständigen Bedarfsmeldungen aufgetreten? Hast du da irgendwie Erfahrungen machen können?\\

0:17:2 --> 0:19:17\\
B:\\
Wenn die Bedarfsmeldungen zu wenig Informationen darüber enthalten, was genau die Mitarbeiter, die da drauf angeboten werden sollen, können sollen, wo die vielleicht auch sitzen sollen, idealerweise, ob das ein Angebot einer einzelnen Arbeitskraft ist oder ob das im Rahmen eines Teams ist. Zu schauen ob der Mitarbeiter oder die Mitarbeiterin auf diese Anfrage passt. Wenn ich jetzt einen Junior Mitarbeiter habe, der ganz frisch bei Adesso angefangen hat, dann sollte ich ihn vielleicht nicht auf ein Ticket anbieten, wo diese Person dann ganz alleine zum Kunden geschickt wird. Das wäre unfair der Person gegenüber und wahrscheinlich auch nicht von Erfolg gekrönt. Wenn auf irgendeiner Ebene was fehlt, führt es dazu, dass entweder falsche Zuordnung gemacht werden, oder dass es einen Kommunikationsoverhead gibt. Oder, dass Versprechungen gemacht werden, die dann hinterher nochmal korrigiert werden müssen, weil es dann doch nicht gut passt. Das heißt es ist schon essentiell, dass die Bedarfsmeldungen gut sind, weil nur dann auch die erste Wahl auch eine gute Wahl ist. Wir haben und deswegen in beide Richtungen Schwierigkeiten. wir haben aber teilweise, dass Mitarbeiter angeboten werden, die nicht so gut passen, wo dann die Informationslage auf der anderen Seite schwierig ist. Das ist dann unter Umständen ein großer Overhead und weil man dann auf der Seite sehr viel Aufwand und unter Umständen durch Interviews oder ähnliches betreiben muss. Was dann auch sehr zeitaufwendig sein kann.\\

0:19:17 --> 0:19:21\\
I:\\
Wie wird denn die Qualität von Bedarfsmeldungen bewertet? Gibt es irgendwelche Kriterien oder Standards die ihr einhaltet?\\

0:19:21 --> 0:24:2\\
B:\\
Es gibt, wie gesagt ein paar feste Felder, die auch in dem JIRA eben festgelegt sind, auch ein paar feste Zustände und Übergänge, die einen groben Rahmen geben. Dann gibt es bei größeren Rahmenvereinbarungen feste Rahmenbedingungen, in denen gesagt wird, so übersetzen wir die Anforderungen des Kunden in eine Bedarfsmeldung. Aber es ist nicht so, dass es eine übergreifende Qualitätssicherung über die Tickets gibt, sondern das ist eher ein Lernprozess. Leute die Tickets erzeugen machen, das entweder schon eine Weile oder es wird dort irgendjemanden zur Seite gestellt, der das schon länger macht. Die muss Kriterien was sind die? Soll Kriterien, dass man dabei schreibt. Nicht nur, dass jemand ein Skill hat, sondern auch wie viel Erfahrung er oder sie in dem Skill. Und noch ein paar weitere Punkte, die die dann ja dazu führen, dass die Ticket Qualität hoffentlich gut ist. Aber sie ist auch nicht durchweg gut, das muss man auch dazu sagen. Es gibt sehr gute Tickets und es gibt welche, die nicht ganz so gut sind. Wobei man bei manchen auch fairerweise sagen muss, dass die unter Umständen dann auch aus Rahmenvereinbarungen kommen, wo wir eigentlich nur Anfragen durch Reichen und auch nur teilweise ein zwei Tage Zeit ist, um überhaupt eine Anfrage zu machen und uns nicht sehr viel mehr übrigbleibt als nahezu alles was wir von außen reinbekommen dann 1 zu 1 zu übernehmen. Weil dort nicht genug Raum ist, um einen richtigen Angebotsprozess zu machen. Bei großen Rahmenvereinbarungen gibt es dann auch nochmal auf JIRA Tools, die dabei unterstützen diese Bedarfsmeldungen zu erzeugen, zu tracken bis hin zu einem Angebot zu bringen.\\

0:24:2 --> 0:24:19\\
I:\\
Wie kannst du denn sicherstellen, dass die Bedürfnisse und Anforderungen aller involvierten Leute in Bedarfs Bildung berücksichtigt werden?\\

0:24:19 --> 0:25:43\\
B:\\
Ich glaub, es gibt nicht die eine Person, die das sicherstellen kann. Das ist eine gemeinsame Verantwortung. Damit dann am Ende sichergestellt ist, dass das gut funktioniert, gibt es sicherlich ein paar Punkte, wo man positiv darauf einwirken kann. Wir machen regelmäßig Bereichsmeetings, wo dann auch ein Agenda Punkt im Normalfall ist wie das Staffing läuft. Wir haben auch wöchentlich Jour Fixe, wo wir dann über das Tagesgeschäft reden und was die aktuell laufende Staffing-Prozesse sind? Wie sieht die Auslastung aus usw.? Dort kann man schauen wo denn Probleme sind und dann entsprechend kommunizieren. Viele von den Bedarfen werden gar nicht im JIRA eingetragen, sondern es werden direkt entsprechende Mitarbeiter angesprochen.\\
\newpage
\section{Paraphrasierung und Reduktion}
\label{sec:reduktion}
%\begin{center}
\begin{longtable}{| p{0.5cm} | p{0.5cm} | p{4cm} | p{4cm} | p{4cm} |}
	%{
	%	| >{\hsize=.2\hsize}X 
	%	| >{\hsize=.2\hsize}X
	%	| >{\raggedright\arraybackslash}X
	%	| >{\raggedright\arraybackslash}X
	%	| >{\raggedright\arraybackslash}X | }
	\hline
	Fall & Nr. & Paraphrase & Generalisierung & Reduktion \\
	\hline
	\hline
	\endhead
	%\sout{test generalisierung}
	1 & 1 & Bedarfsmeldungen beschreiben Software-Entwicklungsprojekte, angefangen von Projekten in dem ein adessi in einem Kundenprojekt arbeitet über gemischte Teams bis hin zur kompletten Lieferung. & Bedarfsmeldungen Beschreiben Software-Entwicklungsprojekte wie Kundenprojekte, bei dem in gemischten Teams bis zur Auslieferung gearbeitet wird & \multirow{3}{4cm}{K1 Arten von Projekten \\ -tttttt \\ -ttttttt \\ -t \\ \
		
		K2 Kommunikationswege \\ -tttttt \\ -ttttttt \\ \
		
		K3 Wichtige Informationen \\ -tttttt \\ -ttttttt \\ \
		
		K4 Detaillierungsgrad \\ -tttttt \\ -ttttttt \\ \
		
		K5 Herausforderungen \\ -tttttt \\ -ttttttt \\ \
		
		K6 Qualitätsbewertung \\ -tttttt \\ -ttttttt \\ \
		
		K7 Qualitätsverbesserung \\ -tttttt \\ -ttttttt \\ \
		
		K8 Auswirkungen \\ -tttttt \\ -ttttttt \\} \\
	\cline{1-4}
	1 & 2 & Initial über den Maitre, der das Staffing übernimmt, teilweise auch über das eigene Netzwerk zwischen Führungskräften. Am Ende über das Staffing Jira. & Maitre, Führungskräftenetzwerk, am ende Staffing Jira & \\
	\cline{1-4}
	1 & 3 & Senioritätslevel, Tagessatz, Remote/on Site Einsatz, Dauer, Technischer Stack (Muss- und Kann Kriterien), Einarbeitungszeiträume, Lieferverpflichtung. & Senioritätslevel, Tagessatz, Remote/on Site Einsatz, Dauer, Technischer Stack (Muss- und Kann Kriterien), Einarbeitungszeiträume, Lieferverpflichtung & \\
	\cline{1-4}
	1 & 4 & Bedarfsmeldungen sollten aussagefähig sein, welche technischen Kompetenzen wichtig sind, welche Tagessätze, ob Remote möglich ist und die Dauer. & Aussagefähigkeit von Bedarfsmeldungen durch technischen Kompetenzen, Tagessätze, Remote Möglichkeit und Dauer & \\ 
	\cline{1-4}
	1 & 5 & Der Anforderer muss ggf. Fragen mehrfach beantworten. Dadurch verliert man ggf. den Überblick. & Überblick verlieren durch Mehrfachbeantwortung von Fragen & \\ 
	\cline{1-4}
	1 & 6 & Typische Stakeholder sind Sales/PL: Anforderer mit den technischen Informationen, Maitre: kümmert sich um das Staffing. & Stakeholder sind Sales/PL: Anforderer mit technischen Informationen, Maitre & \\ 
	\cline{1-4}
	1 & 7 & Bewertung der Qualität erfolgt gar nicht. & keine Qualitätsbewertung & \\ 
	\cline{1-4}
	1 & 8 & Qualitätsverbesserung durch klarere Vorgaben und weniger Freitext, aktuell: durch Nachfragen und Bitten um nachträgliche Pflege, ggf. durch Reviewprozesse. & Qualitätsverbesserung durch klare Vorgaben, weniger Freitext, Nachfragen und nachträgliche Pflege (Reviewprozess) & \\ 
	\cline{1-4}
	1 & 9 & Auswirkungen fehlender Informationen sind, dass das Staffing länger dauert und ggf. die Stellen durch andere Dienstleister besetzt werden. & Fehlinformationsauswirkungen sind längerer Staffingprozess und Umbesetzung durch anderen Dienstleister & \\ 
	\cline{1-4}
	1 & 10 & Gute Abstimmungen bevor die Bedarfsmeldung erstellt wird, ggf. durch ein Quality-Gate (Review). & Abstimmung vor Bedarfsmeldungserstellung durch Quality-Gate (Review) & \\ 
	\cline{1-4}
	2 & 11 & In Bedarfsmeldungen sind die typischen Stakeholder Fachverantwortliche, Entscheider und höhere Management-Ebenen wie CEO oder Geschäftsführer. & Stakeholder sind Fachverantwortliche, Entscheider, CEO oder Geschäftsführer & \\ 
	\cline{1-4}
	2 & 12 & Projekte lassen sich in Time Material Projekte und Festpreis Projekte unterteilen, wobei die Art der Projekte die Ressourcenplanung beeinflusst. & Projektunterscheidung ind Time Material und Festpreis Projekte & \\
	\cline{1-4} 
	2 & 13 & Die Dokumentation der Bedarfsmeldungen variiert je nach Kundenanforderungen, von einfachen Word-Dokumenten bis zu detaillierten Präsentationen. & Dokumentation von Bedarfsmeldung variieren (von Word-Dokument bis Präsentation) & \\ 
	\cline{1-4}
	2 & 14 & Wichtig in Bedarfsmeldungen sind das Tech Stack, der Aufwand und die zeitlichen Erwartungen. & Wichtig sind Tech Stack, Aufwand und zeitliche Erwartungen & \\ 
	\cline{1-4}
	2 & 15 & Senioritätslevel ist zweitrangig und abhängig von der Projektsituation. & Senioritätslevel zweitrangig & \\ 
	\cline{1-4}
	2 & 16 & Unklare oder unvollständige Bedarfsmeldungen führen zu erhöhtem Aufwand und möglichen Verzögerungen, da Missverständnisse geklärt werden müssen. & Erhöhter Aufwand und Verzögerungen durch unklare/unvollständige Bedarfsmeldungen & \\ 
	\cline{1-4}
	2 & 17 & Die Qualität von Bedarfsmeldungen kann verbessert werden, indem man sicherstellt, dass das Verständnis auf allen Seiten übereinstimmt und Missverständnisse vermieden werden. & Qualitätssteigerung von Bedarfsmeldungen durch Verständnisübereinstimmung und Beseitigung von Missverständnisse & \\ 
	\cline{1-4}
	2 & 18 & Unklare Informationen in Bedarfsmeldungen erhöhen den Aufwand und führen zu Nacharbeit und potenziellen Konflikten zwischen den Beteiligten. & Unklare Informationen erhöhen Aufwand und resultieren in Nacharbeiten und Konflikten & \\ 
	\cline{1-4}
	2 & 19 & Erfahrung ist entscheidend, um sicherzustellen, dass die Bedürfnisse und Anforderungen aller Beteiligten angemessen berücksichtigt werden. & Erfahrung ist zur Sicherstellung von Bedürfnissen entscheidend & \\ 
	\cline{1-4}
	3 & 20 & Bedarfsmeldungen können entweder vom Kunden direkt kommen oder von uns selbst formuliert werden. & Bedarfsmeldungen können selbst formuliert oder vom Kunden stammen  & \\ 
	\cline{1-4}
	3 & 21 & Bedarfsmeldungen enthalten wirtschaftlich relevante Informationen wie vertragliche Konditionen, Tagessatz, Einsatzbeginn und voraussichtlicher Einsatzzeitraum. & Bedarfsmeldungsinfomationen sind vertragliche Konditionen, Tagessatz, Einsatzbeginn und Einsatzzeitraum & \\ 
	\cline{1-4}
	3 & 22 & Die typische Struktur einer Bedarfsmeldung beinhaltet eine Überschrift, eine allgemeine Beschreibung des Einsatzkontexts, eine Liste der geforderten Fähigkeiten und Erfahrungen sowie deren Gewichtung. & Bedarfsmeldungen enthalten eine Überschrift, eine Beschreibung des Einsatzkontexts, eine Liste der Fähigkeiten und Erfahrungen und deren Gewichtung & \\ 
	\cline{1-4}
	3 & 23 & Unklare oder unvollständige Bedarfsmeldungen können zu Missverständnissen führen und den Staffing-Prozess verlangsamen. & Unvollständige Bedarfsmeldungen führen zu Missverständnissen und langsamen Staffing-Prozess & \\ 
	\cline{1-4}
	3 & 24 & Projekte können entweder vom Kunden gesteuert werden oder von uns im Kundenauftrag durchgeführt werden. & Bedarfe sind entweder Kundengesteuert oder Kundenaufträge & \\ 
	\cline{1-4}
	3 & 25 & Die Art der Projektvergabe (Time and Material oder Festpreis) beeinflusst die Anforderungen und den Ablauf. & Beeinflussung der Anforderungen und Ablauf durch Art der Projektvergabe & \\ 
	\cline{1-4}
	3 & 26 & Der Staffing-Prozess bei adesso ist manuell und nicht normiert, aber es gibt eine grobe thematische Struktur, die sich etabliert hat. & Keine feste, aber grobe Struktur zur Erfassung des Staffing-Prozesses & \\ 
	\cline{1-4}
	3 & 27 & Die Schnelligkeit im Staffing-Prozess ist ein Wettbewerbsvorteil. Eine schnelle und präzise Bearbeitung von Bedarfsmeldungen ist entscheidend für den Erfolg. & schnelle und präzise Bearbeitung von Bedarfsmeldungen ist Erfolgsentscheidend & \\ 
	\cline{1-4}
	3 & 28 & Versierte Personen wie Delivery Manager oder Account Manager sollten den Prozess der Anforderungserhebung und Dokumentation begleiten und moderieren, um sicherzustellen, dass alle relevanten Informationen erfasst werden. & Begleitung des Bedarfsmeldungsdokumentierung durch Delivery Manager oder Account Manager & \\ 
	\cline{1-4}
	4 & 29 & Die Erstellung von Bedarfsmeldungen erfolgt durch Projektleiter und interne Maitre. Diese stimmen den Bedarf ab und sind für das Staffing verantwortlich. & Staffingverantwortliche sind Projektleiter und Maitre & \\ 
	\cline{1-4}
	4 & 30 & Rahmenbedingungen wie Kunde, Einsatzbeginn, Laufzeit, Ort und Tagessatz sowie notwendige Skills und Aufgabenbeschreibungen sind unverzichtbar in Bedarfsmeldungen. & Unverzichtbare Bedarfsmeldungsinformationen sind Rahmenbedingungen, Einsatzbeginn, Laufzeit, Ort und Tagessatz, Skills und Aufgabenbeschreibungen & \\ 
	\cline{1-4}
	4 & 31 & Bedarfsmeldungen werden über Jira erfasst und mit allen beteiligten Parteien geteilt. & Bedarfsmeldungserfassung in Jira & \\ 
	\cline{1-4}
	4 & 32 & Unklare oder unvollständige Bedarfsmeldungen führen dazu, dass das angebotene Personal möglicherweise nicht passt oder verfügbar ist. & Unvollständige/Unklare Bedarfsmeldungen führen zu unpassenden Personal & \\ 
	\cline{1-4}
	4 & 33 & Technische Validierung der Daten erfolgt über Jira, weitergehende Validierung im Vier-Augen-Prinzip durch die Projektleitung. & Qualitätssicherung erfolgt über Jira und der Projektleitung & \\ 
	\cline{1-4}
	4 & 34 & Verbesserungsvorschlag: Strukturiertere Erfassung der Daten und KI-gestützte Prüfungen für Freitextinformationen. & Qualitätssicherungsvorschlag durch strukturierte Datenerfassung und KI-gestützte Prüfungen & \\ 
	\cline{1-4}
	4 & 35 & Fehlende Informationen in Bedarfsmeldungen können dazu führen, dass Mitarbeitende den Aufgaben nicht gewachsen sind, was den Projekterfolg gefährdet. & fehlende Informationen führen zu fehlplatzierten Mitarbeitern und gefährdeten Projekterfolg & \\ 
	\cline{1-4}
	4 & 36 & Durch intensives Lesen, Rückfragen und die Berücksichtigung aller involvierten Personen werden Bedürfnisse und Anforderungen sichergestellt. & Sicherstellung der Anforderungen und Bedürfnisse durch intensives Lesen, Rückfragen und Berücksichtigung involvierter Personen & \\ 
	\cline{1-4}
	5 & 37 & Bedarfsmeldungen werden normalerweise von Sales oder dem internen Maitre erstellt, basierend auf Kundenanforderungen. & Bedarfsmeldungserstellung durch Maitre oder Sales basierend auf Kundenanforderungen. & \\ 
	\cline{1-4}
	5 & 38 & Der Ersteller der Bedarfsmeldung legt auch die Kriterien für den benötigten Mitarbeiter fest. & Festlegung der Kiterien durch den Ersteller der Bedarfsmeldung & \\ 
	\cline{1-4}
	5 & 39 & Die Bedarfsmeldungen werden in JIRA-Tickets überführt, die verschiedene Felder enthalten, wie Tagessatz, Startdatum, Volumen und Aufgaben. Es gibt auch Freitextfelder und Excel-Dokumente zur Selbstbeurteilung des Kandidaten. & Überführung der Bedarfsmeldung in Jira in die Felder Tagessatz, Startdatum, Volumen, Aufgaben, Freitext und Selbstbeurteilung & \\ 
	\cline{1-4}
	5 & 40 & Wichtige Informationen in einer Bedarfsmeldung sind das Startdatum, die Dauer und der Prozentsatz der Einsatzzeit des Mitarbeiters sowie die Muss- und Soll-Kriterien, um die Passgenauigkeit des Mitarbeiters sicherzustellen. & Wichtige Bedarfsmeldungsinformationen sind Startdatum, Prozentsatz, Einsatzzeit und Muss-/Soll- Kriterien & \\ 
	\cline{1-4}
	5 & 41 & Unklare oder unvollständige Bedarfsmeldungen führen zu falschen Zuordnungen, Kommunikationsaufwand und notwendigen Korrekturen. Dies kann auch zu einem erheblichen Overhead und Zeitaufwand durch Interviews führen. & Unklare/Unvollständige Bedarfsmeldungen führen zu falscher Zuordnung, Kommunikations-/Zeitaufwand und Korrekturen & \\ 
	\cline{1-4}
	5 & 42 & Es gibt feste Felder und Zustände in Jira, die einen groben Rahmen bieten, sowie feste Rahmenbedingungen bei größeren Vereinbarungen. Es gibt jedoch keine übergreifende Qualitätssicherung, sondern es handelt sich um einen Lernprozess. & Trotz fehlender Qualitätssicherung groben Rahmen durch feste Felder in Jira und Erfahrung & \\ 
	\cline{1-4}
	5 & 43 & Die Bedürfnisse und Anforderungen aller Beteiligten werden durch regelmäßige Meetings, wöchentliche Jour Fixe und direkte Kommunikation berücksichtigt. Viele Bedarfe werden direkt angesprochen und nicht im JIRA eingetragen. & Bedarfe durch regelmäßigen Meetings, Jour Fixe und Kommunikation besprochen und berücksichtigt & \\ 
	\hline
	\caption{Erstellung der Kategorien aus den Experteninterviews.\label{tab:kategorien}}\\
\end{longtable}
%\captionof{table}{Kategorien}
%\label{tab:Erstellung der Kategorien aus den Experteninterviews}
%\end{center}


%%\addcontentsline{toc}{section}{Eidesstattliche Erklärung}
\hiddenchapter{Eidesstattliche Erklärung}
Hiermit erkläre ich, dass ich die vorliegende Arbeit selbstständig
und ohne Benutzung anderer als der angegebenen Hilfsmittel angefertigt
sowie die aus fremden Quellen direkt oder indirekt übernommenen
Gedanken als solche kenntlich gemacht habe. \\ \\
Die Arbeit wurde bisher in gleicher oder ähnlicher Form keiner
anderen Prüfungsbehörde vorgelegt und auch noch nicht veröffentlicht. \\ \\
Dortmund, den \today \\ \\ \\
\begin{tabular}{@{}l@{}}\hline
	Ricardo Valente de Matos
\end{tabular}
\setcounter{page}{6}

%\bibliographystyle{IEEEtranS}
%\bibliography{Literatur}
\raggedright
\printbibliography

\end{document}
