\chapter{Ähnlichkeitsmetriken}
\label{chap:lösungsansatz}

Ähnlichkeitsmetriken:

Überprüfe die Genauigkeit der Ähnlichkeitsmetriken, die im Recommender-System verwendet werden. Dazu gehören beispielsweise Kosinus-Ähnlichkeit, Pearson-Korrelation, Jaccard-Ähnlichkeit oder andere, je nach Kontext.

Top-N-Empfehlungen:
Evaluieren Sie, wie gut das Recommender-System in der Lage ist, relevante Elemente unter den Top-N-Empfehlungen zu platzieren. Dies ist eine gängige Metrik, um die praktische Anwendbarkeit des Systems zu bewerten.

--------------------------
Repräsentation der Merkmale:
Untersuche, wie gut die Merkmale (Features) der Elemente im System repräsentiert sind. Eine gute Ähnlichkeitsberechnung hängt oft von der Qualität und Relevanz der Merkmale ab.


Diversität der Empfehlungen:

Prüfe, ob die Ähnlichkeitsbasierten Empfehlungen zu vielfältig sind. Eine zu starke Konzentration auf ähnliche Elemente könnte zu eintönigen Empfehlungen führen.
Benutzerbewertungen und Feedback:

Integriere Benutzerbewertungen und -feedback in die Evaluierung, um sicherzustellen, dass die Ähnlichkeitsberechnungen den tatsächlichen Vorlieben der Benutzer entsprechen.
Cold Start-Szenarien:

Teste das System unter Bedingungen des "Cold Start", um sicherzustellen, dass es auch effektive Empfehlungen machen kann, wenn es nur begrenzte Daten gibt.
Auswirkungen von Merkmalen:

Analysiere, wie sich das Hinzufügen oder Entfernen von Merkmalen auf die Empfehlungen auswirkt. Dies kann helfen, die Sensitivität des Systems gegenüber verschiedenen Merkmalen zu verstehen.
Nutzerinteraktion:

%Untersuche, wie gut das Recommender-System auf Veränderungen in der Benutzerinteraktion reagiert. Dies könnte Änderungen in den Präferenzen der Benutzer oder neue Interaktionen mit dem System einschließen.
Es ist wichtig, die spezifischen Anforderungen deines Recommender-Systems zu berücksichtigen und die Evaluierungsmethoden entsprechend anzupassen. Kombiniere mehrere Metriken, um ein umfassenderes Bild der Leistung des Systems zu erhalten.

%-Welche Art und Weisen des Testens und der Überwachung existieren (preprocessing/similarity visualisieren, Ergebnis und Input gegenüberstellen, etc.)

%Keywords visualisieren: word cloud, Circle packing, The horn of plenty, Treemap, donut chart, Grid of bar charts

%Daten gegenüberstellen: Alle Eingabedaten und alle Mitarbeiterinformationen nebeneinander
%Vielleicht mit einer „similarity“-Matrix.
%Vielleicht Gegenüberstellung von Ergebnissen eines supervised und unsupervised Ansatzes. (Matrix)
%Visualisierung von Daten:
%1.	Kreisdiagramm
%2.	Balkendiagramm
%3.	Säulendiagramm
%4.	Kurvendiagramm
%5.	Punktediagramm
%Vergleichstypen:
%1.	Strukturvergleich(Welcher Anteil verschiedene Komponenten macht an einer Gesamtheit (z. B. in Prozent) aus)
%2.	Rangfolgevergleich(Die verschiedenen Objekte können im Vergleich zueinander z. B. kleiner, größer, besser, schlechter oder gleich sein)
%3.	Häufigkeitsvergleich(Größenklassen bilden)
%4.	Korrelationsvergleich(Vergleich, ob zwischen zwei Variablen ein quantitativer Zusammenhang bzw. 


\section{Genauigkeit der Ähnlichkeit}

\section{Qualität und Relevanz der Merkmale}

\section{Eintönige Empfehlungen}

\section{Benutzerbewertungen und -feedback}

\section{Cold Start}

\section{Sensitivität des Systems}
\newpage