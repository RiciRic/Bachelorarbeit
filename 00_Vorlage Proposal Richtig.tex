%Dokumentklasse
\documentclass[a4paper,12pt]{scrreprt}
\usepackage[left= 3.5cm,right = 2cm, bottom = 2 cm]{geometry}
\addtolength{\footskip}{-0.5cm}
\usepackage[onehalfspacing]{setspace}
% ============= Packages =============

% Dokumentinformationen
\usepackage[hyphens]{url}
\usepackage[
pdfsubject={},
pdfauthor={Ricardo Valente de Matos},
pdfkeywords={},	
%Links nicht einrahmen
hidelinks,
breaklinks=true
]{hyperref}
% Standard Packages
\usepackage[utf8]{inputenc}
\usepackage[ngerman]{babel}
\usepackage[T1]{fontenc}
\usepackage{graphicx, subfig}
\graphicspath{{img/}}
\usepackage{fancyhdr}
\usepackage{lmodern}
\usepackage{color}

\usepackage{dirtree}

\usepackage[style=ieee, sorting=nty, urldate =comp, backend=bibtex]{biblatex}
\addbibresource{Literatur.bib}

%\usepackage[numbers]{natbib}
%\bibpunct{(}{)}{;}{a}{,}{,}

\usepackage{listings}

% zusätzliche Schriftzeichen der American Mathematical Society
\usepackage{amsfonts}
\usepackage{amsmath}
\usepackage{float}

\usepackage{tabularx}
\usepackage{multirow}
\usepackage{enumitem}

%nicht einrücken nach Absatz
\setlength{\parindent}{0pt}
\RedeclareSectionCommand[beforeskip=0pt]{chapter}
\usepackage{listings}
\usepackage{color}
\definecolor{lightgray}{rgb}{.9,.9,.9}
\definecolor{darkgray}{rgb}{.4,.4,.4}
\definecolor{purple}{rgb}{0.65, 0.12, 0.82}

\lstdefinelanguage{JavaScript}{
	keywords={typeof, new, true, false, catch, function, return, null, catch, switch, var, if, in, while, do, else, case, break},
	keywordstyle=\color{blue}\bfseries,
	ndkeywords={class, export, boolean, throw, implements, import, this},
	ndkeywordstyle=\color{darkgray}\bfseries,
	identifierstyle=\color{black},
	sensitive=false,
	comment=[l]{//},
	morecomment=[s]{/*}{*/},
	commentstyle=\color{purple}\ttfamily,
	stringstyle=\color{red}\ttfamily,
	morestring=[b]',
	morestring=[b]"
}

\lstset{
	language=JavaScript,
	backgroundcolor=\color{lightgray},
	extendedchars=true,
	basicstyle=\footnotesize\ttfamily,
	showstringspaces=false,
	showspaces=false,
	numbers=left,
	numberstyle=\footnotesize,
	numbersep=9pt,
	tabsize=2,
	breaklines=true,
	showtabs=false,
	captionpos=b
}


% ============= Kopf- und Fußzeile =============
\pagestyle{fancy}
%
\lhead{}
\chead{}
\rhead{\slshape \leftmark}
%%
\lfoot{}
\cfoot{\thepage}
\rfoot{}
%%
\renewcommand{\headrulewidth}{0.4pt}
\renewcommand{\footrulewidth}{0pt}

% ============= Package Einstellungen & Sonstiges ============= 
%Besondere Trennungen
\hyphenation{De-zi-mal-tren-nung}

\newcommand{\hiddenchapter}[1]{
	\chapter*{{#1}}
}

%-------------

\newcommand{\todo}[1]{\textcolor{red}{ToDo:} #1\marginpar{<--hier}}

% ============= Dokumentbeginn =============

\begin{document}
%Seiten ohne Kopf- und Fußzeile sowie Seitenzahl
\pagestyle{empty}

\begin{center}
	\begin{tabular}{p{\textwidth}}
		
		\begin{center}
			\textbf{\Large{Thesis zur Erlangung des akademischen Grades Bachelor of Science (B. Sc.)}}
		\end{center} \\ \\
		
		\begin{center}
			\LARGE{\textsc{
					%\textit{\emph{adesso Staffing Advisor Lab}}\\
					%Konzeption und prototypische Entwicklung der Struktur und Architektur einer Softwareplattform für Transparenz in KI-Anwendungen
					Automatisierung der Informationsgewinnung in Bedarfsmeldungen
			}}
		\end{center}
		
		\\
		
		
		
		\begin{center}
			von
		\end{center}
		
		\begin{center}
			\large{\textbf{Ricardo Valente de Matos}}
		\end{center}
	
	\begin{center}
		\large{geboren am 30.10.1999} \\
		\large{Matrikelnummer: 7203677} \\
		\large{im Studiengang Wirtschaftsinformatik \\
			der Fachhochschule Dortmund \\ im Fachbereich Informatik}
	\end{center}
		
		
		\\
		
		\\
		
		\begin{center}
			\begin{tabular}{lll}
				\textbf{Erstprüfer:} & & Prof. Dr.-Ing. Guy Vollmer\\
				\textbf{Zweitprüfer:} & & Stephan Schmeißer, M. Sc., Adessoplatz 1, 44269 Dortmund\\
			\end{tabular}
		\end{center}
	
	\\ \\
	
	\begin{center}
		\large{Dortmund, den \today}
	\end{center}
		
	\end{tabular}
\end{center}

%\setcounter{page}{1}
%\pagestyle{plain}

%letztes kapitel zusammenfassung und ausblick
\pagestyle{fancy}
\pagenumbering{Roman}
%\tableofcontents


%\listoffigures
%\lstlistoflistings
\newpage
%-NOCH ZU ERLEDIGEN-\\

%\chapter*{Lesehinweis}
%Aus Gründen der besseren Lesbarkeit werden Wörter und Wortgruppen, die hervorgehoben werden oder mehrfach auftauchen, durch \emph{kursiven} Text kenntlich gemacht. Zudem wird in dieser Projektarbeit die Sprachform des generischen Maskulinums angewandt. Sämtliche Ausführungen sind jedoch geschlechtsunabhängig und beziehen sich damit auf alle Geschlechter.
\newpage

\setcounter{page}{1}
\pagestyle{fancy}
\pagenumbering{arabic}
\setcounter{chapter}{0}
\newpage

\hiddenchapter{Motivation}
Die Suche nach qualifizierten Mitarbeitern ist für Unternehmen von entscheidender Bedeutung, um wettbewerbsfähig zu bleiben und langfristigen Erfolg zu sichern. In einer Zeit, in der der Arbeitsmarkt zunehmend global und dynamisch wird, stehen Organisationen vor der Herausforderung, aus einer Vielzahl von Mitarbeitern diejenigen zu identifizieren, die am besten zu einem spezifischen Projekt im Unternehmen passen. Hier setzt die Entwicklung eines Recommender Systems zur Mitarbeiterempfehlung an. Ein solches System kann Unternehmen dabei unterstützen, den Prozess der Mitarbeiterrekrutierung und -auswahl zu optimieren. Durch die Berücksichtigung verschiedener Kriterien wie Qualifikationen, Fähigkeiten, Erfahrungen kann das Recommender-System dazu beitragen, die Auswahl effektiv zu filtern und diejenigen herauszufiltern, die am besten zu einem Projekt im Unternehmen passen. Ein solches System bietet außerdem den Vorteil, den Prozess der Mitarbeiterempfehlung zu automatisieren und zu beschleunigen. Dies ermöglicht Unternehmen, schneller auf offene Stellen zu reagieren und potenzielle Kandidaten zeitnah zu identifizieren. Dadurch wird die Effizienz der Mitarbeitersuche verbessert und die Qualität der Einstellungsentscheidungen erhöht.\\

Das Potenzial von Recommender Systems wurde auch bei \emph{adesso} entdeckt und nun wird nach und nach Wege gesucht, KI-gestützte Systeme in die eigenen Prozesse zu integrieren. Im internen Projekt \emph{adesso Staffing Advisor} wird an einem Recommender-System zur Mitarbeiterempfehlung für ausgewählte Projekte gearbeitet. Die Umsetzung der Recommender Systems bedient sich verschiedener KI-basierten Ansätze. Ein ganz entscheidender Schritt im Prozess der Mitarbeiterempfehlung ist die Vorverarbeitung der Bedarfsmeldungen. Diese sind eine wertvolle Informationsquelle, die Fachkräften helfen kann, die Empfehlungen effizienter zu gestalten, um dadurch wettbewerbsfähig zu bleiben. Allerdings sind diese oft umfangreich und komplex, was ihre effektive Nutzung erschwert.\\

Deshalb ist es entscheidend, effiziente Methoden und Techniken des Information Retrieval anzuwenden, um so relevante Informationen schnell und präzise aus Bedarfsmeldungen zu extrahieren. Die Extraktion wichtiger Schlüsselwörter, Phrasen und Themen ermöglicht es einen besseren Einblick in die Ziele, Methoden und Ergebnisse der Projekte zu bekommen. Dadurch können fundierte Entscheidungen bezüglich der Personalbesetzung getroffen und Ressourcen effizient genutzt werden.\\
\hiddenchapter{Problemstellung}
In einer immer stärker vernetzten und informationsreichen Welt stehen Organisationen vor der Herausforderung, relevante Informationen effizient aus umfangreichen Bedarfsmeldungen zu extrahieren. Obwohl diese Beschreibungen wichtige Einblicke in Ziele, Methoden und Ergebnisse liefern, können sie aufgrund ihres Umfangs und ihrer Komplexität schwer durchsuchbar und analysierbar sein. Die manuelle Identifizierung und Extraktion relevanter Inhalte ist zeitaufwendig und fehleranfällig. Daher stellt sich die Problemstellung: \\

Wie können wir effektive Methoden und Techniken des Information Retrieval und Data-Mining nutzen, um automatisiert relevante Inhalte aus Bedarfsmeldungen im spezifischen Software Entwicklungs-Kontext zu extrahieren und somit die Effizienz, Genauigkeit und Geschwindigkeit der Informationsgewinnung für Führungskräfte zu verbessern.
\hiddenchapter{Ziele und Ergebnisse der Arbeit}
Diese Ausarbeitung präsentiert eine umfassende Untersuchung zur Entwicklung eines automatisierten Systems zur Extraktion relevanter Inhalte aus Bedarfsmeldungen im Software-Entwicklungs-Kontext.
\begin{itemize}
	\item Die erste Phase dieser Ausarbeitung besteht darin, eine klare Erwartungshaltung hinsichtlich der Anforderungen und Bedürfnisse der Stakeholder zu entwickeln. Hierfür werden Interviews mit Führungskräften durchgeführt, um die Erwartungen bezüglich einer \glqq{}perfekten\grqq{} Bedarfsmeldung herauszuarbeiten. Diese dient als Grundlage für die weiteren Entwicklungs- und Evaluierungsphasen.
	\item Im Anschluss erfolgt eine eingehende Analyse der Techniken <was für Techniken> des Information Retrieval und Data-Mining, um die besten Ansätze zur Extraktion relevanter Inhalte zu identifizieren. Diese Analyse bildet die Grundlage für die Konzeptionierung einer Vorverarbeitung, das eine Kombination der erforschten Ergebnisse darstellt. Die Implementierung dieses Modells erfolgt durch den Aufbau einer Pipeline in Python, die eine effiziente Verarbeitung und Extraktion der Bedarfsmeldungen ermöglicht.
	\item Zur Evaluierung der Leistungsfähigkeit des entwickelten Systems werden reale Bedarfsmeldungen und Mitarbeiterinformationen verwendet. Dabei wird überprüft, inwiefern das Ergebnis der definierten Erwartungshaltung entspricht. Mithilfe von den Metriken \emph{Precision}, \emph{Recall} und \emph{F1-Score} werden Abweichungen, Ähnlichkeiten und Anpassungen in Parametern analysiert, um Erkenntnisse darüber zu gewinnen, wie das System inhaltlich abschneidet und verbessert werden kann.
	\item (Schließlich wird eine vergleichende Untersuchung mit einem auf Large Language Model basierenden Vorverarbeitungsansatz durchgeführt. Dabei werden die Performance, Zeit und Ergebnisqualität des entwickelten Systems mit diesem alternativen Ansatz verglichen. Dieser Vergleich dient dazu, die Stärken und Schwächen des entwickelten Systems zu identifizieren und gegebenenfalls weitere Verbesserungen vorzunehmen.)
\end{itemize}

\hiddenchapter{Vorgehen und Zeitplan}
%handelt sich um Projektarbeit, peile bis  ende august bsw an
Ziel ist es die Arbeit im Mai fertig zu stellen. Die einzelnen Monatsziele können aus der nachfolgenden Tabelle entnommen werden. \\ \\
\begin{tabularx}{1\textwidth} { 
		| >{\raggedright\arraybackslash}X 
		| >{\raggedright\arraybackslash}X | }
	\hline
	Februar
	& \begin{itemize}
		\item Durchführung der Interviews mit Fachkräften
		\item Zusammentragung aller relevanter Information Retrieval- und Preprocessing-Ansätze 
	\end{itemize}\\
	\hline
	März
	& \begin{itemize}
		\item Durchführung der Interviews mit Fachkräften
		\item Formulierung der Anforderungen für Bedarfsmeldungen
	\end{itemize}\\
	\hline
	April
	& \begin{itemize}
		\item Entwicklung des Eigenen Preprocessing-Modells
		\item Evaluierung der Ergebnisse
	\end{itemize}\\
	\hline
	Mai
	& \begin{itemize}
		\item Schluss schreiben
		\item Korrekturen
	\end{itemize}\\
	\hline
\end{tabularx}
\newpage

\renewcommand\contentsname{Aufbau der Arbeit}
\todo{Aufbau der Arbeit anpassen}
\tableofcontents

\cite{kobayashi2000information}

\cite{singhal2001modern}

\cite{croft2000combining}

\cite{horesh2016information}

\cite{belkin1992information}

information filtering
\cite{lanquillon2001enhancing}

preprocessing
\cite{alasadi2017review}

-------
spam-filter
\cite{shafi2017review}
\cite{khorsi2007overview}
\cite{tretyakov2004machine}
----
TF-IDF (Term Frequency-Inverse Document Frequency): TF-IDF ist eine statistische Methode, die verwendet wird, um die Relevanz eines Begriffs in einem Dokument relativ zu einem Korpus von Dokumenten zu bestimmen. Wörter mit höheren TF-IDF-Werten werden als potenzielle Schlüsselwörter betrachtet.
\cite{bafna2016document}
\cite{ramos2003using}

Text-Ranking-Algorithmen: Text-Ranking-Algorithmen wie TextRank oder YAKE (Yet Another Keyword Extractor) verwenden Graphen-basierte Methoden, um Schlüsselwörter in einem Text zu identifizieren. Diese Algorithmen bewerten die Wichtigkeit von Wörtern basierend auf ihrer Verbindung zu anderen Wörtern im Text und extrahieren Schlüsselwörter entsprechend ihrer Rangfolge.
\cite{mihalcea2004textrank}
\cite{zhang2020empirical}
\cite{pay2019ensemble}

N-Gramm-Analyse: N-Gramme sind Sequenzen von N aufeinanderfolgenden Wörtern in einem Text. Durch die Analyse von N-Grammen können häufig auftretende Phrasen oder Begriffe identifiziert werden, die potenzielle Schlüsselwörter darstellen.
\cite{pirk2019implementierung}


Part-of-Speech (POS) Tagging: POS-Tagging wird verwendet, um die grammatischen Kategorien von Wörtern in einem Text zu bestimmen. Durch die Berücksichtigung von Wörtern mit bestimmten POS-Tags wie Substantiven oder Adjektiven können relevante Schlüsselwörter extrahiert werden.
\cite{kumawat2015pos}
\cite{nakagawa2007hybrid}

bekommen wir ein unsupervised learning ansatz der 

Regelbasierte Ansätze: Regelbasierte Ansätze verwenden vordefinierte Regeln oder Muster, um Schlüsselwörter zu identifizieren. Dies kann beispielsweise das Extrahieren von Wörtern sein, die häufig im Text vorkommen oder bestimmten Mustern entsprechen.

Hybride Ansätze: Hybride Ansätze kombinieren verschiedene Methoden und Techniken, um eine genauere Extraktion von Schlüsselwörtern zu ermöglichen. Zum Beispiel könnte eine Kombination aus TF-IDF-Gewichtung und Text-Ranking-Algorithmen verwendet werden, um eine robuste Schlüsselwortextraktion zu erreichen.


transformation von bedarfsmeldung zu guter bedarfsmeldung, was ist der fokus von der bedarfsmeldung, wie gut machen die ansätze das, und muss man das dann noch weiter verarbeiten, haben wir alles was wir brauchen mit nur einem algorithmus, inferenz falls parameter fehlt, gibt es einen der alles löst,

ergebnis der arbeit: diese modelle in der reihenfolge kommen am nähesten an die bedarfsmeldung

ausblick: die keyword extraction auch für die profile nutzen 

was muss ich jetzt machen: gucken wie ich das inhaltlich genau machen will, also pipeline genauch checken, quellen von der bachelorarbeit checken

\chapter{Einleitung}
\label{chap:einleitung}
In einer globalisierten und dynamischen Wirtschaftswelt sind Unternehmen zunehmend auf Projekte angewiesen, um ihre Ziele zu erreichen und Wettbewerbsvorteile zu erlangen. Die Personalbeschaffung für solche Projekte erfordert oft spezialisiertes Fachwissen und vielfältige Fähigkeiten, um erfolgreich umgesetzt zu werden. Es ist entscheidend für den Projekterfolg, dass die Personalbeschaffung die passenden Mitarbeiter für ausgewählte Projekte findet. Hier setzt die Entwicklung eines Recommender Systems zur Mitarbeiterempfehlung an. Ein solches System kann Unternehmen dabei unterstützen, den Prozess der Mitarbeiterrekrutierung und -auswahl zu optimieren. Durch die Berücksichtigung verschiedener Kriterien wie Qualifikationen, Fähigkeiten und Erfahrungen kann das Recommender-System dazu beitragen, die Auswahl effektiv zu filtern und diejenigen herauszufiltern, die am besten zu einem Projekt im Unternehmen passen. Ein solches System bietet außerdem den Vorteil, den Prozess der Mitarbeiterempfehlung zu automatisieren und zu beschleunigen. Dies ermöglicht Unternehmen, schneller auf offene Stellen zu reagieren und potenzielle Kandidaten zeitnah zu identifizieren. Dadurch wird die Effizienz der Mitarbeitersuche verbessert und die Qualität der Einstellungsentscheidungen erhöht.\\

Das Potenzial von Recommender Systems wurde auch bei \emph{adesso} entdeckt und nun wird nach und nach Wege gesucht, KI-gestützte Systeme in die eigenen Prozesse zu integrieren. Im internen Projekt \emph{adMatch} wird an einem Recommender-System zur Mitarbeiterempfehlung für ausgewählte Projekte gearbeitet. Die Umsetzung der Recommender Systems bedient sich verschiedener KI-basierten Ansätze. Als IT-Dienstleister wird \emph{adesso} von Kunden unter anderem mit der Entwicklung individueller Softwarelösungen beauftragt. Derzeit verbringen Führungskräfte jedoch viel Zeit damit, interne Mitarbeiterinnen und Mitarbeiter manuell für Kundenprojekte zu suchen und diese dann aufgrund ihrer Erfahrungen und Fähigkeiten auszuwählen und entsprechend einzusetzen. Ein ganz entscheidender Schritt im Prozess der Mitarbeiterempfehlung ist die Vorverarbeitung der \emph{Bedarfsmeldungen}. Diese beinhalten Informationen zu Projekten und sind eine wertvolle Informationsquelle, die Führungskräften helfen kann, die Empfehlungen effizienter zu gestalten, um dadurch wettbewerbsfähig zu bleiben. Allerdings sind diese oft umfangreich, unsortiert und komplex, was ihre effektive Nutzung erschwert. Dieser Prozess soll durch eine KI-Lösung unterstützt werden. Da es sich bei der Personalsuche um einen geschäftskritischen Prozess handelt, ist der Spielraum für Fehler gering. Im internen Projekt \emph{adMatch} wird eine durch Large Language Model-gestützte Anwendung entwickelt, die Führungskräfte bei der Suche nach geeignetem Personal für ausgewählte Projekte unterstützt. Der Ansatz des Large Language Modeling ist jedoch nicht deterministisch. Es besteht die Gefahr, dass bei gleichem Input unterschiedliche Ergebnisse erzielt werden. Somit versucht \emph{adesso} durch den Einsatz von Methoden und Technologien neben dem Large Language Model-Ansatz deterministische Ergebnisse zu erzielen, die auf einem ähnlichen Niveau liegen. Ein Ansatz ist die Anwendung von effizienten Methoden und Techniken des Information Retrieval, um so relevante Informationen schnell und präzise aus \emph{Bedarfsmeldungen} zu extrahieren. Die Extraktion wichtiger Schlüsselwörter, Phrasen und Themen ermöglicht es einen besseren Einblick in die Ziele, Methoden und Ergebnisse der Projekte zu bekommen. Dadurch können fundierte Entscheidungen bezüglich der Personalbesetzung getroffen und Ressourcen effizient genutzt werden.\\
\section{Problemstellung}
\label{sec:problemstellung}
Der Staffing-Prozess kann ausschließlich von ausgewählten Mitarbeitenden von adesso durchgeführt werden. Um das Entlastungspotenzial für Führungskräfte durch das Gesamtsystem eines Recommender Systems für Mitarbeiterempfehlungen zu realisieren, ist eine technische Abbildung des Prozesses erforderlich. Dazu sind mehrere Schritte notwendig. Eine Informationsgewinnung aus den unstrukturierten Projekt- und Mitarbeiterdaten ist unerlässlich, um schließlich den Ähnlichkeitsvergleich für die Empfehlungen durchführen zu können. Diese Ausarbeitung befasst sich mit dem Schritt der Strukturierung und Informationsextraktion der \emph{Bedarfsmeldungen}. Somit steht \emph{adesso} vor der Herausforderung, relevante Informationen effizient aus umfangreichen \emph{Bedarfsmeldungen} zu extrahieren. Obwohl diese Beschreibungen wichtige Einblicke in Ziele, Methoden und Ergebnisse liefern, können sie aufgrund ihres Umfangs und ihrer Komplexität schwer durchsuchbar und analysierbar sein. Die manuelle Identifizierung und Extraktion relevanter Inhalte ist zeitaufwendig und fehleranfällig. Daher stellt sich die Problemstellung: \\

Wie können wir effektive Methoden und Techniken des Information Retrieval und Data-Mining nutzen, um automatisiert relevante Inhalte aus \emph{Bedarfsmeldungen} im spezifischen Kontext der Software Entwicklung zu extrahieren und somit die Effizienz, Genauigkeit und Geschwindigkeit der Informationsgewinnung für Führungskräfte zu verbessern.\\

In der Vergangenheit wurden bereits Methoden im Bereich des automatisierten Recruitings untersucht. Im Projektgeschäft sehen wir uns mit einem Problem konfrontiert, dessen Umfang jedoch präziser definiert werden kann, da die Kandidatenauswahl einem begrenzten Pool unterliegt. Besondere Relevanz hat hierbei die Erstellung einer Standardisierung der \emph{Bedarfsmeldung}, da diese häufig unstrukturiert und mit fehlenden Informationen vorliegt.
\section{Ziele und Ergebnisse der Arbeit}
\label{sec:zieleundergebnis}
Diese Ausarbeitung präsentiert eine umfassende Untersuchung zur Entwicklung eines automatisierten Systems zur Extraktion relevanter Inhalte aus \emph{Bedarfsmeldungen} im Software-Entwicklungs-Kontext.
\begin{itemize}
	\item In der Ausarbeitung wird zunächst ein Konzept einer standardisierten \emph{Bedarfsmeldung} erarbeitet. Dazu wird eine klare Erwartungshaltung hinsichtlich der Anforderungen und Bedürfnisse der Stakeholder entwickeln. Hierfür werden Interviews mit Führungskräften durchgeführt, um die Erwartungen bezüglich einer \glqq{}perfekten\grqq{} \emph{Bedarfsmeldung} herauszuarbeiten. Dieses Konzept dient als Grundlage für die weiteren Entwicklungs- und Evaluierungsphasen.
	\item Es wird an einer ausführbaren prototypischen Software gearbeitet, die \emph{Bedarfsmeldungen} effizient verarbeitet und wichtige Informationen extrahiert. Hierfür wird eine Pipeline in Python aufgebaut und strukturell durch ein Use-Case- und UML-Aktivitätsdiagramme dokumentiert. Es werden Modelle des Information Retrieval und Data-Mining implementiert die dazu beitragen, eine \emph{Bedarfsmeldung} in die vorher definierte Struktur umzuformen. Dabei erfolgt zunächst eine eingehende Analyse der Techniken \emph{TF-IDF}, \emph{N-Gramm}, \emph{Named Entity Recognition}, \emph{POS-Tagging}, und Hybride Ansätze, um die besten Ansätze zur Extraktion relevanter Inhalte zu identifizieren. Diese Analyse bildet die Grundlage für die Umsetzung des Software-Prototypen, das eine Kombination der erforschten Ergebnisse darstellt.
	\item Um die Leistungsfähigkeit des entwickelten Systems zu evaluieren, werden Testfälle für reale \emph{Bedarfsmeldungen} definiert. Dabei wird überprüft, inwieweit das Ergebnis den Erwartungen entspricht. Mit Hilfe einer manuellen Überprüfung werden Abweichungen, Ähnlichkeiten und Anpassungen analysiert, um Erkenntnisse über die inhaltliche Leistung des Systems und die Techniken zu gewinnen, die allein oder in Kombination mit mehreren Ansätzen die wichtigsten Informationen aus den semi-strukturierten \emph{Bedarfsmeldungen} herausfiltern. Da die Dauer eine entscheidende Rolle spielt, werden auch Zeit und Leistung gemessen. %Diese Ergebnisse werden mit einem neuen Vorverarbeitungsansatz verglichen, der auf dem Large Language Model basiert. Die Performance, Zeit und Ergebnisqualität des entwickelten Systems soll im Vergleich mit diesem alternativen Ansatz die Stärken und Schwächen des entwickelten Systems aufzeigen, um daraus gegebenenfalls weitere Verbesserungsmöglichkeiten zu identifizieren.
\end{itemize}
\section{Aufbau der Arbeit}
\todo{Anpassen}\\

\textbf{Kapitel \ref{chap:einleitung} \nameref{chap:einleitung}} befasst sich mit der Problemstellung und die Zielsetzung der Ausarbeitung. Dazu wird der Aufbau der Arbeit erläutert.\\

In \textbf{Kapitel \ref{chap:erwartungshaltung} \nameref{chap:erwartungshaltung}} werden Experteninterviews durchgeführt und analysiert, um eine standardisierte \emph{Bedarfsmeldung} zu entwickeln. \\

\textbf{Kapitel \ref{chap:konzeption} \nameref{chap:konzeption}} befasst sich mit der Grundsätzlichen Idee eines Recommender Systems zur Mitarbeiterempfehlung und der Historie von Recommender Systems. Außerdem wird auf abstrakter Ebene das zu Entwickelnde System konzipiert und Anforderungen zusammengetragen.\\

\textbf{Kapitel \ref{sec:literaturueberblick} \nameref{sec:literaturueberblick}} wird Literatur zu Methodiken und Ansätze für die Nutzung von Extraktionsmechanismen von Schlüsselwörtern analysiert. \\

In \textbf{Kapitel \ref{chap:implementierung} \nameref{chap:implementierung}} wird auf Basis der erforschten Ergebnisse aus Kapitel \ref{chap:erwartungshaltung} und \ref{sec:literaturueberblick} Implementationsdetails des System zur Strukturierung von \emph{Bedarfsmeldungen} dargestellt. \\

In \textbf{Kapitel \ref{chap:evaluation} \nameref{chap:evaluation}} wird das in Kapitel \ref{chap:implementierung} entwickelte System evaluiert. \\

Das abschließende \textbf{Kapitel \ref{chap:ergebnisseausblick} \nameref{chap:ergebnisseausblick}} fasst die wichtigsten Ergebnisse zusammen und gibt einen Ausblick auf mögliche weiterführende Forschungen und Anpassungsmöglichkeiten des entwickelten Systems. \\
\newpage


\chapter{Grundlagen}
\label{chap:literaturüberblick}
In diesem Kapitel werden die für das Thema notwendigen Grundlagen und bereits erforschten Themengebiete im Kontext von Recommender Systemen und Informationsverarbeitung behandelt, die für das weitere Verständnis der Arbeit notwendig sind. Es wird ein Einblick in die Art und Weise gegeben, wie andere Autoren Information Retrieval und Filtering einsetzen und kombinieren.
\todo{dieser satz passt nicht mehr so richtig}
\section{Kontext}
Als IT-Dienstleister wird \emph{adesso} von Kunden unter anderem mit der Entwicklung individueller Softwarelösungen beauftragt. Derzeit verbringen Führungskräfte jedoch viel Zeit damit, interne Mitarbeiterinnen und Mitarbeiter manuell für Kundenprojekte zu suchen und diese dann aufgrund ihrer Erfahrungen und Fähigkeiten auszuwählen und entsprechend einzusetzen. Dieser Prozess soll durch eine KI-Lösung unterstützt werden. Da es sich bei der Personalsuche um einen geschäftskritischen Prozess handelt, ist der Spielraum für Fehler gering. Im internen Projekt \emph{adesso Staffing Advisor} wird eine durch Large Language Model-gestützte Anwendung entwickelt, die Führungskräfte bei der Suche nach geeignetem Personal für ausgewählte Projekte unterstützt. Der Ansatz des Large Language Modeling ist jedoch nicht deterministisch. Es besteht die Gefahr, dass bei gleichem Input unterschiedliche Ergebnisse erzielt werden. Daher versucht \emph{adesso} durch den Einsatz von Methoden und Technologien neben dem Large Language Model-Ansatz deterministische Ergebnisse zu erzielen, die auf einem ähnlichen Niveau liegen.\\

Die vorliegende Ausarbeitung befasst sich mit der Informationsgewinnung in \emph{Bedarfsmeldungen}.
\todo{gucken wie ich das hier mache}
\section{Recommender Systems Historie und aktueller Stand der Forschung}
Auch wenn die Erstellung eines Recommender Systems nicht Gegenstand der vorliegenden Ausarbeitung ist, stellt die Nutzung von Information Retrieval und Filtering ein entscheidener Schritt in Richtung eines funktionierenden Recommender Systems dar. Das Verständnis der Funktionsweise eines Recommender Systems sowie dessen Entwicklung in den vergangenen Jahren ist daher für das Verständnis des Teilbereichs dieser Thematik von Nutzen.\\

Recommender Systems existieren bereits seit vielen Jahren. Im Jahr 1992 führten Belkin und Croft eine Analyse und einen Vergleich des Information Retrievals und Filtering durch \cite{dong2022brief}. Das Information Retrieval behandelt die grundlegende Technologie der Suchmaschine \cite{dong2022brief}. Das Recommender System basiert hauptsächlich auf der Technologie des Information Filtering. Im selben Jahr präsentierte Goldberg das Tapestry-System, welches das erste System zur Informationsfilterung darstellt, das auf kollaboratives Filtern durch menschliche Bewertung basiert. Die Mehrheit der frühen Empfehlungsmodelle basiert auf kollaborativer Empfehlungen, wobei K-Nearest-Neighbor (KNN)-Modelle eine besondere Rolle einnehmen. Diese Modelle prognostizieren die Nachbarn eines Zielnutzers, indem sie eine Ähnlichkeit zwischen den vorherigen Präferenzen und den Präferenzen der anderen Nutzer berechnen \cite{dong2022brief}. Die Studie von Goldberg inspirierte einige Forscher des Massachusetts Institute of Technology (MIT) und der University of Minnesota (UMN) dazu, einen Nachrichtenempfehlungsdienst mit dem Namen \emph{GroupLens} zu entwickeln. Die Hauptkomponente dieses Dienstes ist ein Modell zur kollaborativen Filterung zwischen Nutzern \cite{dong2022brief}. Das gleichnamige Forschungslabor kann somit als Pionier auf dem Gebiet der Recommender Systems bezeichnet werden. Die dort durchgeführten Forschungen bilden die Grundlage für nachfolgende Musik- und Video-Ähnlichkeitsempfehlungen \cite{dong2022brief}. \\

Recommender Systeme haben in den letzten Jahren verschiedene Definitionen erhalten. Eine dieser Definitionen wird in dem Artikel von Resnick und Varian (1997) sinngemäß so beschrieben, dass ein typisches Recommender System Empfehlungen durch Personen als Eingabe erhält, die das System dann zusammenschließt und an geeignete Empfänger weiterleitet \cite{burke2011recommender}. In einigen Fällen besteht die primäre Transformation in der Zusammenführung, in anderen Fällen liegt die Fähigkeit des Systems darin, gute Übereinstimmungen zwischen Empfehlungsgebern und Empfehlungsempfängern herzustellen \cite{burke2011recommender}. Empfehlungssysteme stellen ein Instrument zur Interaktion mit umfangreichen und vielschichtigen Informationen dar. Sie ermöglichen eine personalisierte Sicht auf diese Informationen, indem sie die für den Nutzer wahrscheinlich relevanten Inhalte aufbereiten \cite{burke2011recommender}. Besonders im Handelsverkehr im Internet sind Recommender Systeme ein häufiger Einsatzgebiet. Dabei werden Recommender Systeme als Werkzeuge zum Suchen und Filtern von Informationen verwendet, die dem Benutzer Vorschläge unterbreiten, die für ihn nützlich sein könnten. Sie sind in einer Vielzahl von Internetanwendungen weit verbreitet und helfen den Nutzern, bessere Entscheidungen bei der Suche nach Nachrichten, Musik, Urlaubsangeboten oder Geldanlagen zu treffen \cite{ricci2014recommender}. Eine spezifisches Recommender System konzentriert sich normalerweise auf eine Art von Themengebiet wie z. B. Filme oder Nachrichten \cite{ricci2014recommender}. Darüber hinaus sind sie zu einem entscheidenden Faktor in der Entscheidungsfindung von Organisationen geworden \cite{chartron2014general}. Unternehmen wie \emph{adesso} bauen immer weiter auf Recommender System unterstützte System auf, um Prozesse zu beschleunigen oder zu vereinfachen. Grundsätzlich können die Methoden in die Typen (i)\emph{collaborative Filtering-based} (kollaborative Empfehlungssysteme), (ii)\emph{content-based} (inhaltsbasierte Empfehlungssysteme), (iii)\emph{knowledge-based} (wissensbasiert Empfehlungssysteme) und (iv)\emph{hybrid} (hybride Empfehlungssysteme) unterteilt werden.\\

Jede Empfehlungsmethode hat ihre Vorteile und Grenzen \cite{lu2020recommender}. Insbesondere das inhaltsbasierte Empfehlungssystem bring eine hohe Relevanz für das Mitarbeiterempfehlungssystem. Die Grundprinzipien inhaltsbasierter Empfehlungssysteme sind zum einen die Analyse der Beschreibung der von einem bestimmten Benutzer bevorzugten \emph{Items}, um die gemeinsamen Hauptattribute (Präferenzen) zu identifizieren, die diese \emph{Items} unterscheiden. Diese Präferenzen werden in einem \emph{Benutzerprofil} gespeichert \cite{lu2020recommender}. Zusätzlich werden die Eigenschaften jedes \emph{Items} mit dem \emph{Benutzerprofil} verglichen, so dass nur \emph{Items} empfohlen werden, die eine hohe Ähnlichkeit mit dem \emph{Benutzerprofil} aufweisen \cite{lu2020recommender}. Bei der Idee der Mitarbeiterempfehlung kann also die \emph{Bedarfsmeldung} mit den benötigten Projektskills und Anforderung als \emph{Benutzerprofil} angesehen werden. Die Mitarbeiterprofile sind dabei die \emph{Items}. Die Attribute werden verglichen (Skills der Mitarbeiter mit den Skills und Anforderungen der \emph{Bedarfsmeldung}) und ähnliche \emph{Items} werden vorgeschlagen. Mit Hilfe traditioneller Methoden des Information Retrievals, wie z.B. dem Kosinus-Ähnlichkeitsmaß, werden dann Empfehlungen generiert \cite{lu2020recommender}. Darüber hinaus generieren sie Empfehlungen mit Hilfe von statistischen und maschinelle Lernverfahren, die in der Lage sind, Nutzerinteressen aus historischen Nutzerdaten zu lernen \cite{lu2020recommender}.
\section{Information Retrieval und Information Filtering}
%Diese Arbeit beschreibt den Unterschied zwischen Information Filtering und Information Retrieval\cite{belkin1992information}
Im Allgemeinen wird einem Informationssystem die Funktion zugeschrieben, den Benutzer zu den Dokumenten zu führen, die seinen Informationsbedarf am besten decken \cite{belkin1992information}. Allgemeiner ausgedrückt ist das Ziel eines Informationssystems, dem Benutzer Informationen aus der Wissensressource zur Verfügung zu stellen, die ihm helfen, ein Problem zu lösen \cite{belkin1992information}. Auf der anderen Seite ist unter Filtern das Entfernen von Daten aus einem eingehenden Datenstrom zu verstehen und nicht das Auffinden von Daten in diesem Datenstrom \cite{belkin1992information}. Filtersysteme verarbeiten große Datenmengen \cite{belkin1992information}. Typische Anwendungen betreffen Gigabytes von Text oder weitaus größere Mengen anderer Medien \cite{belkin1992information}. Während es bei dem Information Retrieval typischerweise um die einmalige Nutzung des Systems durch eine Person mit einem einmaligen Ziel und einer einmaligen Anfrage geht, befasst sich die Informationsfilterung mit der wiederholten Nutzung des Systems durch eine oder mehrere Personen mit langfristigen Zielen oder Interessen \cite{belkin1992information}.
\section{Data-Mining}
Data Mining ist ein interdisziplinäres Teilgebiet der Informatik, das sich mit der rechnergestützten Entdeckung von Mustern in großen Datenbeständen befasst \cite{jain2013data}. Ziel dieses fortgeschrittenen Analyseverfahrens ist es, Informationen aus einem Datensatz zu extrahieren und in eine für die weitere Verwendung verständliche Struktur umzuwandeln \cite{jain2013data}. Die verwendeten Methoden liegen an der Schnittstelle zwischen künstlicher Intelligenz, maschinellem Lernen, Statistik, Datenbanksystemen und Business Intelligence \cite{jain2013data}. Beim Data Mining geht es um die Lösung von Problemen durch die Analyse von Daten, die bereits in Datenbanken vorhanden sind \cite{jain2013data}.
\section{NLP}
\emph{NLP} (Natural Language Processing) stellt einen zentralen Aspekt im Bereich der künstlichen Intelligenz sowie der Computerwissenschaften dar \cite{kang2020natural}. Studien in diesem Bereich umfassen Theorien und Methoden, die eine Kommunikation zwischen Menschen und Computern in natürlicher Sprache ermöglichen \cite{kang2020natural}. \emph{NLP} vereint die Gebiete Informatik, Linguistik und Mathematik mit dem primären Ziel, menschliche Sprache in Befehle zu übersetzen, die von Computern ausgeführt werden können \cite{kang2020natural}.

\chapter{Verwandte Arbeiten}
\label{chap:verwandtearbeiten}

\newpage

\chapter{Adesso Staffing Advisor}
\label{chap:staffingadvisor}

\section{Aufbau des Projekts}
\label{sec:ausgangssituation}

\section{Preprocessing}
\label{sec:preprocessing}

\subsection{Keyword-Extraction}

\subsection{Normalizing}

\subsection{Large Language Models}

\section{KI-Modelle}
\label{sec:similaritycalculation}

\subsection{spacy}

\subsection{sbert}

\section{Nutzung von Daten}
\label{chap:ui}

-Welche Informationen der Ergebnisse des KI-Ansatzes sind vorhanden und werden gebraucht. (preprocessing, Ergebnis, similarity-Werte)

\subsection{Welche Daten werden in die KI gegeben}

\subsection{Welche Daten werden von den KI-Ansätzen erstellt}

\subsection{Welche Daten werden von der KI zurückgegeben}

\newpage

\chapter{Ähnlichkeitsmetriken}
\label{chap:lösungsansatz}

Ähnlichkeitsmetriken:

Überprüfe die Genauigkeit der Ähnlichkeitsmetriken, die im Recommender-System verwendet werden. Dazu gehören beispielsweise Kosinus-Ähnlichkeit, Pearson-Korrelation, Jaccard-Ähnlichkeit oder andere, je nach Kontext.

Top-N-Empfehlungen:
Evaluieren Sie, wie gut das Recommender-System in der Lage ist, relevante Elemente unter den Top-N-Empfehlungen zu platzieren. Dies ist eine gängige Metrik, um die praktische Anwendbarkeit des Systems zu bewerten.

--------------------------
Repräsentation der Merkmale:
Untersuche, wie gut die Merkmale (Features) der Elemente im System repräsentiert sind. Eine gute Ähnlichkeitsberechnung hängt oft von der Qualität und Relevanz der Merkmale ab.


Diversität der Empfehlungen:

Prüfe, ob die Ähnlichkeitsbasierten Empfehlungen zu vielfältig sind. Eine zu starke Konzentration auf ähnliche Elemente könnte zu eintönigen Empfehlungen führen.
Benutzerbewertungen und Feedback:

Integriere Benutzerbewertungen und -feedback in die Evaluierung, um sicherzustellen, dass die Ähnlichkeitsberechnungen den tatsächlichen Vorlieben der Benutzer entsprechen.
Cold Start-Szenarien:

Teste das System unter Bedingungen des "Cold Start", um sicherzustellen, dass es auch effektive Empfehlungen machen kann, wenn es nur begrenzte Daten gibt.
Auswirkungen von Merkmalen:

Analysiere, wie sich das Hinzufügen oder Entfernen von Merkmalen auf die Empfehlungen auswirkt. Dies kann helfen, die Sensitivität des Systems gegenüber verschiedenen Merkmalen zu verstehen.
Nutzerinteraktion:

%Untersuche, wie gut das Recommender-System auf Veränderungen in der Benutzerinteraktion reagiert. Dies könnte Änderungen in den Präferenzen der Benutzer oder neue Interaktionen mit dem System einschließen.
Es ist wichtig, die spezifischen Anforderungen deines Recommender-Systems zu berücksichtigen und die Evaluierungsmethoden entsprechend anzupassen. Kombiniere mehrere Metriken, um ein umfassenderes Bild der Leistung des Systems zu erhalten.

%-Welche Art und Weisen des Testens und der Überwachung existieren (preprocessing/similarity visualisieren, Ergebnis und Input gegenüberstellen, etc.)

%Keywords visualisieren: word cloud, Circle packing, The horn of plenty, Treemap, donut chart, Grid of bar charts

%Daten gegenüberstellen: Alle Eingabedaten und alle Mitarbeiterinformationen nebeneinander
%Vielleicht mit einer „similarity“-Matrix.
%Vielleicht Gegenüberstellung von Ergebnissen eines supervised und unsupervised Ansatzes. (Matrix)
%Visualisierung von Daten:
%1.	Kreisdiagramm
%2.	Balkendiagramm
%3.	Säulendiagramm
%4.	Kurvendiagramm
%5.	Punktediagramm
%Vergleichstypen:
%1.	Strukturvergleich(Welcher Anteil verschiedene Komponenten macht an einer Gesamtheit (z. B. in Prozent) aus)
%2.	Rangfolgevergleich(Die verschiedenen Objekte können im Vergleich zueinander z. B. kleiner, größer, besser, schlechter oder gleich sein)
%3.	Häufigkeitsvergleich(Größenklassen bilden)
%4.	Korrelationsvergleich(Vergleich, ob zwischen zwei Variablen ein quantitativer Zusammenhang bzw. 


\section{Genauigkeit der Ähnlichkeit}

\section{Qualität und Relevanz der Merkmale}

\section{Eintönige Empfehlungen}

\section{Benutzerbewertungen und -feedback}

\section{Cold Start}

\section{Sensitivität des Systems}
\newpage

%\chapter{Umsetzung der Visualisierungsmethoden}
\label{chap:implementation}
-Wie können diese Art und Weisen mit dem „adesso Staffing Advisor“ Lab implementiert werden

\section{World Cloud}

\newpage

\chapter{Evaluation}
\label{chap:auslieferung}
%\section{SSO ADFS}

-Evaluation der Art und Weisen der KI-Test- und Überwachungsmethoden (Wie hilfreich sind die unterschiedlichen Methoden, vielleicht mit einem Bewertungssystem im „adstaff lab“)
Dashboard mit vielen verschiedenen Methoden der Visualisierung. Jede Methode hat einen eigenen Bereich, wo z.B. Sterne vergeben werden können.


\newpage

\chapter{Zusammenfassung und Ausblick}
\label{chap:ergebnisse}

ergebnis der arbeit: diese modelle in der reihenfolge kommen am nähesten an die bedarfsmeldung

\subsection*{Ausblick}
-die keyword extraction auch für die profile nutzen 
-genauer untersuchen wie einzelne ansätze mit anderen nlp vortrainierten datensätzen abschneidet
-in bezug zu recommender systems beschreiben wie damit weiter gemacht werden könnte
\newpage


\hiddenchapter{Eigenständigkeitserklärung}
Hiermit versichere ich, dass ich die vorliegende Arbeit selbständig angefertigt und mich keiner fremden Hilfe bedient sowie keine anderen als die angegebenen Quellen und Hilfsmittel benutzt habe. Alle Stellen, die wörtlich oder sinngemäß veröffentlichten oder nicht veröffentlichten Schriften und anderen Quellen entnommen sind, habe ich als solche kenntlich gemacht.\\

Diese Arbeit hat in gleicher oder ähnlicher Form noch keiner Prüfungsbehörde vorgelegen.\\
\section*{Erklärung zu eingesetzten Hilfsmitteln}
\begin{enumerate}
	\item Korrekturservice der Fachhochschule bzw. des Fachbereichs genutzt:
	\begin{todolist}
		\item Ja
		\item[\wontfix] Nein
	\end{todolist}
	\item Einsatz eines externen (kommerziellen) Korrekturservice:
	\begin{todolist}
		\item Ja
		\item[\wontfix] Nein
	\end{todolist}
	\item Folgende Personen haben die Arbeit zusätzlich Korrektur gelesen:
	\begin{itemize}
		\item Natascha Schröder
	\end{itemize}
	\item Nutzung von Sprachmodellen für die Texterstellung (z.B. ChatGPT), wenn ja, welche und in welchen Abschnitten:
	\begin{todolist}
		\item Ja
		\item[\wontfix] Nein
	\end{todolist}
	\item Sprachübersetzungstools (z.B. Google Übersetzer, DeepL), wenn ja, welche und in welchen Abschnitten:
	\begin{todolist}
		\item[\wontfix] Ja
		\item Nein
	\end{todolist}
	\begin{itemize}
		\item DeepL, Im Kapitel Literaturüberblick für das bessere Verständnis der Literatur und zur unterstützten Anfertigung des englischen Abstracts
	\end{itemize}
	\item Einsatz von Software zur Sprachkorrektur (z.B. Grammarly), wenn ja, welche und in welchen Abschnitten:
	\begin{todolist}
		\item Ja
		\item[\wontfix] Nein
	\end{todolist}
	\item Einsatz anderer Hilfsmittel:
	\begin{itemize}
		\item 
	\end{itemize}
	\item Ich stimme dem möglichen Einsatz von Software zur Plagiatserkennung zu:
	\begin{todolist}
		\item[\wontfix] Ja
		\item Nein
	\end{todolist}
\end{enumerate}

Ich bestätige, dass obige Aussagen vollständig und nach bestem Wissen ausgefüllt wurden.\\

Dortmund, den \today \\ \\ \\
\begin{tabular}{@{}l@{}}\hline
	Ricardo Valente de Matos
\end{tabular}

%\bibliographystyle{IEEEtranS}
%\bibliography{Literatur}
\raggedright
\printbibliography

\end{document}
