%Dokumentklasse
\documentclass[a4paper,12pt]{scrreprt}
%\usepackage[left= 3.5cm,right = 2cm, bottom = 2 cm]{geometry}
\usepackage[left= 4.5cm,right = 1.5cm, bottom = 2.5 cm]{geometry}
\addtolength{\footskip}{-0.5cm}
\usepackage[onehalfspacing]{setspace}
% ============= Packages =============

% Dokumentinformationen
\usepackage[hyphens]{url}
\usepackage[
pdfsubject={},
pdfauthor={Ricardo Valente de Matos},
pdfkeywords={},	
%Links nicht einrahmen
hidelinks,
breaklinks=true
]{hyperref}
% Standard Packages
\usepackage[utf8]{inputenc}
\usepackage[ngerman]{babel}
\usepackage[T1]{fontenc}
\usepackage{graphicx, subfig}
\graphicspath{{img/}}
\usepackage{fancyhdr}
\usepackage{lmodern}
\usepackage{color}

\usepackage{dirtree}

\usepackage[style=ieee, sorting=nty, urldate =comp, backend=bibtex]{biblatex}
\addbibresource{Literatur.bib}

%\usepackage[numbers]{natbib}
%\bibpunct{(}{)}{;}{a}{,}{,}

\usepackage{listings}

% zusätzliche Schriftzeichen der American Mathematical Society
\usepackage{amsfonts}
\usepackage{amsmath}
\usepackage{float}

\usepackage{tabularx}
\usepackage{multirow}
\usepackage{enumitem}

%nicht einrücken nach Absatz
\setlength{\parindent}{0pt}
\RedeclareSectionCommand[beforeskip=0pt]{chapter}
\usepackage{listings}
\usepackage{color}
\definecolor{lightgray}{rgb}{.9,.9,.9}
\definecolor{darkgray}{rgb}{.4,.4,.4}
\definecolor{purple}{rgb}{0.65, 0.12, 0.82}

\lstdefinelanguage{JavaScript}{
	keywords={typeof, new, true, false, catch, function, return, null, catch, switch, var, if, in, while, do, else, case, break},
	keywordstyle=\color{blue}\bfseries,
	ndkeywords={class, export, boolean, throw, implements, import, this},
	ndkeywordstyle=\color{darkgray}\bfseries,
	identifierstyle=\color{black},
	sensitive=false,
	comment=[l]{//},
	morecomment=[s]{/*}{*/},
	commentstyle=\color{purple}\ttfamily,
	stringstyle=\color{red}\ttfamily,
	morestring=[b]',
	morestring=[b]"
}

\lstset{
	language=JavaScript,
	backgroundcolor=\color{lightgray},
	extendedchars=true,
	basicstyle=\footnotesize\ttfamily,
	showstringspaces=false,
	showspaces=false,
	numbers=left,
	numberstyle=\footnotesize,
	numbersep=9pt,
	tabsize=2,
	breaklines=true,
	showtabs=false,
	captionpos=b
}


% ============= Kopf- und Fußzeile =============
\pagestyle{fancy}
%
\lhead{}
\chead{}
\rhead{\slshape \leftmark}
%%
\lfoot{}
\cfoot{\thepage}
\rfoot{}
%%
\renewcommand{\headrulewidth}{0.4pt}
\renewcommand{\footrulewidth}{0pt}

% ============= Package Einstellungen & Sonstiges ============= 
%Besondere Trennungen
\hyphenation{De-zi-mal-tren-nung}

\newcommand{\hiddenchapter}[1]{
	\chapter*{{#1}}
}

%-------------

\newcommand{\todo}[1]{\textcolor{red}{ToDo:} #1\marginpar{<--hier}}

% ============= Dokumentbeginn =============

\begin{document}
%Seiten ohne Kopf- und Fußzeile sowie Seitenzahl
\pagestyle{empty}

\begin{center}
	\begin{tabular}{p{\textwidth}}
		
		\begin{center}
			\textbf{\Large{Bachelorarbeit}}
		\end{center} \\ \\
		
		\begin{center}
			\LARGE{\textsc{
					%\textit{\emph{adesso Staffing Advisor Lab}}\\
					%Konzeption und prototypische Entwicklung der Struktur und Architektur einer Softwareplattform für Transparenz in KI-Anwendungen
					...
			}}
		\end{center}
		
		\\
		
		
		
		\begin{center}
			von
		\end{center}
		
		\begin{center}
			\large{\textbf{Ricardo Valente de Matos}}
		\end{center}
	
	\begin{center}
		\large{Matrikelnummer: 7203677} \\
		\large{im Studiengang Wirtschaftsinformatik \\
			der Fachhochschule Dortmund \\}
	\end{center}
		
		
		\\
		
		\\
		
		\begin{center}
			\begin{tabular}{lll}
				\textbf{Erstprüfer:} & & Prof. Dr.-Ing. Guy Vollmer\\
				\textbf{Zweitprüfer:} & & Stephan Schmeißer, M. Sc., Adessoplatz 1, 44269 Dortmund\\
			\end{tabular}
		\end{center}
	
	\\ \\
	
	\begin{center}
		\large{Dortmund, den \today}
	\end{center}
		
	\end{tabular}
\end{center}

%\setcounter{page}{1}
%\pagestyle{plain}

%letztes kapitel zusammenfassung und ausblick
\pagestyle{fancy}
\pagenumbering{Roman}
%\tableofcontents

%\listoffigures
%\lstlistoflistings
\newpage
%-NOCH ZU ERLEDIGEN-\\

%\chapter*{Lesehinweis}
%Aus Gründen der besseren Lesbarkeit werden Wörter und Wortgruppen, die hervorgehoben werden oder mehrfach auftauchen, durch \emph{kursiven} Text kenntlich gemacht. Zudem wird in dieser Projektarbeit die Sprachform des generischen Maskulinums angewandt. Sämtliche Ausführungen sind jedoch geschlechtsunabhängig und beziehen sich damit auf alle Geschlechter.
\newpage

\setcounter{page}{1}
\pagestyle{fancy}
\pagenumbering{arabic}
\setcounter{chapter}{0}
\newpage

\hiddenchapter{Motivation}
In einer globalisierten und dynamischen Wirtschaftswelt sind Unternehmen zunehmend auf Projekte angewiesen, um ihre Ziele zu erreichen und Wettbewerbsvorteile zu erlangen. Die Personalbeschaffung für solche Projekte erfordert oft spezialisiertes Fachwissen und vielfältige Fähigkeiten, um erfolgreich umgesetzt zu werden. Es ist entscheidend für den Projekterfolg, dass die Personalbeschaffung die passenden Mitarbeiter für ausgewählte Projekte findet. Hier setzt die Entwicklung eines Recommender Systems zur Mitarbeiterempfehlung an. Ein solches System kann Unternehmen dabei unterstützen, den Prozess der Mitarbeiterrekrutierung und -auswahl zu optimieren. Durch die Berücksichtigung verschiedener Kriterien wie Qualifikationen, Fähigkeiten und Erfahrungen kann das Recommender-System dazu beitragen, die Auswahl effektiv zu filtern und diejenigen herauszufiltern, die am besten zu einem Projekt im Unternehmen passen. Ein solches System bietet außerdem den Vorteil, den Prozess der Mitarbeiterempfehlung zu automatisieren und zu beschleunigen. Dies ermöglicht Unternehmen, schneller auf offene Stellen zu reagieren und potenzielle Kandidaten zeitnah zu identifizieren. Dadurch wird die Effizienz der Mitarbeitersuche verbessert und die Qualität der Einstellungsentscheidungen erhöht.\\

Das Potenzial von Recommender Systems wurde auch bei \emph{adesso} entdeckt und nun wird nach und nach Wege gesucht, KI-gestützte Systeme in die eigenen Prozesse zu integrieren. Im internen Projekt \emph{adesso Staffing Advisor} wird an einem Recommender-System zur Mitarbeiterempfehlung für ausgewählte Projekte gearbeitet. Die Umsetzung der Recommender Systems bedient sich verschiedener KI-basierten Ansätze. Ein ganz entscheidender Schritt im Prozess der Mitarbeiterempfehlung ist die Vorverarbeitung der Bedarfsmeldungen. Diese sind eine wertvolle Informationsquelle, die Führungskräften helfen kann, die Empfehlungen effizienter zu gestalten, um dadurch wettbewerbsfähig zu bleiben. Allerdings sind diese oft umfangreich, unsortiert und komplex, was ihre effektive Nutzung erschwert. Deshalb ist es entscheidend, effiziente Methoden und Techniken des Information Retrieval anzuwenden, um so relevante Informationen schnell und präzise aus Bedarfsmeldungen zu extrahieren. Die Extraktion wichtiger Schlüsselwörter, Phrasen und Themen ermöglicht es einen besseren Einblick in die Ziele, Methoden und Ergebnisse der Projekte zu bekommen. Dadurch können fundierte Entscheidungen bezüglich der Personalbesetzung getroffen und Ressourcen effizient genutzt werden.\\
\hiddenchapter{Problemstellung}
Um das Entlastungspotenzial für Führungskräfte durch das Gesamtsystem eines Recommender Systems für Mitarbeiterempfehlungen zu realisieren, sind mehrere Schritte notwendig. Eine Informationsgewinnung aus den unstrukturierten Projekt- und Mitarbeiterdaten ist unerlässlich, um schließlich den Ähnlichkeitsvergleich für die Empfehlungen durchführen zu können. Diese Ausarbeitung befasst sich mit dem ersten Schritt der Strukturierung und Informationsextraktion der vorhandenen Bedarfsmeldungen. Somit steht \emph{adesso} vor der Herausforderung, relevante Informationen effizient aus umfangreichen Bedarfsmeldungen zu extrahieren. Obwohl diese Beschreibungen wichtige Einblicke in Ziele, Methoden und Ergebnisse liefern, können sie aufgrund ihres Umfangs und ihrer Komplexität schwer durchsuchbar und analysierbar sein. Die manuelle Identifizierung und Extraktion relevanter Inhalte ist zeitaufwendig und fehleranfällig. Daher stellt sich die Problemstellung: \\

Wie können wir effektive Methoden und Techniken des Information Retrieval und Data-Mining nutzen, um automatisiert relevante Inhalte aus Bedarfsmeldungen im spezifischen Software Entwicklungs-Kontext zu extrahieren und somit die Effizienz, Genauigkeit und Geschwindigkeit der Informationsgewinnung für Führungskräfte zu verbessern.\\

In der Vergangenheit wurden bereits Methoden im Bereich des automatisierten Recruitings untersucht. Im Projektgeschäft sehen wir uns mit einem Problem konfrontiert, dessen Umfang jedoch präziser definiert werden kann, da die Kandidatenauswahl einem begrenzten Pool unterliegt. Besondere Relevanz hat hierbei die Erstellung einer Standardisierung der Bedarfsmeldung, da diese häufig unstrukturiert und mit fehlenden Informationen vorliegt.
\hiddenchapter{Ziele und Ergebnisse der Arbeit}
Diese Ausarbeitung präsentiert eine umfassende Untersuchung zur Entwicklung eines automatisierten Systems zur Extraktion relevanter Inhalte aus Bedarfsmeldungen im Software-Entwicklungs-Kontext.
\begin{itemize}
	\item In der Ausarbeitung wird zunächst ein Konzept einer standardisierten Bedarfsmeldung erarbeitet. Dazu wird eine klare Erwartungshaltung hinsichtlich der Anforderungen und Bedürfnisse der Stakeholder entwickeln. Hierfür werden Interviews mit Führungskräften durchgeführt, um die Erwartungen bezüglich einer \glqq{}perfekten\grqq{} Bedarfsmeldung herauszuarbeiten. Dieses Konzept dient als Grundlage für die weiteren Entwicklungs- und Evaluierungsphasen.
	\item Es wird an einer ausführbaren prototypischen Software gearbeitet, die Bedarfsmeldungen effizient verarbeitet und wichtige Informationen extrahiert. Hierfür wird eine Pipeline in Python aufgebaut und strukturell durch Use-Case- und UML-Diagramme dokumentiert. Es werden Modelle des Information Retrieval und Data-Mining implementiert. Dabei erfolgt zunächst eine eingehende Analyse der Techniken \emph{TF-IDF}, \emph{Text-Ranking-Algorithmen}, \emph{N-Gramm-Analyse}, \emph{POS-Tagging}, \emph{Named Entity Recognition}, Regelbasierte Ansätze und Hybride Ansätze, um die besten Ansätze zur Extraktion relevanter Inhalte zu identifizieren. Diese Analyse bildet die Grundlage für die Konzeptionierung des Software-Prototypen, das eine Kombination der erforschten Ergebnisse darstellt.
	\item Um die Leistungsfähigkeit des entwickelten Systems zu evaluieren, werden Testfälle für reale Bedarfsmeldungen definiert. Dabei wird überprüft, inwieweit das Ergebnis den Erwartungen entspricht. Mit Hilfe einer manuellen Überprüfung werden Abweichungen, Ähnlichkeiten und Anpassungen analysiert, um Erkenntnisse über die inhaltliche Leistung des Systems und die Techniken zu gewinnen, die allein oder in Kombination mit mehreren Ansätzen die wichtigsten Informationen herausfiltern. Da die Dauer eine entscheidende Rolle spielt, werden auch Zeit und Leistung gemessen. Diese Ergebnisse werden mit einem neuen Vorverarbeitungsansatz verglichen, der auf dem Large Language Model basiert. Die Performance, Zeit und Ergebnisqualität des entwickelten Systems soll im Vergleich mit diesem alternativen Ansatz die Stärken und Schwächen des entwickelten Systems aufzeigen, um daraus gegebenenfalls weitere Verbesserungsmöglichkeiten zu identifizieren.
\end{itemize}

\hiddenchapter{Vorgehen und Zeitplan}
Ziel ist es die Arbeit im Juli fertig zu stellen. Die einzelnen Monatsziele können aus der nachfolgenden Tabelle entnommen werden. \\ \\
\begin{tabularx}{1\textwidth} { 
		| >{\raggedright\arraybackslash}X 
		| >{\raggedright\arraybackslash}X | }
	\hline
	Ende April
	& \begin{itemize}
		\item Durchführung der Interviews mit Führungskräften
		\item Zusammentragung der gesamten Literatur
		\item Formulierung der Anforderungen für Bedarfsmeldungen
		\item Implementierung der Software
	\end{itemize}\\
	\hline
	Mai
	& \begin{itemize}
		\item Fertigstellung der Implementierung
		\item Durchführung der Tests mit echten Bedarfsmeldungen
		\item Ausarbeitung schreiben
	\end{itemize}\\
	\hline
	Juni
	& \begin{itemize}
		\item Evaluierung der Ergebnisse
		\item Ausarbeitung zu Ende schreiben
	\end{itemize}\\
	\hline
	Ende Juli
	& \begin{itemize}
		\item Schluss schreiben
		\item Korrekturen
	\end{itemize}\\
	\hline
\end{tabularx}
\newpage

\renewcommand\contentsname{Aufbau der Arbeit}
\tableofcontents

\newpage

\hiddenchapter{Interviewfragen mit Führungskräften zur Identifizierung von Stakeholder-Erwartungen}
\begin{enumerate}
	\item Wer sind die typischen Stakeholder bei der Erstellung von Bedarfsmeldungen und welche
	Rolle spielen sie?
	\item Welche Art von Projekten sind typischerweise in Ihrem Unternehmen an der Tagesordnung?
	Können Sie uns Beispiele für verschiedene Arten von Projekten geben, die adesso
	durchführt?
	\item Wie werden Projektbedarfe und -anforderungen innerhalb von adesso typischerweise
	kommuniziert und dokumentiert?
	\item Welche Informationen halten Sie in einer Bedarfsmeldung für besonders wichtig oder
	unverzichtbar?
	\item Wie detailliert sollten Projektbeschreibungen Ihrer Meinung nach sein? Sind bestimmte
	Schlüsselaspekte oder -informationen in jeder Bedarfsmeldung enthalten?
	\item Wie wird die Qualität von Bedarfsmeldungen bei adesso bewertet? Gibt es bestimmte
	Kriterien oder Standards, anhand derer Bedarfsmeldungen beurteilt werden?
	\item Wie können Sie die Qualität und Klarheit von Bedarfsmeldungen verbessern?
	\item Welche Herausforderungen oder Schwierigkeiten sind bei unklaren oder unvollständigen
	Bedarfsmeldungen aufgetreten?
	\item Welche Auswirkungen haben unklare oder fehlende Informationen in Bedarfsmeldungen
	auf die Effizienz und den Erfolg von Projekten?
	\item Wie können Sie sicherstellen, dass die Bedürfnisse und Anforderungen aller relevanten
	Stakeholder in einer Bedarfsmeldung angemessen berücksichtigt werden?
\end{enumerate}

\newpage
\setcounter{page}{1}
\pagestyle{fancy}
\pagenumbering{arabic}
\setcounter{chapter}{0}

\chapter{Einleitung}
\label{chap:einleitung}

\newpage
\section{Problemstellung}
\label{sec:problemstellung}

\newpage
\section{Ziele und Ergebnisse der Arbeit}
\label{sec:zieleundergebnis}

\section{Aufbau der Arbeit}

\newpage

%\section{Verwandte Arbeiten}
\chapter{Literaturüberblick}
\label{sec:literaturueberblick}
Es gibt eine Reihe an verwandten Arbeiten, die sich mit unterschiedlichen Aspekten des \emph{Staffing}-Prozesses und der Nutzung von Information Retrieval und Filtering zur Informationsgewinnung beschäftigen. Dennoch beschäftigt sich keine Arbeit mit dem spezifischen Problem der Informationsgewinnung aus \emph{Bedarfsmeldungen}, in der Art, wie sie \emph{adesso} verwendet. Es wird ein Einblick in die Art und Weise gegeben wie andere Autoren Information Retrieval einsetzen und kombinieren.\\
\section{Techniken}
Nachfolgend werden Details zu verschiedenen Arbeiten absatzweise dargestellt. Die Quellen stehen jeweils am Anfang jedes Absatzes und beziehen sich aufgetroffene Aussagen innerhalb der Absätze. Diese befassen sich mit unterschiedlichen Techniken, die Lösungen für die Implementierungsebene des zu entwickelnden Systems bieten.
\subsection*{Automatisiertes Staffing}
In der Arbeit \citetitle{horesh2016information}\cite{horesh2016information} beschreiben \citeauthor{horesh2016information} einen Ansatz zur Ableitung von Unternehmensdaten und digitalen Fußabdrücken von Mitarbeitern. Mit Hilfe eines Big-Data-Workflows, der die Komponenten Information Retrieval und Suche, Datenfusion, Matrixvervollständigung und ordinale Regression nutzt, können Informationen zur Expertise automatisch zusammengeführt und für die Nutzung durch Experten aufbereitet werden. Das System soll Fähigkeiten, Talente und Fachwissen der Mitarbeiter in einem breiten Bereich wie cloud computing oder cybersecurity einschätzen. Beim Ansatz des Information Retrieval und -fusion wird eine Liste von Suchbegriffen erstellt, die sich auf das breite Fachgebiet der Mitarbeiter beziehen. Die Suche wird nach jedem dieser Abfragebegriffe durchgeführt, um Zusammenhänge zwischen Mitarbeiter und Datenquellen zu finden. Die verschiedenen Zusammenhänge werden zusammengefügt, gewichtet und nach der Abfrage sortiert. Die Mitarbeiter werden nach Daten gewichtet und bewertet, um einen einzigen Wert (sehr niedrig, niedrig, moderat, etwas, begrenzt) für ihr Fachwissen in diesem breiten Bereich zu erhalten.\\
\subsection*{Vorverarbeitung}
Die Autoren \citeauthor{jain2013data} der Arbeit \citetitle{jain2013data}\cite{jain2013data} beschreiben Data-Mining als ein interdisziplinäres Teilgebiet der Informatik, das sich mit der rechnergestützten Entdeckung von Mustern in großen Datenbeständen befasst \cite{jain2013data}. Ziel dieses fortgeschrittenen Analyseverfahrens ist es, Informationen aus einem Datensatz zu extrahieren und in eine für die weitere Verwendung verständliche Struktur umzuwandeln \cite{jain2013data}. Die verwendeten Methoden liegen an der Schnittstelle zwischen künstlicher Intelligenz, maschinellem Lernen, Statistik, Datenbanksystemen und Business Intelligence \cite{jain2013data}. Beim Data Mining geht es um die Lösung von Problemen durch die Analyse von Daten, die bereits in Datenbanken vorhanden sind \cite{jain2013data}.\\

In der Arbeit \citetitle{alasadi2017review}\cite{alasadi2017review} zeigen die Autoren \citeauthor{alasadi2017review} Wege und Schritte zur Aufbereitung von Datensätzen auf. Die Arbeit umfasst Data-Mining Vorverarbeitungsmethoden, um die Qualität der Daten zu verbessern. Diese weisen wichtige Schritte auf, um die Effizienz in der Datensammlung zu verbessern. Die Datenvorverarbeitung (preprocessing) stellt eine der essenziellen Data-Mining-Aufgaben dar, die die Vorbereitung und Umwandlung von Daten in eine geeignete Form umfasst. Die Datenvorverarbeitung zielt unter anderem darauf ab, die Datenmenge zu reduzieren und Daten zu standardisieren. Die Datenvorverarbeitung umfasst eine Reihe von Techniken, darunter Datenbereinigung, -integration, -transformation und -reduktion. \\

In der Arbeit mit dem Titel \citetitle{kroha2000preprocessing}\cite{kroha2000preprocessing} von dem Autor \citeauthor{kroha2000preprocessing} wird der Teil des Anforderungsspezifikationsprozesses diskutiert, der zwischen der textuellen Anforderungsdefinition und den dazugehörigen Diagrammen der Anforderungsspezifikation liegt. Es wird die These aufgestellt, dass die Erstellung einer textuellen Anforderungsbeschreibung, die das Verständnis des Analysten für das Problem darstellt, die Effizienz der Anforderungsvalidierung durch den Benutzer verbessert. Die vorliegende Idee ist laut dem Autor aus dem Problem entstanden, dass Software-Entwickler nicht immer über die erforderlichen Kenntnisse in den fachlichen Abläufen der Themengebiete verfügen, die für die Erstellung der Software relevant sind. Im Rahmen der Anforderungsdefinition erfolgt eine textuelle Verfeinerung, die als Anforderungsbeschreibung bezeichnet werden kann. Bei der Arbeit mit dem unterstützten Werkzeug \emph{Tessi} ist der Analytiker durch die genannten Vorgaben gezwungen, Anforderungen zu vervollständigen und zu erklären sowie die Rollen der Wörter im Text im Sinne der objektorientierten Analyse zu spezifizieren. Im Rahmen der Vorverarbeitung erfolgt eine Transformation der Requirements durch Templates.
\subsection*{Schlüsselwörter identifizieren}
In der Arbeit \citetitle{ramos2003using}\cite{ramos2003using} haben die Autoren \citeauthor{ramos2003using} die \emph{TF-IDF}-Methode (\emph{Term Frequency-Inverse Document Frequency}) zur Ermittlung der Häufigkeit von Wörtern in einem bestimmten Dokument im Vergleich zum Anteil dieses Wortes im gesamten Dokumenten ermittelt. Die Berechnung erlaubt eine Einschätzung der Relevanz eines bestimmten Wortes in einem bestimmten Dokument. Die Grundidee des Ansatzes besteht darin, dass Wörter, die in einem einzigen Dokument oder in einer kleinen Gruppe von Dokumenten häufig vorkommen, tendenziell höhere \emph{TF-IDF}-Werte aufweisen als häufig vorkommende Wörter wie Artikel und Präpositionen. \emph{TF-IDF} stellt laut den Autoren ein effizientes Verfahren zum Abgleich von Wörtern in einer Anfrage mit Dokumenten dar. Bei Eingabe einer Abfrage zu einem bestimmten Thema durch einen Benutzer kann \emph{TF-IDF} relevante Informationen zu dieser Abfrage in Dokumenten finden. Trotz der Stärken von \emph{TF-IDF}, sind auch seine Grenzen zu berücksichtigen. In Bezug auf Synonyme ist zu beachten, dass \emph{TF-IDF} nicht auf die Beziehung zwischen den Wörtern eingeht. Des Weiteren werden laut den Autoren unterschiedliche Schreibweisen von Wörtern nicht berücksichtigt, was dazu führen kann, dass Wörter fälschlicherweise als nicht so häufig auftauchend deklariert werden, obwohl sie mit leicht abgewandelter Schreibweise häufiger vorkommen.\\
Die Autoren haben die \emph{TF-IDF}-Methode anhand von 1400 Dokumenten getestet. Dazu wurden die \emph{TF-IDF}-Werte berechnet und die ersten 100 Dokumente zurückgegeben. Die zurückgegebenen Dokumente werden in absteigender Reihenfolge zurückgegeben, wobei die Dokumente mit höheren Gewichtssummen zuerst erscheinen. Um die Ergebnisse zu vergleichen, wurde zu mehreren Dokumenten die Anzahl einer bestimmten Anfrage ermittelt.\\

Die Autoren \citeauthor{bafna2016document} der Arbeit \citetitle{bafna2016document}\cite{bafna2016document} haben im Rahmen der Textanalyse die \emph{TF-IDF}-Methode angewandt, um häufige Begriffe zu eliminieren und lediglich die relevantesten Begriffe aus einem Textkorpus zu extrahieren. In der Untersuchung wird der \emph{TF-IDF}-Algorithmus zusammen mit dem \emph{Fuzzy K-means} und dem \emph{hierarchischen} Algorithmus verwendet. In einem ersten Schritt werden Experimente mit einem kleinen Datensatz durchgeführt und eine Clusteranalyse vorgenommen. Im Anschluss erfolgt die Anwendung des vorher besten ermittelten Algorithmus auf den erweiterten Datensatz. In Kombination mit den verschiedenen Clustern der verwandten Dokumente werden der resultierende \emph{Silhouettenkoeffizient}, die \emph{Entropie} sowie der \emph{F1-Score} dargestellt, um das Verhalten des Algorithmus für den Datensatz zu veranschaulichen.\\
\subsection*{Wortketten bilden}
Die Autoren \citeauthor{majumder2002n} beschreiben in der Arbeit \citetitle{majumder2002n}\cite{majumder2002n} \emph{N-Gramme} als Folgen von Zeichen oder Wörtern, die aus einem Text extrahiert werden. Diese lassen sich in zwei Kategorien unterteilen: i) zeichenbasiert und ii) wortbasiert. Ein Zeichen-\emph{N-Gramm} bezeichnet eine Folge von n aufeinanderfolgenden Zeichen, die aus einem Wort extrahiert werden. Die Hauptmotivation hinter diesem Ansatz besteht darin, dass ähnliche Wörter einen hohen Anteil an \emph{N-Grammen} gemeinsam haben werden. In der Regel umfasst ein \emph{N-Gramm} lediglich die am häufigsten auftretenden Wortpaare und verwendet einen Backoff-Mechanismus, um die Wahrscheinlichkeit zu berechnen, die bei der Suche nach dem gewünschten Wortpaar nicht erfolgreich war. Die Analyse von \emph{N-Grammen} erlaubt die Identifikation häufig vorkommender Phrasen oder Begriffe, die als potenzielle Schlüsselwörter bezeichnet werden können. In der Durchführung des Experimentes werden von den Autoren \emph{N-Gramme} verwendet, um die indische Sprache aus mehrsprachigen Dokumentensammlungen zu identifizieren. Dazu werden zunächst \emph{N-Gramm}-Profile für 10 indische Sprachen erstellt. Ein \emph{N-Gramm}-Häufigkeitsprofil wird durch Zählen aller \emph{N-Gramme} in einer Reihe von Dokumenten in einer bestimmten Sprache und deren Sortierung in absteigender Reihenfolge erstellt. Im Falle der Identifizierung einer neuen Dokumentsprache wird ein \emph{N-Gramm}-Profil des Dokuments erstellt und anschließend der Abstand zwischen dem neuen Dokumentprofil und den Sprachprofilen berechnet. Der Abstand wird mit dem \emph{out-of-place measure} zwischen den beiden Profilen berechnet. Der kürzeste Abstand wird ausgewählt und es wird vorhergesagt, ob das bestimmte Dokument zu dieser Sprache gehört. Zur Vermeidung einer Fehlklassifizierung wurde ein Schwellenwert eingeführt, bei dessen Überschreitung das System die Aussage trifft, dass die Sprache des Dokuments nicht bestimmt werden kann.
\subsection*{Wortgruppen identifizieren}
Die Autoren \citeauthor{kumawat2015pos} der Arbeit \citetitle{kumawat2015pos}\cite{kumawat2015pos} beschreiben unterschiedliche Ansätze von \emph{POS-Tagging} (\emph{Part-of-Speech-Tagging}), um einen für indische Sprache zu erstellen. Sie beschreiben, dass die Katalogisierung von Wortarten (\emph{POS}) einen Prozess bezeichnet, bei dem jedem einzelnen Wort eines Satzes ein Wortart-Tag oder ein anderes philologisches Klassenzeichen zugeordnet wird. Die Vorverarbeitungsaufgabe des \emph{Taggings} von Sprachbestandteilen stellt laut den Autoren einen essenziellen Schritt in \emph{NLP} dar.\\

\emph{NLP} (Natural Language Processing) stellt einen zentralen Aspekt im Bereich der künstlichen Intelligenz sowie der Computerwissenschaften dar \cite{kang2020natural}. Studien in diesem Bereich umfassen Theorien und Methoden, die eine Kommunikation zwischen Menschen und Computern in natürlicher Sprache ermöglichen \cite{kang2020natural}. \emph{NLP} vereint die Gebiete Informatik, Linguistik und Mathematik mit dem primären Ziel, menschliche Sprache in Befehle zu übersetzen, die von Computern ausgeführt werden können \cite{kang2020natural}.\\

Die Zuordnung von Wortarten stellt laut den Autoren \citeauthor{kumawat2015pos} der Arbeit \citetitle{kumawat2015pos}\cite{kumawat2015pos} eine grundlegende Aufgabe bei der Verarbeitung natürlicher Sprache dar. Die Erstellung erfolgt unter Zuhilfenahme linguistischer Theorien, zufälliger Muster sowie einer Kombination aus beidem. Ein \emph{POS-Tagger} ist definiert als ein Teil einer Software, der jedem Wort einer Sprache, das er liest, eine Wortart zuordnet. Die Ansätze des \emph{POS-Tagging} lassen sich in drei Kategorien unterteilen: \emph{regelbasiertes Tagging}, \emph{statistisches Tagging} und \emph{hybrides Tagging}. Im Rahmen der Zuweisung von \emph{POS-Tags} zu Wörtern im regelbasierten \emph{POS}-System erfolgt die Verwendung einer Reihe von handgeschriebenen Regeln in Kombination mit Kontextinformationen. Der Nachteil dieses Systems besteht darin, dass es nicht funktioniert, wenn der Text nicht bekannt ist. Das Problem besteht darin, dass das System nicht in der Lage ist, den passenden Text vorherzusagen. Um eine höhere Effizienz und Genauigkeit in diesem System zu erreichen, ist es daher empfehlenswert, einen umfassenden Satz von handkodierten Regeln zu verwenden. Die Häufigkeit und Wahrscheinlichkeit sind in dem statistischen Ansatz einbezogen. Der grundlegende statistische Ansatz basiert auf der am häufigsten verwendeten Markierung für ein bestimmtes Wort in den annotierten Trainingsdaten. Diese Information wird auch zur Markierung dieses Wortes im Text verwendet.\\
Das Experiment aus der Arbeit umfasst einen Datensatz aus 20000 Sätzen mit manuellen Tags in der Sprache Marathi. Damit kann das entwickelte \emph{POS-Tagging}-System getestet werden. 

%\section{Text-Ranking-Algorithmus}
%Text-Ranking-Algorithmen: Text-Ranking-Algorithmen wie TextRank oder YAKE (Yet Another Keyword Extractor) verwenden graphenbasierte Methoden, um Schlüsselwörter in einem Text zu identifizieren. Die Algorithmen bewerten die Wichtigkeit von Wörtern basierend auf ihrer Verbindung zu anderen Wörtern im Text und extrahieren Schlüsselwörter entsprechend ihrer Rangfolge.\\ \cite{mihalcea2004textrank}\cite{zhang2020empirical}\cite{pay2019ensemble}\\

%Die Autoren \citeauthor{mihalcea2004textrank} verwenden in der Arbeit \citetitle{mihalcea2004textrank}\cite{mihalcea2004textrank} die Methode \emph{TextRank}, um Schlüsselwörter in einem natürlichen Text zu ermitteln. Dabei handelt es sich um ein graphenbasierten Text-Ranking-Algorithmus, der eine Möglichkeit darstellt, die Wichtigkeit eines Knotens innerhalb eines Graphen zu bestimmen. Dabei werden globale Informationen berücksichtigt, die rekursiv aus dem gesamten Graphen berechnet werden. Im Gegensatz zu anderen Methoden, die sich lediglich auf lokale, knotenspezifische Informationen stützen, ermöglicht dies eine objektivere Bewertung. Der \emph{TextRank}-Algorithmus zur Schlüsselwortextraktion
%ist ein \emph{unsupervised}-Ansatz. Zunächst wird der Text im Vorverarbeitungsschritt \emph{tokenisiert} und mit \emph{Part-of-Speech-Tags} annotiert, der für die Anwendung syntaktischer Filter notwendig ist. Um ein Wachstum des Graphen durch Hinzufügen aller Kombinationen von Sequenzen, die aus mehr als einem \emph{N-Gramm} bestehen, zu vermeiden, werden nur einzelne Wörter zum Hinzufügung in den Graphen in Betracht gezogen. Anschließend werden alle lexikalischen Einheiten, in den Graphen eingefügt und es wird eine Kante zwischen den lexikalischen Einheiten eingefügt. Nach der Erstellung des Graphen wird jedem Knoten eine Punktzahl mit einen Anfangswert von 1 gegeben. Der Ranking-Algorithmus wird anschließend in mehreren Iterationen auf den Graphen angewendet. Im Anschluss an die Ermittlung der Punktzahlen für jeden Knoten im Graphen erfolgt eine Sortierung der Knoten in umgekehrter Reihenfolge ihrer Punktzahl. Die Knoten mit der höchsten Punktzahl werden für die Nachbearbeitung ausgewählt. Bei Nachbearbeitung werden alle vom \emph{TextRank}-Algorithmus als potenzielle Schlüsselwörter im Text markiert. Anschließend werden Sequenzen benachbarter Schlüsselwörter zusammengefasst.\\

%\cite{zhang2020empirical}\cite{pay2019ensemble}
\subsection*{Datum und Zeitangaben identifizieren}
% \cite{nadeau2007survey}\cite{partalidou2019design}\\
Die Arbeit \citetitle{mansouri2008named}\cite{mansouri2008named} der Autoren \citeauthor{mansouri2008named} vergleichen verschiedene \emph{NER}-Ansätze (Named Entity Recognition) auf die Genauigkeit, um die Stärken und Schwächen einzelner Methoden zu identifizieren. \emph{NER} stellt einen Teilbereich der Informationsextraktion dar und umfasst die Verarbeitung sowohl strukturierter als auch unstrukturierter Dokumente und die Identifizierung von Wörtern, die sich auf Personen, Orte, Organisationen und Unternehmen beziehen. \emph{NER} stellt eine grundlegende Aufgabe eines Systems im Bereich des \emph{NLP} dar. \emph{NER} umfasst laut den Autoren zwei Aufgaben. Die erste Aufgabe besteht in der Identifizierung von Eigennamen im Text. Die zweite Aufgabe ist die Klassifizierung dieser Namen in eine Reihe von vordefinierten Kategorien. Dazu zählen (i)Personennamen, (ii)Organisationen, wie Unternehmen, Regierungsorganisationen, Ausschüsse usw., (iii)Orte, wie Städte, Länder, Flüsse usw., (iv)Datumsangaben und (v)Zeitangaben.\\
%Die Arbeit vergleicht Ansätze mithilfe von den Methoden \emph{Precision}, \emph{Recall} und \emph{F1-Score}
\subsection*{Hybride Ansätze}
In der Untersuchung \citetitle{croft2000combining}\cite{croft2000combining} wird von den Autoren \citeauthor{croft2000combining} die Entwicklung von Kombinationen im Bereich des Information Retrievals analysiert. Dabei werden sowohl experimentelle Ergebnisse als auch die Retrieval-Modelle, die als formale Rahmen für die Kombination vorgeschlagen wurden, berücksichtigt. In der Untersuchung wird aufgezeigt, dass Kombinationsansätze für die Informationssuche als Kombination der Ergebnisse mehrerer Klassifikatoren auf der Grundlage einer oder mehrerer Darstellungen modelliert werden können. Zudem wird dargelegt, dass dieses einfache Modell Erklärungen für viele der experimentellen Ergebnisse liefern kann.\\

Die Arbeit \citetitle{chiny2021lstm}\cite{chiny2021lstm} von den Autoren \citeauthor{chiny2021lstm} kombiniert drei Ansätze des Information Retrievals mit dem Ziel, relevante Informationen aus Produktreviews zu extrahieren. Der Ansatz \emph{TF-IDF} wird mit einem sogenannten \emph{CLASSIFIER} Model kombiniert. Das Klassifikationsmodell verarbeitet drei Eingaben der Modelle \emph{LSTM}, \emph{VADER} und \emph{TF-IDF}. Die Werte dieser Eingaben liegen im Bereich von [0,1]. Die Ausgabe des Klassifikationsmodells ist binär und gibt eine Vorhersage des vollständigen Textes der Modelleingabe aus (positiv oder negativ). Aus einem Datensatz wurden 5000 zufällige Bewertungen ausgewählt, die sich von den für die LSTM- und \emph{TF-IDF}-Modelle verwendeten Trainings- und Testdatensätzen unterscheiden. Die Autoren \citeauthor{chiny2021lstm} haben das System durch die Eingabe des globalen Modells laufen lassen, um die Vorhersagen zu erhalten, die von den \emph{LSTM}-, \emph{VADER}- und \emph{TF-IDF}-Modellen zurückgegeben werden. Anschließend teilten sie diese Ergebnisse in zwei Stapel auf (75 \% für die Trainingsmenge und 25 \% für die Testmenge), um das binäre Klassifizierungsmodell zu trainieren und zu bewerten. Die Evaluation erfolgt durch den Einsatz der Methoden \emph{Precision}, \emph{Recall} und \emph{F1-Score}.\\

Die Arbeit \citetitle{suhasini2021hybrid}\cite{suhasini2021hybrid} von den Autoren \citeauthor{suhasini2021hybrid} befasst sich mit der Filterung von Falschmeldungen. In diesem Beitrag werden hybride Verfahren zur Gewinnung von Merkmalen untersucht, die in dem Gebiet noch nicht gründlich erforscht wurden. Die Anwendung von Hybridsystemen hat sich in einer Vielzahl von Anwendungsbereichen als nützlich erwiesen und zeigen eine Tendenz, die Fehlerquote zu reduzieren, indem sie Techniken wie \emph{TF-IDF} und \emph{N-Grams} verwenden. Es wurden Experimente unter Verwendung von Echtzeit-Twitterdaten durchgeführt. Der Datensatz umfasste ca. 5.800 Tweets, die sich auf Donald-Drummond-Geschichten bezogen. Die Sammlung und Verarbeitung der Tweets erfolgt mit Python. Der Datensatz umfasste Original-Tweets, die als gefälscht und echt gekennzeichnet wurden. Die Genauigkeit der Prognose wurde anhand der verschiedenen Nachrichten evaluiert, die für das Training verwendet wurden und am Ende mit \emph{Precision}, \emph{Recall} und \emph{F1-Score} evaluiert.\\

Im Rahmen der Studie mit dem Titel \citetitle{darmawan2015hybrid}\cite{darmawan2015hybrid} wurde von den Autoren \citeauthor{darmawan2015hybrid} ein hybrider Algorithmus zur Extraktion von Schlüsselwörtern und Kosinusähnlichkeit zur Verbesserung der Satzkohäsion bei der Textzusammenfassung vorgeschlagen. Die vorgeschlagene Methode basiert auf einer Komprimierung von 50 \%, 30 \% und 20 \%, um Kandidaten für die Zusammenfassung zu erstellen. Die Auswertung des Ergebnisses mittels \emph{t-Test} zeigt, dass die vorgeschlagene Methode den Kohäsionsgrad signifikant erhöht.\\
Der Ablauf umfasst die Analyse eines Dokuments mithilfe eines Extraktionsalgorithmus sowie die Berechnung der \emph{TF-IDF}-Werte für jeden Begriff. Anschließend werden alle \emph{TF-IDF}-Werte für jeden Satz summiert. Im nächsten Schritt werden alle Sätze anhand der Summe von \emph{TF-IDF} eingestuft. Das Kompressionsverhältnis bestimmt die Position des Satzrangs. In dieser Studie wird eine Kompression von 50 \% verwendet, was bedeutet, dass die Satzzusammenfassung um 50 \% des Originaltextes gekürzt wird. Nach der Auswahl des Satzes wird dessen Berechnung durchgeführt. Die Ähnlichkeit wird mit der \emph{Cosinus-Ähnlichkeitsmethode} berechnet. Anschließend werden alle Sätze anhand ihrer \emph{Cosinus-Ähnlichkeit} von der höchsten zur niedrigsten sortiert. Der resultierende Text mit neuer Satzanordnung stellt die finale Zusammenfassung dar.\\

\subsection*{Pipeline}
In der Arbeit \citetitle{pirk2019implementierung} \cite{pirk2019implementierung} wird von dem Autor \citeauthor{pirk2019implementierung} eine Pipeline entwickelt, die die \emph{N-Gramm}-Analyse verwendet, um Schlagwörter aus einem Text zu extrahieren und mit verschiedenen Ansätzen von Word-Clouds zu visualisieren. Der Fokus dieser Studie liegt dennoch eher auf der Visualisierung als auf der Informationsgewinnung eines Textes.\\

Die Arbeit \citetitle{lavin2019analyzing}\cite{lavin2019analyzing} von dem Autor \citeauthor{lavin2019analyzing} präsentiert eine Anleitung zur Erstellung einer Pipeline mit Python und \emph{TF-IDF}. Darüber hinaus wird die Relevanz von \emph{TF-IDF} als Vorverarbeitung beim maschinellen Lernen erörtert. Im Vergleich zur rohen Termhäufigkeit weist \emph{TF-IDF} in der Regel einen höheren Vorhersagewert auf. Die Gewichtung von Themenwörtern wird erhöht, um die Bedeutung von Wörtern zu erhöhen, während die Gewichtung von hochfrequenten Funktionswörtern verringert wird. Es werden Verfahren zur Vorverarbeitung von Texten vorgestellt, die eine Umformung in die gewünschte Darstellungsform ermöglichen. Zudem werden Methoden zur Interpretation der Ergebnisse des \emph{TF-IDF}-Verfahrens erörtert. Die vorliegende Arbeit widmet sich zunächst einer detaillierten Betrachtung der zugrundeliegenden Algorithmen und ihrer Funktionsweise. Im Anschluss erfolgt die Implementierung in Python. Die Verwendung der Bibliothek \emph{sklearn} ist dabei von zentraler Bedeutung.\\

In dem Beitrag mit dem Titel \citetitle{partalidou2019design}\cite{partalidou2019design} wird von den Autoren \citeauthor{partalidou2019design} ein maschineller Lernansatz für die Bereiche \emph{POS-Tagging} und \emph{NER} für die griechische Sprache unter Verwendung von \emph{spaCy} erarbeitet und evaluiert. Die Verarbeitung natürlicher Sprache wirft insbesondere bei der Analyse unüblicher Sprachen wie Griechisch Schwierigkeiten auf.\\ Der Datensatz wurde aus Texten einer griechischen Zeitung extrahiert. Die Artikel der Zeitung wurden in verschiedene Kategorien wie beispielsweise Sport, Gesundheit, Wirtschaft und politische Nachrichten eingeteilt. Die Daten bestehen aus einer Reihe von XML-Dateien, die Informationen auf der Ebene von Absätzen, Sätzen und Wörtern enthalten. Im Rahmen der Evaluation wurden verschiedene Parameter getestet, um das optimale Ergebnis zu erzielen. Der Datensatz wurde gemischt und in einen Trainingssatz, einen Testsatz und einen Validierungssatz aufgeteilt. Zur Evaluierung wurden die Methoden des \emph{Precision}, \emph{Recalls} und \emph{F1-Scores} angewendet.

\section{Ergebnis}
In der Literatur finden sich verschiedene Ansätze zur Extraktion relevanter Stichpunkte aus einem Volltext. Eine Methode zur Ermittlung wichtiger Schlüsselwörter in Texten stellt die \emph{TF-IDF}-Methode dar. Innerhalb einer oder mehrerer \emph{Bedarfsmeldungen} lassen sich mit dieser Methode häufig auftauchende Wörter ermitteln. Es besteht somit die Möglichkeit, Wörter aus einer \emph{Bedarfsmeldung} mit anderen \emph{Bedarfsmeldungen} zu vergleichen und die Häufigkeit der Wörter zu berechnen, um somit potenzielle Schlüsselwörter zu ermitteln. %Des Weiteren können graphenbasierte Methoden wie \emph{TextRank} zur Identifizierung von Schlüsselwörtern in Texten herangezogen werden. 
Ein Ansatz zur Bildung von Wortketten ist die Nutzung von \emph{N-Grammen}. Dabei lassen sich Sätze und Absätze zur weiteren Verarbeitung separieren, wodurch Wörter und ihre Zusammenhänge identifizieren lassen. Auch grammatische Kategorien von Wörtern können Rückschlüsse auf potenzielle Schlüsselwörter zulassen. Die \emph{POS-Tagging}-Methode stellt eine Möglichkeit dar, um dieses Ziel zu erreichen. Des Weiteren kann \emph{NER} dazu beitragen Zeitangaben zu identifizieren. Hybride, die sich durch besondere Zusammensetzungen auszeichnen, zeigen Überlegenheiten gegenüber den einzelnen Ansätzen, die für sich genommen jeweils nur eine Teilkompetenz abdecken.

%spam-filter (Empfinde das Thema eventuell als zu unpassend)\\
%-Überblick über verfügbare Methoden, Herausforderungen und zukünftige Forschungsrichtungen im Bereich der Spam-Erkennung, Filterung und Eindämmung von SMS-Spam. Dabei werden auch Methodiken der keyword frequency ratio und Herunterbrechung auf keyword components behandelt \cite{shafi2017review}\\

%In diesem Beitrag werden Studien zu Technologien vorgestellt, die für die Suche und das Abrufen von Informationen im Web nützlich sind. Es wird aufgezeigt, dass Information Retrieval und Ranking im Web-Kontext anders Funktioniert als in einer statischen Datenbank. \cite{kobayashi2000information}\\

%konzeptbasiertes recruiting --> neuer Ansatz

%\section{Relevante Methoden und Techniken im Bereich Information Retrieval und Data-Mining}
%\label{sec:relevante-methoden}
\newpage







\chapter{Entwicklung einer klaren Erwartungshaltung}
\label{chap:erwartungshaltung}
Dieses Kapitel befasst sich mit der Methodologie und Durchführung von Experteninterviews mit dem Ziel 

\section{Beschreibung der Interviews mit Führungskräften zur Identifizierung von Stakeholder-Erwartungen}
\label{sec:beschreibung-der-interviews}

Im Rahmen der vorliegenden Ausarbeitung werden halbstrukturierte Interviews mit Experten aus dem Bereich <> durchgeführt. Sinn und Zweck von Experteninterviews ist die Rekonstruktion spezifischer Wissensbestände oder besonders exklusiver, detaillierter oder umfassender Kenntnisse über bestimmte Wissensbestände und Praktiken.\\

Der Begriff \emph{Experte} bezeichnet eine Person, die über einen privilegierten Zugang zu Informationen verfügt. Die Expertise eines Experten ist jedoch nicht allein durch die Informationen definiert, über die er exklusiv verfügt. Auch die Verantwortung für Problemlösungsentscheidungen ist ein entscheidender Faktor. Diesbezüglich ist Kompetenz erforderlich, die mit Verantwortung und mit Fähigkeiten sowie mit der Bereitschaft, Verantwortung zu übernehmen, verbunden ist. Dabei ist zu beachten, dass Verantwortung, Fähigkeiten und Bereitschaft in der Regel zusammenfallen.\\

Die Experteninterviews zielen darauf ab, qualitative Daten zu erheben, was durch die Transkription der Interviews in Text gelingt. Die Interviews werden als Einzelinterviews durchgeführt, wodurch der Fokus besonders gut auf das spezifische Wissen jedes Befragten gerichtet werden kann. Jeder Interviewpartner reagiert individuell aufgrund seines eigenen Vorwissens auf die Interviewfragen und beeinflusst daher nicht die Aussagen anderer Interviewteilnehmer. Die Ergebnisse der Interviews bilden die Grundlage für die Formulierung der Anforderungen einer optimalen Bedarfsmeldung, welche als Basis für das zu entwickelnde System eingesetzt wird. Jedes Interview wird zu Dokumentationszwecken aufgezeichnet. Um die Interviews strukturiert für die qualitative Inhaltsanalyse vorzubereiten, ist es erforderlich, sie vorher in schriftliche Transkripte umzuwandeln. Im Rahmen dieses Schritts werden der Interviewer mit einem „I“ und die jeweils befragte Person mit „B“ gekennzeichnet. Im Rahmen des Transkriptionsprozesses werden die Aussagen der Interviews anonymisiert. Dies bedeutet, dass sämtliche personenbezogenen Informationen, wie Vor- und Nachname, durch neutrale Bezeichnungen ersetzt werden.\\

Im Vorfeld der Durchführung der Interviews wurde eine Überprüfung der inhaltlichen Verständlichkeit der Fragen sowie ihrer Beantwortbarkeit vorgenommen. Zudem wurde Feedback zur Reihenfolge eingeholt. Zu diesem Zweck wurden die Fragen vorab an eine Führungsperson geschickt und schriftlich beantwortet. Die Fragen wurden den Experten vorab inklusive Kontext des Interviews geschickt, damit diese sich bei Bedarf Gedanken vorab machen können. Zur zeitlichen Begrenzung wird das Interview auf zehn Fragen reduziert. Trotz der vorgegebenen Strukturierung des Interviews wird Raum für spontane Fragen gelassen, um eine natürliche Gesprächsführung zu ermöglichen. Die Fragen dienen als Orientierungshilfe und Leitfaden durch das Interview. Der Leitfaden ist dabei lediglich als inhaltliche Richtlinie zu verstehen, von der situativ abgewichen werden kann.\\

Ablauf der Interviews...

\section{Übersicht der Experten}

\begin{tabularx}{1\textwidth} { 
		| >{\raggedright\arraybackslash}X 
		| >{\raggedright\arraybackslash}X
		| >{\raggedright\arraybackslash}X
		| >{\raggedright\arraybackslash}X | }
	\hline
	Nr. und Datum
	& Profil der Befragten & Durchführungsart & Dauer\\
	\hline
	1. ... & CC-Leiter & schriftlicher Vorabtest & -\\
	\hline
	2. 29.04.2024 & CC-Leiter & Video-Interview & ...min\\
	\hline
	3. 29.04.2024 & CC-Leiter & Video-Interview & ...min\\
	\hline
	4. 30.04.2024 & CC-Leiter & Video-Interview & ...min\\
	\hline
	5. 6.05.2024 & CC-Leiter & Video-Interview & ...min\\
	\hline
\end{tabularx}\\

Die erste Spalte der Tabelle zeigt die vergebene Nummerierung sowie das Datum der Durchführung. Dies dient der vereinfachten Referenzierung innerhalb der nachfolgenden Analyse.
Die zweite Spalte der Tabelle enthält die Rolle bzw. Tätigkeit der Befragten. Aus den Angaben wird ersichtlich, dass zwei von fünf durchgeführten Interviews mit ...




%Zur Beantwortung der Forschungsfragen 2 und 3 dieser Arbeit werden
%Experteninterviews mit E-Learning Experten durchgeführt. Nachdem für die
%Beantwortung der Forschungsfrage eins bereits auf die Theorie eingegangen wurde, soll
%nun diese durch praktische Erfahrungen ergänzt werden. In Forschungsfrage 2 werden
%die Anforderungen der Praktiker an einen kultursensitiven Leitfaden herausgearbeitet.
%Um diese Anforderungen angemessen herauszuarbeiten ist eine ausführliche
%Vorbereitung notwendig. Diese Vorbereitung wird in diesem Kapitel erläutert. Zu
%Beginn dieses Kapitels wird allgemein auf die qualitative Forschung eingegangen und
%im Anschluss daran auf die Vorbereitung der Interviews sowie die Vorgehensweise bei
%der Durchführung der Interviews. Das Kapitel schließt mit einer kurzen Einordnung der
%Interviewpartner, bezüglich Haupttätigkeitsfeld im E-Learning und der
%Mitarbeiteranzahl des Unternehmens, ab. 


%Methodik erklären, siehe Wirtschaftsinformatik Bachelorarbeit (S.34)
%genau die Schritte der Fragen erklären. Warum diese Reihenfolge

%nochmal eine bedarfsmeldung genau erklären und die schritte wie bedarfsmeldung erhalten, gepflegt etc wird erklären

%\begin{enumerate}
%	\item Welche Art von Projekten sind typischerweise in Ihrem Unternehmen an der Tagesordnung? Können Sie uns Beispiele für verschiedene Arten von Projekten geben, die \emph{adesso} durchführt?
%	\item Wie werden Projektbedarfe und -anforderungen innerhalb von \emph{adesso} typischerweise kommuniziert und dokumentiert?
%	\item Welche Informationen halten Sie in einer Bedarfsmeldung für besonders wichtig oder unverzichtbar?
%	\item Wie detailliert sollten Projektbeschreibungen Ihrer Meinung nach sein? Sind bestimmte Schlüsselaspekte oder -informationen in jeder Bedarfsmeldung enthalten?
%	\item Welche Herausforderungen oder Schwierigkeiten sind bei unklaren oder unvollständigen Bedarfsmeldungen aufgetreten?
%	\item Wer sind die typischen Stakeholder bei der Erstellung von Bedarfsmeldungen und welche Rolle spielen sie?
%	\item Wie wird die Qualität von Bedarfsmeldungen bei \emph{adesso} bewertet? Gibt es bestimmte Kriterien oder Standards, anhand derer Bedarfsmeldungen beurteilt werden?
%	\item Wie können Sie die Qualität und Klarheit von Bedarfsmeldungen verbessern?
%	\item Welche Auswirkungen haben unklare oder fehlende Informationen in Bedarfsmeldungen auf die Effizienz und den Erfolg von Projekten?
%	\item Wie können Sie sicherstellen, dass die Bedürfnisse und Anforderungen aller relevanten Stakeholder in einer Bedarfsmeldung angemessen berücksichtigt werden?
%\end{enumerate}
\newpage
g
\newpage
g
\newpage
g
\newpage

\section{Analyse der Ergebnisse und Entwicklung einer klaren Erwartungshaltung für die Bedarfsmeldungen}

1. Transkription der Interviews:\\
Falls du die Interviews aufgezeichnet hast, transkribiere sie vollständig und genau. Dadurch hast du eine schriftliche Version der Aussagen der Experten, die du leichter analysieren kannst.\\

2. Codierung der Daten:\\
Gehe durch die transkribierten Interviews und markiere oder kodiere relevante Themen, Aussagen oder Muster. Verwende dabei Codes oder Kategorien, die sich auf deine Forschungsfragen beziehen.\\

3. Thematische Analyse:\\
Führe eine thematische Analyse durch, indem du die kodierten Daten systematisch durchgehst und nach wiederkehrenden Themen oder Mustern suchst. Identifiziere Gemeinsamkeiten, Unterschiede oder interessante Einsichten, die sich aus den Aussagen der Experten ergeben.\\

4. Triangulation:\\
Vergleiche die Ergebnisse der Experteninterviews mit anderen Quellen, wie beispielsweise der Literatur, Fallstudien oder empirischen Daten. Durch die Triangulation kannst du die Glaubwürdigkeit und Validität deiner Ergebnisse erhöhen.\\

5. Interpretation der Ergebnisse:\\
Interpretiere die identifizierten Themen oder Muster im Kontext deiner Forschungsfragen und -ziele. Versuche zu verstehen, welche Bedeutung oder Implikationen die Aussagen der Experten für deine Forschung haben könnten.\\

6. Reflexion und Kritik:\\
Reflektiere kritisch über die Aussagen der Experten und die gewonnenen Erkenntnisse. Berücksichtige mögliche Einschränkungen oder Bias in den Interviews und betrachte die Ergebnisse aus verschiedenen Perspektiven.\\

7. Integration in die Gesamtanalyse:\\
Integriere die Ergebnisse der Experteninterviews in deine Gesamtanalyse deiner Bachelorarbeit. Verknüpfe sie mit anderen Forschungsergebnissen, theoretischen Konzepten oder empirischen Daten, um ein umfassendes Verständnis deines Forschungsthemas zu entwickeln.\\

8. Darstellung der Ergebnisse:\\
Präsentiere die wichtigsten Ergebnisse und Erkenntnisse aus den Experteninterviews in deiner Bachelorarbeit. Verwende geeignete Zitate oder Beispiele, um die Aussagen der Experten zu veranschaulichen und deine Argumentation zu unterstützen.
\cite{maguire2002user}

im anhang sind die transskripte
wenn man nicht ne größere anzahl an infos hat gucken ob man das halb automatisch evaluieren. Vielleicht kategorisieren. Infos die wichtig sind gucken ob die dann auch nach dem preprocessing drin sind. Regressive tests schreiben.

transformation von bedarfsmeldung zu guter bedarfsmeldung, was ist der fokus von der bedarfsmeldung, wie gut machen die ansätze das, und muss man das dann noch weiter verarbeiten, haben wir alles was wir brauchen mit nur einem algorithmus, inferenz falls parameter fehlt, gibt es einen der alles löst

Fragen in das proposal aufnehmen, führungskraft vorher fragen ob die fragen nice sind.
\newpage
g
\begin{enumerate}
	\item Wer sind die typischen Stakeholder bei der Erstellung von Bedarfsmeldungen und welche
	Rolle spielen sie?
	\item Welche Art von Projekten sind typischerweise in Ihrem Unternehmen an der Tagesordnung?
	Können Sie uns Beispiele für verschiedene Arten von Projekten geben, die adesso
	durchführt?
	\item Wie werden Projektbedarfe und -anforderungen innerhalb von adesso typischerweise
	kommuniziert und dokumentiert?
	\item Welche Informationen halten Sie in einer Bedarfsmeldung für besonders wichtig oder
	unverzichtbar?
	\item Wie detailliert sollten Projektbeschreibungen Ihrer Meinung nach sein? Sind bestimmte
	Schlüsselaspekte oder -informationen in jeder Bedarfsmeldung enthalten?
	\item Wie wird die Qualität von Bedarfsmeldungen bei adesso bewertet? Gibt es bestimmte
	Kriterien oder Standards, anhand derer Bedarfsmeldungen beurteilt werden?
	\item Wie können Sie die Qualität und Klarheit von Bedarfsmeldungen verbessern?
	\item Welche Herausforderungen oder Schwierigkeiten sind bei unklaren oder unvollständigen
	Bedarfsmeldungen aufgetreten?
	\item Welche Auswirkungen haben unklare oder fehlende Informationen in Bedarfsmeldungen
	auf die Effizienz und den Erfolg von Projekten?
	\item Wie können Sie sicherstellen, dass die Bedürfnisse und Anforderungen aller relevanten
	Stakeholder in einer Bedarfsmeldung angemessen berücksichtigt werden?
\end{enumerate}
\newpage
g
\newpage

\chapter{Analyse der Techniken des Information Retrieval und Data-Mining}
\label{chap:staffingadvisor}

\section{Beschreibung der untersuchten Techniken und Ansätze}

\subsection{TF-IDF}
Im Rahmen der Textanalyse wird die TF-IDF-Technik angewendet, welche die häufigsten Begriffe eliminiert und lediglich die relevantesten Begriffe aus einem Textkorpus extrahiert \cite{bafna2016document}. Die TF-IDF-Methode dient der Ermittlung der Häufigkeit von Wörtern in einem bestimmten Dokument im Vergleich zum Anteil dieses Wortes im gesamten Dokumenten \cite{ramos2003using}. Die Berechnung erlaubt eine Einschätzung der Relevanz eines bestimmten Wortes in einem bestimmten Dokument \cite{ramos2003using}. Die Grundidee des Ansatzes besteht darin, dass Wörter, die in einem einzigen Dokument oder in einer kleinen Gruppe von Dokumenten häufig vorkommen, tendenziell höhere TF-IDF-Werte aufweisen als häufig vorkommende Wörter wie Artikel und Präpositionen \cite{ramos2003using}. TF-IDF stellt ein effizientes Verfahren zum Abgleich von Wörtern in einer Anfrage mit Dokumenten dar \cite{ramos2003using}. Bei Eingabe einer Abfrage zu einem bestimmten Thema durch einen Benutzer kann TF-IDF relevante Informationen zu dieser Abfrage in Dokumenten finden \cite{ramos2003using}. In Bezug auf die Bedarfsmeldungen besteht somit die Möglichkeit, Wörter aus einer Bedarfsmeldung mit anderen Bedarfsmeldungen zu vergleichen und die Häufigkeit ihrer Verwendung zu ermitteln, um somit potenzielle Schlüsselwörter zu ermitteln.\\

Trotz der Stärken von TF-IDF, sind auch seine Grenzen zu berücksichtigen. In Bezug auf Synonyme ist zu beachten, dass TF-IDF nicht auf die Beziehung zwischen den Wörtern eingeht. Des Weiteren werden unterschiedliche Schreibweisen von Wörtern nicht berücksichtigt, was dazu führen kann, dass Wörter fälschlicherweise als nicht so häufig auftauchend deklariert werden, obwohl sie mit leicht abgewandelter Schreibweise häufiger vorkommen.
\subsection{Text-Ranking-Algorithmen}
Text-Ranking-Algorithmen: Text-Ranking-Algorithmen wie TextRank oder YAKE (Yet Another Keyword Extractor) verwenden graphenbasierte Methoden, um Schlüsselwörter in einem Text zu identifizieren. Die Algorithmen bewerten die Wichtigkeit von Wörtern basierend auf ihrer Verbindung zu anderen Wörtern im Text und extrahieren Schlüsselwörter entsprechend ihrer Rangfolge.\\ \cite{mihalcea2004textrank}\cite{zhang2020empirical}\cite{pay2019ensemble}\\

Ein graphenbasierter Rangordnungsalgorithmus stellt eine Möglichkeit dar, die Wichtigkeit eines Knotens innerhalb eines Graphen zu bestimmen. Dabei werden globale Informationen berücksichtigt, die rekursiv aus dem gesamten Graphen berechnet werden. Im Gegensatz zu anderen Methoden, die sich lediglich auf lokale, knotenspezifische Informationen stützen, ermöglicht dies eine objektivere Bewertung.\cite{mihalcea2004textrank}

\subsection{N-Gramm}
N-Gramme sind Folgen von Zeichen oder Wörtern, die aus einem Text extrahiert werden. N-Gramme lassen sich in zwei Kategorien unterteilen: i) zeichenbasiert und ii) wortbasiert. Ein Zeichen-N-Gramm bezeichnet eine Folge von n aufeinanderfolgenden Zeichen, die aus einem Wort extrahiert werden. Die Hauptmotivation hinter diesem Ansatz besteht darin, dass ähnliche Wörter einen hohen Anteil an N-Grammen gemeinsam haben werden. In der Regel umfasst ein N-Gramm lediglich die am häufigsten auftretenden Wortpaare und verwendet einen Backoff-Mechanismus, um die Wahrscheinlichkeit zu berechnen, die bei der Suche nach dem gewünschten Wortpaar nicht erfolgreich war. \cite{majumder2002n} Die Analyse von N-Grammen erlaubt die Identifikation häufig vorkommender Phrasen oder Begriffe, die als potenzielle Schlüsselwörter bezeichnet werden können.

\subsection{POS-Tagging}
%Part-of-Speech (POS) Tagging: POS-Tagging wird genutzt, um die grammatischen Kategorien von Wörtern in einem Text zu bestimmen. Durch die Berücksichtigung von Wörtern mit bestimmten POS-Tags wie Substantiven oder Adjektiven können relevante Schlüsselwörter extrahiert werden.\\ \cite{kumawat2015pos}\cite{nakagawa2007hybrid}\\

Die Katalogisierung von Wortarten (POS) bezeichnet einen Prozess, bei dem jedem einzelnen Wort eines Satzes ein Wortart-Tag oder ein anderes philologisches Klassenzeichen zugeordnet wird. Die Vorverarbeitungsaufgabe des Taggings von Sprachbestandteilen stellt einen essenziellen Schritt in der Verarbeitung natürlicher Sprache dar. Die Zuordnung von Wortarten stellt eine grundlegende Aufgabe bei der Verarbeitung natürlicher Sprache dar. Die Erstellung erfolgt unter Zuhilfenahme linguistischer Theorien, zufälliger Muster sowie einer Kombination aus beidem. Ein Part-of-Speech-Tagger (POS-Tagger) ist definiert als ein Teil einer Software, der jedem Wort einer Sprache, das er liest, eine Wortart zuordnet. Die Ansätze des POS-Tagging lassen sich in drei Kategorien unterteilen: regelbasiertes Tagging, statistisches Tagging und hybrides Tagging. Im Rahmen der Zuweisung von POS-Tags zu Wörtern im regelbasierten POS-System erfolgt die Verwendung einer Reihe von handgeschriebenen Regeln in Kombination mit Kontextinformationen. Der Nachteil dieses Systems besteht darin, dass es nicht funktioniert, wenn der Text nicht bekannt ist. Das Problem besteht darin, dass das System nicht in der Lage ist, den passenden Text vorherzusagen. Um eine höhere Effizienz und Genauigkeit in diesem System zu erreichen, ist es daher empfehlenswert, einen umfassenden Satz von handkodierten Regeln zu verwenden. Die Häufigkeit und Wahrscheinlichkeit sind in den statistischen Ansatz einbezogen. Der grundlegende statistische Ansatz basiert auf der am häufigsten verwendeten Markierung für ein bestimmtes Wort in den annotierten Trainingsdaten. Diese Information wird auch zur Markierung dieses Wortes im unannotierten Text verwendet.\cite{kumawat2015pos}

\subsection{Named Entity Recognition}
Named Entity Recognition (NER) \cite{mansouri2008named} \cite{nadeau2007survey}\cite{partalidou2019design}\\

\subsection{Regelbasierte Ansätze}
Regelbasierte Ansätze: Regelbasierte Ansätze verwenden vordefinierte Regeln oder Muster, um Schlüsselwörter zu identifizieren. Dies kann beispielsweise das Extrahieren von Wörtern sein, die häufig im Text vorkommen oder bestimmten Mustern entsprechen.\\

\subsection{Hybride Ansätze}
Hybride Ansätze: Hybride Ansätze kombinieren verschiedene Methoden und Techniken, um eine genauere Extraktion von Schlüsselwörtern zu ermöglichen. (Z.B. Kombination aus TF-IDF-Gewichtung und Text-Ranking-Algorithmen verwendet).\cite{darmawan2015hybrid} \\

\subsection{Data-Mining}
Data-Mining: \cite{jun2001review}\cite{jain2013data}

\subsection{preprocessing}
preprocessing: \cite{garcia2016big}

\subsection{data-fusion}
data-fusion: \cite{famili1997data} \cite{frank2005comparing} \cite{bohne2013data}
\newpage
g
\newpage
g
\newpage
g
\newpage

\section{Bewertung und Auswahl der besten Ansätze für die Extraktion relevanter Inhalte aus Bedarfsmeldungen}

\newpage
g
\newpage

\chapter{Umsetzung}
\label{chap:implementierung}
Auf Basis welcher Methodiken und Ansätze die relevanten Informationen extrahiert werden können, wurde in Kapitel \ref{sec:literaturueberblick} dargestellt. Nun gilt es diese Ansätze in einem System zu implementieren. In diesem Kapitel werden die Funktionalitäten des Systems zusammengetragen. Dabei werden verwendete Technologien und Implementationsaspekte genauer beschrieben.
\section{Beschreibung}
Das System durchläuft verschiedene Schritte, weswegen es als Pipeline implementiert wird. Damit alle Ansätze und Methoden zur Extraktion von Informationen gut funktionieren und vergleichbar bleiben, wird eine Übersetzungsfunktion der \emph{Bedarfsmeldungen} hinzugefügt. Auch wenn die \emph{Bedarfsmeldungen} in den meisten Fällen auf Deutsch sind, hilf es diese zu übersetzen, damit keine Unterschiede in der Ergebnisqualität resultiert, da einige Methoden und Ansätze auf Basis von englischen Trainingssätzen trainiert wurden. Schließlich müssen alle aus Kapitel \ref{sec:literaturueberblick} untersuchten Ansätze implementiert und nutzbar sein. Sie sollen die Möglichkeit haben \emph{Bedarfsmeldungen} als Input zu erhalten und eine strukturierte \emph{Bedarfsmeldung} als Ausgabe zurückzugeben. Zur vereinfachten Entwicklung soll das System modular sein, damit Methoden und Ansätze nach belieben durchgetauscht und verwendet werden können.
\section{Konkreter Ablauf der Pipeline}
Dieses Kapitel beschreibt den Ablauf der Pipeline. Dazu wird das Aktivitätsdiagramm aus der Abbildung \ref{fig:ablaufsystemabstrakt} verfeinert und mit allen analysierten Komponenten aus Kapitel \ref{sec:literaturueberblick} genauer beschrieben. Die Abbildung \ref{fig:flowchart} zeigt das UML-Aktivitätsdiagramm der Python Pipeline zur Strukturierung von \emph{Bedarfsmeldungen}. Im ersten Schritt \emph{unstrukturierte Bedarfsmeldung strukturieren} im Ablauf der Pipeline wird eine \emph{Bedarfsmeldung} ausgewählt und in die festgelegte \emph{Bedarfsmeldungsstruktur} aus Kapitel \ref{sec:strukturierungbedarfsmeldung} umgebaut. Die Felder \emph{Einsatzbeginn} und \emph{Einsatzende} werden zu einem Feld \emph{Einsatz} zusammengefügt. Neben den Feldern \emph{Aufgaben} und \emph{Skills} bleiben alle weiteren Felder bis zum letzten Punkt Ausgabe unverändert. Die Felder \emph{Aufgaben} und \emph{Skills} werden jeweils in die Schleife weitergeleitet und auf Stichpunkte reduziert.
\begin{figure}[H]
	\centering  
	\includegraphics[width=\linewidth]{Abbildungen/flowchart.png}
	\caption{UML-Aktivitätsdiagramm der Pipeline.}
	\label{fig:flowchart}
\end{figure}\mbox{} \\
Der Prozess zur Reduktion durchläuft mehrere Schritte. Zu Beginn werden die Volltexte aus den Feldern im Punkt \emph{ins Englische übersetzen} übersetzt. Anschließend erfolgt im Schritt \emph{Datums-Daten und Zeiten mit NER extrahieren} zum einen die Extrahierung der zeitbezogenen Daten. Darauffolgend wird der Volltext im Punkt \emph{Preprocessing} für die weitere Nutzung vorverarbeitet. Der Grund warum die Extraktion mit \emph{NER} nicht nach der Vorverarbeitung durchgeführt werden kann ist, da im Vorverarbeitungsschritt alle Nummern und somit auch alle zeitbezogenen Daten entfernt werden. Die Resultate des \emph{NER} werden zum Ende hin zurück ins deutsche übersetzt und der fertigen Stichpunktliste beigefügt. Zur Erstellung der Stichpunktliste werden relevante Schlüsselwörter durch die \emph{TF-IDF}-Methode ermittelt. Anschließend wird der vorverarbeitete Text zu \emph{Bi-Grammen} umgeformt. Weshalb für die \emph{n-Gramme} \emph{Bi-Gramme} verwendet werden, wird in Kapitel \ref{sec:ngram} näher erläutert. Beim Punkt \emph{Bi-Gramme auf Schlüsselwörter filtern} werden alle \emph{Bi-Gramme} entfernt, die kein Schlüsselwort aus der \emph{TF-IDF}-Methode enthalten. Somit erhält der Schritt \emph{untypische Wortartenkombinationen mit POS-Tagging entfernen} eine reduzierte Liste mit \emph{Bi-Grammen}, bei dem mindestens eines der beiden \emph{Bi-Gramm}-Wörter ein Schlüsselwort darstellt. Jedes Wort der \emph{Bi-Gramm}-Liste durchläuft eine weitere Filterung. Im Englischen existieren Wortarten, die typischerweise nicht nebeneinander stehen. Durch Entfernung dieser \emph{Bi-Gramme} durch POS-Tagging-Kombinationen erfolgt eine weitere Trennung der Wortketten in der \emph{Bi-Gramm}-Liste. Wörter die nicht zusammengehören, verlieren dadurch die Verbindung zueinander. Im darauf folgenden Schritt \emph{Bi-Gramm zu Stichpunkten überführen} werden die \emph{Bi-Gramme} jeweils zu einem String zusammengefügt, bei dem im darauf Folgendem \emph{Bi-Gramm} ebenfalls ein relevantes Wort enthalten sind. Dadurch formt sich eine Liste mit Stichpunkten bestehend aus eins bis x vielen Wörtern, die Bezug zueinander haben. Im Schritt \emph{ins Deutsche übersetzen} wird die Liste mit Stichpunkten zurück ins Deutsche übersetzt. Dieser Ablauf wird für die beiden Felder \emph{Aufgaben} und \emph{Skills} durchlaufen, da diese Volltextfelder darstellen.
\section{Implementationsdetails}
Dieses Kapitel beschreibt den technischen Entwicklungsprozess zur Umsetzung der Anforderungen des Systems. Die Implementierung fokussiert sich auf die Umsetzungen von Technologien und Funktionsweisen verschiedener Anforderungen. %Zudem wird die Struktur des Projektes aufgezeigt.
\subsection{Pipeline}
Die Pipeline wurde in der Programmiersprache Python umgesetzt. Python hat sich zu einer der populärsten interpretierten Programmiersprachen entwickelt \cite{mckinney2012python}. Die Programmiersprache eignet sich insbesondere für die Erstellung kleiner Programme und Skripte, die zur Automatisierung von Aufgaben eingesetzt werden können \cite{mckinney2012python}. Python hat eine große und aktive Community für wissenschaftliche Berechnungen und Datenanalysen hervorgebracht und hat sich in den letzten Jahren zu einer der wichtigsten Sprachen für Data Science, maschinelles Lernen und allgemeine Softwareentwicklung in Wissenschaft und Industrie entwickelt \cite{mckinney2012python}. Python unterstützt Modularität, wodurch ein Teil der Anforderungen somit abgedeckt werden kann. Die Implementierungen der einzelnen Verfahren aus den Anforderungen werden nicht manuell, sondern auf Basis von bereits existierende Bibliotheken umgesetzt.
\subsection{Projektstruktur}
\label{sec:projektstruktur}
Das Projekt wird in einem git Repository gespeichert und versioniert. %Die Projektstruktur ist ohne zusätzliche Konfigurationsdateien wie folgt aufgebaut:
%\dirtree{%
%	.1 .git/.
%	.2 modules/.
%	.3 ner.py.
%	.3 nGram.py.
%	.3 posTagging.py.
%	.3 preprocessing.py.
%	.3 readRequirements.py.
%	.3 textRankingAlgorithm.py.
%	.3 tfIdf.py.
%	.3 transformRequirements.py.
%	.3 translate.py.
%	.2 requirements/.
%	.3 jiraTickets.json.
%	.3 preprocessedRequirements.json.
%	.3 ....
%	.2 app.py.
%	.2 ....
%}
Die einzelnen Module sind in dem Verzeichnis \url{modules/} in separaten \url{.py} Dateien gelagert. Die \emph{Bedarfsmeldungen} werden im \url{requirements/}-Verzeichnis gespeichert. Alle relevanten \emph{Bedarfsmeldungen} sind in einer \url{jiraTickets.json}-Datei als Liste gespeichert. Die Datei \url{app.py} ist der Kern der Pipeline. Diese importiert alle Module und implementiert die Struktur der Pipeline. Das Projekt kann über den Befehl \lstinline{> py app.py} ausgeführt werden.
\subsection{Modulimplementationen}
Nachfolgend werden Implementationsdetails zu den einzelnen Modulen gegeben. Dabei werden verwendete Bibliotheken und Code-Details näher erläutert.
\paragraph{Strukturierung von Bedarfsmeldungen}\mbox{}\\
Die \emph{Bedarfsmeldungen} wurden über die Jira-Schnittstelle extrahiert und im Verzeichnis \url{requirements/jiraTickets.json} gespeichert. Zur Eingrenzung der Datenmenge wurde ein Filter angewendet, der nur die \emph{offenen} und \emph{eskalierten} \emph{Bedarfsmeldungen} zurückgibt. Diese sind die relevanten und noch aktuellen \emph{Bedarfsmeldungen}. Zudem wurden alle unrelevanten Felder mit einem weiteren Filter herausgenommen. Die Daten aus der Jira-API bestehen namentlich aus \emph{customfields} mit einer angehangenden ID. Der Softwareprototyp lädt in dem Modul \emph{readRequirements.py} die unstrukturierten Fields und formt diese in dem Modul \emph{transformRequirements.py} in \emph{Bedarfsmeldungs}-Objekte um.
\begin{center}
	\begin{tabularx}{1\textwidth} { 
			| >{\raggedright\arraybackslash}X 
			| >{\raggedright\arraybackslash}X
			| >{\raggedright\arraybackslash}X | }
		\hline
		Display-Felder & Jira-API-Felder & Objekt-Felder \\
		\hline
		\hline
		Überschrift & summary & header\\
		\hline
		Rolle & customfield\_15321 & role\\
		\hline
		Aufgaben & customfield\_10288 & tasks\\
		\hline
		Skills & customfield\_10296 & skills\\
		\hline
		Skill-Level & customfield\_15322 & skillLevel\\
		\hline
		Kunde & customfield\_10279 & customer\\
		\hline
		Einsatzort & customfield\_10297 & location\\
		\hline
		Beginn & customfield\_10293 & timePeriod\\
		\hline
		Ende & customfield\_10294 & timePeriod\\
		\hline
		Tagessatz & customfield\_10298 & dailyRate\\
		\hline
	\end{tabularx}\\
	\captionof{table}{Übersicht der Datenfelder}
	\label{tab:jiradaten}
\end{center}
In der Tabelle \ref{tab:jiradaten} ist in der ersten Spalte eine Übersicht der Datenfelder, wie diese in der Abbildung \ref{fig:jiraafter} mit dem Mockup einer standardisierten \emph{Bedarfsmeldung} auftauchen. In der zweiten Spalte sind die dazugehörigen Feldernamen, die in der \url{jiraTickets.json} enthalten sind. Die dritte Spalte spiegelt die jeweiligen Namen innherhalb des  \emph{Bedarfsmeldungs}-Objekts im Prototypen wieder. Das  \emph{Bedarfsmeldungs}-Objekt dient der strukturierten Handhabung der \emph{Bedarfsmeldungs}-Daten innerhalb des Systems.

%Um einen Freitext aus einer einzelnen \emph{Bedarfsmeldung} für die Pipeline zu laden werden Daten mit der Endung \url{.txt} verwendet.
%\begin{lstlisting}[caption={Implementation der Methode read() des Moduls \emph{readRequirements.py}}, label=lst:read]
%	def read(filename):
%		path = os.path.join(PATH, filename)
%		file = open(path, "r", encoding="utf-8")
%		content = file.read()
%		file.close()
%		return content
%\end{lstlisting}
%Im Listing \ref{lst:read} ist die Implementierung der Methode zum laden eines Freitextes dargestellt. Durch ein mitgelieferten Parameter \emph{filename} wird der Name der Datei in Zeile 2 an den Pfad angehängt. Mit der Methode \lstinline{open()}
%aus Zeile 3 kann die Datei geladen werden. Die Methode erhält die Parameter Pfad, \emph{'r'} (read) und das encoding \emph{'utf-8'}. Das encoding ist dabei Entscheidend, damit die unstrukturierten \emph{Bedarfsmeldungen} laden können. Bei der Pflege in Jira wird wenig Wert auf eine einheitliche Struktur. Somit können Zeichen enthalten sein, die beim öffnen nicht erkannt werden und eine Fehlermeldung wird zurückgegeben. Zur Vermeidung dieses Fehlers wird das encoding festgelegt. Nach dem laden durch die Methode \lstinline{read()} wird der Inhalt der \url{.txt} Datei in der Variable \emph{content} gespeichert und zurückgegeben.
\paragraph{Übersetzung}\mbox{}\\
Das Modul \emph{translate.py} ist dazu da, um die \emph{Bedarfsmeldungen} zu übersetzen. Für die Übersetzung wurde die Python Bibliothek \emph{deep-translator} verwendet. Diese bietet Implementationen unterschiedlicher Übersetzungs-APIs von diversen Anbietern. Der Vorteil ist dabei die vereinfachte Möglichkeit Anbieter bei Bedarf zu wechseln.
%\begin{lstlisting}[caption={Implementation des Moduls \emph{translate.py}}, label=lst:translate]
%	from deep_translator import GoogleTranslator
%	
%	def translate(text):
%		translated = GoogleTranslator(source='auto', target='en').translate(text)
%		return translated
%\end{lstlisting}
%Das Listing \ref{lst:translate} zeigt die Implementation des Moduls. 
Es wurde sich für den Google Translator entschieden, da hierfür kein API-Key benötigt wird. Die Methode erhält einen Text als Parameter. Der Google Translator erhält die Parameter \emph{source} und \emph{target}, bei dem \emph{source} angibt in welcher Sprache der Eingabetext ist. Durch Angabe von \emph{'auto'} wird die Sprache ermittelt. Der Grund dafür ist, dass grundsätzlich anderssprachige \emph{Bedarfsmeldung} enthalten sein können. Der Parameter \emph{target} ist die Zielsprache in die der Input übersetzt werden soll. Die Zielsprache ist hier Englisch (\emph{'en'}).
\paragraph{NER}\mbox{}\\
Um die Methode \emph{NER} zu Implementieren wurde die Bibliothek \emph{spaCy} und das Modell \emph{en\_core\_web\_sm} verwendet. 
%\begin{lstlisting}[caption={Implementation des Moduls \emph{ner.py}}, label=lst:ner]
%	import spacy
%	nlp = spacy.load("en_core_web_sm")
%	ner_categories = ["ORG","FAC","GPE","PRODUCT", "EVENT", "LANGUAGE", "DATE", "QUANTITY"]
%	def useNER(requirement):
%		tokenized = nlp(requirement)
%		entities = []
%		for ent in tokenized.ents:
%			if ent.label_ in ner_categories:
%				entities.append((ent.text, ent.label_))
%	return entities
%\end{lstlisting}
Die zu extrahierende Kategorie ist Datum (DATE). Innerhalb der Methode \lstinline{useNER()} wird der Volltext als Parameter übergeben und zu Tokens umgeformt. Anschließend werden alle Tokens durchlaufen und nach ihren Kategorien überprüft. Ist ein Token die definierte Kategorie, wird der Token in eine separate Liste gespeichert und zurückgegeben. Die extrahierten Tokens werden aus dem Volltext entfernt, um am ende keine doppelten Stichpunkte zu erhalten.
\paragraph{Preprocessing}\mbox{}\\
Vor der weiteren Nutzung der Daten innerhalb einer \emph{Bedarfsmeldung}, ist es erforderlich diese von irrelevanten Wörtern, Zeichen und Formatierungen zu befreien. % Zur Entfernung von Satzzeichen wurde die Bibliothek \emph{string} verwendet.
%\begin{lstlisting}[caption={Implementation der Methode removePunctuation() des Moduls \emph{preprocessing.py}}, label=lst:punctuation]
%	import string
%	def removePunctuation(text):
%		content=""
%		for i in text: 
%			if i not in string.punctuation:
%				content+=i    
%		return content
%\end{lstlisting}
%Im Listing \ref{lst:punctuation} ist die Implementierung dargestellt. Die \emph{Bedarfsmeldung} wird in Zeile 2 als Parameter übergeben. In Zeile 4 erfolgt ein Durchlauf jedes Zeichens innerhalb der \emph{Bedarfsmeldung}. Falls innerhalb der Schleife das Aktuelle Zeichen kein Satzzeichen enthält wird dieses in die Variable \emph{content} zwischengespeichert. Nach Abschluss der Schleife wird die Variable zurückgegeben. 
Zur Eliminierung wiederaufgetretener Wörter, die keine Relevanz für den Informationsgehalt aufweisen, wurde die Bibliothek \emph{nltk} verwendet. Diese beinhaltet eine Liste an sogenannten \emph{stopwords}.
%\begin{lstlisting}[caption={Implementation der Methode removeStopwords() des Moduls \emph{preprocessing.py}}, label=lst:stopwords]
%	from nltk.corpus import stopwords
%	def removeStopwords(text):
%		words=[word for word in text.split(" ") if word not in set(stopwords.words('english'))]
%		return " ".join(str(word) for word in words)
%\end{lstlisting}
%Im Listing \ref{lst:stopwords} ist die Implementierung zur Entfernung von \emph{stopwords} dargestellt. In Zeile 3 
Es werden alle mit einem Leerzeichen getrennten Wörter aus der übergebenen \emph{Bedarfsmeldung} in einer Liste aufgeteilt. Dabei wird jeder Listeneintrag mit der \emph{stopword}-Liste von \emph{nltk} verglichen. Stimmt das Wort nicht mit einem Eintrag der \emph{stopwords} überein, wird diese in die Liste \emph{words} hinzugefügt. Zum Schluss werden die Wörter wieder zu einem String zusammengetragen und zurückgegeben. \\

Um weitere Formatierungen und ungewünschte Zeichen zu entfernen wird die Bibliothek \emph{re} verwendet. Diese kann Regular Expression-Patterns anwenden und Bereiche, die zum Pattern passen entfernen.
%\begin{lstlisting}[caption={Implementation der Methode removeTags() und removeSpecialCharactersAndDigits() des Moduls \emph{preprocessing.py}}, label=lst:re]
%	import re
%	def removeTags(text):
%		return re.sub("</?.*?>"," <> ",text)
%	
%	def removeSpecialCharactersAndDigits(text):
%		return re.sub("(\\d|\\W)+"," ",text)
%\end{lstlisting}
%Im Listing \ref{lst:re} sind Implementationsdetails zur Entfernung von Tags und ungewollte Zeichen dargestellt. 
Die Methode \lstinline{removeTags()} erhält die Expression \lstinline{</?.*?>}. Dabei werden \lstinline{<Tags>} ermittelt und mit der Methode \lstinline{sub()} entfernt. Zur Entfernung von Ziffern und Nicht-Alphanummerischen Zeichen wird die Expression \lstinline{(\\d|\\W)+} angewendet. Als Ergebnis des Preprocessing wird ein gesäuberter String ohne Zeichen und Tags zurückgegeben. Zum schluss werden alle Wörter in Kleinbuchstaben umgewandelt. Der Grund dafür ist, dass somit die Vergleichbarkeit der Wörter gefördert wird. Es könnte sonst vorkommen, dass Beispielsweise beim Schlüsselwörterabgleich zwei Wörter nicht als identisch identifiziert werden, da sie einmal mit großem und eimal mit kleinem Anfangsbuchstaben geschrieben wurde.
\paragraph{TF-IDF}\label{tfidf}\mbox{}\\
Vorbereitend für die \emph{TF-IDF}-Methode wurden die \emph{Skills} und \emph{Aufgaben} aus allen \emph{Bedarfsmeldungen} ins Englische übersetzt, Preprocessed und in einem String zusammengefasst. Der Grund dafür ist, dass für die \emph{TF-IDF}-Methode ein Textkorpus benötigt wird, woraus die Schlüsselwörter durch \emph{Term Frequency} ermittelt werden. Damit dieser Prozess nur einmal erfolgen muss, wurden die Ergebnisse in die \url{requirements/preprocessedRequirements.json} gespeichert. Für die Implementierung des \emph{TF-IDF} wurden die Bibliotheken \emph{sklearn} und \emph{numpy} verwendet. Der Textkorpus wird dem \emph{TfidfVectorizer} von \emph{sklearn} beigefügt und die \emph{TF-IDF}-Werte werden berechnet. Anschließend werden alle durchschnittlichen \emph{TF-IDF}-Werte für jedes Wort im gesamten Textkorpus berechnet. Daraus wird eine Liste mit Wörtern und ihren durchschnittlichen \emph{TF-IDF}-Werten erstellt und in absteigender Reihenfolge sortiert. Durch ein vordefinierten Score Threshold können darunterliegende Schlüsselwörter entfernt werden. Dies ist relevant, um Wörter mit niedrigem Scoring und diese Beispielsweise nur einmal im Textkorpus auftauchen nicht als Schlüsselwörter erfasst werden. Als Ergebnis wird eine Liste mit Schlüsselwörtern zurückgegeben.

%\begin{lstlisting}[caption={Implementation des Moduls \emph{tfIdf.py}}, label=lst:postagging]
%import modules.readRequirements as readRequirements
%from sklearn.feature_extraction.text import TfidfVectorizer
%import numpy as np
%def useTfIdf(text):
%	object = readRequirements.loadJson("preprocessedRequirements.json")
%	documents = [object['tasks'], object['skills']]
%	vectorizer = TfidfVectorizer()
%	tfidf_matrix = vectorizer.fit_transform(documents)
%	feature_names = vectorizer.get_feature_names_out()
%	avg_tfidf_scores = np.mean(tfidf_matrix.toarray(), axis=0)
%	tfidf_scores = list(zip(feature_names, avg_tfidf_scores))
%	sorted_tfidf_scores = sorted(tfidf_scores, key=lambda x: x[1], reverse=True)
%	score_threshold = 0.1
%	filtered_keywords = [word for word, score in sorted_tfidf_scores if score > score_threshold]
%	return filtered_keywords
%\end{lstlisting}

%\paragraph{TextRank}\mbox{}\\
%Für die Implementation von \emph{TextRank} wurde die Bibliothek \emph{spaCy} und \emph{pytextrank} verwendet. Dieses bietet die NLP-Pipeline \emph{en\_core\_web\_sm}, das mit Internettext vortrainiert wurde und Vokabeln, Syntax und Entitäten enthält.
%\begin{lstlisting}[caption={Implementation des Moduls \emph{textRankingAlgorithm.py}}, label=lst:textrank]
%	import spacy
%	import pytextrank
%	def useTextRank(requirement):
%		nlp = spacy.load("en_core_web_sm")
%		nlp.add_pipe("textrank")
%		tokenized = nlp(requirement)
%		for phrase in tokenized._.phrases:
%			print(phrase.text)
%			print(phrase.rank, phrase.count)
%			print(phrase.chunks)
%\end{lstlisting}
%Das Listing \ref{textrank} zeigt die Implementierung des \emph{textRankingAlgorithm.py} Moduls. In Zeile 3 wird die \emph{Bedarfsmeldung} als Parameter übergeben. Die Pipeline \emph{en\_core\_web\_sm} wird in Zeile 4 geladen und in Zeile 5 werden die \emph{TextRank} Elemente der Pipeline hinzugefügt. Anschließend wird in Zeile 6 die \emph{Bedarfsmeldung} der Pipeline eingefügt und in Tokens umgewandelt.\\
%\todo{muss noch fertiggestellt werden}\\
\paragraph{N-Gramm}\label{sec:ngram}\mbox{}\\
Zur Bildung von Wortketten wie in Kapitel \ref{sec:anforderungsanalyse} beschrieben, ist es notwendig \emph{bi-Gramme} zu verwenden. Dadurch können immer zwei nebeneinanderstehenden Wörter isoliert betrachtet werden. Bei einem höheren n der \emph{n-Gramme} würde somit das Prinzip einer Wortkette, wie sie im System benötigt wird nicht funktionieren. Die Methode der \emph{n-Gramme} wurde mit der Bibliothek \emph{nltk} implementiert.
%\begin{lstlisting}[caption={Implementation des Moduls \emph{nGram.py}}, label=lst:ngram]
%	from nltk import ngrams
%	n = 2
%	def useNGram(text):
%		nGramList = []
%		nGrams = ngrams(text.split(), n)
%		for grams in nGrams:
%			nGramList.append(grams)
%		return nGramList
%\end{lstlisting}
%Das Listing \ref{lst:ngram} zeigt die Implementation des Moduls \emph{nGram.py}. 
Eine Variable \emph{n} ist definiert, die die Größe eines \emph{n-Gramms} widerspiegelt. Da in der Pipeline \emph{Bi-Gramme} benötigt werden, liegt der Wert von n auf 2. Mit der Methode \lstinline{ngrams()}
können die \emph{n-Gramme} generiert werden. Diese erhalten einen \emph{String} und die Variable \emph{n} als Parameter. Der zu überführende Text wird als Parameter übergeben und durch die Methode \lstinline{split()}
in einer List auf die einzelnen Wörter umgeformt. Die einzelnen Tupel mit den \emph{Bi-Grammen} werden am Ende zurückgegeben.
\paragraph{Entfernung von Bi-Grammen ohne Schlüsselwörter}\mbox{}\\
Zur Entfernung von Bi-Gramm-Tupel, die keine Schlüsselwörter enthalten, wurde die Methode \lstinline{containsKeywords()} erstellt.
\begin{lstlisting}[caption={Implementation der Filterung für Schlüsselwörter in einer Bi-Gramm Liste}, label=lst:bigram]
	def containsKeywords(biGramList, keywordsList):
		filteredBiGram = []
		for tupel in biGramList:
			if any(word in tupel for word in keywordsList):
				filteredBiGram.append(tupel)
		return filteredBiGram
\end{lstlisting}
Das Listing \ref{lst:bigram} implementiert diese Methode. Als Parameter wird die \emph{Bi-Gramm}-Liste und die Schlüsselwörterliste übergeben. In Zeile 3 ist zu sehen, dass jeder Tupel traversiert wird. Dabei wird in Zeile 4 jedes Wort im Tupel mit der Schlüsselwörterliste verglichen. Wenn das aktuelle Wort in der Liste enthalten ist, so wird dieser Tupel in eine neue Liste aufgenommen. Diese wird am Ende zurückgegeben, wodurch eine gefilterte Liste mit ausschließlich enthaltenen Schlüsselwörtern vorhanden ist.
\paragraph{POS-Tagging}\label{postagging} \mbox{}\\
In der Englischen Sprache existieren verschiedene Wortgruppenkombinationen, die zusammen ungewöhnlich klingen und dadurch beim sprechen und schreiben nicht oder nur selten verwendet werden. Wenn diese Kombinationen in den \emph{Bi-Grammen} identifiziert und entfernt werden, können somit Verbindungsketten unterbrochen werden. Dadurch stehen Wörter und spätere Stichpunkte nicht nebeneinander, die sprachlich im betrachteten Kontext wenig Sinn machen. \emph{Bedarfsmeldungen} wurden im professionell seriösen Kontext angefertigt, wodurch einige nebeneinanderstehenden Wörter im Englischen nicht häufig zusammen auftauchen.
\begin{center}
	\begin{tabularx}{1\textwidth} { 
			| >{\raggedright\arraybackslash}X 
			| >{\raggedright\arraybackslash}X
			| >{\raggedright\arraybackslash}X | }
		\hline
		Wortarten & POS-Tagging Abkürzungen & Beispiel \\
		\hline
		\hline
		Nomen + Adjektiv & NN JJ & "time beautiful"\\
		\hline
		Verb + Adjektiv & VB JJ & "think beautiful"\\
		\hline
		Verb + Pronomen & VB PRP & "think him"\\
		\hline
		Verb + Verb & VB VB & "think understand"\\
		\hline
		Adjektiv + Adjektiv & JJ JJ & beautiful strange\\
		\hline
		Adjektiv + Verb & JJ VB & beautiful think\\
		\hline
		Adjektiv + Adverb & JJ RB & "beautiful fast"\\
		\hline
		Adverb + Adjektiv & RB JJ & "fast beautiful"\\
		\hline
	\end{tabularx}\\
	\captionof{table}{Untypische Wortartenkombinationen in der englischen Sprache.}
	\label{tab:wortkombinationen}
\end{center}
In der Tabelle \ref{tab:wortkombinationen} sind einige Beispiele für solche untypischen Zusammensetzungen. In Spalte eins sind die jeweiligen Wortarten beschrieben. Die zweite Spalte zeigt die dazugehörigen Tags, wie diese mit \emph{POS-Tagging} ermittelt werden. Die dritte Spalte zeigt jeweils ein Beispiel dieser Wortkombinationen. Die Identifikation erfolgte durch Betrachtung der Grundregeln der Englischen Grammatik aus der Arbeit \citetitle{ogden1930basic}\cite{ogden1930basic} und den Syntaxregeln aus der Arbeit \citetitle{tayal2014syntax}\cite{tayal2014syntax}. Diese Regeln beinhalten Kombinationen aus Wortarten, die zusammen Erlaubt sind. Dadurch konnten Kombinationen ausgeschlossen werden. Schließlich wurden durch trial and error weitere Kombinationen identifiziert, die schließlich in die Tabelle aufgenommen wurden. Es kann nicht ausgeschlossen werden, dass weitere Kombinationen und Sonderfälle existieren. Dies stellt ein Optimierungspotenzial außerhalb dieser Ausarbeitung dar. Somit werden nur die Kombinationen aus der Tabelle \ref{tab:wortkombinationen} im System beachtet. Dennoch ist dieses so konzipiert, dass weitere Kombinationen in das System hinzugefügt werden können.\\

Zur Implementierung der Methode \emph{POS-Tagging} wurde die Bibliothek \emph{nltk} verwendet. Die \emph{Tags} werden mit dem vortrainierten Modell \emph{averaged\_perceptron\_tagger} ermittelt.
%\begin{lstlisting}[caption={Implementation des Moduls \emph{posTagging.py}}, label=lst:postagging]
%	import nltk
%	nltk.download('averaged_perceptron_tagger')
%	def usePosTagging(text):
%		tokens = nltk.word_tokenize(text)
%		pos_tags = nltk.pos_tag(tokens)
%		return pos_tags
%\end{lstlisting}
%Im Listing \ref{lst:postagging} sind Details zur Implementierung dargestellt. 
Als Parameter wird ein String übergeben, der zwei Wörter aus den \emph{Bi-Grammen} enthält. Dieser String wird in Tokens umgewandelt. Diese werden in die Methode \lstinline{pos_tag()}
übergeben und eine Liste mit Tupeln wird zurückgegeben, bei dem das Wort und der dazugehörige \emph{Tag} enthalten ist.
\paragraph{Entfernung von Bi-Grammen mit untypischen Wortartenkombinationen}\mbox{}\\
Die \emph{Bi-Gramm}-Liste muss im nächsten Schritt weiter reduziert werden. Hierbei werden alle \emph{POS-Tagging}-Kombinationen aus der Tabelle \ref{tab:wortkombinationen} mit den Tupeln aus der \emph{Bi-Gramm}-Liste verglichen.
\begin{lstlisting}[caption={Implementation der Filterung von Wortartenkombinationen}, label=lst:wortarten]
	combinationsToRemove = ["NN JJ", "VB JJ", "VB PRP", "VB VB", "JJ JJ", "JJ VB", "JJ RB", "RB JJ"]
	def removeWordCombination(biGramList):
		filteredList = []
		for left, right in biGramList:
			biGramString = f"{left} {right}"
			biGramStringsWithTags = usePosTagging(biGramString)
			tagString = ' '.join(tag for _, tag in biGramStringsWithTags)
			if tagString not in combinationsToRemove:
				filteredList.append((left, right))
		return filteredList
\end{lstlisting}
Das Listing \ref{lst:wortarten} implementiert diesen Schritt. Dazu werden in Zeile 1 alle Kombinationen als String in eine Liste zwischengespeichert. Als Parameter wird die zu filternde \emph{Bi-Gramm}-Liste übergeben. Diese wird in Zeile 4 traversiert und die Tupel mit den beiden Wörtern aus den \emph{Bi-Grammen} wird in Zeile 5 als String zusammengefügt. Dieser String wird in die \emph{POS-Tagging}-Methode übergeben und als Rückgabewert erhält das System eine Liste mit zwei Tupeln. Diese Tupel beinhalten jeweils eines der Wörter und den dazugehörigen Tag aus dem \emph{POS-Tagging}. Die beiden Tags werden in Zeile 7 zusammengefügt, sodass ein String entsteht, wie es in der \emph{combinationsToRemove}-Liste in Zeile 1 ist. In Zeile 8 wird überprüft ob die Kombination aus Tags eines aus der Liste in Zeile 1 entspricht. Ist dies nicht der Fall, wird der Tupel in eine neue Liste eingefügt, die am Ende als Ausgabe zurückgegeben wird.
\paragraph{Überführung von Bi-Grammen in Stichpunkte}\label{par:stichpunkte}\mbox{}\\
Da der Volltext aus den Bedarfsmeldungen nicht einzelne Wörter, sondern inhaltlich aufeinander bezogene Stichpunkte entstehen lassen soll, werden die um die Schlüsselwörter herumliegenden Wörter aneinander gefügt. Dazu werden diejenigen \emph{Bi-Gramme} zu einem Satz zusammengefügt, die im ersten Tupel auf der rechten Seite und im zweiten Tupel auf der linken Seite das gleiche Wort enthalten. Hinzu kommt eine Überprüfung ob das darauffolgende linke Wort ein Schlüsselwort ist. Somit werden nur die Wörter zu Sätzen verkettet, die aus Schlüsselwörtern bestehen. In dem Fall, bei dem im darauf folgenden Tupel kein Schlüsselwort auf der linken sondern nur auf der rechten besteht, entsteht ein neuer Stichpunkt. Als Beispiel können folgende \emph{Bi-Gramme} betrachtet werden: (\grqq expertise\grqq, \grqq aws\grqq) (\grqq aws\grqq, \grqq technologie\grqq) (\grqq technologie\grqq, \grqq pflicht\grqq) (\grqq pflicht\grqq, \grqq englisch\grqq) (\grqq englisch\grqq, \grqq skills\grqq). Als Schlüsselwörter wurden die Wörter \grqq aws\grqq, \grqq pflicht\grqq, \grqq englisch\grqq ermittelt. Die Idee ist es, die zusammengehörigen \emph{Bi-Gramme} an diesen Schlüsselwörtern zusammenzufügen. Daraus werden die Stichpunkte \grqq expertise aws technologie\grqq und \grqq technologie pflicht englisch skills\grqq.
\begin{lstlisting}[caption={Umformung der Bi-Gramm Liste in Stichpunkte}, label=lst:stichpunkte]
	def combineWords(filteredBiGram, keywords):
		combinedWords = []
		i = 0
		while i < len(filteredBiGram):
			keyPhrase = recursiveCombine(filteredBiGram, i, "", keywords, "")
			combinedWords.append(keyPhrase)
			wordsCount = len(keyPhrase.split(" "))
			wordsCount = max(1, wordsCount - 1)
			i += wordsCount
			if(i > len(filteredBiGram)):
				break
		return combinedWords
	
	def recursiveCombine(filteredBiGram, index, currentString, keywords, lastRightItem):
		left, right = filteredBiGram[index]
		if not currentString:
			currentString = left
		else:
			currentString += " " + left
		if lastRightItem in {left, ""}:
			if right in keywords and index + 1 < len(filteredBiGram):
				currentString = recursivCombine(filteredBiGram, index + 1, currentString, keywords, right)
			else:
				currentString += " " + right
		return currentString
\end{lstlisting}
Die Implementierung zu dieser Idee ist im Listing \ref{lst:stichpunkte} dargestellt. Die Methode \lstinline{combineWords} in Zeile 1 erhält die \emph{Bi-Gramm}-Liste und die Schlüsselwörterliste als Parameter. Die \emph{Bi-Gramm}-Liste wird traversiert und ein string aus zusammengehörigen Wörtern wird rekursiv in Zeile 5 mit der Methode \lstinline{recursiveCombine} erstellt. Dazu wird in jedem rekursiven Schritt überprüft, ob das darauf folgende rechte wort ein Schlüsselwort und das gleiche wie das linke Wort im aktuellen Tupel entspricht. Ist dies der Fall werden die \emph{bi-Gramm} an der Stelle zusammengefügt. Dies wiederholt sich bis dieser Fall nicht mehr eintritt. Dadurch ist die Rekursion vorbei und es wird in Zeile 7 bis 11 geprüft wie viele Wörter in einem String gelandet sind. Das ist wichtig um herauszufinden an welcher Stelle in der \emph{bi-Gramm}-Liste weiter gearbeitet werden muss. Sind alle \emph{bi-Gramme} zusammengesetzt worden, wird eine Liste mit allen zusammengesetzten Stichpunkten zurückgegeben.
\section{Erklärung des Systems anhand eines Beispiels}
Nachfolgend wird zum besseren Verständnis die Schritte zur Verarbeitung eines Volltextes anhand eines konkreten Beispiels dargestellt.\\
\begin{enumerate}
	\itemsep-0.15em
	\item Zu Beginn haben wir folgenden Volltext:\\ \textbf{Für verschiedene IT-Projekte (mit internen und externen Kunden) benötigen wir Unterstützung von einem Projektleiter, mit Erfahrungen in der Planung, Koordination und Durchführung von IT-Migrations- und Implementierungsprojekten im Umfeld von Rechenzentrumsservices. Mindestens 2 Jahre Erfahrung sind erwünscht.}
	\item Dieser Text wird übersetzt:\\ \textbf{For various IT projects (with internal and external customers) we need support from a project manager with experience in planning, coordinating and executing IT migration and implementation projects in the area of data center services. At least 2 years of experience is desired.}
	\item Nun werden aus dem Text Datums- und Zeitangaben extrahiert und aus dem Volltext entnommen: \grqq At least 2 years\grqq \\ Der neue Text beinhaltet: \textbf{For various IT projects (with internal and external customers) we need support from a project manager with experience in planning, coordinating and executing IT migration and implementation projects in the area of data center services. of experience is desired.}
	\item Als nächstes wird der Text Vorverarbeitet:\\ \textbf{for various it projects internal external customers need support project manager experience planning coordinating executing it migration implementation projects area data center services experience desired}
	\item Die Schlüsselwörter bestehen unter anderem aus:\\ \textbf{planning, coordinating, migration, implementation, data, center, services, ...}
	\item Aus dem Volltext werden diese \emph{bi-Gramme} erstellt:\\ \textbf{('for', 'various'), ('various', 'it'), ('it', 'projects'), ('projects', 'internal'), ('internal', 'external'), ('external', 'customers'), ('customers', 'need'), ('need', 'support'), ('support', 'project'), ('project', 'manager'), ('manager', 'experience'), ('experience', 'planning'), ('planning', 'coordinating'),
	('coordinating', 'executing'), ('executing', 'it'), ('it', 'migration'), ('migration', 'implementation'), ('implementation', 'projects'), ('projects', 'area'), ('area', 'data'), ('data', 'center'), ('center', 'services'), ('services', 'experience'), ('experience', 'desired')}
	\item Anschließend werden alle \emph{bi-Gramme} ohne Schlüsselwort entfernt:\\ \textbf{('support', 'project'), ('project', 'manager'), ('manager', 'experience'), ('experience', 'planning'), ('planning', 'coordinating'), ('coordinating', 'executing'), ('it', 'migration'), ('migration', 'implementation'), ('implementation', 'projects'), ('area', 'data'), ('data', 'center'), ('center', 'services'), ('services', 'experience'), ('experience', 'desired')}
	\item Der nächste Schritt beinhaltet die Entfernung von untypischen \emph{POS-Tagging}-Kombinationen. In diesem Beispiel sind alle Kombinationen zulässig, weswegen die \emph{bi-Gramm}-Liste unverändert bleibt.
	\item Nun werden die \emph{bi-Gramme} nach dem Verfahren aus Kapitel \ref{par:stichpunkte} in folgende Stichpunkte zusammengetragen:\\ \textbf{support project manager, manager experience planning coordinating executing, it migration implementation projects, area data Center services experience desired}.\\ \\ Hinzu kommen die Daten aus dem \emph{NER}-Schritt:\\ \textbf{At least 2 years}
	\item Nach der Übersetzung ist das Ergebnis: \begin{itemize}\itemsep-0.3em\item Unterstützung des Projektleiters \item Manager Erfahrung Planung Koordinierung Durchführung \item Projekte zur Implementierung von IT-Migrationen \item Erfahrung im Bereich Rechenzentrumsdienstleistungen erwünscht \item Mindestens 2 Jahre \end{itemize}
\end{enumerate}

%\chapter{Umsetzung der Visualisierungsmethoden}
\label{chap:implementation}
-Wie können diese Art und Weisen mit dem „adesso Staffing Advisor“ Lab implementiert werden

\section{World Cloud}

\newpage

\chapter{Evaluierung des entwickelten Systems}
\label{chap:evaluation}

vielleicht erklären warum precision, recall, f1 score nicht gehen -\\

nicht überlegen wie evaluieren sonder was will ich evaluieren,\\
was sind die fragen die ich beantworten möchte, was sind die aussagen die ich machen will. hypothesen belegen, wiederlegen\\
was möchte ich zeigen, (den expertenprozess abbilden, expertenprozess ist ideal, mein prozess hat diese abweichung)\\

z.b. erwartungshaltung formulieren und mit cosine similarity gucken was näher dran ist,

wie machen das andere ansätze,

\section{Evaluationsmetrik}
-cosine similarity
-performance, zeit

\section{Beschreibung des verwendeten Datensatzes}
\todo{darauf eingehen welche Felder in der JSON von JIRA sind und worauf sich genau konzentriert wird.}\\
\newpage

\section{Beschreibung des verwendeten Datensatzes}

überlegung ob tfidf unterschied macht alle bedarfsmeldungen mit einer zu vergleichen und daraus wichtige wörter identifizieren oder eine für sich alleine reicht.

gucken was tokenisierung wirklich macht
\section{Präsentation und Diskussion der Ergebnisse}
\newpage
g
\newpage
g
\newpage
g
\newpage
Zeit und Leistung Übersicht
\newpage
g
\newpage

\section{Vergleich des Systems mit einem Large Language Model-Ansatz}
\newpage
g
\newpage
g
\newpage

g
\newpage

\section{Analyse von Abweichungen, Ähnlichkeiten und Verbesserungspotenzialen des Systems}
\newpage
g
\newpage

\chapter{Zusammenfassung und Ausblick}
\label{chap:ergebnisseausblick}

ergebnis der arbeit: diese modelle in der reihenfolge kommen am nähesten an die bedarfsmeldung

\subsection*{Ausblick}
-keywords präzisieren und keywordskatalog anlegen
-genauer untersuchen wie einzelne ansätze mit anderen nlp vortrainierten datensätzen abschneidet
-in bezug zu recommender systems beschreiben wie damit weiter gemacht werden könnte, die keyword extraction auch für die profile nutzen 
\newpage


%%\addcontentsline{toc}{section}{Eidesstattliche Erklärung}
\hiddenchapter{Eidesstattliche Erklärung}
Hiermit erkläre ich, dass ich die vorliegende Arbeit selbstständig
und ohne Benutzung anderer als der angegebenen Hilfsmittel angefertigt
sowie die aus fremden Quellen direkt oder indirekt übernommenen
Gedanken als solche kenntlich gemacht habe. \\ \\
Die Arbeit wurde bisher in gleicher oder ähnlicher Form keiner
anderen Prüfungsbehörde vorgelegt und auch noch nicht veröffentlicht. \\ \\
Dortmund, den \today \\ \\ \\
\begin{tabular}{@{}l@{}}\hline
	Ricardo Valente de Matos
\end{tabular}
\setcounter{page}{6}

%\bibliographystyle{IEEEtranS}
%\bibliography{Literatur}
\raggedright
\printbibliography

\end{document}
