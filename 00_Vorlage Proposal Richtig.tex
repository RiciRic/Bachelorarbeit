%Dokumentklasse
\documentclass[a4paper,12pt]{scrreprt}
\usepackage[left= 3.5cm,right = 2cm, bottom = 2 cm]{geometry}
\addtolength{\footskip}{-0.5cm}
\usepackage[onehalfspacing]{setspace}
% ============= Packages =============

% Dokumentinformationen
\usepackage[hyphens]{url}
\usepackage[
pdfsubject={},
pdfauthor={Ricardo Valente de Matos},
pdfkeywords={},	
%Links nicht einrahmen
hidelinks,
breaklinks=true
]{hyperref}
% Standard Packages
\usepackage[utf8]{inputenc}
\usepackage[ngerman]{babel}
\usepackage[T1]{fontenc}
\usepackage{graphicx, subfig}
\graphicspath{{img/}}
\usepackage{fancyhdr}
\usepackage{lmodern}
\usepackage{color}

\usepackage{dirtree}

\usepackage[style=ieee, sorting=nty, urldate =comp, backend=bibtex]{biblatex}
\addbibresource{Literatur.bib}

%\usepackage[numbers]{natbib}
%\bibpunct{(}{)}{;}{a}{,}{,}

\usepackage{listings}

% zusätzliche Schriftzeichen der American Mathematical Society
\usepackage{amsfonts}
\usepackage{amsmath}
\usepackage{float}

\usepackage{tabularx}
\usepackage{multirow}
\usepackage{enumitem}

%nicht einrücken nach Absatz
\setlength{\parindent}{0pt}
\RedeclareSectionCommand[beforeskip=0pt]{chapter}
\usepackage{listings}
\usepackage{color}
\definecolor{lightgray}{rgb}{.9,.9,.9}
\definecolor{darkgray}{rgb}{.4,.4,.4}
\definecolor{purple}{rgb}{0.65, 0.12, 0.82}

\lstdefinelanguage{JavaScript}{
	keywords={typeof, new, true, false, catch, function, return, null, catch, switch, var, if, in, while, do, else, case, break},
	keywordstyle=\color{blue}\bfseries,
	ndkeywords={class, export, boolean, throw, implements, import, this},
	ndkeywordstyle=\color{darkgray}\bfseries,
	identifierstyle=\color{black},
	sensitive=false,
	comment=[l]{//},
	morecomment=[s]{/*}{*/},
	commentstyle=\color{purple}\ttfamily,
	stringstyle=\color{red}\ttfamily,
	morestring=[b]',
	morestring=[b]"
}

\lstset{
	language=JavaScript,
	backgroundcolor=\color{lightgray},
	extendedchars=true,
	basicstyle=\footnotesize\ttfamily,
	showstringspaces=false,
	showspaces=false,
	numbers=left,
	numberstyle=\footnotesize,
	numbersep=9pt,
	tabsize=2,
	breaklines=true,
	showtabs=false,
	captionpos=b
}


% ============= Kopf- und Fußzeile =============
\pagestyle{fancy}
%
\lhead{}
\chead{}
\rhead{\slshape \leftmark}
%%
\lfoot{}
\cfoot{\thepage}
\rfoot{}
%%
\renewcommand{\headrulewidth}{0.4pt}
\renewcommand{\footrulewidth}{0pt}

% ============= Package Einstellungen & Sonstiges ============= 
%Besondere Trennungen
\hyphenation{De-zi-mal-tren-nung}

\newcommand{\hiddenchapter}[1]{
	\chapter*{{#1}}
}

%-------------

\newcommand{\todo}[1]{\textcolor{red}{ToDo:} #1\marginpar{<--hier}}

% ============= Dokumentbeginn =============

\begin{document}
%Seiten ohne Kopf- und Fußzeile sowie Seitenzahl
\pagestyle{empty}

\begin{center}
	\begin{tabular}{p{\textwidth}}
		
		\begin{center}
			\textbf{\Large{Bachelorarbeit}}
		\end{center} \\ \\
		
		\begin{center}
			\LARGE{\textsc{
					%\textit{\emph{adesso Staffing Advisor Lab}}\\
					%Konzeption und prototypische Entwicklung der Struktur und Architektur einer Softwareplattform für Transparenz in KI-Anwendungen
					...
			}}
		\end{center}
		
		\\
		
		
		
		\begin{center}
			von
		\end{center}
		
		\begin{center}
			\large{\textbf{Ricardo Valente de Matos}}
		\end{center}
	
	\begin{center}
		\large{Matrikelnummer: 7203677} \\
		\large{im Studiengang Wirtschaftsinformatik \\
			der Fachhochschule Dortmund \\}
	\end{center}
		
		
		\\
		
		\\
		
		\begin{center}
			\begin{tabular}{lll}
				\textbf{Erstprüfer:} & & Prof. Dr.-Ing. Guy Vollmer\\
				\textbf{Zweitprüfer:} & & Stephan Schmeißer, M. Sc., Adessoplatz 1, 44269 Dortmund\\
			\end{tabular}
		\end{center}
	
	\\ \\
	
	\begin{center}
		\large{Dortmund, den \today}
	\end{center}
		
	\end{tabular}
\end{center}

%\setcounter{page}{1}
%\pagestyle{plain}

%letztes kapitel zusammenfassung und ausblick
\pagestyle{fancy}
\pagenumbering{Roman}
%\tableofcontents


%\listoffigures
%\lstlistoflistings
\newpage
%-NOCH ZU ERLEDIGEN-\\

%\chapter*{Lesehinweis}
%Aus Gründen der besseren Lesbarkeit werden Wörter und Wortgruppen, die hervorgehoben werden oder mehrfach auftauchen, durch \emph{kursiven} Text kenntlich gemacht. Zudem wird in dieser Projektarbeit die Sprachform des generischen Maskulinums angewandt. Sämtliche Ausführungen sind jedoch geschlechtsunabhängig und beziehen sich damit auf alle Geschlechter.
\newpage

\setcounter{page}{1}
\pagestyle{fancy}
\pagenumbering{arabic}
\setcounter{chapter}{0}
\newpage

\hiddenchapter{Motivation}
Die Suche nach qualifizierten Mitarbeitern ist für Unternehmen von entscheidender Bedeutung, um wettbewerbsfähig zu bleiben und langfristigen Erfolg zu sichern. In einer Zeit, in der der Arbeitsmarkt zunehmend global und dynamisch wird, stehen Organisationen vor der Herausforderung, aus einer Vielzahl von Mitarbeitern diejenigen zu identifizieren, die am besten zu einem spezifischen Projekt im Unternehmen passen. Hier setzt die Entwicklung eines Recommender Systems zur Mitarbeiterempfehlung an. Ein solches System kann Unternehmen dabei unterstützen, den Prozess der Mitarbeiterrekrutierung und -auswahl zu optimieren. Durch die Berücksichtigung verschiedener Kriterien wie Qualifikationen, Fähigkeiten, Erfahrungen kann das Recommender-System dazu beitragen, die Auswahl effektiv zu filtern und diejenigen herauszufiltern, die am besten zu einem Projekt im Unternehmen passen. Ein solches System bietet außerdem den Vorteil, den Prozess der Mitarbeiterempfehlung zu automatisieren und zu beschleunigen. Dies ermöglicht Unternehmen, schneller auf offene Stellen zu reagieren und potenzielle Kandidaten zeitnah zu identifizieren. Dadurch wird die Effizienz der Mitarbeitersuche verbessert und die Qualität der Einstellungsentscheidungen erhöht.\\

Das Potenzial von Recommender Systems wurde auch bei \emph{adesso} entdeckt und nun wird nach und nach Wege gesucht, KI-gestützte Systeme in die eigenen Prozesse zu integrieren. Im internen Projekt \emph{adesso Staffing Advisor} wird an einem Recommender-System zur Mitarbeiterempfehlung für ausgewählte Projekte gearbeitet. Die Umsetzung der Recommender Systems bedient sich verschiedener KI-basierten Ansätze. Ein ganz entscheidender Schritt im Prozess der Mitarbeiterempfehlung ist die Vorverarbeitung der Bedarfsmeldungen. Diese sind eine wertvolle Informationsquelle, die Fachkräften helfen kann, die Empfehlungen effizienter zu gestalten, um dadurch wettbewerbsfähig zu bleiben. Allerdings sind diese oft umfangreich und komplex, was ihre effektive Nutzung erschwert.\\

Deshalb ist es entscheidend, effiziente Methoden und Techniken des Information Retrieval anzuwenden, um so relevante Informationen schnell und präzise aus Bedarfsmeldungen zu extrahieren. Die Extraktion wichtiger Schlüsselwörter, Phrasen und Themen ermöglicht es einen besseren Einblick in die Ziele, Methoden und Ergebnisse der Projekte zu bekommen. Dadurch können fundierte Entscheidungen bezüglich der Personalbesetzung getroffen und Ressourcen effizient genutzt werden.\\
\hiddenchapter{Problemstellung}
In einer immer stärker vernetzten und informationsreichen Welt stehen Organisationen vor der Herausforderung, relevante Informationen effizient aus umfangreichen Bedarfsmeldungen zu extrahieren. Obwohl diese Beschreibungen wichtige Einblicke in Ziele, Methoden und Ergebnisse liefern, können sie aufgrund ihres Umfangs und ihrer Komplexität schwer durchsuchbar und analysierbar sein. Die manuelle Identifizierung und Extraktion relevanter Inhalte ist zeitaufwendig und fehleranfällig. Daher stellt sich die Problemstellung: \\

Wie können wir effektive Methoden und Techniken des Information Retrieval und Data-Mining nutzen, um automatisiert relevante Inhalte aus Bedarfsmeldungen im spezifischen Software Entwicklungs-Kontext zu extrahieren und somit die Effizienz, Genauigkeit und Geschwindigkeit der Informationsgewinnung für Führungskräfte zu verbessern.
\hiddenchapter{Ziele und Ergebnisse der Arbeit}
Diese Ausarbeitung präsentiert eine umfassende Untersuchung zur Entwicklung eines automatisierten Systems zur Extraktion relevanter Inhalte aus Bedarfsmeldungen im Software-Entwicklungs-Kontext.
\begin{itemize}
	\item Die erste Phase dieser Ausarbeitung besteht darin, eine klare Erwartungshaltung hinsichtlich der Anforderungen und Bedürfnisse der Stakeholder zu entwickeln. Hierfür werden Interviews mit Führungskräften durchgeführt, um die Erwartungen bezüglich einer \glqq{}perfekten\grqq{} Bedarfsmeldung herauszuarbeiten. Diese dient als Grundlage für die weiteren Entwicklungs- und Evaluierungsphasen.
	\item Im Anschluss erfolgt eine eingehende Analyse der Techniken <was für Techniken> des Information Retrieval und Data-Mining, um die besten Ansätze zur Extraktion relevanter Inhalte zu identifizieren. Diese Analyse bildet die Grundlage für die Konzeptionierung einer Vorverarbeitung, das eine Kombination der erforschten Ergebnisse darstellt. Die Implementierung dieses Modells erfolgt durch den Aufbau einer Pipeline in Python, die eine effiziente Verarbeitung und Extraktion der Bedarfsmeldungen ermöglicht.
	\item Zur Evaluierung der Leistungsfähigkeit des entwickelten Systems werden reale Bedarfsmeldungen und Mitarbeiterinformationen verwendet. Dabei wird überprüft, inwiefern das Ergebnis der definierten Erwartungshaltung entspricht. Mithilfe von den Metriken \emph{Precision}, \emph{Recall} und \emph{F1-Score} werden Abweichungen, Ähnlichkeiten und Anpassungen in Parametern analysiert, um Erkenntnisse darüber zu gewinnen, wie das System inhaltlich abschneidet und verbessert werden kann.
	\item (Schließlich wird eine vergleichende Untersuchung mit einem auf Large Language Model basierenden Vorverarbeitungsansatz durchgeführt. Dabei werden die Performance, Zeit und Ergebnisqualität des entwickelten Systems mit diesem alternativen Ansatz verglichen. Dieser Vergleich dient dazu, die Stärken und Schwächen des entwickelten Systems zu identifizieren und gegebenenfalls weitere Verbesserungen vorzunehmen.)
\end{itemize}

\hiddenchapter{Vorgehen und Zeitplan}
%handelt sich um Projektarbeit, peile bis  ende august bsw an
Ziel ist es die Arbeit im Mai fertig zu stellen. Die einzelnen Monatsziele können aus der nachfolgenden Tabelle entnommen werden. \\ \\
\begin{tabularx}{1\textwidth} { 
		| >{\raggedright\arraybackslash}X 
		| >{\raggedright\arraybackslash}X | }
	\hline
	Februar
	& \begin{itemize}
		\item Durchführung der Interviews mit Fachkräften
		\item Zusammentragung aller relevanter Information Retrieval- und Preprocessing-Ansätze 
	\end{itemize}\\
	\hline
	März
	& \begin{itemize}
		\item Durchführung der Interviews mit Fachkräften
		\item Formulierung der Anforderungen für Bedarfsmeldungen
	\end{itemize}\\
	\hline
	April
	& \begin{itemize}
		\item Entwicklung des Eigenen Preprocessing-Modells
		\item Evaluierung der Ergebnisse
	\end{itemize}\\
	\hline
	Mai
	& \begin{itemize}
		\item Schluss schreiben
		\item Korrekturen
	\end{itemize}\\
	\hline
\end{tabularx}
\newpage

\renewcommand\contentsname{Aufbau der Arbeit}
\todo{Aufbau der Arbeit anpassen}
\tableofcontents

\cite{kobayashi2000information}

\cite{singhal2001modern}

\cite{croft2000combining}

\cite{horesh2016information}

\cite{belkin1992information}

information filtering
\cite{lanquillon2001enhancing}

preprocessing
\cite{alasadi2017review}

-------
spam-filter
\cite{shafi2017review}
\cite{khorsi2007overview}
\cite{tretyakov2004machine}
----
TF-IDF (Term Frequency-Inverse Document Frequency): TF-IDF ist eine statistische Methode, die verwendet wird, um die Relevanz eines Begriffs in einem Dokument relativ zu einem Korpus von Dokumenten zu bestimmen. Wörter mit höheren TF-IDF-Werten werden als potenzielle Schlüsselwörter betrachtet.
\cite{bafna2016document}
\cite{ramos2003using}

Text-Ranking-Algorithmen: Text-Ranking-Algorithmen wie TextRank oder YAKE (Yet Another Keyword Extractor) verwenden Graphen-basierte Methoden, um Schlüsselwörter in einem Text zu identifizieren. Diese Algorithmen bewerten die Wichtigkeit von Wörtern basierend auf ihrer Verbindung zu anderen Wörtern im Text und extrahieren Schlüsselwörter entsprechend ihrer Rangfolge.
\cite{mihalcea2004textrank}
\cite{zhang2020empirical}
\cite{pay2019ensemble}

N-Gramm-Analyse: N-Gramme sind Sequenzen von N aufeinanderfolgenden Wörtern in einem Text. Durch die Analyse von N-Grammen können häufig auftretende Phrasen oder Begriffe identifiziert werden, die potenzielle Schlüsselwörter darstellen.
\cite{pirk2019implementierung}


Part-of-Speech (POS) Tagging: POS-Tagging wird verwendet, um die grammatischen Kategorien von Wörtern in einem Text zu bestimmen. Durch die Berücksichtigung von Wörtern mit bestimmten POS-Tags wie Substantiven oder Adjektiven können relevante Schlüsselwörter extrahiert werden.
\cite{kumawat2015pos}
\cite{nakagawa2007hybrid}

bekommen wir ein unsupervised learning ansatz der 

Regelbasierte Ansätze: Regelbasierte Ansätze verwenden vordefinierte Regeln oder Muster, um Schlüsselwörter zu identifizieren. Dies kann beispielsweise das Extrahieren von Wörtern sein, die häufig im Text vorkommen oder bestimmten Mustern entsprechen.

Hybride Ansätze: Hybride Ansätze kombinieren verschiedene Methoden und Techniken, um eine genauere Extraktion von Schlüsselwörtern zu ermöglichen. Zum Beispiel könnte eine Kombination aus TF-IDF-Gewichtung und Text-Ranking-Algorithmen verwendet werden, um eine robuste Schlüsselwortextraktion zu erreichen.


transformation von bedarfsmeldung zu guter bedarfsmeldung, was ist der fokus von der bedarfsmeldung, wie gut machen die ansätze das, und muss man das dann noch weiter verarbeiten, haben wir alles was wir brauchen mit nur einem algorithmus, inferenz falls parameter fehlt, gibt es einen der alles löst,

ergebnis der arbeit: diese modelle in der reihenfolge kommen am nähesten an die bedarfsmeldung

ausblick: die keyword extraction auch für die profile nutzen 

was muss ich jetzt machen: gucken wie ich das inhaltlich genau machen will, also pipeline genauch checken, quellen von der bachelorarbeit checken

\chapter{Einleitung}
\label{chap:einleitung}

\newpage
\section{Problemstellung}
\label{sec:problemstellung}

\newpage
\section{Ziele und Ergebnisse der Arbeit}
\label{sec:zieleundergebnis}

\section{Aufbau der Arbeit}

\newpage

\chapter{Grundlagen}
\label{chap:grundlagen}

\section{Künstliche Intelligenz}
\label{sec:kunstlicheintelligenz}

1. Starke KI beinhaltet Problemlösungen genereller Art. Das, was am Ehesten an sowas heran kommt ist ChatGPT. Dennoch ist das Konzept einer starken KI ein Produkt aus Science-Fiction. Die Idee ist, dass die Maschine eine Art Bewusstsein hat und ein selbstständiges Verständnis unterschiedlicher Wissensbereiche entwickelt. 
2. Schwache KI beinhaltet meist die Problemlösung einer konkreten Art. KI ist ein Konstrukt aus komplexen Algorithmen. Wenn von KI gesprochen wird, ist immer eine schwache KI gemeint.

\section{Recommender Systems}
\label{sec:recommendersystems}

\section{Warum Testen und Überwachen der KI}
\label{sec:testen}

1. KI-Systeme übernehmen bereits kritische Aufgaben. Identifizierung von Unfällen, Feuer oder Naturkatastrophen sind Aufgaben, die von einer KI schneller, besser und effizienter erledigt werden kann. Bei kritischen Prozessen ist es wichtig, dass die KI die vorgesehenden Leistungen erbringt.
\newpage







\chapter{Verwandte Arbeiten}
\label{chap:verwandtearbeiten}

\newpage

\chapter{Adesso Staffing Advisor}
\label{chap:staffingadvisor}

\section{Aufbau des Projekts}
\label{sec:ausgangssituation}

\section{Preprocessing}
\label{sec:preprocessing}

\subsection{Keyword-Extraction}

\subsection{Normalizing}

\subsection{Large Language Models}

\section{KI-Modelle}
\label{sec:similaritycalculation}

\subsection{spacy}

\subsection{sbert}

\section{Nutzung von Daten}
\label{chap:ui}

-Welche Informationen der Ergebnisse des KI-Ansatzes sind vorhanden und werden gebraucht. (preprocessing, Ergebnis, similarity-Werte)

\subsection{Welche Daten werden in die KI gegeben}

\subsection{Welche Daten werden von den KI-Ansätzen erstellt}

\subsection{Welche Daten werden von der KI zurückgegeben}

\newpage

\chapter{Ähnlichkeitsmetriken}
\label{chap:lösungsansatz}

Ähnlichkeitsmetriken:

Überprüfe die Genauigkeit der Ähnlichkeitsmetriken, die im Recommender-System verwendet werden. Dazu gehören beispielsweise Kosinus-Ähnlichkeit, Pearson-Korrelation, Jaccard-Ähnlichkeit oder andere, je nach Kontext.

Top-N-Empfehlungen:
Evaluieren Sie, wie gut das Recommender-System in der Lage ist, relevante Elemente unter den Top-N-Empfehlungen zu platzieren. Dies ist eine gängige Metrik, um die praktische Anwendbarkeit des Systems zu bewerten.

--------------------------
Repräsentation der Merkmale:
Untersuche, wie gut die Merkmale (Features) der Elemente im System repräsentiert sind. Eine gute Ähnlichkeitsberechnung hängt oft von der Qualität und Relevanz der Merkmale ab.


Diversität der Empfehlungen:

Prüfe, ob die Ähnlichkeitsbasierten Empfehlungen zu vielfältig sind. Eine zu starke Konzentration auf ähnliche Elemente könnte zu eintönigen Empfehlungen führen.
Benutzerbewertungen und Feedback:

Integriere Benutzerbewertungen und -feedback in die Evaluierung, um sicherzustellen, dass die Ähnlichkeitsberechnungen den tatsächlichen Vorlieben der Benutzer entsprechen.
Cold Start-Szenarien:

Teste das System unter Bedingungen des "Cold Start", um sicherzustellen, dass es auch effektive Empfehlungen machen kann, wenn es nur begrenzte Daten gibt.
Auswirkungen von Merkmalen:

Analysiere, wie sich das Hinzufügen oder Entfernen von Merkmalen auf die Empfehlungen auswirkt. Dies kann helfen, die Sensitivität des Systems gegenüber verschiedenen Merkmalen zu verstehen.
Nutzerinteraktion:

%Untersuche, wie gut das Recommender-System auf Veränderungen in der Benutzerinteraktion reagiert. Dies könnte Änderungen in den Präferenzen der Benutzer oder neue Interaktionen mit dem System einschließen.
Es ist wichtig, die spezifischen Anforderungen deines Recommender-Systems zu berücksichtigen und die Evaluierungsmethoden entsprechend anzupassen. Kombiniere mehrere Metriken, um ein umfassenderes Bild der Leistung des Systems zu erhalten.

%-Welche Art und Weisen des Testens und der Überwachung existieren (preprocessing/similarity visualisieren, Ergebnis und Input gegenüberstellen, etc.)

%Keywords visualisieren: word cloud, Circle packing, The horn of plenty, Treemap, donut chart, Grid of bar charts

%Daten gegenüberstellen: Alle Eingabedaten und alle Mitarbeiterinformationen nebeneinander
%Vielleicht mit einer „similarity“-Matrix.
%Vielleicht Gegenüberstellung von Ergebnissen eines supervised und unsupervised Ansatzes. (Matrix)
%Visualisierung von Daten:
%1.	Kreisdiagramm
%2.	Balkendiagramm
%3.	Säulendiagramm
%4.	Kurvendiagramm
%5.	Punktediagramm
%Vergleichstypen:
%1.	Strukturvergleich(Welcher Anteil verschiedene Komponenten macht an einer Gesamtheit (z. B. in Prozent) aus)
%2.	Rangfolgevergleich(Die verschiedenen Objekte können im Vergleich zueinander z. B. kleiner, größer, besser, schlechter oder gleich sein)
%3.	Häufigkeitsvergleich(Größenklassen bilden)
%4.	Korrelationsvergleich(Vergleich, ob zwischen zwei Variablen ein quantitativer Zusammenhang bzw. 


\section{Genauigkeit der Ähnlichkeit}

\section{Qualität und Relevanz der Merkmale}

\section{Eintönige Empfehlungen}

\section{Benutzerbewertungen und -feedback}

\section{Cold Start}

\section{Sensitivität des Systems}
\newpage

%\chapter{Umsetzung der Visualisierungsmethoden}
\label{chap:implementation}
-Wie können diese Art und Weisen mit dem „adesso Staffing Advisor“ Lab implementiert werden

\section{World Cloud}

\newpage

\chapter{Evaluierung des entwickelten Systems}
\label{chap:evaluation}

vielleicht erklären warum precision, recall, f1 score nicht gehen -\\

nicht überlegen wie evaluieren sonder was will ich evaluieren,\\
was sind die fragen die ich beantworten möchte, was sind die aussagen die ich machen will. hypothesen belegen, wiederlegen\\
was möchte ich zeigen, (den expertenprozess abbilden, expertenprozess ist ideal, mein prozess hat diese abweichung)\\

z.b. erwartungshaltung formulieren und mit cosine similarity gucken was näher dran ist,

wie machen das andere ansätze,

\section{Evaluationsmetrik}
-cosine similarity
-performance, zeit

\section{Beschreibung des verwendeten Datensatzes}
\todo{darauf eingehen welche Felder in der JSON von JIRA sind und worauf sich genau konzentriert wird.}\\
\newpage

\section{Beschreibung des verwendeten Datensatzes}

überlegung ob tfidf unterschied macht alle bedarfsmeldungen mit einer zu vergleichen und daraus wichtige wörter identifizieren oder eine für sich alleine reicht.

gucken was tokenisierung wirklich macht
\section{Präsentation und Diskussion der Ergebnisse}
\newpage
g
\newpage
g
\newpage
g
\newpage
Zeit und Leistung Übersicht
\newpage
g
\newpage

\section{Vergleich des Systems mit einem Large Language Model-Ansatz}
\newpage
g
\newpage
g
\newpage

g
\newpage

\section{Analyse von Abweichungen, Ähnlichkeiten und Verbesserungspotenzialen des Systems}
\newpage
g
\newpage

\chapter{Zusammenfassung und Ausblick}
\label{chap:ergebnisseausblick}

ergebnis der arbeit: diese modelle in der reihenfolge kommen am nähesten an die bedarfsmeldung

\subsection*{Ausblick}
-keywords präzisieren und keywordskatalog anlegen
-genauer untersuchen wie einzelne ansätze mit anderen nlp vortrainierten datensätzen abschneidet
-in bezug zu recommender systems beschreiben wie damit weiter gemacht werden könnte, die keyword extraction auch für die profile nutzen 
\newpage


%%\addcontentsline{toc}{section}{Eidesstattliche Erklärung}
\hiddenchapter{Eidesstattliche Erklärung}
Hiermit erkläre ich, dass ich die vorliegende Arbeit selbstständig
und ohne Benutzung anderer als der angegebenen Hilfsmittel angefertigt
sowie die aus fremden Quellen direkt oder indirekt übernommenen
Gedanken als solche kenntlich gemacht habe. \\ \\
Die Arbeit wurde bisher in gleicher oder ähnlicher Form keiner
anderen Prüfungsbehörde vorgelegt und auch noch nicht veröffentlicht. \\ \\
Dortmund, den \today \\ \\ \\
\begin{tabular}{@{}l@{}}\hline
	Ricardo Valente de Matos
\end{tabular}
\setcounter{page}{6}

%\bibliographystyle{IEEEtranS}
%\bibliography{Literatur}
\raggedright
\printbibliography

\end{document}
